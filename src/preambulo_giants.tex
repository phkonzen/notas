%Este trabalho está licenciado sob a Licença Atribuição-CompartilhaIgual 4.0 Internacional Creative Commons. Para visualizar uma cópia desta licença, visite http://creativecommons.org/licenses/by-sa/4.0/ ou mande uma carta para Creative Commons, PO Box 1866, Mountain View, CA 94042, USA.


% Giants

\newcommand{\arnoldi}{\endnote{Walter Edwin Arnoldi, 1917 - 1995, engenheiro americano estadunidense. Fonte: \href{https://pt.wikipedia.org/wiki/Walter_Edwin_Arnoldi}{Wikipédia}.}}

\newcommand{\bernoulli}{\endnote{Jacob Bernoulli, 1655-1705, matemático suíço. Fonte: \href{https://pt.wikipedia.org/wiki/Jakob_Bernoulli}{Wikipédia: Jakob Bernoulli}.}}

\newcommand{\bhaskara}{\endnote{Bhaskara Akaria, 1114 - 1185, matemático e astrônomo indiano. Fonte: \href{https://pt.wikipedia.org/wiki/Bhaskara\_II}{Wikipédia: Bhaskara II}.}}

\newcommand{\boole}{\endnote{George Boole, 1815 - 1864, matemático britânico. Fonte: \href{https://pt.wikipedia.org/wiki/George_Boole}{Wikipédia: George Boole}.}}

\newcommand{\burgers}{\endnote{Jan Burgers, 1895 - 1981, físico neerlandês. Fonte: \href{https://pt.wikipedia.org/wiki/Jan_Burgers}{Wikipédia: Jan Burgers}.}}

\newcommand{\cauchy}{\endnote{Augustin-Louis Cauchy, 1789-1857, matemático francês. Fonte: \href{https://pt.wikipedia.org/wiki/Augustin-Louis_Cauchy}{Wikipédia: Augustin-Louis Cauchy}.}}

\newcommand{\cotes}{\endnote{Roger Cotes, 1682 - 1716, matemático inglês. Fonte: \href{https://pt.wikipedia.org/wiki/Roger_Cotes}{Wikipédia: Roger Cotes}.}}

\newcommand{\cramer}{\endnote{Gabriel Cramer, 1704 - 1752, matemático suíço. Fonte: \href{https://pt.wikipedia.org/wiki/Gabriel_Cramer}{Wikipédia: Gabriel Cramer}.}}

\newcommand{\descartes}{\endnote{René Descartes, 1596 - 1650, matemático e filósofo francês.  Fonte: \href{https://pt.wikipedia.org/wiki/Ren\%C3\%A9_Descartes}{Wikipédia: René Descartes}.}}

\newcommand{\dirichlet}{\endnote{Johann Peter Gustav Lejeune Dirichlet, 1805 - 1859, matemático alemão. Fonte: \href{https://pt.wikipedia.org/wiki/Johann_Peter_Gustav_Lejeune_Dirichlet}{Wikipédia: Johann Peter Gustav Lejeune Dirichlet}.}}

\newcommand{\euclides}{\endnote{Euclides de Alexandria, 300 a.C., matemático grego. Fonte: \href{https://pt.wikipedia.org/wiki/Euclides}{Wikipédia: Euclides}.}}

\newcommand{\euler}{\endnote{Leonhard Paul Euler, 1707-1783, matemático e físico suíço. Fonte: \href{https://pt.wikipedia.org/wiki/Ronald_Fisher}{Wikipédia: Ronald Fisher}.}}

\newcommand{\fibonacci}{\endnote{Leonardo Fibonacci, 1170 - 1250, matemático italiano. Fonte: \href{https://pt.wikipedia.org/wiki/Leonardo_Fibonacci}{Wikipédia: Leonardo Fibonacci}.}}

\newcommand{\fisher}{\endnote{Ronald Aylmer Fisher, 1890-1962, biólogo inglês. Fonte: \href{https://pt.wikipedia.org/wiki/Ronald_Fisher}{Wikipédia: Ronald Fisher}.}}

\newcommand{\galerkin}{\endnote{Boris Galerkin, 1871 - 1945, engenheiro e matemático soviético. Fonte: \href{https://pt.wikipedia.org/wiki/Boris_Galerkin}{Wikipédia}.}}

\newcommand{\gauss}{\endnote{Johann Carl Friedrich Gauss, 1777 - 1855, matemático alemão. Fonte: \href{https://pt.wikipedia.org/wiki/Carl_Friedrich_Gauss}{Wikipédia: Carl Friedrich Gauss}.}}

\newcommand{\gram}{\endnote{Jørgen Pedersen Gram, 1850 - 1916, matemático dinamarquês. Fonte: \href{https://pt.wikipedia.org/wiki/J\%C3\%B8rgen_Pedersen_Gram}{Wikipédia}.}}

\newcommand{\green}{\endnote{George Green, 1793 - 1841, matemático britânico. Fonte: \href{https://pt.wikipedia.org/wiki/George_Green}{Wikipédia:George Green }.}}

\newcommand{\heron}{\endnote{Heron de Alexandria, 10 - 80, matemático grego. Fonte: \href{https://pt.wikipedia.org/wiki/Heron_de_Alexandria}{Wikipédia: Heron de Alexandria}.}}

\newcommand{\hessenberg}{\endnote{Karl Adolf Hessenberg, 1904 - 1959, engenheiro e matemático alemão. Fonte: \href{https://pt.wikipedia.org/wiki/Karl_Hessenberg}{Wikipédia}.}}

\newcommand{\householder}{\endnote{Alston Scott Householder, 1904 - 1993, matemático americano estadunidense. Fonte: \href{https://pt.wikipedia.org/wiki/Alston_Scott_Householder}{Wikipédia}.}}

\newcommand{\jacobi}{\endnote{Carl Gustav Jakob Jacobi, 1804 - 1851, matemático alemão. Fonte: \href{https://pt.wikipedia.org/wiki/Carl_Gustav_Jakob_Jacobi}{Wikipédia: Carl Gustav Jakob Jacobi}.}}

\newcommand{\lovelace}{\endnote{Augusta Ada Byron King, Condessa de Lovelace, 1815 - 1852, matemática inglesa. Fonte: \href{https://pt.wikipedia.org/wiki/Ada_Lovelace}{Wikipédia: Ada Lovelace}.}}

\newcommand{\krylov}{\endnote{Alexei Nikolajewitsch Krylov, 1863 - 1945, engenheiro e matemático russo. Fonte: \href{https://pt.wikipedia.org/wiki/Alexei_Krylov}{Wikipédia}.}}

\newcommand{\kutta}{\endnote{Martin Wilhelm Kutta, 1867 - 1944, matemático alemão. Fonte: \href{https://pt.wikipedia.org/wiki/Martin_Wilhelm_Kutta}{Wikipédia: Martin Wilhelm Kutta}.}}

\newcommand{\lagrange}{\endnote{Joseph-Louis Lagrange, 1736 - 1813, matemático italiano. Fonte: \href{https://pt.wikipedia.org/wiki/Joseph-Louis_Lagrange}{Wikipédia: Joseph-Louis Lagrange}.}}

\newcommand{\laplace}{\endnote{Pierre-Simon Laplace, 1749 - 1827, matemático francês. Fonte: \href{https://pt.wikipedia.org/wiki/Pierre-Simon_Laplace}{Wikipédia: Pierre-Simon Laplace}.}}

\newcommand{\lipschitz}{\endnote{Rudolf Otto Sigismund Lipschitz, 1832 - 1903, matemático alemão. Fonte: \href{https://pt.wikipedia.org/wiki/Rudolf_Lipschitz}{Wikipédia}.}}

\newcommand{\neumann}{\endnote{Carl Gottfried Neumann, 1832 - 1925, matemático alemão. Fonte: \href{https://pt.wikipedia.org/wiki/Carl_Neumann}{Wikipédia: Carl Neumann}.}}

\newcommand{\newton}{\endnote{Isaac Newton, 1642 - 1727, matemático, físico, astrônomo, teólogo e autor inglês. Fonte: \href{https://pt.wikipedia.org/wiki/Isaac_Newton}{Wikipédia: Isaac Newton}.}}

\newcommand{\pitagoras}{\endnote{Pitágoras de Samos, c.570, c. 495 a.C., matemático grego jônico. Fonte: \href{https://pt.wikipedia.org/wiki/Pit\%C3\%A1goras}{Wikipédia:Pitágoras}.}}

\newcommand{\petrov}{\endnote{Georgi Iwanowitsch Petrov, 1912 - 1987, engenheiro soviético. Fonte: \href{https://de.wikipedia.org/wiki/Georgi_Iwanowitsch_Petrow}{Wikipedia}.}}

\newcommand{\poisson}{\endnote{Siméon Denis Poisson, 1781 - 1840, matemático francês. Fonte: \href{https://pt.wikipedia.org/wiki/Sim\%C3\%A9on_Denis_Poisson}{Wikipédia:Siméon Denis Poisson}.}}

\newcommand{\riemann}{\endnote{Georg Friedrich Bernhard Riemann, 1826 - 1866, matemático alemão. Fonte: \href{https://pt.wikipedia.org/wiki/Bernhard_Riemann}{Wikipédia: Bernhard Riemann}.}}

\newcommand{\robin}{\endnote{Victor Gustave Robin, 1855 - 1897, matemático francês. Fonte: \href{https://en.wikipedia.org/wiki/Victor_Gustave_Robin}{Wikipedia: Victor Gustave Robin}.}}

\newcommand{\rossum}{\endnote{Guido van Rossum, 1956-, matemático e programador de computadores holandês. Fonte: \href{https://pt.wikipedia.org/wiki/Guido\_van\_Rossum}{Wikipédia: Guido van Rossum}.}}

\newcommand{\rothe}{\endnote{Erich Hans Rothe, 1895 - 1988, matemático alemão. Fonte: \href{https://pt.wikipedia.org/wiki/Erich_Rothe}{Wikipédia}.}}

\newcommand{\runge}{\endnote{Carl David Tolmé Runge, 1856 - 1927, matemático alemão. Fonte: \href{https://pt.wikipedia.org/wiki/Carl_Runge}{Wikipédia: Carl Runge}.}}

\newcommand{\seidel}{\endnote{Philipp Ludwig von Seidel, 1821 - 1896, matemático alemão. Fonte: \href{https://pt.wikipedia.org/wiki/Philipp_Ludwig_von_Seidel}{Wikipédia: Philipp Ludwig von Seidel}.}}

\newcommand{\simpson}{\endnote{Thomas Simpson, 1710 - 1761, matemático britânico. Fonte: \href{https://pt.wikipedia.org/wiki/Thomas_Simpson}{Wikipédia: Thomas Simpson}.}}

\newcommand{\schmidt}{\endnote{Erhard Schmidt, 1876 - 1959, matemático alemão. Fonte: \href{https://pt.wikipedia.org/wiki/Erhard_Schmidt}{Wikipédia}.}}

\newcommand{\schwarz}{\endnote{Karl Hermann Amandus Schwarz, 1843-1921, matemático alemão. Fonte: \href{https://pt.wikipedia.org/wiki/Hermann_Amandus_Schwarz}{Wikipédia: Hermann Amandus Schwarz}.}}

\newcommand{\taylor}{\endnote{Brook Taylor, 1685 - 1731, matemático britânico. Fonte: \href{https://pt.wikipedia.org/wiki/Brook_Taylor}{Wikipédia:Brook Taylor}.}}

\newcommand{\thomas}{\endnote{Llewellyn Hilleth Thomas, 1903 - 1992, físico e matemático aplicado britânico. Fonte: \href{https://pt.wikipedia.org/wiki/Llewellyn_Thomas}{Wikipedia}.}}

\newcommand{\vandermonde}{\endnote{Alexandre-Théophile Vandermonde, 1735 - 1796, matemático francês. Fonte: \href{https://pt.wikipedia.org/wiki/Alexandre-The\%C3\%B3phile_Vandermonde}{Wikipédia: Alexandre-Theóphile Vandermonde}.}}

\newcommand{\vonNeumann}{\endnote{John von Neumann, 1903 - 1957, matemático húngaro, naturalizado estadunidense. Fonte: \href{https://pt.wikipedia.org/wiki/John_von_Neumann}{Wikipédia: John von Neumann}.}}