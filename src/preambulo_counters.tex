\theoremstyle{plain}
\newtheorem{teo}{Teorema}[section]
\newtheorem{teorema}{Teorema}[section]

\newtheorem{lem}{Lema}[section]
\newtheorem{lema}{Lema}[section]

\newtheorem{prop}{Proposição}[section]
\newtheorem{proposicao}{Proposição}[section]

\newtheorem{corol}{Corolário}[section]
\newtheorem{corolario}{Corolário}[section]

\theoremstyle{definition}
\newtheorem{defn}{Definição}[section]
\newtheorem{definicao}{Definição}[section]

% observação
\newtheorem{obs}{Observação}[section]
\AtBeginEnvironment{obs}{%
  \pushQED{\qed}\renewcommand{\qedsymbol}{$\triangle$}%
}
\AtEndEnvironment{obs}{\popQED\endex}

% exemplo
\newtheorem{ex}{Exemplo}[section]
\AtBeginEnvironment{ex}{%
  \pushQED{\qed}\renewcommand{\qedsymbol}{$\triangle$}%
}
\AtEndEnvironment{ex}{\popQED\endex}


\newenvironment{dem}{\begin{proof}}{\end{proof}}
\newenvironment{demonstracao}{\begin{proof}}{\end{proof}}



%Exercícios Resolvidos

\newtheorem{exeresol}{ER}[section]

\newcommand{\exerref}[1]{E.\ref{#1}}
\newcommand{\exeresolref}[1]{ER.\ref{#1}}


%%% PDF %%%
\ifisbook
% exerícios para notas de aula
\usepackage[answerdelayed,lastexercise]{exercise}
\renewcommand{\ExerciseName}{E.}
\usepackage{chngcntr}
\counterwithin{Exercise}{section}
\counterwithin{Answer}{section}
\renewcommand{\ExerciseHeader}{\noindent\textbf{\ExerciseName\ExerciseHeaderNB.}~}
\renewcommand{\AnswerHeader}{\textbf{\ExerciseName\ExerciseHeaderNB.}~}
\newenvironment{exer}{\begin{Exercise}}{\end{Exercise}}
\newenvironment{resp}{\begin{Answer}}{\end{Answer}}
\fi


%%% HTML %%%
\ifishtml
\newtheorem{exer}{E.}[section]
\newtheorem*{resp}{Resposta}  
\newtheorem*{resol}{Solução}
\fi
