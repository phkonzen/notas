

\chapter{Técnicas de extrapolação}\label{cap_extrapl}
\thispagestyle{fancy}

Neste capítulo, estudamos algumas técnicas de extrapolação, as quais serão usadas nos próximos capítulos.

\section{Extrapolação de Richardson}\label{cap_extrapl_sec_Richardson}

Seja $F_1(h)$ uma aproximação de $I$ tal que
\begin{equation}\label{eq:extrapl_aux1}
  I = F_1(h) + \underbrace{k_1h + k_2h^2 + k_3h^3 + O(h^4)}_{\text{erro de truncamento}}.
\end{equation}
Então, dividindo $h$ por $2$, obtemos
\begin{equation}\label{eq:extrapl_aux2}
  I = F_1\left(\frac{h}{2}\right) + k_1\frac{h}{2} + k_2\frac{h^2}{4} + k_3\frac{h^3}{8} + O(h^4).
\end{equation}
Agora, de forma a eliminarmos o termo de ordem $h$ das expressões acima, subtraímos \eqref{eq:extrapl_aux1} de $2$ vezes~\eqref{eq:extrapl_aux2}, o que nos leva a
\begin{equation}\label{eq:extrapl_aux3}
  I = \underbrace{\left[F_1\left(\frac{h}{2}\right) + \left(F_1\left(\frac{h}{2}\right) - F_1(h)\right)\right]}_{F_2(h)} - k_2\frac{h^2}{2} - k_3\frac{3h^3}{4} + O(h^4).
\end{equation}
Ou seja, denotando
\begin{equation}
  F_2(h) := F_1\left(\frac{h}{2}\right) + \left(F_1\left(\frac{h}{2}\right) - F_1(h)\right)
\end{equation}
temos que $N_2(h)$ é uma aproximação de $I$ com erro de truncamento da ordem de $h^2$, uma ordem a mais de $N_1(h)$. Ou seja, esta combinação de aproximações de ordem de truncamento $h$ nos fornece uma aproximação de ordem de truncamento $h^2$.

Analogamente, consideremos a aproximação de $I$ por $N_2(h/2), i.e.$
\begin{equation}\label{eq:extrapl_aux4}
  I = F_2\left(\frac{h}{2}\right) - k_2\frac{h^2}{8} - k_2\frac{3h^3}{32} + O(h^4)
\end{equation}
Então, subtraindo~\eqref{eq:extrapl_aux3} de $4$ vezes~\eqref{eq:extrapl_aux4} de, obtemos
\begin{equation}\label{eq:extrapl_aux5}
  I = \underbrace{\left[3F_2\left(\frac{h}{2}\right) + \left(F_2\left(\frac{h}{2}\right) - F_2(h)\right)\right]}_{F_3(h)} + k_3\frac{3h^3}{8} + O(h^4).
\end{equation}
Observemos, ainda, que $N_3(h)$ pode ser reescrita na forma
\begin{equation}
  F_3(h) = F_2\left(\frac{h}{2}\right) + \frac{F_2\left(\frac{h}{2}\right) - F_2(h)}{3},
\end{equation}
a qual é uma aproximação de ordem $h^3$ para $I$.

Para fazermos mais um passo, consideramos a aproximação de $I$ por $F_3(h/2)$, i.e.
\begin{equation}\label{eq:extrapl_aux6}
  I = F_3\left(\frac{h}{2}\right) + k_3\frac{3h^3}{64} + O(h^4).
\end{equation}
E, então, subtraindo~\eqref{eq:extrapl_aux5} de $8$ vezes~\eqref{eq:extrapl_aux6}, temos
\begin{equation}
  I = \underbrace{\left[F_3\left(\frac{h}{2} \right)+ \left(\frac{F_3\left(\frac{h}{2}\right)-F_3(h)}{7}\right)\right]}_{F_4(h)} + O(h^4).
\end{equation}
Ou seja,
\begin{equation}
  F_4(h) = \left[F_3\left(\frac{h}{2}\right) + \frac{F_3\left(\frac{h}{2}\right)-F_3(h)}{7}\right]
\end{equation}
é uma aproximação de $I$ com erro de truncamento da ordem $h^4$. Estes cálculos nos motivam o seguinte teorema.

\begin{teo}\label{teo:Richardson}
  Seja $F_1(h)$ uma aproximação de $I$ com erro de truncamento da forma
  \begin{equation}
    I-F_1(h) = \sum_{i=1}^n k_1h^i + O(h^{n+1}).
  \end{equation}
Então, para $j\geq 2$,
\begin{equation}
  F_j(h) := F_{j-1}\left(\frac{h}{2}\right) + \frac{F_{j-1}\left(\frac{h}{2}\right) - F_{j-1}(h)}{2^{j-1}-1}
\end{equation}
é uma aproximação de $I$ com erro de truncamento da forma
\begin{align}
  I-F_{j}(h) &= \sum_{i=j}^n (-1)^{j-1}\frac{\left(2^{i-1}-1\right)\prod_{l=1}^{j-2}\left(2^{i-l-1}-1\right)}{2^{(j-1)(i-j+1)}d_j}k_ih^i \nonumber \\
           & + O(h^{n+1}),
\end{align}
onde $d_{j}$ é dado recursivamente por $d_{j+1}=2^{j-1}d_j$, com $d_2=1$.
\end{teo}
\begin{dem}
  Fazemos a demonstração por indução. O resultado para $j=2$ segue de~\eqref{eq:extrapl_aux3}. Assumimos, agora, que vale
  \begin{align}
    I-F_{j}(h) &= (-1)^{j-1}\frac{\left(2^{j-1}-1\right)\prod_{l=1}^{j-2}\left(2^{j-l-1}-1\right)}{2^{(j-1)}d_j}k_jh^j \nonumber \\
              &+ \sum_{i=j+1}^n (-1)^{j-1}\frac{\left(2^{i-1}-1\right)\prod_{l=1}^{j-2}\left(2^{i-l-1}-1\right)}{2^{(j-1)(i-j+1)}d_j}k_ih^i \nonumber \\
              & + O(h^{n+1}).\label{eq:extrapl_aux7}
  \end{align}
para $j\geq 2$. Então, tomamos
\begin{align}
  I-F_{j}\left(\frac{h}{2}\right) &= (-1)^{j-1}\frac{\left(2^{j-1}-1\right)\prod_{l=1}^{j-2}\left(2^{j-l-1}-1\right)}{2^{(j-1)}d_j}k_j\frac{h^j}{2^j} \nonumber \\
              &+ \sum_{i=j+1}^n (-1)^{j-1}\frac{\left(2^{i-1}-1\right)\prod_{l=1}^{j-2}\left(2^{i-l-1}-1\right)}{2^{(j-1)(i-j+1)}d_j}k_i\frac{h^i}{2^i} \nonumber \\
              & + O(h^{n+1}). \label{eq:extrapl_aux8}
\end{align}
Agora, subtraímos~\eqref{eq:extrapl_aux7} de $2^{j}$ vezes~\eqref{eq:extrapl_aux8}, o que nos fornece
\begin{align}
  I &= \left[F_{j}\left(\frac{h}{2}\right) + \frac{F_{j}\left(\frac{h}{2}\right) - F_{j}(h)}{2^{j}-1}\right] \nonumber\\
    &+ \sum_{i=j+1}^n (-1)^{(j+1)-1}\frac{\left(2^{i-1}-1\right)\prod_{l=1}^{(j+1)-2}\left(2^{i-l-1}-1\right)}{2^{((j+1)-1)(i-(j+1)+1)}2^{j-1}d_j}k_ih^i\nonumber \\
              & + O(h^{n+1}).
\end{align}
\end{dem}

\begin{corol}
  Seja $F_1(h)$ uma aproximação de $I$ com erro de truncamento da forma
  \begin{equation}
    I-F_1(h) = \sum_{i=1}^n k_1h^{2i} + O(h^{2n+2}).
  \end{equation}
Então, para $j\geq 2$,
\begin{equation}
  F_j(h) := F_{j-1}\left(\frac{h}{2}\right) + \frac{F_{j-1}\left(\frac{h}{2}\right) - F_{j-1}(h)}{4^{j-1}-1}
\end{equation}
é uma aproximação de $I$ com erro de truncamento da forma
\begin{align}
  I-F_{j}(h) &= \sum_{i=j}^n (-1)^{j-1}\frac{\left(4^{i-1}-1\right)\prod_{l=1}^{j-2}\left(4^{i-l-1}-1\right)}{4^{(j-1)(i-j+1)}d_j}k_ih^{2i} \nonumber \\
           & + O(h^{n+1}),
\end{align}
onde $d_{j}$ é dado recursivamente por $d_{j+1}=4^{j-1}d_j$, com $d_2=1$.
\end{corol}
\begin{dem}
  A demonstração é análoga ao do Teorema~\ref{teo:Richardson}.
\end{dem}

\begin{ex}
  Dada uma função $f(x)$, consideremos sua aproximação por diferenças finitas progressiva de ordem $h$, i.e.
  \begin{align}
    \underbrace{f'(x)}{I} &= \underbrace{\frac{f(x+h)-f(x)}{h}}_{F_1(h)}\nonumber\\
    &+ \frac{f''(x)}{2}h + \frac{f'''(x)}{6}h^2 + O(h^3).
  \end{align}
Estão, considerando a primeira extrapolação de Richardson, temos
\begin{align}
  F_2(h) &= F_1\left(\frac{h}{2}\right) + \left(F_1\left(\frac{h}{2}\right) - F_1(h)\right)\\
  &= 4\frac{f(x+h/2)-f(x)}{h} - \frac{f(x+h)-f(x)}{h}\\
  &= \frac{-f(x+h)+4f(x+h/2)-3f(x)}{h},
\end{align}
a qual é a fórmula de diferenças finitas progressiva de três pontos com passo $h/2$, i.e. $D_{+,(h/2)^2}f(x)$ (veja, Fórmula~\eqref{eq:dfp_3pts}).
\end{ex}

\begin{ex}
  Dada uma função $f(x)$, consideremos sua aproximação por diferenças finitas central de ordem $h^2$, i.e.
  \begin{align}
    \underbrace{f'(x)}{I} &= \underbrace{\frac{f(x+h)-f(x-h)}{2h}}_{F_1(h)} \nonumber \\
                          &- \frac{f'''}{6}h^2 - \frac{f^{(5)}(x)}{120}h^4 + O(h^6).
  \end{align}
Estão, considerando a primeira extrapolação de Richardson, temos
\begin{align}
  F_2(h) &= F_1\left(\frac{h}{2}\right) + \frac{\left(F_1\left(\frac{h}{2}\right) - F_1(h)\right)}{3}\\
  &= \frac{1}{6h}\left[f(x-h)-8f(x-h/2)+8f(x+h/2)-f(x+h)\right]
\end{align}
a qual é a fórmula de diferenças finitas central de cinco pontos com passo $h/2$, i.e. $D_{+,(h/2)^4}f(x)$ (veja, Fórmula~\eqref{eq:dfc_5pts}).
\end{ex}

\subsection{Sucessivas extrapolações}

Sucessivas extrapolações de Richardson podem ser computadas de forma robusta com o auxílio de uma tabela. Seja $F_1(h)$ uma dada aproximação de uma quantidade de interesse $I$ com erro de truncamento da forma
\begin{equation}
  I-F_1(h) = k_1h + k_2h^2 + k_3h^3 + \cdots + k_nh^n + O(h^{n+1}).
\end{equation}
Então, as sucessivas extrapolações $F_2(h)$, $F_3(h)$, $\dotsc$, $F_n(h)$ podem ser organizadas na seguinte forma tabular
\begin{equation}
  T = \left[\begin{array}{lllll}
    F_1(h)\\
    F_1(h/2) & F_2(h) \\
    F_1(h/2^2) & F_2(h/2) & F_3(h) \\
    \vdots & \vdots & \vdots \\
    F_1(h/2^n) & F_2(h/2^{n-1}) & F_3(h/2^{n-2}) & \cdots & F_n(h)
  \end{array}\right]
\end{equation}
Desta forma, temos que
\begin{equation}
  F_j\left(\frac{h}{2^{i-1}}\right) = t_{i,j-1} + \frac{t_{i,j-1}-t_{i-1,j-1}}{2^{j-1}-1}
\end{equation}
com $j=2, 3, \dotsc, n$ e $j\geq i$, onde $t_{i,j}$ é o elemento da $i$-ésima linha e $j$-ésima coluna da matriz $T$.

\begin{ex}\label{ex:Richardson_suc_h}
  Consideremos o problema de aproximar a derivada da função $f(x) = \sen(x)$ no ponto $\pi/3$. Usando a fórmula de diferenças finitas progressiva de ordem $h$ obtemos
  \begin{align}
    f'\left(\frac{\pi}{3}\right) &= \underbrace{\frac{f\left(\frac{\pi}{3}+h\right)-f\left(\frac{\pi}{3}\right)}{h}}_{F_1(h) := D_{+,h}f(\pi/3)} \nonumber\\
          &+ \frac{f''(x)}{2}h + \frac{f'''(x)}{6}h^2 + \cdots \label{eq:ex_Richardson_suc_h}
  \end{align}
Na Tabela~\ref{tab:ex_Richardson_suc_h} temos os valores das aproximações de $f'(\pi/3)$ computadas via sucessivas extrapolações de Richardson a partir de \eqref{eq:ex_Richardson_suc_h} com $h=0.1$.

\begin{table}[h!]
  \centering
  \caption{Resultados referente ao Exemplo~\ref{ex:Richardson_suc_h}.}
  \begin{tabular}{cccc}\hline
    $O(h)$ & $O(h^2)$ & $O(h^3)$ & $O(h^4)$\\ \hline
    $4,55902\E-1$ \\
    $4,78146\E-1$ & $5,00389\E-1$ \\
    $4,89123\E-1$ & $5,00101\E-1$ & $5,00005\E-1$ \\
    $4,94574\E-1$ & $5,00026\E-1$ & $5,00001\E-1$ & $5,00000\E-1$ \\\hline
  \end{tabular}
  \label{tab:ex_Richardson_suc_h}
\end{table}

\ifisoctave
No \verb+GNU Octave+, podemos fazer estes cálculos com o seguinte código:
\begin{verbatim}
#funcao
f = @(x) sin(x);
x=pi/3;

#aproximacao de ordem 1
dfp = @(x,h) (f(x+h)-f(x))/h;
h=0.1;

#tabela c/ sucessivas extrapolacoes
T=zeros(4,4);
for i=1:4
  T(i,1) = dfp(x,h/2^(i-1));
endfor
for j=2:4
  for i=j:4
    T(i,j) = T(i,j-1) ... 
           + (T(i,j-1)-T(i-1,j-1))/(2^(j-1)-1);
  endfor
endfor
\end{verbatim}
\fi
\end{ex}

\begin{ex}\label{ex:Richardson_suc_h2}
  Novamente, consideremos o problema de aproximar a derivada da função $f(x) = \sen(x)$ no ponto $\pi/3$. A fórmula de diferenças finitas central de ordem $h^2$ tem a forma
  \begin{align}
    f'\left(\frac{\pi}{3}\right) &= \underbrace{\frac{f\left(\frac{\pi}{3}+h\right)-f\left(\frac{\pi}{3}-h\right)}{2h}}_{F_1(h) := D_{0,h^2}f(\pi/3)} \nonumber\\
          &- \frac{f'''(x)}{6}h^2 + \frac{f^{(5)}(x)}{120}h^4 - \cdots \label{eq:ex_Richardson_suc_h2}
  \end{align}
Na Tabela~\ref{tab:ex_Richardson_suc_h2} temos os valores das aproximações de $f'(\pi/3)$ computadas via sucessivas extrapolações de Richardson a partir de \eqref{eq:ex_Richardson_suc_h2} com $h=1$.

\begin{table}[h!]
  \centering
  \caption{Resultados referente ao Exemplo~\ref{ex:Richardson_suc_h2}.}
  \begin{tabular}{cccc}\hline
    $O(h^2)$ & $O(h^4)$ & $O(h^6)$ & $O(h^8)$\\ \hline
    $4,20735\E-1$ \\
    $4,79426\E-1$ & $4,98989\E-1$ \\
    $4,94808\E-1$ & $4,99935\E-1$ & $4,99998\E-1$ \\
    $4,98699\E-1$ & $4,99996\E-1$ & $5,00000\E-1$ & $5,00000\E-1$ \\\hline
  \end{tabular}
  \label{tab:ex_Richardson_suc_h2}
\end{table}

\ifisoctave
No \verb+GNU Octave+, podemos fazer estes cálculos com o seguinte código:
\begin{verbatim}
#funcao obj.
f = @(x) sin(x);
x=pi/3;

#aprox. O(h^2)
h=1;
dfp = @(x,h) (f(x+h)-f(x-h))/(2*h);

#tabela c/ sucessivas extrapolacoes
T=zeros(4,4);
for i=1:4
  T(i,1) = dfp(x,h/2^(i-1));
endfor
for j=2:4
  for i=j:4
    T(i,j) = T(i,j-1) ... 
           + (T(i,j-1)-T(i-1,j-1))/(4^(j-1)-1);
  endfor
endfor
\end{verbatim}
\fi
\end{ex}

\subsection{Exercícios}

\begin{exer}
  Mostre que a primeira extrapolação de Richardson de
  \begin{equation}
    D_{-,h}f(x) = \frac{f(x)-f(x-h)}{h}
  \end{equation}
é igual a
\begin{equation}
  D_{-,(h/2)^2}f(x) = \frac{3f(x)-4f(x-h)+f(x-2h)}{h}.
\end{equation}
\end{exer}

\begin{exer}\label{exer:df_fun}
  Considere o problema de aproximar a derivada de 
  \begin{equation}
    f(x) = \frac{\sen(x+2) - e^{-x^2}}{x^2 + \ln(x+2)}+x
  \end{equation}
no ponto $x=2,5$. Para tanto, use de sucessivas extrapolações de Richardson a partir da aproximação por diferenças finitas:
\begin{enumerate}[a)]
\item progressiva de ordem $h$, com $h=0,5$.
\item regressiva de ordem $h$, com $h=0,5$.
\item central de ordem $h^2$, com $h=0,5$.
\end{enumerate}
Nas letras a) e b), obtenha as aproximações de ordem $h^3$ e, na letra $c)$ obtenha a aproximação de ordem $h^6$.
\end{exer}
\begin{resp}
  \ifisoctave 
  \href{https://github.com/phkonzen/notas/blob/master/src/MatematicaNumerica/cap_deriv/dados/exer_Richardson_fun/exer_Richardson_fun.m}{Código.} 
  \fi
  a)~$1,05919$; b)~$1,05916$; c)~$1,05913$
\end{resp}
