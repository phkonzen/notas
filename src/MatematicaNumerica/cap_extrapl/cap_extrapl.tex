%Este trabalho está licenciado sob a Licença Atribuição-CompartilhaIgual 4.0 Internacional Creative Commons. Para visualizar uma cópia desta licença, visite http://creativecommons.org/licenses/by-sa/4.0/deed.pt_BR ou mande uma carta para Creative Commons, PO Box 1866, Mountain View, CA 94042, USA.

\chapter{Técnicas de extrapolação}\label{cap_extrapl}
\thispagestyle{fancy}

Neste capítulo, estudamos algumas técnicas de extrapolação, as quais serão usadas nos próximos capítulos.

\section{Extrapolação de Richardson}\label{cap_extrapl_sec_Richardson}

Seja $F_1(h)$ uma aproximação de $I$ tal que
\begin{equation}\label{eq:extrapl_aux1}
  I = F_1(h) + \underbrace{k_1h + k_2h^2 + k_3h^3 + O(h^4)}_{\text{erro de truncamento}}.
\end{equation}
Então, dividindo $h$ por $2$, obtemos
\begin{equation}\label{eq:extrapl_aux2}
  I = F_1\left(\frac{h}{2}\right) + k_1\frac{h}{2} + k_2\frac{h^2}{4} + k_3\frac{h^3}{8} + O(h^4).
\end{equation}
Agora, de forma a eliminarmos o termo de ordem $h$ das expressões acima, subtraímos \eqref{eq:extrapl_aux1} de $2$ vezes~\eqref{eq:extrapl_aux2}, o que nos leva a
\begin{equation}\label{eq:extrapl_aux3}
  I = \underbrace{\left[F_1\left(\frac{h}{2}\right) + \left(F_1\left(\frac{h}{2}\right) - F_1(h)\right)\right]}_{F_2(h)} - k_2\frac{h^2}{2} - k_3\frac{3h^3}{4} + O(h^4).
\end{equation}
Ou seja, denotando
\begin{equation}
  F_2(h) := F_1\left(\frac{h}{2}\right) + \left(F_1\left(\frac{h}{2}\right) - F_1(h)\right)
\end{equation}
temos que $N_2(h)$ é uma aproximação de $I$ com erro de truncamento da ordem de $h^2$, uma ordem a mais de $N_1(h)$. Ou seja, esta combinação de aproximações de ordem de truncamento $h$ nos fornece uma aproximação de ordem de truncamento $h^2$.

Analogamente, consideremos a aproximação de $I$ por $N_2(h/2), i.e.$
\begin{equation}\label{eq:extrapl_aux4}
  I = F_2\left(\frac{h}{2}\right) - k_2\frac{h^2}{8} - k_2\frac{3h^3}{32} + O(h^4)
\end{equation}
Então, subtraindo~\eqref{eq:extrapl_aux3} de $4$ vezes~\eqref{eq:extrapl_aux4} de, obtemos
\begin{equation}\label{eq:extrapl_aux5}
  I = \underbrace{\left[3F_2\left(\frac{h}{2}\right) + \left(F_2\left(\frac{h}{2}\right) - F_2(h)\right)\right]}_{F_3(h)} + k_3\frac{3h^3}{8} + O(h^4).
\end{equation}
Observemos, ainda, que $N_3(h)$ pode ser reescrita na forma
\begin{equation}
  F_3(h) = F_2\left(\frac{h}{2}\right) + \frac{F_2\left(\frac{h}{2}\right) - F_2(h)}{3},
\end{equation}
a qual é uma aproximação de ordem $h^3$ para $I$.

Para fazermos mais um passo, consideramos a aproximação de $I$ por $F_3(h/2)$, i.e.
\begin{equation}\label{eq:extrapl_aux6}
  I = F_3\left(\frac{h}{2}\right) + k_3\frac{3h^3}{64} + O(h^4).
\end{equation}
E, então, subtraindo~\eqref{eq:extrapl_aux5} de $8$ vezes~\eqref{eq:extrapl_aux6}, temos
\begin{equation}
  I = \underbrace{\left[F_3\left(\frac{h}{2} \right)+ \left(\frac{F_3\left(\frac{h}{2}\right)-F_3(h)}{7}\right)\right]}_{F_4(h)} + O(h^4).
\end{equation}
Ou seja,
\begin{equation}
  F_4(h) = \left[F_3\left(\frac{h}{2}\right) + \frac{F_3\left(\frac{h}{2}\right)-F_3(h)}{7}\right]
\end{equation}
é uma aproximação de $I$ com erro de truncamento da ordem $h^4$. Estes cálculos nos motivam o seguinte teorema.

\begin{teo}
  Seja $F_1(h)$ uma aproximação de $I$ com erro de truncamento da forma
  \begin{equation}
    I-F_1(h) = \sum_{i=1}^n k_1h^i + O(h^{n+1}).
  \end{equation}
Então, para $j\geq 2$,
\begin{equation}
  F_j(h) := F_{j-1}\left(\frac{h}{2}\right) + \frac{F_{j-1}\left(\frac{h}{2}\right) - F_{j-1}(h)}{2^{j-1}-1}
\end{equation}
é uma aproximação de $I$ com erro de truncamento da forma
\begin{equation}
  I-F_{j}(h) = \sum_{i=j}^n (-1)^{j-1}\frac{\left(2^{i-1}-1\right)\prod_{l=1}^{j-2}\left(2^{i-l-1}-1\right)}{2^{(j-1)(i-j+1)}d_j}k_ih^i + O(h^{n+1}),
\end{equation}
onde $d_{j}$ é dado recursivamente por $d_{j+1}=2^{j-1}d_j$, com $d_2=1$.
\end{teo}
\begin{dem}
  Fazemos a demonstração por indução. O resultado para $j=2$ segue de~\eqref{eq:extrapl_aux3}. Assumimos, agora, que vale
  \begin{align}
    I-F_{j}(h) &= (-1)^{j-1}\frac{\left(2^{j-1}-1\right)\prod_{l=1}^{j-2}\left(2^{j-l-1}-1\right)}{2^{(j-1)}d_j}k_jh^j \nonumber \\
              &+ \sum_{i=j+1}^n (-1)^{j-1}\frac{\left(2^{i-1}-1\right)\prod_{l=1}^{j-2}\left(2^{i-l-1}-1\right)}{2^{(j-1)(i-j+1)}d_j}k_ih^i \nonumber \\
              & + O(h^{n+1}).\label{eq:extrapl_aux7}
  \end{align}
para $j\geq 2$. Então, tomamos
\begin{align}
  I-F_{j}\left(\frac{h}{2}\right) &= (-1)^{j-1}\frac{\left(2^{j-1}-1\right)\prod_{l=1}^{j-2}\left(2^{j-l-1}-1\right)}{2^{(j-1)}d_j}k_j\frac{h^j}{2^j} \nonumber \\
              &+ \sum_{i=j+1}^n (-1)^{j-1}\frac{\left(2^{i-1}-1\right)\prod_{l=1}^{j-2}\left(2^{i-l-1}-1\right)}{2^{(j-1)(i-j+1)}d_j}k_i\frac{h^i}{2^i} \nonumber \\
              & + O(h^{n+1}). \label{eq:extrapl_aux8}
\end{align}
Agora, subtraímos~\eqref{eq:extrapl_aux7} de $2^{j}$ vezes~\eqref{eq:extrapl_aux8}, o que nos fornece
\begin{align}
  I &= \left[F_{j}\left(\frac{h}{2}\right) + \frac{F_{j}\left(\frac{h}{2}\right) - F_{j}(h)}{2^{j}-1}\right] \nonumber\\
    &+ \sum_{i=j+1}^n (-1)^{(j+1)-1}\frac{\left(2^{i-1}-1\right)\prod_{l=1}^{(j+1)-2}\left(2^{i-l-1}-1\right)}{2^{((j+1)-1)(i-(j+1)+1)}2^{j-1}d_j}k_ih^i\nonumber \\
              & + O(h^{n+1}).
\end{align}
\end{dem}


\emconstrucao