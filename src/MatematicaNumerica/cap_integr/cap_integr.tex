%Este trabalho está licenciado sob a Licença Atribuição-CompartilhaIgual 4.0 Internacional Creative Commons. Para visualizar uma cópia desta licença, visite http://creativecommons.org/licenses/by-sa/4.0/deed.pt_BR ou mande uma carta para Creative Commons, PO Box 1866, Mountain View, CA 94042, USA.

\chapter{Integração}\label{cap_integr}
\thispagestyle{fancy}

Neste capítulo, discutimos os métodos numéricos fundamentais para a aproximação de integrais definidas de funções. Tais métodos são chamados de \emph{quadraturas numéricas} e têm a forma
\begin{equation}
  \int_a^b f(x)\,dx \approx \sum_{i=1}^n f(x_i)w_i,
\end{equation}
onde $x_i$ e $w_i$ são, respectivamente, o $i$-ésimo nodo e o $i$-ésimo peso da quadratura, $i=1, 2, \dotsc, n$.

\section{Regras de Newton-Cotes}\label{cap_integr_sec_NC}

Dada uma função $f(x)$ e um intervalo $[a, b]$, denotamos por
\begin{equation}
  I := \int_a^b f(x)\,dx.
\end{equation}
a integral de $f(x)$ no intervalo $[a, b]$. A ideia das regras de Newton-Cotes e aproximar $I$ pela integral de um polinômio interpolador de $f(x)$ por pontos previamente selecionados.

Seja, então, $p(x)$ o polinômio interpolador de grau $n$ de $f(x)$ pelos dados pontos $\{(x_i, f(x_i))\}_{i=1}^{n+1}$, com $x_1 < x_2 < \cdots < x_{n+1}$ e $x_i\in [a, b]$ para todo $i=1, 2, \dotsc, n+1$. Então, pelo teorema de Lagrange, temos
\begin{equation}
  f(x) = p(x) + R_{n+1}(x),
\end{equation}
onde
\begin{equation}
  p(x) = \sum_{i=1}^{n+1} f(x_i)\prod_{\overset{j=1}{j\neq i}}^{n+1} \frac{(x-x_j)}{x_i-x_j}
\end{equation}
e
\begin{equation}
  R_{n+1}(x) = \frac{f^{(n+1)}(\xi)}{(n+1)!}\prod_{j=1}^{n+1}(x-x_j),
\end{equation}
onde $\xi = \xi(x)$ pertencente ao intervalo $[x_1, x_{n+1}]$. Deste modo, temos
\begin{align}
  I &:= \int_a^b f(x)\\
  &= \int_a^b p(x)\,dx + \int_a^b R_{n+1}(x)\,dx\\
  &= \underbrace{\sum_{i=1}^{n+1} f(x_i)\int_a^b \prod_{\overset{j=1}{j\neq i}}^{n+1} \frac{(x-x_j)}{x_i-x_j)}\,dx}_{\text{quadratura}} + \underbrace{\int_a^b R_{n+1}(x)\,dx}_{\text{erro de truncamento}}
\end{align}
Ou seja, nas quadraturas (regras) de Newton-Cotes, os nodos são as abscissas dos pontos interpolados e os pesos são as integrais dos polinômios de Lagrange associados.

Na sequência, abordaremos as regras de Newton-Cotes mais usuais e estimaremos o erro de truncamento caso a caso. Para uma abordagem mais geral, recomenda-se consultar~\cite[Ch 7.,Sec. 1.1]{Isaacson1994a}.

\subsection{Regras de Newton-Cotes fechadas}

As regras de Newton-Cotes fechadas são aqueles que a quadratura incluem os extremos do intervalo de integração, i.e. os nodos extremos são $x_1=a$ e $x_{n+1}=b$.

\subsubsection{Regra do trapézio}

A regra do trapézio é obtida tomando-se os nodos $x_1=a$ e $x_2=b$. Então, denotando $h:=b-a$\footnote{Neste capítulo, $h$ é escolhido como a distância entre os nodos.}, os pesos da quadratura são:
\begin{align}
  w_1 &= \int_a^b \frac{x-b}{a-b}\,dx \\
  &= \frac{(b-a)}{2} = \frac{h}{2}
\end{align}
e
\begin{align}
  w_2 &= \int_a^b \frac{x-a}{b-a}\,dx \\
  &= \frac{(b-a)}{2} = \frac{h}{2}.
\end{align}
Agora, estimamos o erro de truncamento com
\begin{align}
  E &:= \int_a^b R_2(x)\,dx\\
  &= \int_a^b \frac{f''(\xi(x))}{2}(x-a)(x-b)\,dx\\
  &\leq C\left|\int_a^b (x-a)(x-b)\,dx\right|\\
  &= C\frac{(b-a)^3}{6} = O(h^3).
\end{align}

Portanto, a \emph{regra do trapézio}\index{regra do!trapézio} é dada por
\begin{equation}
  \int_a^b f(x)\,dx = \frac{h}{2}(f(a) + f(b)) + O(h^3).
\end{equation}

\begin{ex}\label{ex:int_trap}
  Consideremos o problema de computar a integral de $f(x)=xe^{-x^2}$ no intervalo $[0, 1/4]$. Analiticamente, temos
  \begin{align}
    I = \int_0^{1/4} xe^{-x^2}\,dx &= \left. -\frac{e^{-x^2}}{2} \right|_0^{1/4}\\
    &= \frac{1-e^{-1/4}}{2} = 3,02935\E-2.
  \end{align}
Agora, usando a regra do trapézio, obtemos a seguinte aproximação para $I$
\begin{align}
  I &\approx \frac{h}{2}(f(0) + f(1/2))\\
  &= \frac{1/4}{2}\left(0 + \frac{1}{4}e^{-(1/4)^2}\right) = 2,93567\E-2.
\end{align}

\ifisoctave
Podemos obter a aproximação dada pela regra do trapézio no \verb+GNU Octave+ com o seguinte código:
\begin{verbatim}
f = @(x) x*exp(-x^2);
a=0;
b=0.25;
h=b-a;
Itrap = (h/2)*(f(a)+f(b));
printf("%1.5E\n",Itrap)
\end{verbatim}
\fi
\end{ex}

\subsubsection{Regra de Simpson}

A regra de Simpson é obtida escolhendo-se os nodos $x_1=a$, $x_2=(a+b)/2$ e $x_3=b$. Com isso e denotando $h=(b-a)/2$, calculamos os seguintes pesos:
\begin{align}
  w_1 &= \int_a^b\frac{(x-x_2)(x-x_3)}{(x_1-x_2)(x_1-x_3)}\,dx\\
  &= \frac{(b-a)}{6} = \frac{h}{6},
\end{align}
\begin{align}
  w_2 &= \int_a^b\frac{(x-x_1)(x-x_3)}{(x_2-x_1)(x_2-x_3)}\,dx\\
  &= 4\frac{(b-a)}{6} = 4\frac{h}{6}
\end{align}
e
\begin{align}
  w_3 &= \int_a^b\frac{(x-x_1)(x-x_2)}{(x_3-x_1)(x_3-x_2)}\,dx\\
  &= \frac{(b-a)}{6} = \frac{h}{6}.
\end{align}
Isto nos fornece a chamada \emph{regra de Simpson}\index{regra de Simpson}
\begin{equation}\label{eq:aux_Simpson}
  I \approx \frac{h}{6}\left[f(a) + 4f\left(\frac{a+b}{2}\right) + f(b)\right]
\end{equation}

Nos resta estimar o erro de truncamento da regra de Simpson. Para tanto, consideramos a expansão em polinômio de Taylor de grau 3 de $f(x)$ em torno do ponto $x_2$, i.e.
\begin{align}
  f(x) &= f(x_2) + f'(x_2)(x-x_2) + \frac{f''(x_2)}{2}(x-x_2)^2 \nonumber\\
  &+ \frac{f'''(x_2)}{6}(x-x_2)^3 \nonumber\\
  &+ \frac{f^{(4)}(\xi_1(x))}{24}(x-x_2)^4,
\end{align}
donde
\begin{align}
  \int_a^b f(x)\,dx &= 2hf(x_2) + \frac{h^3}{3}f''(x_2) \nonumber\\
  &+ \frac{1}{24}\int_a^bf^{(4)}(\xi_1(x))(x-x_2)^4\,dx.\label{eq:aux_int_sim1}
\end{align}
Daí, usando da fórmula de diferenças finitas central de ordem $h^2$, temos
\begin{equation}\label{eq:aux_int_sim2}
  f''(x_2) = \frac{f(x_1) - 2f(x_2) + f(x_3)}{h^2} + O(h^2).
\end{equation}
Ainda, o último termo da equação~\eqref{eq:aux_int_sim1} pode ser estimado por
\begin{align}
  \left|\frac{1}{24}\int_a^bf^{(4)}(\xi_1(x))(x-x_2)^4\,dx\right| &\leq C\left|\int_a^b (x-x_2)^4\,dx\right|\\
  &= C(b-a)^5 = O(h^5).\label{eq:aux_int_sim3}
\end{align}\label{eq:aux_int_sim3}
Então, de \eqref{eq:aux_int_sim1}, \eqref{eq:aux_int_sim2} e \eqref{eq:aux_int_sim3}, temos
\begin{equation}
  \int_a^b f(x)\,dx = \frac{h}{3}\left[f(a) + 4f\left(\frac{a+b}{2}\right) + f(b)\right] + O(h^5),
\end{equation}
o que mostra que a \emph{regra de Simpson tem erro de truncamento da ordem $h^5$}.

\begin{ex}\label{ex:int_simp}
  Aproximando a integral dada no Exemplo~\ref{ex:int_trap} pela a regra de Simpson, temos
  \begin{align}
    \int_0^{1/4} f(x)\,dx &\approx \frac{1/8}{3}\left[f(0) + 4f\left(\frac{1}{8}\right) + f\left(\frac{1}{4}\right)\right]\\
    &= \frac{1}{24}\left[\frac{1}{2}e^{-(1/8)^2} + \frac{1}{4}e^{-(1/4)^2}\right]\\
    &= 3,02959\E-2.
  \end{align}

\ifisoctave
Podemos computar a aproximação dada pela regra de Simpson no \verb+GNU Octave+ com o seguinte código:
\begin{verbatim}
f = @(x) x*exp(-x^2);
a=0;
b=1/4;
h=(b-a)/2;
Isimp = (h/3)*(f(a)+4*f((a+b)/2)+f(b));
printf("%1.5E\n",Isimp)
\end{verbatim}
\fi
\end{ex}

\subsection{Regras de Newton-Cotes abertas}

As regras de Newton-Cotes abertas não incluem os extremos dos intervalos como nodos das quadraturas.

\subsubsection{Regra do ponto médio}

A regra do ponto médio\index{regra do!ponto médio} é obtida usando apenas o nodo $x_1=(a+b)/2$. Desta forma, temos
\begin{equation}
  \int_a^b f(x)\,dx = \int_a^b f(x_1)\,dx + \int_a^b f'(\xi(x))(x-x_1)\,dx,
\end{equation}
donde, denotando $h:=(b-a)$, temos
\begin{equation}
  \int_a^b f(x),dx = hf\left(\frac{a+b}{2}\right) + O(h^3).
\end{equation}
Deixa-se para o leitor a verificação do erro de truncamento (veja, Exercício~\ref{exer:trunc_pto_medio}).

\begin{ex}\label{ex:int_pto_medio}
  Aproximando a integral dada no Exemplo~\ref{ex:int_trap} pela a regra do ponto médio, temos
  \begin{align}
    \int_0^{1/4} f(x)\,dx &\approx \frac{1}{4}f\left(\frac{1}{8}\right)\\
    &= \frac{1}{32}e^{-(1/8)^2}\\
    &= 3,07655\E-2
  \end{align}

\ifisoctave
Podemos computar a aproximação dada pela regra do ponto médio no \verb+GNU Octave+ com o seguinte código:
\begin{verbatim}
f = @(x) x*exp(-x^2);
a=0;
b=0.25;
h=b-a;
Ipmd = h*f((a+b)/2);
printf("%1.5E\n",Ipmd)
\end{verbatim}
\fi
\end{ex}

\subsection*{Exercício}

\begin{exer}\label{exer:int_NC_fun}
  Aproxime
  \begin{equation}
    \int_{-1}^0 \frac{\sen(x+2)-e^{-x^2}}{x^2+\ln(x+2)}\,dx
  \end{equation}
usando a:
\begin{enumerate}[a)]
\item regra do ponto médio.
\item regra do trapézio.
\item regra de Simpson.
\end{enumerate}
\end{exer}
\begin{resp}
  \ifisoctave 
  \href{https://github.com/phkonzen/notas/blob/master/src/MatematicaNumerica/cap_integr/dados/exer_int_NC_fun/exer_int_NC_fun.m}{Código.} 
  \fi
  a)~$3,33647sE-1$; b)~$1,71368\E-1$; c)~$2,79554\E-1$
\end{resp}

\begin{exer}\label{exer:int_NC_tab}
  Considere a seguinte tabela de pontos
  \begin{center}
    \begin{tabular}{l|cccccc}
      $i$ & $1$ & $2$ & $3$ & $4$ & $5$ & $6$ \\\hline
      $x_i$ & $2,0$ & $2,1$ & $2,2$ & $2,3$ & $2,4$ & $2,5$ \\
      $y_i$ & $1,86$ & $1,90$ & $2,01$ & $2,16$ & $2,23$ & $2,31$ \\\hline
    \end{tabular}
  \end{center}
Assumindo que $y = f(x)$, calcule:
\begin{enumerate}[a)]
\item $\displaystyle \int_{2,1}^{2,3} f(x)\,dx$ usando a regra do ponto médio.
\item $\displaystyle \int_{2,0}^{2,5} f(x)\,dx$ usando a regra do trapézio.
\item $\displaystyle \int_{2,0}^{2,4} f(x)\,dx$ usando a regra de Simpson.
\end{enumerate}
\end{exer}
\begin{resp}
  \ifisoctave 
  \href{https://github.com/phkonzen/notas/blob/master/src/MatematicaNumerica/cap_integr/dados/exer_int_NC_tab/exer_int_NC_tab.m}{Código.} 
  \fi
  a)~$4,02000\E-1$; b)~$1,04250E+0$; c)~$8,08667\E-1$
\end{resp}

\begin{exer}\label{exer:trunc_pto_medio}
  Mostre que o erro de truncamento da regra do ponto médio é da ordem de $h^3$, onde $h$ é o tamanho do intervalo de integração.
\end{exer}
\begin{resp}
  Use um procedimento semelhante aquele usado para determinar a ordem do erro de truncamento da regra de Simpson.
\end{resp}

\begin{exer}\label{exer:NC_aberta_2pts}
  Obtenha a regra de Newton-Cotes aberta de $2$ pontos e estime seu erro de truncamento.
\end{exer}
\begin{resp}
  \begin{align}
    \displaystyle \int_a^bf(x)\,dx &= \frac{3h}{2}\left[f\left(a+\frac{1}{3}(b-a)\right)\right. \\
    &+ \left. f\left(a + \frac{2}{3}(b-a)\right)\right] + O(h^3), ~h=\frac{(b-a)}{3}
  \end{align}
\end{resp}

\section{Quadratura de Romberg}

\emconstrucao