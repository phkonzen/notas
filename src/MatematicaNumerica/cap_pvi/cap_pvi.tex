%Este trabalho está licenciado sob a Licença Atribuição-CompartilhaIgual 4.0 Internacional Creative Commons. Para visualizar uma cópia desta licença, visite http://creativecommons.org/licenses/by-sa/4.0/deed.pt_BR ou mande uma carta para Creative Commons, PO Box 1866, Mountain View, CA 94042, USA.

\chapter{Problema de valor inicial}\label{cap_pvi}
\thispagestyle{fancy}

Neste capítulo, discutimos sobre técnicas numéricas para aproximar a solução de equações diferenciais ordinárias com valor inicial, i.e. problemas da forma
\begin{align}
  y'(t) = f(t,y(t)),\quad t>t_0,\\
  y(t_0) = y_0.
\end{align}

\section{Método de Euler}

Dado um problema de valor inicial
\begin{align}
  y'(t) = f(t,y(t)),\quad t>t_0,\\
  y(t_0) = y_0,
\end{align}
temos que $f(t,y)$ é a derivada da solução $y(t)$ no tempo $t$. Então, aproximando a derivada pela razão fundamental de passo $h>0$, obtemos
\begin{align}
  \frac{y(t+h)-y(t)}{h} &\approx f(t,y) \\
  \Rightarrow y(t+h) &\approx y(t) + hf(t,y(t)).\label{eq:Euler_aux1}
\end{align}
Ou seja, se conhecermos a solução $y$ no tempo $t$, então \eqref{eq:Euler_aux1} nos fornece uma aproximação da solução $y$ no tempo $t+h$. Observemos que isto poder ser usado de forma iterativa. Da condição inicial $y(t_0)=y_0$, computamos uma aproximação de $y$ no tempo $t_0+h$. Usando esta no lugar de $y(t_0+h)$, \eqref{eq:Euler_aux1} com $t_0+h$ no lugar de $t$ nos fornece uma aproximação para $y(t_0+2h)$ e, assim, sucessivamente.

Mais especificamente, denotando $y^{(i)}$ a aproximação de $y(t^{(i)})$ com $t^{(i)}=t_0+(i-1)h$, $i=1, 2, \dotsc, n$, o \emph{método de Euler} consiste na iteração
\begin{align}
  y^{(1)} &= y_0,\\
  y^{(i+1)} &= y^{(i)} + hf(t^{(i)},y^{(i)}),
\end{align}
com $i=1, 2, \dotsc, n$. O número de iteradas $n$ e o tamanho do passo $h>0$, determinam os tempos discretos $t^{(i)}$ nos quais a solução $y$ será aproximada.

\begin{ex}\label{ex:pvi_Euler}
  Consideremos o seguinte problema de valor inicial
  \begin{align}
    y' - y &= \sen(t), t>0\\
    y(0) &= \frac{1}{2}.
  \end{align}
\end{ex}

\begin{align}
  & y' - y = \sen(t) \\
  & u = e^{-t}y\\
  & \frac{d}{t}(e^{-t}u) = e^{-t}\sen(t)\\
  & e^{-t}y - y_0 = \int_{t_0}^t e^{-s}\sen(s)\,ds\\
  &  y(t) = e^t  - \frac{1}{2}\sen(t) - \frac{1}{2}\cos(t)
\end{align}


\emconstrucao