%Este trabalho está licenciado sob a Licença Atribuição-CompartilhaIgual 4.0 Internacional Creative Commons. Para visualizar uma cópia desta licença, visite http://creativecommons.org/licenses/by-sa/4.0/deed.pt_BR ou mande uma carta para Creative Commons, PO Box 1866, Mountain View, CA 94042, USA.

\chapter{Problema de valor inicial}\label{cap_pvi}
\thispagestyle{fancy}

Neste capítulo, discutimos sobre técnicas numéricas para aproximar a solução de equações diferenciais ordinárias com valor inicial, i.e. problemas da forma
\begin{align}
  y'(t) &= f(t,y(t)),\quad t>t_0,\\
  y(t_0) &= y_0.
\end{align}

\section{Método de Euler}\label{cap_pvi_sec_Euler}

Dado um problema de valor inicial
\begin{align}
  y'(t) &= f(t,y(t)),\quad t>t_0,\label{eq:Euler_pvi_1}\\
  y(t_0) &= y_0,\label{eq:Euler_pvi_2}
\end{align}
temos que $f(t,y)$ é a derivada da solução $y(t)$ no tempo $t$. Então, aproximando a derivada pela razão fundamental de passo $h>0$, obtemos
\begin{align}
  \frac{y(t+h)-y(t)}{h} &\approx f(t,y) \\
  \Rightarrow y(t+h) &\approx y(t) + hf(t,y(t)).\label{eq:Euler_aux1}
\end{align}
Ou seja, se conhecermos a solução $y$ no tempo $t$, então \eqref{eq:Euler_aux1} nos fornece uma aproximação da solução $y$ no tempo $t+h$. Observemos que isto poder ser usado de forma iterativa. Da condição inicial $y(t_0)=y_0$, computamos uma aproximação de $y$ no tempo $t_0+h$. Usando esta no lugar de $y(t_0+h)$, \eqref{eq:Euler_aux1} com $t_0+h$ no lugar de $t$ nos fornece uma aproximação para $y(t_0+2h)$ e, assim, sucessivamente.

Mais especificamente, denotando $y^{(i)}$ a aproximação de $y(t^{(i)})$ com $t^{(i)}=t_0+(i-1)h$, $i=1, 2, \dotsc, n$, o \emph{método de Euler} consiste na iteração
\begin{align}
  y^{(1)} &= y_0,\label{eq:iter_Euler_1}\\
  y^{(i+1)} &= y^{(i)} + hf(t^{(i)},y^{(i)}),\label{eq:iter_Euler_2}
\end{align}
com $i=1, 2, \dotsc, n$. O número de iteradas $n$ e o tamanho do passo $h>0$, determinam os tempos discretos $t^{(i)}$ nos quais a solução $y$ será aproximada.

\begin{ex}\label{ex:Euler_1}
  Consideremos o seguinte problema de valor inicial
  \begin{align}
    y' - y &= \sen(t), t>0\label{eq:Euler_aux2}\\
    y(0) &= \frac{1}{2}.
  \end{align}
  A solução analítica deste é
  \begin{equation}
    y(t) = e^t - \frac{1}{2}\sen(t) - \frac{1}{2}\cos(t).
  \end{equation}
No tempo, $t_f=1$, temos $y(t_f)=e^{t_f} - \sen(1)/2 - \cos(1)/2 = 2,02740$. Agora, computarmos uma aproximação para este problema pelo método de Euler, reescrevemos \eqref{eq:Euler_aux2} na forma
\begin{equation}
  y' = y + \sen(t) =: f(t,y).
\end{equation}
Então, escolhendo $h=0,1$, a iteração do método de Euler \eqref{eq:iter_Euler_1}-\eqref{eq:iter_Euler_2} nos fornece
o método de Euler com passo $h=0,1$
\begin{align}
  y^{(1)} &= 0,5\\
  y^{(2)} &= y^{(1)} + hf(t^{(1)},y^{(1)})\nonumber\\
  &= 0,5 + 0,1[0,5 + \sen(0)]\nonumber\\
  &= 0,55\\
  y^{(3)} &= y^{(2)} + hf(t^{(2)},y^{(2)})\nonumber\\
  &= 0,55 + 0,1[0,55 + \sen(0,1)]\nonumber\\
  &= 6,14983\E-01\\
  &\,~\vdots\nonumber\\
  y^{(11)} &= 1,85259
\end{align}
Na Figura~\ref{fig:ex_Euler_1}, temos os esboços das soluções analítica e numérica.

\begin{figure}[h!]
  \centering
  \includegraphics[width=\textwidth]{./cap_pvi/dados/ex_Euler_1/ex_Euler_1}
  \caption{Esboço das soluções referente ao Exemplo~\ref{ex:Euler_1}.}
  \label{fig:ex_Euler_1}
\end{figure}

\ifisoctave
As aproximações obtidas neste exemplo podem ser computadas no \verb+GNU Octave+ com o seguinte código:
\begin{verbatim}
f = @(t,y) y+sin(t);

h=0.1;
n=11;
t=zeros(n,1);
y=zeros(n,1);

t(1)=0;
y(1)=0.5;

for i=1:n-1
  t(i+1) = t(i)+h;
  y(i+1)=y(i)+h*f(t(i),y(i));
endfor

printf("%1.5E %1.5E\n",t(n),y(n))
tt=linspace(0,1);
ya = @(t) exp(t)-sin(t)/2-cos(t)/2;
plot(tt,ya(tt),'b-',...
     t,y,'r.-');grid
legend("analit.","Euler")
\end{verbatim}
\fi
\end{ex}

\subsection{Análise de consistência e convergência}

O método de Euler com passo $h$ aplicado ao problema de valor inicial \eqref{eq:Euler_pvi_1}-\eqref{eq:Euler_pvi_2}, pode ser escrito da seguinte forma
\begin{align}
  \tilde{y}(t^{(1)};h) &= y_0,\label{eq:MPS_1}\\
  \tilde{y}(t^{(i+1)};h) &= \tilde{y}(t^{(i)};h) + h\Phi(t^{(i)},\tilde{y}(t^{(i)});h),\label{eq:MPS_2}
\end{align}
onde $\tilde{y}(t^{(i)})$ representa a aproximação da solução exata $y$ no tempo $t^{(i)}=t_0+(i-1)h$, $i=1, 2, \ldots$. Métodos que podem ser escritos desta forma, são chamados de métodos de passo simples (ou único). No caso específico do método de Euler, temos
\begin{equation}
  \Phi(t,y;h) := f(t,y(t)).
\end{equation}

Agora, considerando a solução exata $y(t)$ de \eqref{eq:Euler_pvi_1}-\eqref{eq:Euler_pvi_2}, introduzimos
\begin{equation}
  \Delta(t,y;h) := \left\{
    \begin{array}{ll}
      \frac{y(t+h)-y(t)}{h} &, h\neq 0,\\
      f(t,y(t)) &, h=0,
    \end{array}\right.
\end{equation}

Com isso, vamos analisar o chamado \emph{erro de discretização local}
\begin{equation}
  \tau(t,y;h) := \Delta(t,y;h) - \Phi(t,y;h),
\end{equation}
a qual estabelece uma medida quantitativa com que a solução exata $y(t)$ no tempo $t+h$ satisfaz a iteração de Euler.

\begin{defn}\normalfont{(Consistência)}\label{defn:pvi_consistencia}
  Um método de passo simples \eqref{eq:MPS_1}-\eqref{eq:MPS_2} é dito consistente quando
  \begin{equation}
    \lim_{h\to 0}\tau(t,y;h) = 0,
  \end{equation}
ou, equivalentemente, quando
\begin{equation}
  \lim_{h\to 0} \Phi(t,y;h) = f(t,y).
\end{equation}
\end{defn}

\begin{obs}
  Da Definição~\ref{defn:pvi_consistencia}, temos que o método de Euler é consistente.
\end{obs}

A \emph{ordem do erro de discretização local} de um método de passo simples \eqref{eq:MPS_1}-\eqref{eq:MPS_2} é dita ser $p$, quando
\begin{equation}
  \tau(t,y;h) = O(h^p).
\end{equation}

Para determinarmos a ordem do método de Euler, tomamos a expansão em série de Taylor da solução exata $y(t)$ em torno de $t$, i.e.
\begin{equation}
  y(t+h) = y(t) + hy'(t) + \frac{h^2}{2}y''(t) + \frac{h^3}{6}y'''(t+\theta h), ~0<\theta<1.
\end{equation}
Como $y(t)=f(t,y(t))$ e assumindo a devida suavidade de $f$, temos
\begin{align}
  y''(t) &= \frac{d}{dt}f(t,y(t)) \\
         &= f_t(t,y) + f_y(t,y)y'\\
         &= f_t(t,y) + f_y(t,y)f(t,y).
\end{align}
Então,
\begin{equation}
  \Delta(t,y;h) = f(t,y(t)) + \frac{h}{2}[f_t(t,y) + f_y(t,y)f(t,y)] + O(h^2).
\end{equation}
Portanto, para o método de Euler temos
\begin{align}
  \tau(t,y;h) &:= \Delta(t,y;h)-\Phi(t,y;h)\\
              &= \frac{h}{2}[f_t(t,y) + f_y(t,y)f(t,y)] + O(h^2)\\
              &= O(h).
\end{align}
Isto mostra que o método de Euler é um método de ordem $1$.

A análise acima trata apenas da consistência do método de Euler. Para analisarmos a convergência de métodos de passo simples, definimos o \emph{erro de discretização global}
\begin{equation}
  e(t;h_n) := \tilde{y}(t;h_n) - y(t),\quad h_n := \frac{t-t_0}{n}.
\end{equation}
E, com isso, dizemos que o método é \emph{convergente} quando
\begin{equation}
  \lim_{n\to \infty} e(t,h_n) = 0,
\end{equation}
bem como, dizemos que o método tem erro de discretização global de ordem $h^p$ quando $e(t,h_n) = O(h^p)$.

\begin{obs}
  Pode-se mostrar que, assumindo a devida suavidade de $f$, que a ordem do erro de discretização global de um método de passo simples é igual a sua ordem do erro de discretização local (veja, \cite[Cap. 7, Sec. 7.2]{Stoer1993a}). Portanto, o método de Euler é convergente e é de ordem $1$.
\end{obs}

\begin{ex}\label{ex:Euler_1}
  Consideremos o seguinte problema de valor inicial
  \begin{align}
    y' - y &= \sen(t), t>0\label{eq:Euler_aux2}\\
    y(0) &= \frac{1}{2}.
  \end{align}
  Na Tabela~\ref{tab:ex_Euler_2}, temos as aproximações $\tilde{y}(1)$ de $y(1)$ computadas pelo método de Euler com diferentes passos $h$.
 
  \begin{table}[h!]
    \centering
    \begin{tabular}{l|cc}
      $h$ & $\tilde{y}(1)$ & $|\tilde{y}(1)-y(1)|$\\\hline
      $10^{-1}$ & $1,85259$ & $1,7\E-01$ \\
      $10^{-2}$ & $2,00853$ & $1,9\E-02$ \\
      $10^{-3}$ & $2,02549$ & $1,9\E-03$ \\
      $10^{-5}$ & $2,02735$ & $4,8\E-05$ \\
      $10^{-7}$ & $2.02739$ & $1,9\E-07$ \\\hline
    \end{tabular}
    \caption{Resultados referentes ao Exemplo~\ref{ex:Euler_2}}
    \label{tab:ex_Euler_2}
  \end{table}

\ifisoctave
Os resultados mostrados na Tabela~\ref{tab:ex_Euler_2} podem ser computados no \verb+GNU Octave+ com o auxílio do seguinte código:
\begin{verbatim}
f = @(t,y) y+sin(t);

h=1e-2;
n=fix(1/h+1);
t=zeros(n,1);
y=zeros(n,1);

t(1)=0;
y(1)=0.5;

for i=1:n-1
  t(i+1) = t(i)+h;
  y(i+1)=y(i)+h*f(t(i),y(i));
endfor

ya = @(t) exp(t)-sin(t)/2-cos(t)/2;
printf("%1.5E %1.1E\n",y(n),abs(y(n)-ya(1)))
\end{verbatim}
\fi
\end{ex}

\subsection*{Exercícios}

\emconstrucao

\section{Métodos de Runge-Kutta}\label{cap_pvi_sec_RK}

\emconstrucao