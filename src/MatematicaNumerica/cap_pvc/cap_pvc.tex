%Este trabalho está licenciado sob a Licença Atribuição-CompartilhaIgual 4.0 Internacional Creative Commons. Para visualizar uma cópia desta licença, visite http://creativecommons.org/licenses/by-sa/4.0/deed.pt_BR ou mande uma carta para Creative Commons, PO Box 1866, Mountain View, CA 94042, USA.

\chapter{Problema de valor de contorno}\label{cap_pvc}
\thispagestyle{fancy}

Neste capítulo, discutimos sobre a aplicação do método de diferenças finitas para aproximar a solução de problemas de valores de contorno da forma
\begin{align}
  -\alpha u_{xx} &+ \beta u_{x} + \gamma u = f(x),\quad c_1 < x < c_2,\\
  \eta_1 u'(a) &+ \theta_1 u(a) = g_1\\
  -\eta_2 u'(b) &+ \theta_2 u(b) = g_2
\end{align}
onde a incógnita $u = u(x)$ e os coeficientes $\alpha$, $\beta$, $\gamma$, $\eta_1$, $\eta_2$, $\theta_1$ e $\theta_2$ são dados e podem ser função de $x$.

\section{Equação de difusão}

A equação de difusão (ou equação do calor) é uma equação diferencial ordinária de segunda ordem da forma
\begin{equation}\label{eq:diffusao}
  -\alpha u_{xx} = f(x),
\end{equation}
onde, aqui, assumimos definida em um intervalo limitado $c_1 < x < c_2$.

A aplicação do método de diferenças aplicado a esta equação consiste em substituir a derivada $u_{xx}$ por uma fórmula de diferenças finitas adequada. Comumente, para a equação de difusão, escolhemos a aproximação dada pela fórmula de diferenças finitas central para a segunda derivada, i.e.
\begin{equation}
  u_{xx}(x) \approx \frac{u(x-h)-2u(x)+u(x+h)}{h^2},
\end{equation}
onde $c_1 < x < c_2$ e $h>0$ é um passo adequadamente escolhido. Utilizando esta aproximação em \eqref{eq:difusao} para um $x$ fixo, escrevemos o seguinte problema
\begin{equation}
  -\alpha \frac{u(x-h)-2u(x)+u(x+h)}{h^2} = f(x),
\end{equation}
o qual é uma aproximação discreta de \label{eq:difusao} 


\emconstrucao