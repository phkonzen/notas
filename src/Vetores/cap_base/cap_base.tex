%Este trabalho está licenciado sob a Licença Atribuição-CompartilhaIgual 4.0 Internacional Creative Commons. Para visualizar uma cópia desta licença, visite http://creativecommons.org/licenses/by-sa/4.0/deed.pt_BR ou mande uma carta para Creative Commons, PO Box 1866, Mountain View, CA 94042, USA.

\chapter{Bases e coordenadas}\label{cap_base}
\thispagestyle{fancy}

\section{Combinação linear}\label{cap_base_sec_comblin}

\begin{flushright}
  \href{https://archive.org/details/combinacao-linear}{$\blacktriangleright$ Vídeo disponível!}
\end{flushright}

Dados vetores $\vec{u}_1$, $\vec{u}_2$, $\dotsc$, $\vec{u}_n$ e números reais $c_1$, $c_2$, $\dotsc$, $c_n$, com $n$ inteiro positivo, chamamos de
\begin{equation}
  {\color{blue}\vec{u} = c_1\vec{u}_1 + c_2\vec{u}_2 + \cdots + c_n\vec{u}_n}
\end{equation}
uma \emph{combinação linear} de $\vec{u}_1$, $\vec{u}_2$, $\dotsc$, $\vec{u}_n$. Neste caso, também dizemos que $\vec{u}$ é \emph{gerado} pelos vetores $\vec{u}_1$, $\vec{u}_2$, $\dotsc$, $\vec{u}_n$ ou, equivalentemente, que estes vetores \emph{geram} o vetor $\vec{u}$.

\begin{ex}\label{ex:comblinear}
  Sejam dados os vetores $\vec{u}$, $\vec{v}$, $\vec{w}$ e $\vec{z}$. Então, temos:
  \begin{enumerate}[a)]
  \item $\vec{u}_1 = \frac{1}{2}\vec{v} + \sqrt{2}\vec{z}$ é uma combinação linear dos vetores $\vec{v}$ e $\vec{z}$.
  \item $\vec{u_2} = \vec{u} - 2\vec{z}$ é uma outra combinação linear dos vetores $\vec{u}$ e $\vec{z}$.
  \item $\vec{u_3} = 2\vec{u} - \vec{w} + \pi\vec{z}$ é uma combinação linear dos vetores $\vec{u}$, $\vec{w}$ e $\vec{z}$.
  \item $\vec{u_4} = \frac{3}{2}\vec{z}$ é uma combinação linear do vetor $\vec{z}$.
  \end{enumerate}
\end{ex}

\begin{obs}\normalfont{(Interpretação geométrica)}
  \begin{enumerate}[a)]
  \item Uma combinação linear não nula envolvendo um único vetor $\vec{u}$ é um vetor paralelo a $\vec{u}$. De fato, seja
    \begin{equation}
      \vec{v} = c\vec{u},\quad c\neq 0,
    \end{equation}
    i.e. $\vec{v}$ é combinação linear não nula de $\vec{u}$. Então, $\vec{v}$ tem a mesma direção de $\vec{u}$.
  \item Uma combinação linear não nula envolvendo dois vetores $\vec{u}$ e $\vec{v}$ é coplanar a estes vetores. De fato, seja
    \begin{equation}
      \vec{w} = c_1\vec{u} + c_2\vec{v},\quad c_1\cdot c_2 \neq 0,
    \end{equation}
    e $\pi$ o plano determinado pelas representações de $\vec{u} = \overrightarrow{AB}$ e $\vec{v} = \overrightarrow{AC}$. Logo, seguindo a regra do paralelogramo, vemos que $\vec{w}$ tem uma representação no plano determinado pelos segmentos $AB$ e $AC$.
  \end{enumerate}
\end{obs}

\subsection*{Exercícios resolvidos}

\begin{exeresol}\label{exeresol:comblin_geo}
  Com base na figura abaixo, escreva o vetor $\vec{u}$ como combinação linear dos vetores $\vec{i}=\overrightarrow{OA}$ e $\vec{j}=\overrightarrow{OB}$.

  \begin{figure}[H]
    \centering
    \includegraphics[width=0.6\textwidth]{cap_base/dados/fig_comblin_exeresol_geo/fig_comblin_exeresol_geo}
    \caption{ER \ref{exeresol:comblin_geo}.}
    \label{fig:comblin_exeresol_geo}
  \end{figure}
\end{exeresol}
\begin{resol}
  Para escrevermos o vetor $\vec{u}$ como combinação linear dos vetores $\vec{i}$ e $\vec{j}$, devemos determinar números $c_1$ e $c_2$ tais que
  \begin{equation}
    \vec{u} = c_1\vec{i} + c_2\vec{j}.
  \end{equation}
  Com base na Figura \ref{fig:comblin_exeresol_geo}, podemos tomar $c_1 = 3$ e $c_2=2$, i.e. temos
  \begin{equation}
    \vec{u} = 3\vec{i} + 2\vec{j}.
  \end{equation}
\end{resol}

\begin{exeresol}
  Sabendo que $\vec{u}=2\vec{v}$, forneça três maneiras de escrever o vetor nulo $\vec{0}$ como combinação linear dos vetores $\vec{u}$ e $\vec{v}$.
\end{exeresol}
\begin{resol}
  \begin{enumerate}[a)]
  \item
    \begin{gather}
      \vec{u}=2\vec{v}\\
      \vec{0}=2\vec{v}-\vec{u}
    \end{gather}
  \item
    \begin{gather}
      \vec{u}=2\vec{v}\\
      \vec{u}-2\vec{v}=\vec{0}\\
      \vec{0}=\vec{u}-2\vec{v}
    \end{gather}
  \item
    \begin{gather}
      \vec{u}=2\vec{v}\\
      \frac{1}{2}\vec{u}=\vec{v}\\
      \vec{0}=\vec{v}-\frac{1}{2}\vec{u}
    \end{gather}
  \end{enumerate}
\end{resol}

\subsection*{Exercícios}

\begin{exer}\label{exer:comblin_geo1}
  Com base na figura abaixo, escreva $\vec{u}$ como combinação linear dos vetores $\vec{i}=\overrightarrow{OA}$ e $\vec{j}=\overrightarrow{OB}$.

  \begin{figure}[H]
    \centering
    \includegraphics[width=0.4\textwidth]{cap_base/dados/fig_comblin_exer_geo/fig_comblin_exer_geo1}    
    \caption{E \ref{exer:comblin_geo1}.}
    \label{fig:comblin_geo1}
  \end{figure}
\end{exer}
\begin{resp}
  $\vec{u}=\vec{i}+2\vec{j}$
\end{resp}

\begin{exer}\label{exer:comblin_geo2}
  Com base na figura abaixo, escreva $\vec{u}=$ como combinação linear dos vetores $\vec{i}=\overrightarrow{OA}$ e $\vec{j}=\overrightarrow{OB}$.

  \begin{figure}[H]
    \centering
    \includegraphics[width=0.5\textwidth]{cap_base/dados/fig_comblin_exer_geo/fig_comblin_exer_geo2}    
    \caption{E \ref{exer:comblin_geo2}.}
    \label{fig:comblin_geo2}
  \end{figure}
\end{exer}
\begin{resp}
  $\vec{u}=-3\vec{i}+\vec{j}$
\end{resp}

\begin{exer}\label{exer:comblin_geo3}
  Com base na figura abaixo, escreva $\vec{u}=$ como combinação linear dos vetores $\vec{i}=\overrightarrow{OA}$ e $\vec{j}=\overrightarrow{OB}$.

  \begin{figure}[H]
    \centering
    \includegraphics[width=0.4\textwidth]{cap_base/dados/fig_comblin_exer_geo/fig_comblin_exer_geo3}    
    \caption{E \ref{exer:comblin_geo3}.}
    \label{fig:comblin_geo3}
  \end{figure}
\end{exer}
\begin{resp}
  $\vec{u}=\vec{i}-\vec{j}$
\end{resp}

\begin{exer}
  Sabendo que $\vec{u}=3\vec{w}+\vec{v}$, escreva $\vec{w}$ como combinação linear de $\vec{u}$ e $\vec{v}$.
\end{exer}
\begin{resp}
  $\vec{w} = \frac{1}{3}\vec{u} - \frac{1}{3}\vec{v}$
\end{resp}

\begin{exer}
  Sejam $\vec{u}$ e $\vec{v}$ vetores de mesma direção e $\vec{w}$ um vetor não paralelo a $\vec{u}$, todos não nulos. Pode-se escrever $\vec{w}$ como combinação linear de $\vec{u}$ e $\vec{v}$? Justifique sua resposta.
\end{exer}
\begin{resp}
  Não.
\end{resp}

\begin{exer}
  Sejam $\vec{u}$ e $\vec{v}$ ambos não nulos e de mesma direção. Pode-se afirmar que $\vec{u}$ gera $\vec{v}$? Justifique sua resposta.
\end{exer}
\begin{resp}
  Sim.
\end{resp}

\begin{exer}
  Sejam $\vec{u}$ e $\vec{v}$ coplanares com direções diferentes e $\vec{w}$ um vetor não coplanar a $\vec{u}$ e $\vec{v}$, todos não nulos. É possível gerar $\vec{w}$ com $\vec{u}$ e $\vec{v}$?
\end{exer}
\begin{resp}
  Não.
\end{resp}

\begin{exer}
  Sejam $\vec{u}$ e $\vec{v}$ não nulos, coplanares e com direções distintas. Se $\vec{w}$ é um vetor também coplanar a $\vec{u}$ e $\vec{v}$, então $\vec{u}$ e $\vec{v}$ geram $\vec{w}$? Justifique sua resposta.
\end{exer}
\begin{resp}
  Sim.
\end{resp}

\section{Dependência linear}\label{cap_base_sec_deplinear}

Dois ou mais vetores dados são \emph{linearmente dependentes} (l.d.) quando um deles for combinação linear dos demais.

\begin{ex}\label{ex:deplinear}
  No exemplo anterior (Exemplo \ref{ex:comblinear}), temos:
  \begin{enumerate}[a)]
  \item $\vec{u_1}$ é linearmente dependente (l.d.) dos vetores $\vec{v}$ e $\vec{z}$.
  \item $\vec{u_2}$ é l.d. a $\vec{u}$ e $\vec{z}$.
  \item $\vec{u_3}$ depende linearmente dos vetores $\vec{u}$, $\vec{v}$ e $\vec{z}$.
  \item Os vetores $\vec{u_4}$ e $\vec{z}$ são linearmente dependentes.
  \end{enumerate}
\end{ex}

Dois ou mais vetores dados são \emph{linearmente independentes} (l.i.) quando eles não são linearmente dependentes.

\subsection{Observações}\label{cap_base_sec_deplin_subsec_obs}

\subsubsection{Dois vetores}

Dois vetores quaisquer $\vec{u}\neq\vec{0}$ e $\vec{v}\neq\vec{0}$ são l.d. se, e somente se, qualquer uma das seguinte condições é satisfeita:
\begin{enumerate}[a)]
\item um deles é combinação linear do outro, i.e.
  \begin{equation}
    \vec{u} = \alpha\vec{v}\quad\text{ou}\quad\vec{v}=\beta\vec{u};
  \end{equation}
\item $\vec{u}$ e $\vec{v}$ têm a mesma direção;
\item $\vec{u}$ e $\vec{v}$ são paralelos.
\end{enumerate}
De fato, a afirmação a) é a definição de dependência linear. A b) é consequência imediata da a), bem como a c) é equivalente a b). Por fim, se $\vec{u}$ e $\vec{v}$ são vetores paralelos, então um é múltiplo por escalar do outro. Ou seja, c) implica a).

\begin{obs}
  O vetor nulo $\vec{0}$ é l.d. a qualquer vetor $\vec{u}$. De fato, temos
  \begin{equation}
    \vec{0} = 0\cdot\vec{u},
  \end{equation}
  i.e. o vetor nulo é combinação linear do vetor $\vec{u}$.
\end{obs}

\begin{obs}
  Dois vetores não nulos $\vec{u}$ e $\vec{v}$ são l.i. se, e somente se,
  \begin{equation}
    \alpha\vec{u} + \beta\vec{v} = \vec{0} \Rightarrow \alpha=\beta=0.
  \end{equation}
  De fato, se $\alpha\neq 0$, então podemos escrever
  \begin{equation}
    \vec{u} = -\frac{\beta}{\alpha}\vec{v},
  \end{equation}
  i.e. o vetor $\vec{u}$ é combinação linear do vetor $\vec{v}$ e, portanto, estes vetores são l.d.. Isto contradiz a hipótese de eles serem l.i.. Analogamente, se $\beta \neq 0$, então podemos escrever
  \begin{equation}
    \vec{v} = -\frac{\alpha}{\beta}\vec{u}
  \end{equation}
  e, então, teríamos $\vec{u}$ e $\vec{v}$ l.d..
\end{obs}


\subsubsection{Três vetores}

Três vetores quaisquer $\vec{u}$, $\vec{v}$ e $\vec{w}$ são l.d. quando um deles pode ser escrito como combinação linear dos outros dois. Sem perda de generalidade, isto significa que existem constantes $\alpha$ e $\beta$ tais que
\begin{equation}
  \vec{u} = \alpha\vec{v} + \beta\vec{w}.
\end{equation}

\emph{Afirmamos que se $\vec{u}$, $\vec{v}$ e $\vec{w}$ são l.d., então $\vec{u}$, $\vec{v}$ e $\vec{w}$ são coplanares.} Do fato de que dois vetores quaisquer são sempre coplanares, temos que $\vec{u}$, $\vec{v}$ e $\vec{w}$ são coplanares caso qualquer um deles seja o vetor nulo. Suponhamos, agora, que $\vec{u}$, $\vec{v}$ e $\vec{w}$ são não nulos e seja $\pi$ o plano determinado pelos vetores $\vec{v}$ e $\vec{w}$. Se $\alpha = 0$, então $\vec{u} = \beta\vec{w}$ e teríamos uma representação de $\vec{u}$ no plano $\pi$. Analogamente, se $\beta=0$, então $\vec{u} = \alpha\vec{v}$ e teríamos uma representação de $\vec{u}$ no plano $\pi$. Por fim, observamos que se $\alpha,\beta\neq 0$, então $\alpha\vec{v}$ tem a mesma direção de $\vec{v}$ e $\beta\vec{w}$ tem a mesma direção de $\vec{w}$. Isto é, $\alpha\vec{v}$ e $\beta\vec{w}$ admitem representações no plano $\pi$. Sejam $\overrightarrow{AB}$ e $\overrightarrow{BC}$ representações dos vetores $\alpha\vec{v}$ e $\beta\vec{w}$, respectivamente. Os pontos $A$, $B$ e $C$ pertencem a $\pi$, assim como o segmento $AC$. Como $\overrightarrow{AC} = \vec{u} = \alpha\vec{v} + \beta\vec{w}$, concluímos que $\vec{u}$, $\vec{v}$ e $\vec{w}$ são coplanares.

Reciprocamente, \emph{se $\vec{u}$, $\vec{v}$ e $\vec{w}$ são coplanares, então $\vec{u}$, $\vec{v}$ e $\vec{w}$ são l.d..} De fato, se um deles for nulo, por exemplo, $\vec{u}=\vec{0}$, então $\vec{u}$ pode ser escrito como a seguinte combinação linear dos vetores $\vec{v}$ e $\vec{w}$
\begin{equation}
  \vec{u} = 0\vec{v} + 0\vec{w}.
\end{equation}
Neste caso, $\vec{u}$, $\vec{v}$ e $\vec{w}$ são l.d.. Também, se dois dos vetores forem paralelos, por exemplo, $\vec{u}\parallel\vec{v}$, então temos a combinação linear
\begin{equation}
  \vec{u} = \alpha\vec{v} + 0\vec{w}.
\end{equation}
E, então, $\vec{u}$, $\vec{v}$ e $\vec{w}$ são l.d.. Agora, suponhamos que $\vec{u}$, $\vec{v}$ e $\vec{w}$ são não nulos e dois a dois concorrentes (i.e. todos com direções distintas). Sejam, então $\overrightarrow{PA}=\vec{u}$, $\overrightarrow{PB}=\vec{v}$ e $\overrightarrow{PC}=\vec{w}$ representações sobre um plano $\pi$. Sejam $r$ e $s$ as retas determinadas por $PA$ e $PC$, respectivamente. Seja, então, $D$ o ponto de interseção da reta $s$ com a reta paralela a $r$ que passa pelo ponto $B$. Seja, também, $E$ o ponto de interseção da reta $r$ com a reta paralela a $s$ que passa pelo ponto $B$. Sejam, então, $\alpha$ e $\beta$ tais que $\alpha\vec{u}=\overrightarrow{PE}$ e $\beta\vec{w}=\overrightarrow{PD}$. Como $\vec{v} = \overrightarrow{PB} = \overrightarrow{PE} + \overrightarrow{PD} = \alpha\vec{u}+\beta\vec{w}$, temos que $\vec{v}$ é combinação linear de $\vec{u}$ e $\vec{w}$, i.e. $\vec{u}$, $\vec{v}$ e $\vec{w}$ são l.d..

\begin{obs}\label{obs:cbsc_li}
  Três vetores dados $\vec{u}$, $\vec{v}$ e $\vec{w}$ são l.i. se, e somente se, 
  \begin{equation}
    \alpha\vec{u} + \beta\vec{v} + \gamma\vec{w} = 0 \Rightarrow \alpha=\beta=\gamma = 0.
  \end{equation}
  De fato, sem perda de generalidade, se $\alpha\neq 0$, podemos escrever
  \begin{equation}
    \vec{u} = -\frac{\beta}{\alpha}\vec{v} - \frac{\gamma}{\alpha}\vec{w},
  \end{equation}
  e teríamos $\vec{u}$, $\vec{v}$ e $\vec{w}$ vetores l.d..
\end{obs}

\subsubsection{Quatro ou mais vetores}

{\bf Quatro ou mais vetores são sempre l.d..} De fato, sejam dados quatro vetores $\vec{a}$, $\vec{b}$, $\vec{c}$ e $\vec{d}$. Se dois ou três destes forem l.d.entre si, então, por definição, os quatro são l.d.. Assim sendo, suponhamos que três dos vetores sejam l.i. e provaremos que, então, o outro vetor é combinação linear desses três.

Sem perda de generalidade, suponhamos que $\vec{a}$, $\vec{b}$ e $\vec{c}$ são l.i.. Logo, eles não são coplanares. Seja, ainda, $\pi$ o plano determinado pelos vetores $\vec{a}$, $\vec{b}$ e as representações $\vec{a}=\overrightarrow{PA}$, $\vec{b}=\overrightarrow{PB}$, $\vec{c}=\overrightarrow{PC}$ e $\vec{d}=\overrightarrow{PD}$.

\begin{figure}[H]
  \centering
  \includegraphics[width=0.7\textwidth]{./cap_base/dados/fig_4vec_ld/fig_4vec_ld}
  \caption{Quatro vetores são l.d..}
  \label{fig:4vec_ld}
\end{figure}

Consideremos a reta $r$ paralela a $\overrightarrow{PC}$ que passa pelo ponto $D$. Então, seja $E$ o ponto de interseção de $r$ com o plano $\pi$. Vejamos a Figura \ref{fig:4vec_ld}. Observamos que o vetor $\overrightarrow{PE}$ é coplanar aos vetores $\overrightarrow{PA}$ e $\overrightarrow{PB}$ e, portanto, exitem números reais $\alpha$ e $\beta$ tal que
\begin{equation}
  \overrightarrow{PE} = \alpha\overrightarrow{PA} + \beta\overrightarrow{PB}.
\end{equation}
Além disso, como $\overrightarrow{ED}$ tem a mesma direção e sentido de $\overrightarrow{PC} = \vec{c}$, temos que
\begin{equation}
  \overrightarrow{ED} = \gamma\overrightarrow{PC}
\end{equation}
para algum número real $\gamma$. Por fim, observamos que
\begin{align*}
  \overrightarrow{PD} &= \overrightarrow{PE} + \overrightarrow{ED}\\
                      &= \alpha\overrightarrow{PA} + \beta\overrightarrow{PB} + \gamma\overrightarrow{PC}\\
                      &= \alpha\vec{a} + \beta\vec{b} + \gamma\vec{c}.
\end{align*}

\subsection*{Exercícios resolvidos}

\begin{exeresol}
  Se $\vec{u}$ e $\vec{v}$ são l.i. e
  \begin{align}
    \vec{a} &= 2\vec{u} - 3\vec{v},\\
    \vec{b} &= \vec{u} + 2\vec{v},
  \end{align}
  então $\vec{a}$ e $\vec{b}$ são l.d.?
\end{exeresol}
\begin{resol}
  Os vetores $\vec{a}$ e $\vec{b}$ são l.i. se, e somente se,
  \begin{equation}
    \alpha\vec{a} + \beta\vec{b} = \vec{0} \Rightarrow \alpha=\beta=0.
  \end{equation}
  Observemos que
  \begin{align}
    \vec{0} &= \alpha\vec{a}+\beta\vec{b}\\
            &= \alpha(2\vec{u} - 3\vec{v}) + \beta(\vec{u}+2\vec{v})\\
            &= (2\alpha+\beta)\vec{u} + (-3\alpha+2\beta)\vec{v}
  \end{align}
  implica
  \begin{align}
    2\alpha + \beta &= 0 \\
    -3\alpha + 2\beta &= 0
  \end{align}
  Resolvendo este sistema, vemos que $\alpha = \beta = 0$. Logo, concluímos que $\vec{a}$ e $\vec{b}$ são l.i.. 
\end{resol}

\begin{exeresol}
  Sejam $\vec{u}$, $\vec{v}$ e $\vec{w}$ três vetores. Verifique a seguinte afirmação de que se $\vec{u}$ e $\vec{v}$ são l.d., então $\vec{u}$, $\vec{v}$ e $\vec{w}$ são l.d.. Justifique sua resposta.
\end{exeresol}
\begin{resol}
  A afirmação é verdadeira. De fato, se $\vec{u}$ e $\vec{v}$ são l.d., então existe um escalar $\alpha$ tal que
  \begin{equation}
    \vec{u} = \alpha\vec{v}.
  \end{equation}
  Segue que
  \begin{equation}
    \vec{u} = \alpha\vec{v} + 0\vec{w}.
  \end{equation}
  Isto é, $\vec{u}$ é combinação linear de $\vec{v}$ e $\vec{w}$. Então, por definição, $\vec{u}$, $\vec{v}$ e $\vec{w}$ são l.d..
\end{resol}

\begin{exeresol}
  Sejam $\vec{u} = \overrightarrow{AB}$ e $\vec{v} = \overrightarrow{AC}$. Mostre que $A$, $B$ e $C$ são colineares se, e somente se, $\vec{u}$ e $\vec{v}$ são l.d..
\end{exeresol}
\begin{resol}
  Primeiramente, vamos verificar a implicação. Se $A$, $B$ e $C$ são colineares, então os segmentos $AB$ e $AC$ têm a mesma direção. Logo, são l.d. os vetores $\vec{u} = \overrightarrow{AB}$ e $\vec{v} = \overrightarrow{AC}$.

  Agora, verificamos a recíproca. Se $\vec{u} = \overrightarrow{AB}$ e $\vec{v} = \overrightarrow{AC}$ são l.d., então os segmentos $AB$ e $AC$ têm a mesma direção. Como eles são concorrentes, segue que $A$, $B$ e $C$ são colineares.
\end{resol}

\subsection*{Exercícios}

\begin{exer}
  Sendo $\overrightarrow{AB} + 2\overrightarrow{BC} = \vec{0}$, mostre que $\overrightarrow{PA}$, $\overrightarrow{PB}$ e $\overrightarrow{PC}$ são l.d. para qualquer ponto $P$.
\end{exer}
\begin{resp}
  Dica: os vetores $\overrightarrow{AB}$ e $\overrightarrow{BC}$ são l.d..
\end{resp}

\begin{exer}
  Sejam dados três vetores quaisquer $\vec{a}$, $\vec{b}$ e $\vec{c}$. Mostre que os vetores $\vec{u} = 2\vec{a}-\vec{b}$, $\vec{v}=-\vec{a}-2\vec{c}$ e $\vec{w}=\vec{b}+4\vec{c}$ são l.d..
\end{exer}
\begin{resp}
  Dica: Escreva um dos vetores como combinação linear dos outros.
\end{resp}

\begin{exer}
  Sejam $\vec{u} = \overrightarrow{AB}$, $\vec{v} = \overrightarrow{AC}$ e $\vec{w} = \overrightarrow{AD}$. Mostre que $A$, $B$, $C$ e $D$ são coplanares se, e somente se, $\vec{u}$, $\vec{v}$ e $\vec{w}$ são l.d..
\end{exer}
\begin{resp}
  Três vetores são l.d. se, e somente se, eles são coplanares.
\end{resp}

\begin{exer}
  Se $\vec{u}$ e $\vec{v}$ são l.i. e
  \begin{align}
    \vec{a} = 2\vec{u} - \vec{v},\\
    \vec{b} = 2\vec{v} - 4\vec{u},
  \end{align}
  então $\vec{a}$ e $\vec{b}$ são l.i.? Justifique sua resposta.
\end{exer}
\begin{resp}
  Não.
\end{resp}

\begin{exer}
  Verifique se é verdadeira ou falsa cada uma das seguintes afirmações. Justifique sua resposta.
  \begin{enumerate}[a)]
  \item $\vec{u}$, $\vec{v}$, $\vec{w}$ l.d. $\Rightarrow$ $\vec{u}$, $\vec{v}$ l.d..
  \item $\vec{u}$, $\vec{0}$, $\vec{w}$ são l.d..
  \item $\vec{u}$, $\vec{v}$ l.i. $\Rightarrow$ $\vec{u}$, $\vec{v}$ e $\vec{w}$ l.i..
  \item $\vec{u}$, $\vec{v}$, $\vec{w}$ l.d. $\Rightarrow$ $-\vec{u}$, $2\vec{v}$, $-3\vec{w}$ l.d..
  \end{enumerate}
\end{exer}
\begin{resp}
  a) falsa; b) verdadeira; c) falsa; d) verdadeira.
\end{resp}

\section{Bases e coordenadas}\label{cap_base_sec_base}

Seja $V$ o conjunto de todos os vetores no espaço tridimensional. Conforme discutido na Seção \ref{cap_base_sec_deplinear}, se $\vec{a}$, $\vec{b}$ e $\vec{c}$ são l.i., então qualquer vetor $\vec{u}\in V$ pode ser escrito como uma combinação linear destes vetores, i.e. existem números reais $\alpha$, $\beta$ e $\gamma$ tal que
\begin{equation}
  \vec{u} = \alpha\vec{a} + \beta\vec{b} + \gamma\vec{c}.
\end{equation}

A observação acima motiva a seguinte definição: uma {\bf base}\index{base} de $V$ é uma sequência de três vetores l.i. de $V$.

Seja $B = (\vec{a}, \vec{b}, \vec{c})$ uma dada base de $V$. Então, dado qualquer $\vec{v}\in V$, existe um único terno de números reais $\alpha$, $\beta$ e $\gamma$ tais que
\begin{equation}
  \vec{v} = \alpha\vec{a} + \beta\vec{b} + \gamma\vec{c}.\label{clsb_eq0}
\end{equation}
De fato, a existência de $\alpha$, $\beta$ e $\gamma$ segue imediatamente do fato de que $\vec{a}$, $\vec{b}$ e $\vec{c}$ são l.i. e, portanto, $\vec{v}$ pode ser escrito como uma combinação linear destes vetores. Agora, para verificar a unicidade de $\alpha$, $\beta$ e $\gamma$, tomamos $\alpha'$, $\beta'$ e $\gamma'$ tais que
\begin{equation}
  \vec{v} = \alpha'\vec{a} + \beta'\vec{b} + \gamma'\vec{c}.\label{clsb_eq1}
\end{equation}
Subtraindo \eqref{clsb_eq1} de \eqref{clsb_eq0}, obtemos
\begin{equation}
  \vec{0} = (\alpha-\alpha')\vec{a}+(\beta-\beta')\vec{b}+(\gamma-\gamma')\vec{c}.
\end{equation}
Como $\vec{a}$, $\vec{b}$ e $\vec{c}$ são l.i., segue que\footnote{Lembre-se da Observação \ref{obs:cbsc_li}.}
\begin{equation}
  \alpha-\alpha'=0,~\beta-\beta'=0,~\gamma-\gamma'=0,
\end{equation}
i.e. $\alpha=\alpha'$, $\beta=\beta'$ e $\gamma=\gamma'$.

\begin{figure}[H]
  \centering
  \includegraphics[width=0.7\textwidth]{./cap_base/dados/fig_coord/fig_coord}
  \caption{Representação de um vetor $\vec{u} = (u_1, u_2, u_3)_B$ em uma dada base $B=(\vec{a},\vec{b},\vec{c})$.}
  \label{fig:coord}
\end{figure}

Com isso, fixada uma base $B = (\vec{a}, \vec{b}, \vec{c})$, cada vetor $\vec{u}$ é representado de forma única como combinação linear dos vetores da base, digamos
\begin{equation}
  \vec{u} = u_1\vec{a} + u_2\vec{b} + u_3\vec{c},
\end{equation}
onde $u_1$, $u_2$ e $u_3$ são números reais fixos, chamados de {\bf coordenadas}\index{coordenadas} do $\vec{u}$ na base $B$. Ainda, usamos a notação
\begin{equation}
  \vec{u} = (u_1, u_2, u_3)_B,
\end{equation}
para expressar o vetor $\vec{u}$ nas suas coordenadas na base $B$. Vejamos a Figura \ref{fig:coord}.

\begin{ex}
  Fixada uma base $B = (\vec{a}, \vec{b}, \vec{c})$, o vetor $\vec{u}$ de coordenadas $\vec{u}=(-2,\sqrt{2},-3)_B$ é o vetor $\vec{u} = -2\vec{a} + \sqrt{2}\vec{b} - 3\vec{c}$.
\end{ex}

\subsection{Operações de vetores com coordenadas}


Na Seção \ref{cap_vetor_sec_vetor}, definimos as operações de adição, subtração e multiplicação por escalar do ponto de vista geométrico. Aqui, veremos como estas operação são definidas a partir das coordenadas de vetores.

Sejam $B = (\vec{a}, \vec{b}, \vec{c})$ uma base de $V$ e os vetores $\vec{u} = (u_1, u_2, u_3)_B$ e $\vec{v} = (v_1, v_2, v_3)_B$. Isto é, temos
\begin{align}
  \vec{u} &= u_1\vec{a} + u_2\vec{b} + u_3\vec{c},\\
  \vec{v} &= v_1\vec{a} + v_2\vec{b} + v_3\vec{c}.
\end{align}
Então, a {\bf adição} de $\vec{u}$ com $\vec{v}$ é a soma
\begin{align}
  \vec{u}+\vec{v} &= \underbrace{u_1\vec{a} + u_2\vec{b} + u_3\vec{c}}_{\vec{u}} + \underbrace{v_1\vec{a} + v_2\vec{b} + v_3\vec{c}}_{\vec{v}}\\
  &= (u_1+v_1)\vec{a} + (u_2+v_2)\vec{b} + (u_3+v_3)\vec{c},
\end{align}
ou seja
\begin{equation}
  \vec{u}+\vec{v}=(u_1+v_1, u_2+v_2, u_3+v_3)_B.
\end{equation}

\begin{ex}
  Fixada uma base qualquer $B$ e dados os vetores $\vec{u} = (2, -1, -3)_B$ e $\vec{v} = (-1, 4, -5)_B$, temos
  \begin{equation}
    \vec{u}+\vec{v} = \left(2+(-1), -1+4, -3+(-5)\right)_B = (1,3,-8)_B.
  \end{equation}

  \ifispython
  Podemos usar o \verb+SymPy+ para manipularmos vetores em coordenadas. Para computarmos a soma neste exemplo, podemos usar os seguintes comandos:
\begin{verbatim}
from sympy import *
u = Matrix([2,-1,-3])
v = Matrix([-1,4,-5])
u+v
\end{verbatim}
  \fi
\end{ex}

De forma, análoga, o {\bf vetor oposto} ao vetor $\vec{u}$ é
\begin{align}
  -\vec{u} &= -(\underbrace{u_1\vec{a} + u_2\vec{b} + u_3\vec{c}}_{\vec{u}})\\
           &= (-u_1)\vec{a} + (-u_2)\vec{b} + (-u_3)\vec{c},
\end{align}
ou seja,
\begin{equation}
  -\vec{u} = (-u_1, -u_2, -u_3)_B.
\end{equation}

\begin{ex}
  Fixada uma base qualquer $B$ e dado o vetor $\vec{v} = (2, -1, -3)_B$, temos
  \begin{equation}
    -\vec{v} = \left(-2, 1, 3\right)_B.
  \end{equation}

  \ifispython
  Usando o \verb+Sympy+, podemos computar o oposto do vetor $\vec{v}$ com os seguintes comandos:
\begin{verbatim}
from sympy import *
v = Matrix([2,-1,-3])
-v
\end{verbatim}
  \fi
\end{ex}


Lembrando que {\bf subtração} de $\vec{u}$ com $\vec{v}$ é $\vec{u}-\vec{v} := \vec{u} + (-\vec{v})$, segue
\begin{equation}
  \vec{u}-\vec{v} = (u_1-v_1, u_2-v_2, u_3-v_3)_B.
\end{equation}

\begin{ex}
  Fixada uma base qualquer $B$ e dados os vetores $\vec{u} = (2, -1, -3)_B$ e $\vec{v} = (-1, 4, -5)_B$, temos
  \begin{equation}
    \vec{u}-\vec{v} = \left(2-(-1), -1-4, -3-(-5)\right)_B = (3,-5,2)_B.
  \end{equation}

  \ifispython
  Usando o \verb+Sympy+, podemos computar $\vec{u}-\vec{v}$ com os seguintes comandos:
\begin{verbatim}
from sympy import *
u = Matrix([2,-1,-3])
v = Matrix([-1,4,-5])
u-v
\end{verbatim}
  \fi
\end{ex}


Com o mesmo raciocínio, fazemos a {\bf multiplicação de} um dado {\bf número} $\alpha$ pelo {\bf vetor} $\vec{u}$. Vejamos, por definição,
\begin{align}
  \alpha\vec{u} &= \alpha(\underbrace{u_1\vec{a} + u_2\vec{b} + u_3\vec{c}}_{\vec{u}})\\
                &= (\alpha u_1)\vec{a} + (\alpha u_2)\vec{b} + (\alpha u_3)\vec{c},
\end{align}
ou seja,
\begin{equation}
  \alpha\vec{u} = (\alpha u_1,\alpha u_2, \alpha u_3).
\end{equation}

\begin{ex}
  Fixada uma base qualquer $B$ e dado o vetor $\vec{v} = (2, -1, -3)_B$, temos
  \begin{equation}
    \frac{1}{3}\vec{v} = \left(-\frac{2}{3}, \frac{1}{3}, 1\right)_B.
  \end{equation}

    \ifispython
  Usando o \verb+Sympy+, podemos computar o oposto do vetor $\frac{1}{3}\vec{v}$ com os seguintes comandos:
\begin{verbatim}
from sympy import *
v = Matrix([2,-1,-3])
1/3*v
\end{verbatim}
  \fi

\end{ex}

\subsection{Dependência linear}\label{cap_base_sec_base_subsec_dl}

\subsubsection{Dois vetores}

Na Subseção \ref{cap_base_sec_deplin_subsec_obs}, discutimos que dois vetores $\vec{u}$, $\vec{v}$ são l.d. se, e somente se, um for múltiplo do outro, i.e. existe um número real $\alpha$ tal que
\begin{equation}\label{eq:cbsb_vmv}
  \vec{u} = \alpha\vec{v},
\end{equation}
sem perda de generalidade\footnote{Formalmente, pode ocorrer $\vec{v} = \beta\vec{u}$.}.

Fixada uma base $B = (\vec{a}, \vec{b}, \vec{c})$, temos $\vec{u} = (u_1, u_2, u_3)_B$ e $\vec{v} = (v_1, v_2, v_3)_B$. Com isso, a equação \eqref{eq:cbsb_vmv} pode ser reescrita como
\begin{equation}
  (u_1, u_2, u_3)_B = \alpha(v_1, v_2, v_3)_B = (\alpha v_1, \alpha v_2, \alpha v_3)_B,
\end{equation}
donde
\begin{equation}
  u_1 = \alpha v_1,~u_2 = \alpha v_2,~u_3 = \alpha v_3.
\end{equation}
Ou seja, dois vetores são linearmente dependentes se, e somente se, as coordenadas de um deles forem, respectivamente, múltiplas (de mesmo fator) das coordenadas do outro.

\begin{ex}
  Vejamos os seguintes casos:
  \begin{enumerate}[a)]
  \item $\vec{u} = (2,-1,-3)$ e $\vec{v} = \left(1,-\frac{1}{2},-\frac{3}{2}\right)$ são l.d., pois
    \begin{equation}
      2 = 2\cdot \frac{1}{2},~-1 = 2\cdot\left(-\frac{1}{2}\right),~-3 = 2\cdot\left(-\frac{3}{2}\right).
    \end{equation}
  \item $\vec{u} = (2,-1,-3)$ e $\vec{v} = \left(2,-\frac{1}{2},-\frac{3}{2}\right)$ são l.i., pois $u_1 = 1\cdot v_1$, enquanto $u_2 = 2v_2$.
  \end{enumerate}
\end{ex}

\subsubsection{Três vetores}

Na Subseção \ref{cap_base_sec_deplin_subsec_obs}, discutimos que três vetores $\vec{u}$, $\vec{v}$ e $\vec{w}$ são l.i. se, e somente se,
\begin{equation}
  \alpha\vec{u}+\beta\vec{v}+\gamma\vec{w}=\vec{0} \Rightarrow \alpha=\beta=\gamma=0.
\end{equation}

Seja, então, $B = (\vec{a}, \vec{b}, \vec{c})$ uma base de $V$. Então, temos que a equação
\begin{equation}
  \alpha\vec{u}+\beta\vec{v}+\gamma\vec{w} = \vec{0}
\end{equation}
é equivalente a
\begin{equation}
  \alpha(u_1,u_2,u_3)_B+\beta(v_1,v_2,v_3)_B+\gamma(w_1,w_2,w_3)_B=(0, 0, 0)_B.
\end{equation}
Esta por sua vez, nos leva ao seguinte sistema linear
\begin{equation}
  \left\{
    \begin{array}{l}
      u_1\alpha + v_1\beta + w_1\gamma = 0\\
      u_2\alpha + v_2\beta + w_2\gamma = 0\\
      u_3\alpha + v_3\beta + w_3\gamma = 0
    \end{array}
  \right.
\end{equation}
Lembremos que um tal sistema tem solução única (trivial) se, e somente se, o determinante de sua matriz dos coeficientes é não nulo, i.e.
\begin{equation}
  \left|\begin{array}{ccc}
    u_1 & v_1 & w_1\\
    u_2 & v_2 & w_2\\
    u_3 & v_3 & w_3
  \end{array}\right| \neq 0.
\end{equation}

\begin{ex}
  Fixada uma base $B$ de $V$, sejam os vetores $\vec{u} = (2,1,-3)_B$, $\vec{v} = (1,-1,2)_B$ e $\vec{w} = (-2,1,1)_B$. Como
  \begin{align}
    \left|\begin{array}{ccc}
            u_1 & v_1 & w_1\\
            u_2 & v_2 & w_2\\
            u_3 & v_3 & w_3
          \end{array}\right| &=
                               \left|\begin{array}{ccc}
                                       2 & 1 & -2\\
                                       1 & -1 & 1\\
                                       -3 & 2 & 1
                                     \end{array}\right| \\
    &= -2-4-3+6-4-1 = -8\neq 0.
  \end{align}
  Logo, $(\vec{u}, \vec{v}, \vec{w})$ é uma sequência de vetores l.i..
\end{ex}

\subsection{Bases ortonormais}

Uma {\bf base} $B = (\vec{a}, \vec{b}, \vec{c})$ é dita ser {\bf ortonormal} se, e somente se,
\begin{itemize}
\item $\vec{a}$, $\vec{b}$ e $\vec{c}$ são dois a dois ortogonais;
\item $|\vec{a}|=|\vec{b}|=|\vec{c}|=1$.
\end{itemize}

\begin{obs}\normalfont{(Teorema de Pitágoras)}\label{obs:pitagoras}
  Se $\vec{u}\perp\vec{v}$, então $|\vec{u}+\vec{v}|^2=|\vec{u}|^2+|\vec{v}|^2$.
\end{obs}

\begin{prop}\label{prop:bo_norma}
  Seja $B = (\vec{i},\vec{j},\vec{k})$ uma base ortonormal e $\vec{u}=(u_1,u_2,u_3)_B$. Então, $|\vec{u}|=\sqrt{u_1^2+u_2^2+u_3^2}$.
\end{prop}
\begin{dem}
  Temos $|\vec{u}|^2 = |u_1\vec{i}+u_2\vec{j}+u_3\vec{k}|^2$. Seja $\pi$ um plano determinado por dadas representações de $\vec{i}$ e $\vec{j}$. Como $\vec{i}$, $\vec{j}$ e $\vec{k}$ são ortogonais, temos que $\vec{k}$ é ortogonal ao plano $\pi$. Além disso, o vetor $u_1\vec{i}+u_2\vec{j}$ também admite uma representação em $\pi$, logo $u_1\vec{i}+u_2\vec{j}$ é ortogonal a $\vec{k}$. Do Teorema de Pitágoras (Observação \ref{obs:pitagoras}), temos
  \begin{equation}
    |\vec{u}|^2 = |u_1\vec{i}+u_2\vec{j}|^2 + |u_3\vec{k}|^2.
  \end{equation}
  Analogamente, como $\vec{i}\perp\vec{j}$, do Teorema de Pitágoras segue
  \begin{align}
    |\vec{u}|^2 &= |u_1\vec{i}|^2 + |u_2\vec{j}|^2 + |u_3\vec{k}|^2\\
                &= |u_1|^2|\vec{i}| + |u_2|^2|\vec{j}| + |u_3||\vec{k}|^2\\
                &= u_1^2 + u_2^2 + u_3^2.
  \end{align}
  Extraindo a raiz quadrada de ambos os lados da última equação, obtemos o resultado desejado.
\end{dem}

\begin{ex}
  Se $\vec{u} = (-1,2,-\sqrt{2})_B$ e $B$ é uma base ortonormal, então
  \begin{equation}
    |\vec{u}| = \sqrt{(-1)^2 + 2^2 + (-\sqrt{2})^2} = \sqrt{7}.
  \end{equation}
\end{ex}

\subsection*{Exercícios resolvidos}

\begin{exeresol}
  Considere a base $B=(\vec{i}, \vec{j}, \vec{k})$ conforme a figura abaixo. Faça uma representação do vetor $\vec{u}=\left(2,\frac{1}{2},1\right)_B$.
  \begin{figure}[H]
    \centering
    \includegraphics[width=0.6\textwidth]{cap_base/dados/fig_sec_base/repr_geo_exeresol}
  \end{figure}
\end{exeresol}
\begin{resol}
  Primeiramente, observamos que
  \begin{align}
    \vec{u} &= (2,1,1)_B\\
            &= 2\vec{i} + \frac{1}{2}\vec{j} + \vec{k}.
  \end{align}
  Assim sendo, podemos construir uma representação de $\vec{u}$ como dada na figura abaixo. Primeiramente, podemos representar os vetores $2\vec{i}$ e $\frac{1}{2}\vec{j}$ (azul). Então, representamos o vetor $2\vec{i}+\frac{1}{2}\vec{j}$ (cinza). Por fim, temos a representação de $\vec{u}$ (azul).
  \begin{figure}[H]
    \centering
    \includegraphics[width=0.7\textwidth]{cap_base/dados/fig_sec_base/repr_geo_resol}
  \end{figure}
\end{resol}

\begin{exeresol}
  Fixada uma base qualquer $B$ e dados $\vec{u}=(1,-1,2)_B$ e $\vec{v}=(-2,1,-1)_B$, encontre o vetor $\vec{x}$ que satisfaça
  \begin{equation}
    \vec{u}+2\vec{x}=\vec{v}-\left(\vec{x}+\vec{u}\right).
  \end{equation}
\end{exeresol}
\begin{exeresol}
  Primeiramente, podemos manipular a equação de forma a isolarmos $\vec{x}$ como segue
  \begin{gather}
    \vec{u}+2\vec{x}=\vec{v}-\left(\vec{x}+\vec{u}\right)\\
    2\vec{x}=-\vec{u}+\vec{v}-\vec{x}-\vec{u}\\
    3\vec{x}=\vec{v}-2\vec{u}\\
    \vec{x}=\frac{1}{3}\vec{v}-\frac{2}{3}\vec{u}
  \end{gather}
  Agora, sabendo que $\vec{u}=(1,-1,2)_B$ e $\vec{v}=(-2,1,-1)_B$, temos
  \begin{gather}
    \vec{x}=\frac{1}{3}(-2,1,-1)_B-\frac{2}{3}(1,-1,2)_B\\
    \vec{x}=\left(-\frac{2}{3},\frac{1}{3},-\frac{1}{3}\right)_B-\left(\frac{2}{3},-\frac{2}{3},\frac{4}{3}\right)_B\\
    \vec{x}=\left(-\frac{2}{3}-\frac{2}{3},\frac{1}{3}+\frac{2}{3},-\frac{1}{3}-\frac{4}{3}\right)\\
    \vec{x}=\left(-\frac{4}{3},1,-\frac{5}{3}\right)_B.
  \end{gather}
\end{exeresol}

\begin{exeresol}
  Fixada uma base $B$ qualquer, verifique se os vetores $\vec{u}=(1,-1,2)_B$, $\vec{v}=(-2,1,-1)_B$ e $\vec{w}=(-4,3,-5)$ formam uma base para o espaço $V$.
\end{exeresol}
\begin{resol}
  Uma base para o espaço tridimensional $V$ é uma sequência de três vetores l.i.. Logo, para resolver a questão, basta verificar se $(\vec{u},\vec{v},\vec{w})$ é l.i.. Com base na Subseção \ref{cap_base_sec_base_subsec_dl}, basta calcularmos o determinante da matriz cujas colunas são formadas pelas coordenadas dos vetores da sequência, i.e.
  \begin{gather}
    \begin{vmatrix}
      u_1 & v_1 & w_1 \\
      u_2 & v_2 & w_2 \\
      u_3 & v_3 & w_3
    \end{vmatrix}\\
    = \begin{vmatrix}
      1 & -2 & -4 \\
      -1 & 1 & 3 \\
      2 & -1 & -5
    \end{vmatrix}\\
    = -5-4-12-(-8-3-10)\\
    = -21+21 = 0.
  \end{gather}
  Como este determinando é nulo, concluímos que $(\vec{u},\vec{v},\vec{w})$ é l.d. e, portanto, não forma uma base para $V$.
\end{resol}

\subsection*{Exercícios}

\begin{exer}
  Considere a base $B=(\vec{i}, \vec{j}, \vec{k})$ conforme a figura abaixo. Faça uma representação do vetor $\vec{u}=\left(1,-1,\frac{1}{2}\right)_B$.  
  \begin{figure}[H]
    \centering
    \includegraphics[width=0.6\textwidth]{cap_base/dados/fig_sec_base/repr_geo_exer}
  \end{figure}
\end{exer}
\begin{resp}
  \begin{figure}[H]
    \centering
    \includegraphics[width=0.5\textwidth]{cap_base/dados/fig_sec_base/repr_geo_resp}
  \end{figure}  
\end{resp}

\begin{exer}
  Fixada uma base $B=(\vec{i},\vec{j},\vec{k})$ e sabendo que $\vec{v}=(2,0,-3)_B$, escreva $\vec{v}$ como combinação linear de $\vec{i}$, $\vec{j}$ e $\vec{k}$.
\end{exer}
\begin{resp}
  $\vec{v}=2\vec{i}+0\vec{j}-3\vec{k}$
\end{resp}

\begin{exer}
  Fixada uma base $B$ qualquer e $\vec{a}=\left(0,-1,1\right)_B$, $\vec{b}=\left(2,0,-1\right)_B$ e $\vec{c}=\left(\frac{1}{2},-\frac{1}{3},1\right)_B$, calcule:
  \begin{enumerate}[a)]
  \item $6\vec{c}$
  \item $-\vec{b}$
  \item $\vec{c}-\vec{b}$
  \item $2\vec{c}-(\vec{a}-\vec{b})$
  \end{enumerate}
\end{exer}
\begin{resp}
  a)~$6\vec{c}=(3,-2,6)_B$; b)~$-\vec{b}=(-2,0,1)_B$; c)~$\vec{c}-\vec{b}=(-\frac{3}{2},-\frac{1}{3},2)_B$; d)~$2\vec{c}-(\vec{a}-\vec{b})=(3,\frac{1}{3},0)_B$
\end{resp}

\begin{exer}
  Faxada uma base $B$ qualquer, verifique se os seguintes conjuntos de vetores são l.i. ou l.d..
  \begin{enumerate}[a)]
  \item $\vec{i}=(1,0,0)_B$, $\vec{j}=(0,1,0)_B$
  \item $\vec{a}=(1,2,0)_B$, $\vec{b}=(-2,-4,1)_B$
  \item $\vec{a}=(1,2,0)_B$, $\vec{c}=(-2,-4,0)_B$
  \item $\vec{i}=(1,0,0)_B$, $\vec{k}=(0,0,1)_B$
  \item $\vec{j}=(0,1,0)_B$, $\vec{k}=(0,0,1)_B$
  \item $\vec{a}=(1,2,-1)_B$, $\vec{d}=(\frac{1}{2},1,-\frac{1}{2})_B$
  \end{enumerate}
\end{exer}
\begin{resp}
  a)~l.i.; b)~l.i.; c)~l.d.; d)~l.i.; e)~l.i.; f)~l.d.
\end{resp}

\begin{exer}
  Faxada uma base $B$ qualquer, verifique se os seguintes conjuntos de vetores são l.i. ou l.d..
  \begin{enumerate}[a)]
  \item $\vec{i}=(1,0,0)_B$, $\vec{j}=(0,1,0)_B$, $\vec{k}=(0,0,1)_B$
  \item $\vec{a}=(0,-1,1)_B$, $\vec{b}=(2,0,-1)$, $\vec{c}=(\frac{1}{2},-\frac{1}{3},1)_B$
  \item $\vec{u}=(0,-1,1)_B$, $\vec{v}=(2,0,-1)$, $\vec{w}=(2,-1,0)_B$
  \end{enumerate}
\end{exer}
\begin{resp}
  a)~l.i.; b)~l.i.; c)~l.d.
\end{resp}

\begin{exer}
  Seja $B = (\vec{a}, \vec{b}, \vec{c})$ uma base ortogonal, i.e. $\vec{a}$, $\vec{b}$ e $\vec{c}$ são l.i. e dois a dois ortogonais. Mostre que $C = (\vec{a}/|\vec{a}|,\vec{b}/|\vec{b}|,\vec{c}/|\vec{c}|)$ é uma base ortonormal.
\end{exer}
\begin{resp}
  Segue imediatamente do fato de que $\left|\vec{u}/|u|\right|=1$ para qualquer vetor $\vec{u}\neq 0$.
\end{resp}


\section{Mudança de base}\label{cap_base_sec_mudbas}

Sejam $B = (\vec{u}, \vec{v}, \vec{z})$ e $C = (\vec{r}, \vec{s}, \vec{t})$ bases do espaço $V$. Conhecendo as coordenadas de um vetor na base $C$, queremos determinar suas coordenadas na base $B$. Mais especificamente, seja
\begin{align}
  \vec{z} &= (z_1, z_2, z_3)_C\\
  &= z_1\vec{r} + z_2\vec{s} + z_3\vec{t}.
\end{align}
Agora, tendo $\vec{r} = (r_1, r_2, r_3)_B$, $\vec{s} = (s_1, s_2, s_3)_B$ e $\vec{t} = (t_1, t_2, t_3)_B$, então
\begin{align}
  (z_1, z_2, z_3)_C &= z_1(r_1, r_2, r_3)_B \\
                    &+ z_2(s_1, s_2, s_3)_B \\
                    &+ z_3(t_1, t_2, t_3)_B\\
                    &= \underbrace{(r_1z_1+s_1z_2+t_1z_3)}_{z_1'}\vec{u}\\
                    &+ \underbrace{(r_2z_1+s_2z_2+t_2z_3)}_{z_2'}\vec{v}\\
                    &+ \underbrace{(r_3z_1+s_3z_2+t_3z_3)}_{z_3'}\vec{w}
\end{align}
o que é equivalente a
\begin{equation}\label{eq:cbsmb_mcb}
  \begin{bmatrix}
    z_1'\\
    z_2'\\
    z_3'
  \end{bmatrix} =
  \underbrace{\begin{bmatrix}
    r_1 & s_1 & t_1\\
    r_2 & s_2 & t_2\\
    r_3 & s_3 & t_3
  \end{bmatrix}}_{M_{CB}}
  \begin{bmatrix}
    z_1\\
    z_2\\
    z_3
  \end{bmatrix},
\end{equation}
onde $\vec{z} = (z_1', z_2', z_3')_B$.

A matriz $M_{CB}$ é chamada de matriz de mudança de base de $C$ para $B$. Como os vetores $\vec{r}$, $\vec{s}$ e $\vec{t}$ são l.i., temos que a matriz de mudança de base $M_{BC}$ tem determinante não nulo e, portanto é invertível. Portanto, multiplicando por $M_{BC}^{-1}$ pela esquerda em \eqref{eq:cbsmb_mcb}, temos
\begin{equation}
    \begin{bmatrix}
    z_1\\
    z_2\\
    z_3
  \end{bmatrix} =
  \underbrace{\begin{bmatrix}
    r_1 & s_1 & t_1\\
    r_2 & s_2 & t_2\\
    r_3 & s_3 & t_3
  \end{bmatrix}^{-1}}_{M_{BC}}
  \begin{bmatrix}
    z_1'\\
    z_2'\\
    z_3'
  \end{bmatrix},
\end{equation}
ou seja
\begin{equation}
  M_{BC} = (M_{CB})^{-1}. 
\end{equation}

\begin{ex}
  Sejam dadas as bases $B = (\vec{a},\vec{b},\vec{c})$ e $C = (\vec{u},\vec{v},\vec{w})$, com $\vec{u}=(1,2,0)_B$, $\vec{v}=(2,0,-1)_B$ e $\vec{w}=(-1,-3,1)_B$. Seja, ainda, o vetor $\vec{z} = (1,-2,1)_B$. Vamos encontrar as coordenadas de $\vec{z}$ na base $C$.

  Há duas formas de proceder.

  {\bf Método 1.}
  
  A primeira consiste em resolver, de forma direta, a seguinte equação
  \begin{equation}
    (1,-2,1)_B = (x,y,z)_C.
  \end{equation}
  Esta é equivalente a
  \begin{align}
    \vec{a}-2\vec{b}+\vec{c} &= x\vec{u}+y\vec{v}+z\vec{w} \\
            &= x(1,2,0)_B \\
            &+ y(2,0,-1)_B\\
            &+ z(-1,-3,1)_B\\
            &= x(\vec{a}+2\vec{b})\\
            &+ y(2\vec{a}-\vec{c})\\
            &+ z(-\vec{a}-3\vec{b}+\vec{c})\\
            &= (x+2y-z)\vec{a}\\
            &+ (2x-3z)\vec{b}\\
            &+ (-y+z)\vec{c}
  \end{align}
  Isto nos leva ao seguinte sistema linear
  \begin{equation}
    \left\{
      \begin{array}{l}
        x+2y-z=1\\
        2x-3z=-2\\
        -y+z=1
      \end{array}
\right.
\end{equation}
Resolvendo este sistema, obtemos $x=7/5$, $y=3/5$ e $z=8/5$, i.e.
\begin{equation}
  \vec{z} = \left(\frac{7}{5},\frac{3}{5},\frac{8}{5}\right)_C.
\end{equation}

  {\bf Método 2.}

Outra maneira de se obter as coordenadas de $\vec{z}$ na base $C$ é usando a matriz de mudança de base. A matriz de mudança da base $C$ para a base $B$ é
\begin{align}
  M_{CB} &= \begin{bmatrix}
    u_1 & v_1 & w_1 \\
    u_2 & v_2 & w_2 \\
    u_3 & v_3 & w_3
  \end{bmatrix} \\
         &=
  \begin{bmatrix}
    1 & 2 & -1\\
    2 & 0 & -3\\
    0 & -1 & 1
  \end{bmatrix}.
\end{align}
Entretanto, neste exemplo, queremos fazer a mudança de $B$ para $C$. Portanto, calculamos a matriz de mudança de base $M_{BC}$. Segue:
\begin{gather}
  M_{BC} = M_{CB}^{-1} \\
  M_{BC} =
  \begin{bmatrix}
    1 & 2 & -1\\
    2 & 0 & -3\\
    0 & -1 & 1
  \end{bmatrix}^{-1} \\
  M_{BC} = 
  \begin{bmatrix}
    \frac{3}{5} & \frac{1}{5} & \frac{6}{5}\\
    \frac{2}{5} & -\frac{1}{5} & -\frac{1}{5}\\
    \frac{2}{5} & -\frac{1}{5} & \frac{4}{5}
  \end{bmatrix}
\end{gather}
Com esta matriz e denotando $\vec{z}=(x,y,z)_C$, temos
\begin{gather}
  \begin{bmatrix}
    x\\
    y\\
    z
  \end{bmatrix} = 
  \underbrace{\begin{bmatrix}
    \frac{3}{5} & \frac{1}{5} & \frac{6}{5}\\
    \frac{2}{5} & -\frac{1}{5} & -\frac{1}{5}\\
    \frac{2}{5} & -\frac{1}{5} & \frac{4}{5}
  \end{bmatrix}}_{M_{BC}}
  \begin{bmatrix}
    1\\
    -2\\
    1
  \end{bmatrix}\\
  \begin{bmatrix}
    x\\
    y\\
    z
  \end{bmatrix} =
    \begin{bmatrix}
    7/5\\
    3/5\\
    8/5
  \end{bmatrix}
\end{gather}
Logo, temos
\begin{equation}
  \vec{z} = \left(\frac{7}{5},\frac{3}{5},\frac{8}{5}\right)_C.
\end{equation}
\end{ex}

\subsection*{Exercícios resolvidos}

\begin{exeresol}\label{exeresol:cbsmb_mbccb}
  Sejam $B$ e $C$ bases dadas do espaço $V$. Sabendo que a matriz de mudança de base de $B$ para $C$ é
  \begin{equation}
    M =
    \begin{bmatrix}
      1 & 0 & -1 \\
      -1 & 1 & 0 \\
      2 & 3 & 1
    \end{bmatrix},
  \end{equation}
  calcule a matriz de mudança de base de $C$ para $B$.
\end{exeresol}
\begin{resol}
  Sejam $M_{BC}=M$ a matriz de mudança de base de $B$ para $C$ e $M_{CB}$ a matriz de mudança de base de $C$ para $B$. Temos
  \begin{gather}
    M_{CB} = M_{BC}^{-1}\\
    M_{CB} = M^{-1}\\
    M_{CB} =
    \begin{bmatrix}
      1 & 0 & -1 \\
      -1 & 1 & 0 \\
      2 & 3 & 1      
    \end{bmatrix}^{-1}\\
    M_{CB} =
    \begin{bmatrix}
      \frac{1}{6} & - \frac{1}{2} & \frac{1}{6}\\
      \frac{1}{6} & \frac{1}{2} & \frac{1}{6}\\
      - \frac{5}{6} & - \frac{1}{2} & \frac{1}{6}
    \end{bmatrix}
  \end{gather}
\end{resol}

\begin{exeresol}
  Fixadas as mesmas bases do ER \ref{exeresol:cbsmb_mbccb}, determine as coordenadas do vetor $\vec{u}$ na base $C$, sabendo que $\vec{u}=(2,-1,-3)_B$.
\end{exeresol}
\begin{resol}
  Denotando $\vec{u}=(u_1,u_2,u_3)_B$, temos
  \begin{gather}
    \vec{u}_C = M_{BC}\vec{u}_{B} \\
    \begin{bmatrix}
      u_1\\
      u_2\\
      u_3
    \end{bmatrix} =
    \begin{bmatrix}
      1 & 0 & -1 \\
      -1 & 1 & 0 \\
      2 & 3 & 1
    \end{bmatrix}
    \begin{bmatrix}
      2\\
      -1\\
      -3
    \end{bmatrix}\\
    \begin{bmatrix}
      u_1\\
      u_2\\
      u_3
    \end{bmatrix} =
    \begin{bmatrix}
      5\\
      -3\\
      -2
    \end{bmatrix}
  \end{gather}
\end{resol}

\begin{exeresol}
  Considere dadas as bases $A$, $B$ e $C$. Sejam, também, $M_{AB}$ a matriz de mudança de base de $A$ para $B$ e $M_{BC}$ a matriz de mudança de base de $B$ para $C$. Determine a matriz de mudança de base de $A$ para $C$ em função das matrizes $M_{AB}$ e $M_{BC}$.
\end{exeresol}
\begin{resol}
  Para um vetor $\vec{u}$ qualquer, temos
  \begin{gather}
    \vec{u}_B = M_{AB}\vec{u}_A\\
    \vec{u}_C = M_{BC}\vec{u}_B
  \end{gather}
  Logo, temos
  \begin{align}
    \vec{u}_C &= M_{BC}\left(M_{AB}\vec{u}_A\right)\\
              &= \left(M_{BC}M_{AB}\right)\vec{u}_A.
  \end{align}
  Concluímos que $M_{AC} = M_{BC}M_{AB}$.
\end{resol}

\subsection*{Exercícios}

\begin{exer}
  Sejam $A$ e $B$ bases dadas de $V$ (espaço tridimensional). Sabendo que $\vec{v}=(-2,0,1)_A$ e que a matriz de mudança de base
  \begin{equation}
    M_{AB} =
    \begin{matrix}
      1 & 0 & -1 \\
      0 & 2 & -1 \\
      -1 & 1 & 0
    \end{matrix},
  \end{equation}
  determine $\vec{v}_B$, i.e. as coordenadas de $\vec{v}$ na base $B$.
\end{exer}
\begin{resp}
  $\vec{v}=(-3,-1,2)_{B}$
\end{resp}

\begin{exer}
  Sejam $A$ e $B$ bases dadas de $V$ (espaço tridimensional). Sabendo que $\vec{v}=(-2,0,1)_B$ e que a matriz de mudança de base
  \begin{equation}
    M_{AB} =
    \begin{matrix}
      1 & 0 & -1 \\
      0 & 2 & -1 \\
      -1 & 1 & 0
    \end{matrix},
  \end{equation}
  determine $\vec{v}_A$, i.e. as coordenadas de $\vec{v}$ na base $A$.
\end{exer}
\begin{resp}
  $\vec{v}=(0,1,2)_{A}$
\end{resp}

\begin{exer}
  Sejam $B=(\vec{a},\vec{b},\vec{c})$ e $C=(\vec{u},\vec{v},\vec{w})$ bases de $V$ com
  \begin{gather}
    \vec{u} = (0,1,1)_B\\
    \vec{v} = (1,0,1)_B\\
    \vec{w} = (2,1,-1)_B
  \end{gather}
  Forneça a matriz de mudança de base $M_{CB}$.
\end{exer}
\begin{resp}
  $\displaystyle
  \begin{bmatrix}
    0 & 1 & 2\\
    1 & 0 & 1\\
    1 & 1 & -1
  \end{bmatrix}
$
\end{resp}

\begin{exer}
  Sejam $B=(\vec{a},\vec{b},\vec{c})$ e $C=(\vec{u},\vec{v},\vec{w})$ bases de $V$ com
  \begin{gather}
    \vec{a} = (0,1,1)_C\\
    \vec{b} = (1,0,1)_C\\
    \vec{c} = (2,1,-1)_C
  \end{gather}
  Forneça a matriz de mudança de base $M_{CB}$.
\end{exer}
\begin{resp}
  $\displaystyle
  \begin{bmatrix}
    - \frac{1}{4} & \frac{3}{4} & \frac{1}{4}\\
    \frac{1}{2} & - \frac{1}{2} & \frac{1}{2}\\
    \frac{1}{4} & \frac{1}{4} & - \frac{1}{4}
  \end{bmatrix}
$
\end{resp}

\begin{exer}
  Sejam $B=(\vec{a},\vec{b},\vec{c})$ e $C=(\vec{u},\vec{v},\vec{w})$ bases de $V$ com
  \begin{gather}
    \vec{u} = (0,1,1)_B\\
    \vec{v} = (1,0,1)_B\\
    \vec{w} = (2,1,-1)_B
  \end{gather}
  Sabendo que $\vec{d}=(0,-1,2)_{C}$, forneça $\vec{d}_B$, i.e. as coordenadas do vetor $\vec{d}$ na base $B$.
\end{exer}
\begin{resp}
  $\displaystyle\left[\begin{matrix}3\\2\\-3\end{matrix}\right]$
\end{resp}

\begin{exer}
  Sejam $B=(\vec{a},\vec{b},\vec{c})$ e $C=(\vec{u},\vec{v},\vec{w})$ bases de $V$ com
  \begin{gather}
    \vec{u} = (0,1,1)_B\\
    \vec{v} = (1,0,1)_B\\
    \vec{w} = (2,1,-1)_B
  \end{gather}
  Sabendo que $\vec{d}=(1,-2,1)_{B}$, forneça $\vec{d}_C$, i.e. as coordenadas do vetor $\vec{d}$ na base $C$.
\end{exer}
\begin{resp}
  $\displaystyle\left[\begin{matrix}- \frac{3}{2}\\2\\- \frac{1}{2}\end{matrix}\right]$
\end{resp}

\begin{exer}
  Considere dadas as bases $A$, $B$ e $C$ do espaço tridimensional $V$. Sejam, também, $M_{AB}$ a matriz de mudança de base de $A$ para $B$ e $M_{CB}$ a matriz de mudança de base de $C$ para $B$. Determine a matriz de mudança de base de $A$ para $C$ em função das matrizes $M_{AB}$ e $M_{CB}$.  
\end{exer}
\begin{resp}
  $M_{AC}=M_{CB}^{-1}M_{AB}$
\end{resp}