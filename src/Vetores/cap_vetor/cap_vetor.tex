%Este trabalho está licenciado sob a Licença Atribuição-CompartilhaIgual 4.0 Internacional Creative Commons. Para visualizar uma cópia desta licença, visite http://creativecommons.org/licenses/by-sa/4.0/deed.pt_BR ou mande uma carta para Creative Commons, PO Box 1866, Mountain View, CA 94042, USA.

\chapter{Vetores}\label{cap_vetor}
\thispagestyle{fancy}

Neste capítulo, introduzimos os conceitos fundamentais relacionados às definições de vetor e operações básicas envolvendo vetores.

\section{Segmentos orientados}\label{cap_vetor_sec_segorien}

\subsection{Segmento}

\begin{flushright}
  \href{https://archive.org/details/definicao-de-segmento}{$\blacktriangleright$ Vídeo disponível!}
\end{flushright}

Sejam dois pontos $A$ e $B$ sobre uma reta $r$. O conjunto de todos os pontos de $r$ entre $A$ e $B$ é chamado de {\bf segmento}\index{segmento} e denotado por $AB$. A reta $r$ é chamada de reta suporte.

\begin{figure}[H]
  \centering
  \includegraphics[width=0.6\textwidth]{./cap_vetor/dados/fig_segmento/fig_segmento}
  \caption{Esboço de um segmento $AB$.}
  \label{fig:segmento}
\end{figure}

\subsubsection{Norma e direção}

Associado a um segmento $AB$, temos sua \emph{norma}\index{norma} a qual é denotada por $|AB|$ e é definida como a distância entre seus pontos extremos $A$ e $B$. Ou seja, a norma do segmento $AB$ é a medida de seu comprimento ou tamanho.

A {\bf direção}\index{direção} de um segmento $AB$ é a direção de sua reta suporte, i.e. a direção da reta que fica determinada pelos pontos $A$ e $B$. Logo, dois segmentos $AB$ e $CD$ têm a mesma direção, quando suas retas suportes são paralelas ou coincidentes (ou seja, elas têm a mesma direção).

\begin{ex}\label{ex:segmento}
  Consideremos os segmentos esboçados na Figura \ref{fig:ex_segmento}. Os segmentos $AB$ e $CD$ têm as mesmas direções, mas comprimentos diferentes. Já, o segmento $EF$ tem direção diferente dos segmentos $AB$ e $CD$.
  
  \begin{figure}[H]
    \centering
    \includegraphics[width=0.6\textwidth]{./cap_vetor/dados/fig_ex_segmento/fig_ex_segmento}
  \caption{Esboço referente ao Exemplo \ref{ex:segmento}.}
  \label{fig:ex_segmento}
\end{figure}
\end{ex}

\subsubsection{Segmento nulo}

Se $A$ e $B$ são pontos coincidentes, então chamamos $AB$ de {\bf segmento nulo}\index{segmento nulo} e temos $|AB| = 0$. Observamos que a representação geométrica de um segmento nulo é um ponto, tendo em vista que seus pontos extremos são coincidentes. Como existem infinitas retas de diferentes direções que passam por um único ponto, temos que segmentos nulos não têm direção definida.

\subsection{Segmento orientado}

\begin{flushright}
  \href{https://archive.org/details/definicao-segmento-orientado}{$\blacktriangleright$ Vídeo disponível!}
\end{flushright}

Observamos que um dado segmento $AB$ é igual ao segmento $BA$. Agora, podemos associar a noção de {\bf sentido} a um segmento, escolhendo um dos pontos como sua {\bf origem}\index{origem} (ou \emph{ponto de partida}) e o outro como sua {\bf extremidade}\index{extremidade} (ou \emph{ponto de chegada}). Ao fazermos isso, definimos um {\bf segmento orientado}\index{segmento orientado}.

Mais precisamente, um segmento orientado $AB$ é o segmento definido pelos pontos $A$ e $B$, sendo $A$ o ponto de partida (origem) e $B$ o ponto de chegada (extremidade). Veja a Figura \ref{fig:seg_orientado}.

\begin{figure}[H]
  \centering
  \includegraphics[width=0.5\textwidth]{./cap_vetor/dados/fig_seg_orientado/fig_seg_orientado}
  \caption{Esboço de um segmento orientado $AB$.}
  \label{fig:seg_orientado}
\end{figure}

\subsubsection{Norma e direção}

As noções de norma e de direção para segmentos estendem-se diretamente a segmentos orientados. Dizemos que dois dados segmentos orientados não nulos $AB$ e $CD$ têm a {\bf mesma direção} quando as retas $AB$ e $CD$ são paralelas ou coincidentes. A norma de um segmento orientado $AB$ é a norma do segmento $AB$, denotada por $|AB|$. O segmento orientado nulo $AA$ tem norma $|AA|=0$ e não tem direção definida.

\begin{ex}\label{ex:segorien_direcao}
  Consideremos os segmentos orientados esboçados na Figura \ref{fig:ex_segorien_direcao}. Observemos que os segmentos orientados $AB$ e $CD$ têm a mesma direção. Já o segmento orientado $EF$ tem direção diferente dos segmentos $AB$ e $CD$.
  
  \begin{figure}[H]
    \centering
    \includegraphics[width=0.7\textwidth]{./cap_vetor/dados/fig_ex_segorien_direcao/fig_ex_segorien_direcao}
  \caption{Esboço referente ao Exemplo \ref{ex:segorien_direcao}.}
  \label{fig:ex_segorien_direcao}
\end{figure}  
\end{ex}

\subsubsection{Comparação do sentido}

\begin{flushright}
  \href{https://archive.org/details/comparacao-sentido-segmentos-orientados}{$\blacktriangleright$ Vídeo disponível!}
\end{flushright}

Segmentos orientados $AB$ e $CD$ de mesma direção podem ter o mesmo sentido ou sentidos opostos. No caso de suas retas suportes não serem coincidentes, os segmentos orientados $AB$ e $CD$ têm a mesma direção, quando os segmentos $AC$ e $BD$ não se interceptam. E, caso estas se intercetam, os segmentos orientados $AB$ e $CD$ têm sentidos opostos. 

\begin{ex}
  Na Figura \ref{fig:segorien_sentido}, temos que os segmentos $AB$ e $CD$ têm o mesmo sentido. De fato, observamos que eles têm a mesma direção e que os segmentos $AC$ e $BD$ têm interseção vazia.

\begin{figure}[H]
  \centering
  \includegraphics[width=0.7\textwidth]{./cap_vetor/dados/fig_segorien_sentido/fig_segorien_sentido}
  \caption{Segmentos orientados $AB$ e $CD$ de mesmo sentido. Segmentos orientados $EF$ e $GH$ de sentidos opostos.}
  \label{fig:segorien_sentido}
\end{figure}

Na mesma Figura \ref{fig:segorien_sentido}, vemos que os segmentos orientados $EF$ e $GH$ têm sentidos opostos, pois têm a mesma direção e os segmentos $EG$ e $FH$ se interceptam (no ponto $I$). 
\end{ex}

\begin{obs}\label{obs:segorin_sentido_trans}
  A propriedade de segmentos orientados terem o mesmo sentido é transitiva. Ou seja, se $AB$ e $CD$ têm o mesmo sentido e $CD$ e $EF$ têm o mesmo sentido, então $AB$ e $EF$ têm o mesmo sentido.
\end{obs}

Com base na Observação \ref{obs:segorin_sentido_trans}, analisamos o sentido de dois segmentos orientados e colineares escolhendo um deles e construíndo um segmento orientado de mesmo sentido  e não colinear. Então, analisamos o sentido dos segmentos orientados originais com respeito ao introduzido.

\subsubsection{Equipolência}

\begin{flushright}
  \href{https://archive.org/details/segmentos-orientados-equipolentes}{$\blacktriangleright$ Vídeo disponível!}
\end{flushright}

Um segmento orientado não nulo $AB$ é \emph{equipolente}\index{segmento orientado!equipolente} a um segmento orientado $CD$, quando $AB$ tem a \emph{mesma norma}, a \emph{mesma direção} e o \emph{mesmo sentido} de $CD$. Segmentos nulos também são considerados equipolentes entre si. Quando $AB$ é equipolente a $CD$, escrevemos $AB \sim CD$.

\begin{figure}[H]
  \centering
  \includegraphics[width=0.7\textwidth]{./cap_vetor/dados/fig_segequipolentes/fig_segequipolentes}
  \caption{Esboço de dois segmentos orientados $AB$ e $CD$ equipolentes.}
  \label{fig:segequipolentes}
\end{figure}

A relação de equipolência é uma \emph{relação de equivalência}. De fato, temos:
\begin{itemize}
\item \emph{relação reflexiva}: $AB \sim AB$;
\item \emph{relação simétrica}: $AB \sim CD \Rightarrow CD \sim AB$;
\item \emph{relação transitiva}: $AB \sim CD ~ \text{e} ~ CD \sim EF \Rightarrow AB \sim CD$.
\end{itemize}

Com isso, dado um segmento $AB$, definimos a \emph{classe de equipolência} de $AB$ como o conjunto de todos os segmentos equipolentes a $AB$. O segmento $AB$ é um \emph{representante} desta classe.

\subsection*{Exercícios resolvidos}

\begin{exeresol}
  Mostre que dois segmentos orientados $AB$ e $CD$ são equipolentes se, e somente se, os pontos médios de $AD$ e $BC$ são coincidentes.
\end{exeresol}
\begin{resol}
  Começamos mostrando a implicação. Por hipótese, temos que $AB$ e $CD$ são equipolentes. A tese é clara no caso de $AB$ e $CD$ serem coincidentes. Vejamos, então, o caso em que $AB$ e $CD$ não são coincidentes. Desta forma, $ABCD$ determina um paralelogramo de diagonais $AD$ e $BC$. Como as diagonais de um paralelogramo se interceptam em seus pontos médios, temos demonstrado a implicação.

  Agora, mostramos a recíproca. Por hipótese, temos que os pontos médios de $AD$ e $BC$ são coincidentes. Novamente, se $AD$ e $BD$ são coincidentes a conclusão é direta. Consideremos o caso em que $AD$ e $BD$ não são coincidentes. Daí, segue que $AB$ e $CD$ têm o mesmo tamanho e mesma direção. Seja $M$ o ponto médio de $AD$ e $BC$ e $\pi$ o plano determinado pelos segmentos $AB$ e $CD$. Notando que $M$, $B$ e $D$ estão no mesmo semiplano de $\pi$ determinado pela reta $AC$, concluímos que $AB$ e $CD$ são equipolentes.
\end{resol}

\begin{exeresol}
  Mostre que $AB\sim CD$, então $BA\sim DC$.
\end{exeresol}
\begin{resol}
  $AB$ e $BA$ têm o mesmo tamanho e direção. $CD$ e $DC$ têm o mesmo tamanho e direção. Como $AB\sim CD$, temos que $BA$ e $DC$ têm o mesmo tamanho e direção. Por fim, observa-se que $BA$ e $DC$ têm ambos o mesmo sentido oposto de $AB$ e $DC$. 
\end{resol}

\subsection*{Exercícios}

\begin{exer}
  Faça o esboço de dois segmentos $AB$ e $CD$ com $|AB|\neq |CD|$ e cujas retas determinadas por eles sejam coincidentes.
\end{exer}

\begin{exer}
  Faça o esboço de dois segmentos orientados $AB\not\sim CD$ e de mesmo sentido.
\end{exer}

\begin{exer}
  Faça o esboço de dois segmentos orientados colineares, de tamanhos iguais e sentidos opostos.
\end{exer}

\begin{exer}
  Diga se é verdadeira ou falsa a seguinte afirmação: é quadrado todo trapézio retângulo $ABCD$ com segmentos orientados $AD$ e $BC$ equipolentes. Justifique sua afirmação. 
\end{exer}
\begin{resp}
  Verdadeira.
\end{resp}

\begin{exer}
  Mostre que $AB\sim CD$, então $AC\sim BD$.
\end{exer}
\begin{resp}
  Dica: $ABCD$ determina um paralelogramo.
\end{resp}

\begin{exer}
  Mostre que se $AC\sim CB$, então $C$ é ponto médio do segmento $AB$.
\end{exer}

\section{Vetor}\label{cap_vetor_sec_vetor}

\begin{flushright}
  \href{https://archive.org/details/definicao-vetor}{$\blacktriangleright$ Vídeo disponível!}
\end{flushright}

Dado um segmento orientado $AB$, chama-se {\bf vetor} $AB$ e denota-se $\overrightarrow{AB}$, qualquer segmento orientado equipolente a $AB$. Em outras palavras, o vetor $\overrightarrow{AB}$ é a classe de equipolência que tem o segmento orientado $AB$ como um representante. A Figura \ref{fig:vetor} mostra duas representações de um dado vetor $\overrightarrow{AB}$.

\begin{figure}[h!]
  \centering
  \includegraphics[width=0.7\textwidth]{./cap_vetor/dados/fig_vetor/fig_vetor}
  \caption{Esboço de duas representações de um mesmo vetor.}
  \label{fig:vetor}
\end{figure}

Observemos que na Figura \ref{fig:vetorrel}(direita) os vetores foram denotados por $\vec{a}$, $\vec{b}$ e $\vec{c}$, sem alusão aos pontos que definem suas representações como segmentos orientados. Isto é costumeiro, devido a definição de vetor.

O \emph{vetor nulo} é aquele que tem como seu representante um segmento orientado nulo. É denotado por $\vec{0}$.

O {\bf módulo}\index{módulo} (ou {\bf norma}\index{norma}) de um vetor $\vec{v}$ é denotado(a) por $|\vec{v}|$ e é definido como o valor do comprimento de qualquer uma de suas representações. Mais precisamente, se $\overrightarrow{AB}$ é uma representação de $\vec{v}$, então $|\vec{v}| := |\overrightarrow{AB}|$.

\begin{obs}
  $|\vec{v}| = 0$ se, e somente se, $\vec{v} = \vec{0}$.

  Seja $\vec{v} = \overrightarrow{AB}$. Lembrando que $|\overrightarrow{AB}| = |AB|$, i.e. a distância entre os pontos $A$ e $B$, segue que se $\vec{v} = \vec{0}$, então $A$ e $B$ são dois pontos sobrepostos e, portanto, $|\vec{v}| = |AB| = 0$. Reciprocamente, se $|AB| = 0$, então $A$ e $B$ são sobrepostos e $\overrightarrow{AB} = \vec{0}$.
\end{obs}

Dois {\bf vetores} são ditos {\bf paralelos} \index{vetores!paralelos} quando qualquer de suas representações têm a mesma direção. De forma análoga, definem-se {\bf vetores coplanares}\index{vetores!coplanares}, {\bf vetores não coplanares}\index{vetores!não coplanares}, {\bf vetores ortogonais}\index{vetores!ortogonais}, além de conceitos como {\bf ângulo entre dois vetores}\index{ângulo!entre vetores}, etc. Veja a Figura \ref{fig:vetorrel}.

\begin{figure}[h!]
  \centering
  \includegraphics[width=0.5\textwidth]{./cap_vetor/dados/fig_vetorrel/fig_vetorrel}~
  \includegraphics[width=0.5\textwidth]{./cap_vetor/dados/fig_vcolineares/fig_vcolineares}
  \caption{Esquerda: esboços de vetores paralelos e de vetores ortogonais. Direita: esboços de vetores coplanares.}
  \label{fig:vetorrel}
\end{figure}

\subsection{Adição de vetores}

\begin{flushright}
  \href{https://archive.org/details/adicao-de-vetores}{$\blacktriangleright$ Vídeo disponível!}
\end{flushright}

Sejam dados dois vetores $\vec{u}$ e $\vec{v}$. Sejam, ainda, uma representação $\overrightarrow{AB}$ de $\vec{u}$ e uma representação $\overrightarrow{BC}$ do vetor $\vec{v}$. Então, define-se o vetor soma $\vec{u}+\vec{v}$ como o vetor representado por $\overrightarrow{AC}$. Veja a Figura \ref{fig:vadicao}.

\begin{figure}[H]
  \centering
  \includegraphics[width=0.7\textwidth]{./cap_vetor/dados/fig_vadicao/fig_vadicao}
  \caption{Representação geométrica da adição de dois vetores.}
  \label{fig:vadicao}
\end{figure}

\begin{obs}
  Vejamos as seguintes propriedades:
  \begin{enumerate}[a)]
  \item Elemento neutro na adição:
    \begin{equation}
      \vec{u} + \vec{0} = \vec{u}
    \end{equation}
    
    De fato, seja $\vec{u} = \overrightarrow{AB}$. Observamos que podemos representar $\vec{0} = \overrightarrow{BB}$. Logo, temos $\vec{u} + \vec{0} = \overrightarrow{AB} + \overrightarrow{BB} = \overrightarrow{AB} = \vec{u}$.

  \item Associatividade na adição:
    \begin{equation}
      (\vec{u} + \vec{v}) + \vec{w} = \vec{u} + (\vec{v} + \vec{w}).
    \end{equation}

    De fato, sejam $\vec{u} = \overrightarrow{AB}$, $\vec{v} = \overrightarrow{BC}$ e $\vec{w} = \overrightarrow{CD}$. Então, segue
    \begin{align}
      \left(\vec{u} + \vec{v}\right)+\vec{w} &= \left(\overrightarrow{AB}+\overrightarrow{BC}\right)+\overrightarrow{CD} \\
                                             &= \overrightarrow{AC} + \overrightarrow{CD} \\
                                             &= \overrightarrow{AD},
    \end{align}
    bem como,
    \begin{align}
      \vec{u} + \left(\vec{v} + \vec{w}\right) &= \overrightarrow{AB}+\left(\overrightarrow{BC}+\overrightarrow{CD}\right) \\
                                             &= \overrightarrow{AB} + \overrightarrow{BD} \\
                                             &= \overrightarrow{AD}.
    \end{align}
  \item Comutatividade da adição:
    \begin{equation}
      \vec{u} + \vec{v} = \vec{v} + \vec{u}.
    \end{equation}

    Esta propriedade pode ser demonstrada usando a regra do paralelogramo que veremos mais adiante. Veja, também, o Exercício Resolvido \ref{exeresol:vetor_comuta_adicao}.
  \end{enumerate}
\end{obs}

\subsection{Vetor oposto}

\begin{flushright}
  \href{https://archive.org/details/vetor-oposto}{$\blacktriangleright$ Vídeo disponível!}
\end{flushright}

Um \pmb{vetor} $\vec{v}$ é dito ser \pmb{oposto} \index{vetor!oposto} a um dado vetor $\vec{u}$, quando quaisquer representações de $\vec{u}$ e $\vec{v}$ são segmentos orientados de mesmo comprimento e mesma direção, mas com sentidos opostos. Neste caso, denota-se por $-\vec{u}$ o vetor oposto a $\vec{u}$. Veja a Figura \ref{fig:voposto}.

\begin{figure}[H]
  \centering
  \includegraphics[width=0.7\textwidth]{./cap_vetor/dados/fig_voposto/fig_voposto}
  \caption{Representação geométrica de vetores opostos.}
  \label{fig:voposto}
\end{figure}

\begin{obs}
  $|\vec{v}| = |-\vec{v}|$.

  De fato, seja $\vec{v} = \overrightarrow{AB}$. Então, $|\vec{v}| = |AB| = |BA| = |-\vec{v}|$.
\end{obs}

\begin{obs}(Existência do oposto)
  \begin{equation}
    \vec{u} + \left(-\vec{u}\right) = \vec{0}.
  \end{equation}

  De fato, seja $\vec{u} = \overrightarrow{AB}$. Então, $-\vec{u} = -\overrightarrow{AB} = \overrightarrow{BA}$. Segue que
  \begin{align}
    \vec{u} + \left(-\vec{u}\right) &= \overrightarrow{AB} + \left(-\overrightarrow{AB}\right) \\
                                    &= \overrightarrow{AB} + \overrightarrow{BA} \\
                                    &= \overrightarrow{AA} \\
                                    &= \vec{0}.
  \end{align}
\end{obs}

\subsection{Subtração de vetores}

\begin{flushright}
  \href{https://archive.org/details/subtracao-de-vetores}{$\blacktriangleright$ Vídeo disponível!}
\end{flushright}

Sejam dados dois vetores $\vec{u}$ e $\vec{v}$. A subtração de $\vec{u}$ com $\vec{v}$ é denotada por $\vec{u}-\vec{v}$ e é definida pela adição de $\vec{u}$ com $-\vec{v}$, i.e. $\vec{u}-\vec{v}=\vec{u}+(-\vec{v})$. Veja a Figura \ref{fig:vsubtracao}.

\begin{figure}[H]
  \centering
  \includegraphics[width=0.7\textwidth]{./cap_vetor/dados/fig_vsubtracao/fig_vsubtracao}
  \caption{Representação geométrica da subtração de $\vec{u}$ com $\vec{v}$, i.e. $\vec{u}-\vec{v}$.}
  \label{fig:vsubtracao}
\end{figure}

\begin{obs}\normalfont{(Regra do paralelogramo)}
  \begin{flushright}
    \href{https://archive.org/details/regra-do-paralelogramo}{$\blacktriangleright$ Vídeo disponível!}
  \end{flushright}

  Sejam vetores não nulos $\vec{u} = \overrightarrow{AB}$ e $\vec{v} = \overrightarrow{AD}$. Seja, ainda, $C$ o vértice oposto ao $A$ no paralelogramo determinado pelos lados formados pelos segmentos $AB$ e $AD$. Então, temos $\vec{u} + \vec{v} = \overrightarrow{AC}$ e $\vec{u}-\vec{v} = \overrightarrow{DB}$. Veja a Figura \ref{fig:regrapara}.

\begin{figure}[H]
  \centering
  \includegraphics[width=0.7\textwidth]{./cap_vetor/dados/fig_regrapara/fig_regrapara}
  \caption{Regra do paralelogramo para a presentação geométrica da soma e da diferença de vetores.}
  \label{fig:regrapara}
\end{figure}  
\end{obs}

\subsection{Multiplicação de vetor por um escalar}

\begin{flushright}
  \href{https://archive.org/details/multiplicacao-vetor-por-escalar}{$\blacktriangleright$ Vídeo disponível!}
\end{flushright}

A multiplicação de um número real $\alpha>0$ (escalar) por um vetor $\vec{u}$ é denotado por $\alpha\vec{u}$ e é definido pelo vetor de mesma direção e mesmo sentido de $\vec{u}$ com norma $\alpha|\vec{u}|$. Quando $\alpha = 0$, define-se $\alpha\vec{u}=\vec{0}$, i.e. o vetor nulo (geometricamente, representado por qualquer ponto).

\begin{obs}
  \begin{itemize}
  \item Para $\alpha<0$, temos $\alpha\vec{u} = -(-\alpha\vec{u})$.
  \item $|\alpha\vec{u}|=|\alpha||\vec{u}|$.
\end{itemize}
\end{obs}

\begin{figure}[h!]
  \centering
  \includegraphics[width=0.7\textwidth]{./cap_vetor/dados/fig_vescalar/fig_vescalar}
  \caption{Representações geométricas de multiplicações de um vetor por diferentes escalares.}
  \label{fig:vescalar}
\end{figure}

\begin{obs}
  As seguintes propriedades são válidas:
  \begin{enumerate}[a)]
  \item Associatividade da multiplicação por escalar:
    \begin{equation}
      \alpha\left(\beta\vec{u}\right) = (\alpha\beta)\vec{u}
    \end{equation}

    De fato, em primeiro lugar, observamos que $\alpha\left(\beta\vec{u}\right)$ e $(\alpha\beta)\vec{u}$ têm a mesma direção e o mesmo sentido. Por fim, temos
    \begin{align}
      |\alpha\left(\beta\vec{u}\right)| &= |\alpha||\beta\vec{u}| \\
                                        &= |\alpha|\left(|\beta||\vec{u}|\right) \\
                                        &= \left(|\alpha||\beta|\right)|\vec{u}| \\
                                        &= |\alpha\beta||\vec{u}| \\
                                        &= |(\alpha\beta)\vec{u}|.
    \end{align}
    
  \item Distributividade:
    \begin{align}
      &(\alpha + \beta)\vec{u} = \alpha\vec{u} + \beta\vec{u}\\
      &\alpha\left(\vec{u}+\vec{v}\right) = \alpha\vec{u} + \alpha\vec{v}
    \end{align}
  \end{enumerate}
\end{obs}

\subsection{Resumo das propriedades das operações com vetores}

As operações de adição e multiplicação por escalar de vetores têm propriedades importantes. Para quaisquer vetores $\vec{u}$, $\vec{v}$ e $\vec{w}$ e quaisquer escalares $\alpha$ e $\beta$ temos:
\begin{itemize}
\item comutatividade da adição: $\vec{u}+\vec{v}=\vec{v}+\vec{u}$;
\item associatividade da adição: $(\vec{u} + \vec{v}) + \vec{w} = \vec{u} + (\vec{v} + \vec{w})$;
\item elemento neutro da adição: $\vec{u}+\vec{0}=\vec{u}$;
\item existência do oposto: $\vec{u}+(-\vec{u}) = \vec{0}$;
\item associatividade da multiplicação por escalar: $\alpha(\beta\vec{u})=(\alpha\beta)\vec{u}$;
\item distributividade da multiplicação por escalar:
  \begin{align}
    &\alpha(\vec{u}+\vec{v}) = \alpha\vec{u}+\alpha\vec{v},\\
    &(\alpha+\beta)\vec{u} = \alpha\vec{u}+\beta\vec{u};
  \end{align}
\item existência do elemento neutro da multiplicação por escalar: $1\vec{u}=\vec{u}$.
\end{itemize}

\subsection*{Exercícios resolvidos}

\begin{exeresol}\label{exeresol:vetor_comuta_adicao}
  Mostre que $\vec{u} + \vec{v} = \vec{v} + \vec{u}$.
\end{exeresol}
\begin{resol}
  Seja $ABCD$ o paralelogramo com $\vec{u} = \overrightarrow{AB} = \overrightarrow{DC}$ e $\vec{v} = \overrightarrow{AD} = \overrightarrow{BC}$. Logo, pela regra do paralelogramo temos
  \begin{align}
    \vec{u} + \vec{v} &= \overrightarrow{AB} + \overrightarrow{BC} \\
                      &= \overrightarrow{AC} \\
                      &= \overrightarrow{AD} + \overrightarrow{DC} \\
                      &= \vec{v} + \vec{u}.
  \end{align}
\end{resol}

\subsection*{Exercícios}

\begin{exer}\label{exer:vetor_prob_01}
  Na figura abaixo, temos $\vec{u} = \overrightarrow{GJ}$ e $\vec{v} = \overrightarrow{AK}$. Assim sendo, escreva os vetores $\overrightarrow{RS}$, $\overrightarrow{NI}$, $\overrightarrow{AG}$, $\overrightarrow{NQ}$, $\overrightarrow{AT}$ e $\overrightarrow{PE}$ em função de $\vec{u}$ e $\vec{v}$.

  \includegraphics[width=0.7\textwidth]{./cap_vetor/dados/fig_exer_prob_01/fig_exer_prob_01}
\end{exer}

\begin{exer}\label{exer:vetor_prob_02}
  Sejam $\overrightarrow{CA}$, $\overrightarrow{CM}$ e $\overrightarrow{CB}$ os vetores indicados na figura abaixo. Mostre que $\overrightarrow{CM} = \frac{1}{2}\overrightarrow{CA} + \frac{1}{2}\overrightarrow{CB}$.

  \includegraphics[width=0.7\textwidth]{./cap_vetor/dados/fig_exer_prob_02/fig_exer_prob_02}
\end{exer}
\begin{resp}
  Dica: $M$ é o ponto médio do segmento orientado $AB = CB - AC$.
\end{resp}

\begin{exer}
  Seja dado um vetor $\vec{u}\neq 0$. Calcule a norma do vetor $\vec{v}=\vec{u}/|\vec{u}|$\footnote{$\vec{u}/|\vec{u}|$ é chamado de vetor $\vec{u}$ normalizado, ou a normalização do vetor $\vec{u}$.}.
\end{exer}
\begin{resp}
  $|\vec{v}|=1$.
\end{resp}

\begin{exer}
  Diga se é verdadeira ou falsa cada uma das seguintes afirmações. Justifique sua resposta.
  \begin{enumerate}
  \item $\vec{u}+\vec{u} = 2\vec{u}$
  \item $\vec{u}=-\vec{u} \Leftrightarrow \vec{u} = \vec{0}$.
  \end{enumerate}
\end{exer}
\begin{resp}
  a) verdadeira; b) verdadeira.
\end{resp}
