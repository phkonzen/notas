%Este trabalho está licenciado sob a Licença Atribuição-CompartilhaIgual 4.0 Internacional Creative Commons. Para visualizar uma cópia desta licença, visite http://creativecommons.org/licenses/by-sa/4.0/deed.pt_BR ou mande uma carta para Creative Commons, PO Box 1866, Mountain View, CA 94042, USA.

\chapter{Integração}\label{cap_integracao}
\thispagestyle{fancy}

\begin{flushright}
  [Vídeo] | [Áudio] | \href{https://phkonzen.github.io/notas/contato.html}{[Contatar]}
\end{flushright}

Neste capítulo, estudamos métodos numéricos para a integração de funções reais.

\section{Integração Autoadaptativa}\label{cap_integracao_sec_autoadapt}

Vamos considerar o problema de integrar
\begin{equation}
  I(a,b) = \int_a^b f(x)\,dx
\end{equation}
pela \emph{Regra de Simpson}\footnote{Consulte mais sobre a Regra de Simpson em \href{https://phkonzen.github.io/notas/MatematicaNumerica/cap_integr_sec_NC.html}{Seção 10.1 Regras de Newton-Cotes}.}. Em um dado subintervalo $[\alpha, \beta]\subset [a, b]$, temos
\begin{equation}
  I(\alpha, \beta) = \underbrace{\frac{h_0}{3}\left[f(\alpha) + 4f(\alpha+h_0) + f(\beta)\right]}_{S(\alpha,\beta)}-\frac{h_0^5}{90}f^{(4)}(\xi),
\end{equation}
onde $h_0=(\beta-\alpha)/2$ e $\xi\in (\alpha, \beta)$. Ou seja, temos que
\begin{equation}\label{eq:estS1}
  I(\alpha,\beta) - S(\alpha,\beta) = -\frac{h_0^5}{90}f^{(4)}(\xi).
\end{equation}
A ideia é explorarmos esta informação de forma a obtermos uma estimativa para o erro de integração no intervalo $[\alpha,\beta]$ sem necessitar computar $f^{(4)}$.

Aplicando a Regra de Simpson na partição $[\alpha,(\alpha+\beta)/2]\cup [(\alpha+\beta)/2, \beta]$, obtemos
\begin{equation}
  I(\alpha,\beta) - S_2(\alpha,\beta) = -\frac{(h_0/2)^5}{90}\left(f^{(4)}(\xi) + f^{(4)}(\eta)\right),
\end{equation}
onde $\xi\in (\alpha,(\alpha+\beta)/2)$, $\eta\in (\alpha+\beta)/2, \beta)$ e
\begin{equation}
  S_2(\alpha,\beta) = S(\alpha,(\alpha+\beta)/2) + S((\alpha+\beta)/2,\beta).
\end{equation}
Agora, \emph{vamos assumir que} $f^{(4)}(\xi)\approx f^{(4)}(\eta)$ de forma que temos
\begin{equation}\label{eq:estS2}
  I(\alpha,\beta) - S_2(\alpha,\beta) \approx -\frac{1}{16}\frac{h_0^5}{90}f^{(4)}(\xi).
\end{equation}
De \eqref{eq:estS1} e \eqref{eq:estS2}, obtemos
\begin{equation}
  \frac{h_0^5}{90}f^{(4)}(\xi) \approx \frac{16}{15}\underbrace{\left[S(\alpha,\beta)-S_2(\alpha,\beta)\right]}_{\mathcal{E}(\alpha,\beta)}.
\end{equation}
Isto nos fornece a seguinte estimativa {\it a posteriori} do erro
\begin{equation}
  |I(\alpha,\beta)-S_2(\alpha,\beta)| \approx \frac{|\mathcal{E}(\alpha,\beta)|}{15}.
\end{equation}
Na prática, costuma-se utilizar a seguinte estimativa mais restrita
\begin{equation}
  |I(\alpha,\beta)-S_2(\alpha,\beta)| \approx \frac{|\mathcal{E}(\alpha,\beta)|}{10}.
\end{equation}
Para garantir uma precisão global em $[a,b]$ igual a uma dada tolerância, é suficiente impor que
\begin{equation}
  \frac{|\mathcal{E}(\alpha,\beta)|}{10} \leq \epsilon\frac{\beta-\alpha}{b-a}.
\end{equation}

\lstinputlisting[caption=Algoritmo Simpson Autoadaptativo, label={lst:algSimAd}]{./cap_integracao/dados/pySimAd/main.py}

\begin{ex}
  \begin{align}
    \int_{-3}^4\arctg(10x)\,dx &= -3\arctg(30) - \frac{\ln(1601)}{20} + \frac{\ln(901)}{20} + 4\arctg(40)\\
                               &\approx 1.54203622
  \end{align}
\end{ex}

\begin{exer}
  Implemente uma abordagem autoadaptativa usando a Regra do Trapézio. Valide-a e compare com o exemplo anterior.
\end{exer}

\section{Integrais múltiplas}\label{cap_integracao_sec_intmul}

Vamos trabalhar com métodos para a computação de integrais múltiplas
\begin{equation}
  \int\int_R f(x,y)\,dA.
\end{equation}
Em uma região retangular $A=[a,b]\times [c,d]$, podemos reescrevê-la como uma \emph{integral iterada}
\begin{equation}
  \int\int_R f(x,y)\,dA = \int_a^b\int_c^d f(x,y)\,dy\,dx.
\end{equation}

\subsection{Regras de Newton-Cotes}

\subsubsection{Regra do Trapézio}

A Regra do Trapézio\footnote{\href{https://phkonzen.github.io/notas/MatematicaNumerica/cap_integr_sec_NC.html}{Notas de Aula - Matemática Numérica}.} nos fornece
\begin{equation}
  \int_c^d f(x,y)\,dy = \frac{h_y}{2}\left[f(x,c) + f(x,d)\right] - \frac{h_y^3}{12}f''(x,\eta)
\end{equation}
com $h_y = (d-c)$ e $\eta\in (c,d)$. De forma iterada, temos
\begin{align}
  \int_a^b\int_c^d f(x,y)\,dy\,dx &= \frac{h_y}{2}\int_a^b f(x,c)\,dx + \frac{h_y}{2}\int_a^bf(x,d)\,dx\\
                                  &- \frac{h_y^3}{12}\int_a^b f''(x,\eta)\,dx.
\end{align}
Então, à exceção do termo do erro, aplicamos a Regra do Trapézio para as integrais em $x$. Obtemos
\begin{align}
  \int_a^b\int_c^d f(x,y)\,dy\,dx &= \frac{h_y}{2}\frac{h_x}{2}\left[f(a,c) + f(b,c)\right]\\
                                  &+ \frac{h_y}{2}\frac{h_x}{2}\left[f(a,d) + f(b,d)\right]\\
                                  &-\frac{h_y}{2}\frac{h_x^3}{12}f''(\mu',c)\\
                                  &-\frac{h_y}{2}\frac{h_x^3}{12}f''(\mu'',d)\\
                                  &-\frac{h_y^3}{12}\int_a^b f''(x,\eta)\,dx,
\end{align}
com $h_x = (b-a)$, $\mu',\mu''\in (a,b)$. Pelos Teorema do Valor Intermediário e pelo Teorema do Valor Médio, podemos ver que o erro é $O(h_xh_y^3 + h_x^3h_y)$. Por fim, obtemos a Regra do Trapézio para Integrais Iteradas
\begin{align}
  \int_a^b\int_c^d f(x,y)\,dy\,dx &= \frac{h_y}{2}\frac{h_x}{2}\left[f(a,c)+f(b,c)+f(b,d)+f(a,d)\right]\\
                                  &+ O(h_xh_y^3 + h_x^3h_y).
\end{align}

\begin{ex}
  A Regra do Trapézio fornece
  \begin{equation}
    \int_{1.5}^{2}\int_{1}^{1.5}\ln(x + 2y)\,dy\,dx \approx 0.36.
  \end{equation}
  Verifique!
\end{ex}

\subsubsection{Regra de Simpson}

A Regra do Simpson\footnote{\href{https://phkonzen.github.io/notas/MatematicaNumerica/cap_integr_sec_NC.html}{Notas de Aula - Matemática Numérica}.} nos fornece
\begin{align}
  \int_c^d f(x,y)\,dy &= \frac{h_y}{3}\left[f(x,y_1) + 4f(x,y_2) + f(x,y_3)\right] \\
                      &- \frac{h_y^5}{90}f^{(4)}(x,\eta)
\end{align}
com $h_y = (d-c)/2$, $y_j=(j-1)h_y$, $j=1,2,3$, e $\eta\in (c,d)$. De forma iterada, temos
\begin{align}
  \int_a^b\int_c^d f(x,y)\,dy\,dx &= \frac{h_y}{3}\left[\int_a^bf(x,y_1)\,dx + 4\int_a^bf(x,y_2)\,dx + \int_a^bf(x,y_3)\,dx\right] \\
                                  &- \frac{h_y^5}{90}\int_a^bf^{(4)}(x,\eta)\,dx
\end{align}
Então, à exceção do termo do erro, aplicamos a Regra de Simpson para as integrais em $x$. Obtemos
\begin{align}
  \int_a^b\int_c^d f(x,y)\,dy\,dx &= \frac{h_xh_y}{9}\left[f(x_1,y_1) + 4f(x_2,y_1) + f(x_3,y_1)\right] \\
                                  &+ \frac{4h_xh_y}{9}\left[f(x_1,y_2) + 4f(x_2,y_2) + f(x_3,y_2)\right] \\
                                  &+ \frac{h_xh_y}{9}\left[f(x_1,y_3) + 4f(x_2,y_3) + f(x_3,y_3)\right] \\
                                  &- \frac{h_x^5h_y}{270}f^{(4)}(\mu_1,y_1)\\
                                  &- \frac{4h_x^5h_y}{270}f^{(4)}(\mu_2,y_2)\\
                                  &- \frac{h_x^5h_y}{270}f^{(4)}(\mu_3,y_3) \\
                                  &- \frac{h_y^5}{90}\int_a^bf^{(4)}(x,\eta)\,dx
\end{align}
com $h_x = (b-a)/2$, $\mu_1,\mu_2,\mu_3\in (a,b)$. Pelos Teorema do Valor Intermediário e Teorema do Valor Médio, podemos ver que o erro é $O(h_xh_y^5 + h_x^5h_y)$. Por fim, obtemos a \emph{Regra de Simpson para Integrais Iteradas}
\begin{align}
  \int_a^b\int_c^d f(x,y)\,dy\,dx &= \frac{h_xh_y}{9}\left[f(x_1,y_1) + 4f(x_2,y_1) + f(x_3,y_1)\right] \\
                                  &+ \frac{4h_xh_y}{9}\left[f(x_1,y_2) + 4f(x_2,y_2) + f(x_3,y_2)\right] \\
                                  &+ \frac{h_xh_y}{9}\left[f(x_1,y_3) + 4f(x_2,y_3) + f(x_3,y_3)\right] \\
                                  &+ O(h_xh_y^5 + h_x^5h_y).
\end{align}

\begin{ex}
  A Regra de Simpson fornece
  \begin{equation}
    \int_{1.5}^{2}\int_{1}^{1.5}\ln(x + 2y)\,dy\,dx \approx 0.361003.
  \end{equation}
  Verifique!
\end{ex}
