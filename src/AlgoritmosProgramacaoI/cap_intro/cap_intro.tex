\chapter{Introdução}\label{cap_intro}

Vamos começar executando nossas primeiras \emph{linhas de código} na linguagem de programação {\python}. Em um \emph{terminal} {\python} digitamos

\begin{lstlisting}
>>> print('Olá, mundo!')
\end{lstlisting}

Observamos que \lstinline+>>>+ é o símbolo do \lstinline+prompt de entrada+ e digitamos nossa \emph{instrução} logo após ele. Para executarmos a instrução digitada, teclamos \lstinline+<ENTER>+. Uma vez executada, o terminal apresentará as seguintes informações

\begin{lstlisting}
>>> print('Olá, mundo!')
Olá, mundo!
>>> 
\end{lstlisting}

Pronto! O fato do símbolo de \lstinline+prompt de entrada+ ter aparecido novamente, indica que a instrução foi completamente executada e o terminal está pronto para executar uma nova instrução.

A \emph{linha de comando} executada acima pede ao computador para imprimir no \lstinline+prompt de saída+ a frase \lstinline+Olá, mundo!+. O \emph{método} {\PYTHONprint} contém instruções para imprimir \emph{objetos} em um dispositivo de saída, no caso, imprime a frase na tela do computador.

Bem! Talvez imprimir no \lstinline+prompt de saída+ uma frase que digitamos no \lstinline+prompt de entrada+ possa parecer um pouco redundante no momento. Vamos considerar um outro exemplo, computar a soma dos números ímpares entre $0$ e $100$. Podemos fazer isso como segue

\begin{lstlisting}
>>> sum([i for i in range(100) if i%2 != 0])
2500
\end{lstlisting}

Oh! No momento, não se preocupe se não tenha entendido a linha de comando de entrada, ao longo dessas notas de aula isso vai ficando natural. A linha de comando de entrada usa o método {\PYTHONsum} para computar a soma dos elementos da \emph{lista} de números ímpares desejada. A lista é construída de forma \emph{iterada} e \emph{indexada} pela \emph{variável} \lstinline+i+, para \lstinline+i+ no intervalo/faixa de $0$ a $99$, se o resto da divisão de \lstinline+i+ por $2$ não for igual a $0$. Ok! O resultado computado foi $2500$.

De fato, a soma dos números ímpares de $0$ a $100$
\begin{equation}
  (1, 3, 5, \dotsc, 99)
\end{equation}
é a soma dos 50 primeiros elementos da progressão aritmética $a_i = 1 + 2i$, $i=0, 1, \ldots$, i.e.
\begin{align}
  \sum_{i=0}^{49}a_i &= a_0 + a_1 + \cdots + a_{49}\\
                     &= 1 + 3 + \cdots + 99\\
                     &= \frac{50(1 + 99)}{2}\\
                     &= 2500
\end{align}
como já esperado! Em {\python}, esta última conta pode ser computada como segue

\begin{lstlisting}
>>> 50*(1+99)/2
2500.0
\end{lstlisting}
