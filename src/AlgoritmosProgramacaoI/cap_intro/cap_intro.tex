\chapter{Introdução}\label{cap_intro}

Vamos começar executando nossas primeiras \emph{linhas de código} na linguagem de programação {\python}. Em um \emph{terminal} {\python} digitamos

\begin{lstlisting}
>>> print('Olá, mundo!')
\end{lstlisting}

Observamos que \lstinline+>>>+ é o símbolo do \texttt{prompt de entrada} e digitamos nossa \emph{instrução} logo após ele. Para executarmos a instrução digitada, teclamos \texttt{<ENTER>}. Uma vez executada, o terminal apresentará as seguintes informações

\begin{lstlisting}
>>> print('Olá, mundo!')
Olá, mundo!
>>> 
\end{lstlisting}

Pronto! O fato do símbolo de \texttt{prompt de entrada} ter aparecido novamente, indica que a instrução foi completamente executada e o terminal está pronto para executar uma nova instrução.

Alternativamente a terminais, aplicativos \textit{notebooks}, como o {\jupyter}, permitem organizar códigos {\python} em células de programação. Ao longo do texto, salvo explicitado diferente, vamos assumir a utilização de um \textit{notebook} {\jupyter} rodando {\python} 3 \cite{Python2024a}. Células de código são destacadas com linhas enumeradas como, por exemplo,

\begin{lstlisting}
print('Olá, mundo!')
\end{lstlisting}

E quando for o caso, a saída aparece logo na sequência como, no caso,

\begin{verbatim}
Olá, mundo!
\end{verbatim}

A \emph{linha de comando} executada acima pede ao computador para imprimir no \texttt{prompt de saída} a frase \texttt{Olá, mundo!}. O \emph{método}/\emph{função} {\PYTHONprint} contém instruções para imprimir \emph{objetos} em um dispositivo de saída, no caso, imprime a frase na tela do computador.

Vamos considerar um outro exemplo, computar a soma dos números ímpares entre $0$ e $100$. Podemos fazer isso como segue

\begin{lstlisting}
sum([i for i in range(100) if i%2 != 0])
\end{lstlisting}

\begin{verbatim}
2500
\end{verbatim}

Oh! No momento, não se preocupe se não tenha entendido a linha de código acima, ao longo das notas de aula isso vai ficando natural. O código usa a função {\PYTHONsum} para computar a soma dos elementos da \emph{lista} de números ímpares desejada. A lista é construída de forma \emph{iterada} e \emph{indexada} pela \emph{variável} \texttt{i}, para \texttt{i} no intervalo/faixa de $0$ a $99$, se o resto da divisão de \texttt{i} por $2$ não for igual a $0$. 

O resultado computado foi $2500$. De fato, a soma dos números ímpares de $0$ a $100$
\begin{equation}
  (1, 3, 5, \dotsc, 99)
\end{equation}
é a soma dos 50 primeiros elementos da progressão aritmética $a_i = 1 + 2i$, $i=0, 1, \ldots$, i.e.
\begin{align}
  & \sum_{i=0}^{49}a_i = a_0 + a_1 + \cdots + a_{49}\\
  & \text{}\quad = 1 + 3 + \cdots + 99\\
  & \text{}\quad = \frac{50(1 + 99)}{2}\\
  & \text{}\quad = 2500
\end{align}
como já esperado! Em {\python}, esta última conta pode ser computada como segue

\begin{lstlisting}
50*(1+99)/2
\end{lstlisting}

\begin{verbatim}
2500.0
\end{verbatim}
