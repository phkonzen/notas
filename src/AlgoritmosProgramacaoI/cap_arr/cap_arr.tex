% Este trabalho está licenciado sob a Licença Atribuição-CompartilhaIgual 4.0 Internacional Creative Commons. Para visualizar uma cópia desta licença, visite http://creativecommons.org/licenses/by-sa/4.0/deed.pt_BR ou mande uma carta para Creative Commons, PO Box 1866, Mountain View, CA 94042, USA.

\chapter{Arranjos e Matrizes}\label{cap_arr}
\thispagestyle{fancy}

\hl{Um arranjo é uma coleção de objetos} (todos de um mesmo tipo) \hl{em que os elementos são organizados por eixos}. É a estrutura de dados mais utilizada para a alocação de vetores e matrizes, fundamentais na computação matricial.

\section{Arranjos}\label{cap_arr_sec_arr}

\hl{Um arranjo (em inglês, \textit{array}) é uma coleção de objetos (todos do mesmo tipo) em que os elementos são organizados por eixos}. Nesta seção, vamos nos restringir a \hl{\emph{arranjos unidimensionais}} (de apenas um eixo). Esta é a estrutura computacionais usualmente utilizada \hl{para a alocação de vetores}.

\hl{{\numpy} é uma biblioteca {\python} que fornece suporte para a alocação e manipulação de arranjos}. Usualmente, a biblioteca é importada como segue
\begin{lstlisting}
import numpy as np
\end{lstlisting}
Na sequência, vamos assumir que o {\numpy} já está importado como acima.

\subsection{Alocação de Arranjos}

Na linguagem, a \hl{alocação de um arranjo} pode ser feita com o método \hl{{\href{https://numpy.org/doc/stable/reference/generated/numpy.array.html}{\lstinline+np.array(list)+}}}. Como parâmetro de entrada, recebe uma \lstinline+list+ contendo os elementos do arranjo. Por exemplo,
\begin{lstlisting}
>>> v = np.array([-2, 1, 3])
>>> v
array([-2,  1,  3])
>>> type(v)
<class 'numpy.ndarray'>
\end{lstlisting}
aloca o arranho de números inteiros \lstinline+v+. Embora arranjos não sejam vetores, \hl{a modelagem computacional de vetores usualmente é feita utilizando-se {\lstinline+arrays+}}. Por exemplo, em um código {\python}, o vetor
\begin{equation}
  \pmb{v} = (-2, 1, 3)
\end{equation}
pode ser alocado usando-se o \lstinline+array+ \lstinline+v+ acima.

O \hl{tipo dos dados} de um \lstinline+array+ é definido na sua criação. Pode ser feita de forma automática ou explícita pela propriedade \hl{{\href{https://numpy.org/doc/stable/reference/arrays.dtypes.html}{\lstinline+dtype+}}}. Por exemplo,
\begin{lstlisting}
>>> v = np.array([-2, 1, 3])
>>> v.dtype
dtype('int64')
>>> v = np.array([-2., 1, 3])
>>> v.dtype
dtype('float64')
>>> v = np.array([-2, 1, 3], dtype='float')
>>> v.dtype
dtype('float64')
\end{lstlisting}

\begin{ex}
  Aloque o vetor
  \begin{equation}
    \pmb{v} = (\pi, 1, e)
  \end{equation}
  como um \lstinline+array+ do {\numpy}.
\begin{lstlisting}
>>> import numpy as np
>>> v = np.array([np.pi, 1, np.e])
>>> v
array([3.14159265, 1.        , 2.71828183])
\end{lstlisting}
\end{ex}

O {\numpy} conta com métodos úteis para a \hl{\emph{inicialização} de {\lstinline+arrays+}}:
\begin{itemize}
\item \hl{{\lstinline+np.zeros()+}} : arranjo de elementos nulos.
\begin{lstlisting}
>>> np.zeros(3)
array([0., 0., 0.])
\end{lstlisting}
\item \hl{{\lstinline+np.ones()+}} : arranjo de elementos iguais a um.
\begin{lstlisting}
>>> np.ones(2, dtype='int')
array([1, 1])
\end{lstlisting}
\item \hl{{\lstinline+np.empty()+}} : arranjo de elementos não predefinidos.
\begin{lstlisting}
>>> np.empty(3)
array([4.64404327e-310, 0.00000000e+000, 6.93315702e-310])
\end{lstlisting}
\item \hl{{\lstinline+np.linspace(start, stop, num=50)+}} : arranjo de elementos uniformemente espaçados.
\begin{lstlisting}
>>> np.linspace(0, 1, 5)
array([0.  , 0.25, 0.5 , 0.75, 1.  ])
\end{lstlisting}
\end{itemize}

\subsection{Indexação e Fatiamento}

[[tag:construcao]]

\subsection{Operações e Funções Elemento-a-Elemento}

No {\numpy}, temos os \hl{operadores aritméticos elemento-a-elemento} (em ordem de precedência)
\begin{itemize}
\item \hl{{\lstinline!**!}}
\begin{lstlisting}
>>> v = np.array([-2., 1, 3])
>>> w = np.array([1., -1, 2])
>>> v ** w
array([-2.,  1.,  9.])
\end{lstlisting}
\item \hl{{\lstinline!*!}, {\lstinline!/!}, {\lstinline!//!}}, \lstinline!%!
\begin{lstlisting}
>>> v * w
array([-2., -1.,  6.])
>>> v / w
array([-2. , -1. ,  1.5])
>>> v // w
array([-2., -1.,  1.])
>>> v % w
array([ 0., -0.,  1.])
\end{lstlisting}
\item \hl{{\lstinline!+!}, {\lstinline!-!}}
\begin{lstlisting}
>>> v + w
array([-1.,  0.,  5.])
>>> v - w
array([-3.,  2.,  1.])
\end{lstlisting}
\end{itemize}

O {\numpy} também conta com várias funções matemáticas predefinidas, consulte
\begin{center}
  \url{https://numpy.org/doc/stable/reference/routines.math.html}
\end{center}
A aplicação dessas funções correm elemento-a-elemento do \lstinline+array+ de entrada.

\begin{ex}\normalfont{(\hl{Função Vetorial}.)}
  
  [[tag::construcao]]

\end{ex}

\subsection{Vetores e Operações}

[[tag:construcao]]

\subsection{Exercícios}

[[tag:construcao]]