% Este trabalho está licenciado sob a Licença Atribuição-CompartilhaIgual 4.0 Internacional Creative Commons. Para visualizar uma cópia desta licença, visite http://creativecommons.org/licenses/by-sa/4.0/deed.pt_BR ou mande uma carta para Creative Commons, PO Box 1866, Mountain View, CA 94042, USA.

\chapter{Orientação a Objetos}\label{cap_poo}
\thispagestyle{fancy}

Programação Orientação a Objetos (POO) é um paradigma de programação baseado no conceito de classes de objetos. A classe define seus atributos (propriedades e métodos) de seus objetos. Todos os objetos de uma classe têm os mesmos atributos, mas são independentes um dos outros, sendo que cada um é uma instância própria da classe contendo seus próprios valores de seus atributos.

\section{Classe e Objeto}\label{cap_ob_sec_class}

\hl{Uma \emph{classe} é uma forma de estrutura que permite a alocação conjunta de dados e funções}. Em {\python}, a sintaxe de definição de uma classe é
\begin{lstlisting}
class NomeDaClasse:
    <bloco-0>
    <bloco-1>
    ...
    <bloco-2>
\end{lstlisting}
Usualmente, \hl{os blocos de programação consistem de definições de funções (métodos)}. Por exemplo,
\begin{lstlisting}
class MinhaClasse:
    def digaOla(self):
        print('Olá, Mundo!')

obj = MinhaClasse()
obj.digaOla()
\end{lstlisting}
Neste código, temos a definição da classe \lstinline+MinhaClasse+ (linhas 1-3). Esta classe contém o método \lstinline+MinhaClasse.digaOla()+ (linhas 2-3). Obrigatoriamente, \hl{na definição de um método de uma classe deve conter o primeiro parâmetro {\lstinline+self+}}. Um objeto desta classe\footnote{Uma nova instância da classe.} e identificado por \lstinline+obj+ é alocado na linha 5. Na linha 6, este objeto chama seu método \lstinline+obj.digaOla()+.

O método especial \lstinline+__init__()+ é executado na construção de cada nova instância da classe (objeto da classe). Por exemplo,
\begin{lstlisting}
class Brasileira:
    pais = 'Brasil'
    def __init__(self, nome):
        self.nome = nome
        
    def digaOla(self):
        print('\nOlá!')
        print(f'Eu me chamo {self.nome}.')
        print(f'Sou do {self.pais}. :)')

x = Brasileira('Fulane')
x.digaOla()
y = Brasileira('Beltrane')
y.digaOla()
\end{lstlisting}
Aqui, o atributo \lstinline+Brasileira.pais+ é compartilhada entre todas as instâncias da classe (objetos), enquanto que \lstinline+Brasileira.nome+ é um atributo de cada objeto. O método \lstinline+__init()__+ (linhas 3-4) é executada no momento da criação de cada nova instância (linhas 11 e 13).

\begin{ex}
  No seguinte código, começamos a definição de uma classe para a manipulação de triângulos.
\begin{lstlisting}[caption=classTriangulo.py, label=cap_poo_sec_class:cod:classTriangulo]
import matplotlib.pyplot as plt

class Triangulo:
    '''
    Classe Triangulo ABC.
    '''
    num_lados = 3
    def __init__(self, A, B, C):
        # vértices
        self.A = A
        self.B = B
        self.C = C

    def plot(self):
        fig = plt.figure()
        ax = fig.add_subplot()
        # lados
        ax.plot([self.A[0], self.B[0]],
                [self.A[1], self.B[1]], marker='o', color='blue')
        ax.text((self.A[0]+self.B[0])/2,
                (self.A[1]+self.B[1])/2, 'c')
        ax.plot([self.B[0], self.C[0]],
                [self.B[1], self.C[1]], marker='o', color='blue')
        ax.text((self.B[0]+self.C[0])/2,
                (self.B[1]+self.C[1])/2, 'a')
        ax.plot([self.C[0], self.A[0]],
                [self.C[1], self.A[1]], marker='o', color='blue')
        ax.text((self.A[0]+self.C[0])/2,
                (self.A[1]+self.C[1])/2, 'b')
        # vertices
        ax.text(self.A[0], self.A[1], 'A')
        ax.text(self.B[0], self.B[1], 'B')
        ax.text(self.C[0], self.C[1], 'C')
        ax.grid()
        plt.show()

tria = Triangulo((0., 0.),
                 (2., 0.),
                 (1., 1.))
tria.plot()
\end{lstlisting}
\end{ex}

\subsection{Exercícios}

\begin{exer}
  Considere o Código~\ref{cap_poo_sec_class:cod:classTriangulo}. Adicione o método \lstinline+calcLados()+, que computa e aloca o comprimento de cada lado do triângulo.
\end{exer}
\begin{resp}
\begin{lstlisting}
import numpy as np
import matplotlib.pyplot as plt

class Triangulo:
    '''
    Classe Triangulo ABC.
    '''
    num_lados = 3
    def __init__(self, A, B, C):
        # vértices
        self.A = A
        self.B = B
        self.C = C
        # lados
        self.a = 0.
        self.b = 0.
        self.c = 0.

    def calcLados(self):
        self.a = np.sqrt((self.B[0]-self.C[0])**2\
                         + (self.B[1]-self.C[1])**2)
        self.b = np.sqrt((self.A[0]-self.C[0])**2\
                         + (self.A[1]-self.C[1])**2)
        self.c = np.sqrt((self.A[0]-self.B[0])**2\
                         + (self.A[1]-self.B[1])**2)
\end{lstlisting}
\end{resp}

\begin{exer}
  Considere o Código~\ref{cap_poo_sec_class:cod:classTriangulo}. Adicione o método \lstinline+calcPerimetro()+, que computa e retorna o valor do perímetro do triângulo.
\end{exer}
\begin{resp}
\begin{lstlisting}
import numpy as np
import matplotlib.pyplot as plt

class Triangulo:

    ...

    def perimetro(self):
        return self.a + self.b + self.c

    ...
\end{lstlisting}
\end{resp}

\begin{exer}
  Considere o Código~\ref{cap_poo_sec_class:cod:classTriangulo}. Adicione o método \lstinline+calcAngulos()+, que computa e aloca os ângulos do triângulo.
\end{exer}
\begin{resp}
  Dica: use a \href{https://pt.wikipedia.org/wiki/Lei_dos_cossenos}{Lei dos Cossenos}.
\end{resp}

\begin{exer}
  Considere o Código~\ref{cap_poo_sec_class:cod:classTriangulo}. Adicione o método \lstinline+area()+, que computa a área do triângulo.
\end{exer}
\begin{resp}
  Dica: use o \href{https://pt.wikipedia.org/wiki/Teorema_de_Her%C3%A3o}{Teorema de Herão}.
\end{resp}

\begin{exer}
  Similar a classe \lstinline+Triangulo+ (Código~\ref{cap_poo_sec_class:cod:classTriangulo}), implemente uma nova classe \lstinline+Quadrilateros+ com as seguintes propriedades e métodos de quadriláteros $ABCD$:
  \begin{enumerate}[a)]
  \item vértices (\lstinline+tuples+).
  \item lados (\lstinline+floats+).
  \item cálculo do perímetro (método).
  \item cálculo da área (método).
  \item visualização gráfica (método \textit+plot+).
  \end{enumerate}
\end{exer}

\begin{exer}
  Implemente uma classe para a manipulação de polinômios de segundo grau. A classe deve conter as seguintes propriedades e métodos:
  \begin{enumerate}[a)]
  \item coeficientes (\lstinline+floats+).
  \item cálculo do ponto de interseção com o eixo y (método).
  \item cálculo do vértice da parábola associada ao polinômio (método).
  \item cálculo das raízes do polinômio (método).
  \item plotagem do gráfico do polinômio (método).
  \end{enumerate}
\end{exer}
\begin{resp}
  Dica: utilize a notação $p(x) = ax^2 + bx + c$.
\end{resp}
  

