

\chapter*{Prefácio}\label{prefacio}
\addcontentsline{toc}{chapter}{Prefácio}

O site \href{https://www.notaspedrok.com.br}{notaspedrok.com.br} é uma plataforma que construí para o compartilhamento de minhas notas de aula. Essas anotações feitas como preparação de aulas é uma prática comum de professoras/es. Muitas vezes feitas a rabiscos em rascunhos com validade tão curta quanto o momento em que são concebidas, outras vezes, com capricho de um diário guardado a sete chaves. Notas de aula também são feitas por estudantes - são anotações, fotos, prints, entre outras formas de registros de partes dessas mesmas aulas. Essa dispersão de material didático sempre me intrigou e foi o que me motivou a iniciar o site.

Com início em 2018, o site contava com apenas três notas incipientes. De lá para cá, conforme fui expandido e revisando os materais, o site foi ganhando acessos de vários locais do mundo, em especial, de países de língua portugusa. No momento, conta com 13 notas de aula, além de minicursos e uma coleção de vídeos e áudios.

As notas de \emph{Matemática Numérica Paralela} abordam tópicos introdutórios de computação paralela. Com enfase em métodos numéricos, faz-se uma apresentação de programação de multiprocessamento (com OpenMP) e de programação paralela distribuída (com OpenMPI). Códigos exemplos são trabalhos em linguagem C/C++.

Aproveito para agradecer a todas/os que de forma assídua ou esporádica contribuem com correções, sugestões e críticas! ;)

\begin{flushright}
  Pedro H A Konzen

  \url{https://www.notaspedrok.com.br}
\end{flushright}