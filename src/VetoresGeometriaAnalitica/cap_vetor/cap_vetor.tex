%Este trabalho está licenciado sob a Licença Atribuição-CompartilhaIgual 4.0 Internacional Creative Commons. Para visualizar uma cópia desta licença, visite http://creativecommons.org/licenses/by-sa/4.0/deed.pt_BR ou mande uma carta para Creative Commons, PO Box 1866, Mountain View, CA 94042, USA.

\chapter{Vetores}\label{cap_vetor}
\thispagestyle{fancy}

\section{Segmentos orientados}\label{cap_vetor_sec_segorien}

Sejam dois pontos $A$ e $B$ sobre uma reta $r$. O conjunto de todos os pontos de $r$ entre $A$ e $B$ é chamado de {\bf segmento}\index{segmento} $AB$.

\begin{figure}[h!]
  \centering
  \includegraphics[width=0.7\textwidth]{./cap_vetor/dados/fig_segmento/fig_segmento}
  \caption{Esboço de um segmento $AB$.}
  \label{fig:segmento}
\end{figure}

Associado a um segmento $AB$, temos seu {\bf comprimento}\index{comprimento} (ou tamanho), o qual é definido como sendo a {\bf distância}\index{distância} entre os pontos $A$ e $B$. A distância entre os ponto $A$ e $B$ é denotada por $|AB|$ ou $|BA|$.

A {\bf direção} de um segmento $AB$ é a direção da reta que fica determinada pelos pontos $A$ e $B$.

\begin{ex}\label{ex:segmento}
  Consideremos os segmentos esboçados na Figura \ref{fig:ex_segmento}. Os segmentos $AB$ e $CD$ têm as mesmas direções, mas comprimentos diferentes. Já, o segmento $EF$ tem direção diferente dos segmentos $AB$ e $CD$.
  
  \begin{figure}[h!]
    \centering
    \includegraphics[width=0.7\textwidth]{./cap_vetor/dados/fig_ex_segmento/fig_ex_segmento}
  \caption{Esboço referente ao Exemplo \ref{ex:segmento}.}
  \label{fig:ex_segmento}
\end{figure}
\end{ex}

Se $A$ e $B$ são o mesmo ponto, então chamamos $AB$ de {\bf segmento nulo}\index{segmento nulo} e temos $|AB| = 0$. Um segmento nulo não tem direção.

\begin{figure}[h!]
  \centering
  \includegraphics[width=0.7\textwidth]{./cap_vetor/dados/fig_seg_orientado/fig_seg_orientado}
  \caption{Esboço de um segmento orientado $AB$.}
  \label{fig:seg_orientado}
\end{figure}


Observemos que um dado segmento $AB$ é igual ao segmento $BA$. Agora, podemos associar a noção de {\bf sentido} a um segmento, escolhendo um dos pontos como sua {\bf origem}\index{origem} e o outro como sua {\bf extremidade}\index{extremidade}. Ao fazermos isso, definimos um {\bf segmento orientado}\index{segmento orientado}. Mais precisamente, um segmento orientado $AB$ é o segmento definido pelos pontos $A$ e $B$, sendo $A$ a origem e $B$ a extremidade. Veja a Figura \ref{fig:seg_orientado}.

Dizemos que dois dados segmentos orientados não nulos $AB$ e $CD$ têm a {\bf mesma direção} quando as retas $AB$ e $CD$ forem paralelas ou coincidentes.

\begin{ex}\label{ex:segorien_direcao}
  Consideremos os segmentos orientados esboçados na Figura \ref{fig:ex_segorien_direcao}. Observemos que os segmentos orientados $AB$ e $CD$ têm a mesma direção. Já o segmento orientado $EF$ tem direção diferente dos segmentos $AB$ e $CD$.
  
  \begin{figure}[h!]
    \centering
    \includegraphics[width=0.7\textwidth]{./cap_vetor/dados/fig_ex_segorien_direcao/fig_ex_segorien_direcao}
  \caption{Esboço referente ao Exemplo \ref{ex:segorien_direcao}.}
  \label{fig:ex_segorien_direcao}
\end{figure}  
\end{ex}

Sejam dados dois segmentos orientados $AB$ e $CD$ de mesma direção, cujas retas $AB$ e $CD$ não sejam coincidentes. Então, as retas $AB$ e $CD$ determinam um único plano e a reta $AC$ determina dois semiplanos (veja a Figura \ref{fig:segorien_sentido}). Assim sendo, dizemos que os segmentos $AB$ e $CD$ têm {\bf mesmo sentido}\index{mesmo sentido} quando os pontos $B$ e $D$ estão ambos sobre o mesmo semiplano.

\begin{figure}[h!]
  \centering
  \includegraphics[width=0.7\textwidth]{./cap_vetor/dados/fig_segorien_sentido/fig_segorien_sentido}
  \caption{Esboço de dois segmentos orientados $AB$ e $CD$ de mesmo sentido.}
  \label{fig:segorien_sentido}
\end{figure}

Para analisar o sentido de dois segmentos orientados e colineares, escolhemos um deles e construímos um segmento orientado de mesmo sentido a este, mas não colinear. Então, analisamos o sentido dos segmentos orientados originais com respeito ao introduzido.

Dois segmentos orientados não nulos são {\bf equipolentes}\index{equipolentes} quando eles têm o mesmo comprimento, mesma direção e mesmo sentido. Veja o exemplo dado na Figura \ref{fig:segequipolentes}.

\begin{figure}[h!]
  \centering
  \includegraphics[width=0.7\textwidth]{./cap_vetor/dados/fig_segequipolentes/fig_segequipolentes}
  \caption{Esboço de dois segmentos orientados $AB$ e $CD$ equipolentes.}
  \label{fig:segequipolentes}
\end{figure}

\subsection{Exercícios}

\begin{exer}
  Mostre que dois segmentos orientados $AB$ e $CD$ são equipolentes se, e somente se, os pontos médios de $AD$ e $BC$ são coincidentes.
\end{exer}
\begin{resp}
  Propriedades de congruência entre ângulos determinados por retas paralelas cortadas por uma transversal e congruência entre triângulos provam o enunciado.
\end{resp}

\section{Vetores}\label{cap_vetor_sec_vetor}

Dado um segmento orientado $AB$, chama-se {\bf vetor} $AB$ e denota-se $\overrightarrow{AB}$, qualquer segmento orientado equipolente a $AB$. Cada segmento orientado equipolente a $AB$ é um representado de $\overrightarrow{AB}$. A Figura \ref{fig:vetor} mostra duas representações de um dado vetor $\overrightarrow{AB}$.

\begin{figure}[h!]
  \centering
  \includegraphics[width=0.7\textwidth]{./cap_vetor/dados/fig_vetor/fig_vetor}
  \caption{Esboço de duas representações de um mesmo vetor.}
  \label{fig:vetor}
\end{figure}

O {\bf módulo}\index{módulo} (ou {\bf norma}\index{norma}) de um vetor $\overrightarrow{AB}$ é o valor de seu comprimento e é denotado por $|\overrightarrow{AB}|$.

Dois {\bf vetores} são ditos {\bf paralelos} \index{vetores!paralelos} quando qualquer de suas representações têm a mesma direção. De forma análoga, definem-se {\bf vetores coplanares}\index{vetores!coplanares}, {\bf vetores não coplanares}\index{vetores!não coplanares}, {\bf vetores ortogonais}\index{vetores!ortogonais}, além de conceitos como {\bf ângulo entre dois vetores}\index{ângulo!entre vetores}, etc. Veja a Figura \ref{fig:vetorrel}.

\begin{figure}[h!]
  \centering
  \includegraphics[width=0.5\textwidth]{./cap_vetor/dados/fig_vetorrel/fig_vetorrel}~
  \includegraphics[width=0.5\textwidth]{./cap_vetor/dados/fig_vcolineares/fig_vcolineares}
  \caption{Esquerda: esboços de vetores paralelos e de vetores ortogonais. Direita: esboços de vetores coplanares.}
  \label{fig:vetorrel}
\end{figure}

Observemos que na Figura \ref{fig:vetorrel}(direita) os vetores foram denotados por $\vec{a}$, $\vec{b}$ e $\vec{c}$, sem alusão aos pontos que definem suas representações como segmentos orientados. Isto é costumeiro, devido a definição de vetor.

\subsection{Adição de vetores}

Sejam dados dois vetores $\vec{u}$ e $\vec{v}$. Sejam, ainda, uma representação $\overrightarrow{AB}$ qualquer de $u$ e a representação $\overrightarrow{BC}$ do vetor $\vec{v}$. Então, define-se o vetor soma $\vec{u}+\vec{v}$ como o vetor dado por $\overrightarrow{AC}$. Veja a Figura \ref{fig:vadicao}.

\begin{figure}[h!]
  \centering
  \includegraphics[width=0.7\textwidth]{./cap_vetor/dados/fig_vadicao/fig_vadicao}
  \caption{Representação geométrica da adição de dois vetores.}
  \label{fig:vadicao}
\end{figure}

\subsection{Vetor oposto}

Um \pmb{vetor} $\vec{v}$ é dito ser \pmb{oposto} \index{vetor!oposto} a um dado vetor $\vec{u}$, quando quaisquer representações de $\vec{u}$ e $\vec{v}$ são segmentos orientados de mesmo comprimento e mesma direção, mas com sentidos opostos. Neste caso, denota-se por $-\vec{u}$ o vetor oposto a $\vec{u}$. Veja a Figura \ref{fig:voposto}.

\begin{figure}[h!]
  \centering
  \includegraphics[width=0.7\textwidth]{./cap_vetor/dados/fig_voposto/fig_voposto}
  \caption{Representação geométrica de vetores opostos.}
  \label{fig:voposto}
\end{figure}

\subsection{Subtração de vetores}

Sejam dados dois vetores $\vec{u}$ e $\vec{v}$. A subtração de $\vec{u}$ com $\vec{v}$ é denotada por $\vec{u}-\vec{v}$ e é definida pela adição de $\vec{u}$ com $-\vec{v}$, i.e. $\vec{u}-\vec{v}=\vec{u}+(-\vec{v})$. Veja a Figura \ref{fig:vsubtracao}.

\begin{figure}[h!]
  \centering
  \includegraphics[width=0.7\textwidth]{./cap_vetor/dados/fig_vsubtracao/fig_vsubtracao}
  \caption{Representação geométrica da subtração de $\vec{u}$ com $\vec{v}$, i.e. $\vec{u}-\vec{v}$.}
  \label{fig:vsubtracao}
\end{figure}

\subsection{Multiplicação de vetor por um escalar}

A multiplicação de um número real $\alpha>0$ (escalar) por um vetor $\vec{u}$ é denotado por $\alpha\vec{u}$ e é definido pelo vetor de mesma direção e mesmo sentido de $\vec{u}$ com norma $\alpha|\vec{u}|$. Quando $\alpha = 0$, define-se $\alpha\vec{u}=\vec{0}$, i.e. o vetor nulo (geometricamente, representado por qualquer ponto).

\begin{obs}
  \begin{itemize}
  \item Para $\alpha<0$, temos $\alpha\vec{u} = -(-\alpha\vec{u})$.
  \item $|\alpha\vec{u}|=|\alpha||\vec{u}|$.
\end{itemize}
\end{obs}

\begin{figure}[h!]
  \centering
  \includegraphics[width=0.7\textwidth]{./cap_vetor/dados/fig_vescalar/fig_vescalar}
  \caption{Representações geométricas de multiplicações de um vetor por diferentes escalares.}
  \label{fig:vescalar}
\end{figure}

\subsection{Propriedades das operações com vetores}

As operações de adição e multiplicação por escalar de vetores têm propriedades importantes. Para quaisquer vetores $\vec{u}$, $\vec{v}$ e $\vec{w}$ e quaisquer escalares $\alpha$ e $\beta$ temos:
\begin{itemize}
\item comutatividade da adição: $\vec{u}+\vec{v}=\vec{v}+\vec{u}$;
\item associatividade da adição: $(\vec{u} + \vec{v}) + \vec{w} = \vec{u} + (\vec{v} + \vec{w})$;
\item elemento neutro da adição: $\vec{u}+\vec{0}=\vec{u}$;
\item existência do oposto: $\vec{u}+(-\vec{u}) = \vec{0}$;
\item associatividade da multiplicação por escalar: $\alpha(\beta\vec{u})=(\alpha\beta)\vec{u}$;
\item distributividade da multiplicação por escalar:
  \begin{align}
    &\alpha(\vec{u}+\vec{v}) = \alpha\vec{u}+\alpha\vec{v},\\
    &(\alpha+\beta)\vec{u} = \alpha\vec{u}+\beta\vec{u};
  \end{align}
\item existência do elemento neutro da multiplicação por escalar: $1\vec{u}=\vec{u}$.
\end{itemize}

\subsection*{Exercícios}

\begin{exer}\label{exer:vetor_prob_01}
  Na figura abaixo, temos $\vec{u} = \overrightarrow{GJ}$ e $\vec{v} = \overrightarrow{AK}$. Assim sendo, escreva os vetores $\overrightarrow{RS}$, $\overrightarrow{NI}$, $\overrightarrow{AG}$, $\overrightarrow{NQ}$, $\overrightarrow{AT}$ e $\overrightarrow{PE}$ em função de $\vec{u}$ e $\vec{v}$.

  \includegraphics[width=0.7\textwidth]{./cap_vetor/dados/fig_exer_prob_01/fig_exer_prob_01}
\end{exer}

\begin{exer}\label{exer:vetor_prob_02}
  Sejam $\overrightarrow{CA}$, $\overrightarrow{CM}$ e $\overrightarrow{CB}$ os vetores indicados na figura abaixo. Mostre que $\overrightarrow{CM} = \frac{1}{2}\overrightarrow{CA} + \frac{1}{2}\overrightarrow{CB}$.

  \includegraphics[width=0.7\textwidth]{./cap_vetor/dados/fig_exer_prob_02/fig_exer_prob_02}
\end{exer}

\begin{exer}\label{exer:vetor_prob_03}
  Sejam $A$, $B$, $C$, $D$, $E$ e $G$ os pontos dados na figura abaixo. Escreva o vetor $\overrightarrow{DG}$ em função dos vetores $\overrightarrow{AB}$ e $\overrightarrow{AD}$.

  \includegraphics[width=0.7\textwidth]{./cap_vetor/dados/fig_exer_prob_03/fig_exer_prob_03}
\end{exer}

\begin{exer}
  Seja dado um vetor $\vec{u}\neq 0$. Calcule a norma do vetor $\vec{v}=\vec{u}/|\vec{u}|$\footnote{$\vec{u}/|\vec{u}|$ é chamado de vetor $\vec{u}$ normalizado, ou a normalização do vetor $\vec{u}$.}.
\end{exer}
\begin{resp}
  $|\vec{v}|=1$.
\end{resp}
