%%%%%%%%%%%%%%%%%%%%%%%%%%%%%%%%%
%   Predefinicoes
%%%%%%%%%%%%%%%%%%%%%%%%%%%%%%%%%

\newif\ifisbook         % book?
\newif\ifishtml         % html?
\newif\ifisdraft          % draft?
\newif\ifismini         % mini?

\newif\ifisoctave       % Códigos em octave?
\newif\ifispython       % Códigos em python
\newif\ifismaxima       % Códigos em maxima?
\newif\ifiscc           % Códigos em C/C++?

\def\tfn{config.knd}     % Arquivo que guarda as definições do tipo de saída
\def \tdata{}          % Definições do tipo de saída: book, slide ou html.

\openin1=\tfn\relax    % Leitura das definições de saída
\read1 to \tdata
\closein1

\tdata                 % Definições de saída

%%%%%%%%%%%%%%%%%%%%%%%%%%%%%%%%%
%   Opcões de Linguagem
%%%%%%%%%%%%%%%%%%%%%%%%%%%%%%%%%
\usepackage[latin1,utf8]{inputenc}
\usepackage[T1]{fontenc}
\usepackage[portuguese]{babel}

% %%%%%%%%%%%%%%%%%%%%%%%%%%%%%%%%%
% %   Bibliografia
% %%%%%%%%%%%%%%%%%%%%%%%%%%%%%%%%%
% \bibliographystyle{abbrv}


%%%% copy and paste from PDF (correctly) %%%%
\usepackage{lmodern}
\usepackage{textcomp}


% %%%%%%%%%%%%%%%%%%%%%%%%%%%%%%%%%
% %   Parágrafos e indentação
% %%%%%%%%%%%%%%%%%%%%%%%%%%%%%%%%%
% \setlength{\parindent}{0pt}
% \nonzeroparskip
\usepackage[skip=1em, indent=0em]{parskip}


%%%% ams-latex %%%%
\usepackage{amsmath}
% \usepackage{amsmath}
\usepackage{amssymb}
\usepackage{amsthm}

%%% quotes %%%
\usepackage{upquote}

%%% bold symbols %%%
\usepackage{bm}

%%% landscape environment%%%
\usepackage{lscape}

%%%% float H option%%%%
\usepackage{float}

%%%% graphics %%%%
\usepackage[export]{adjustbox}
\usepackage{graphicx}
\usepackage{caption}
\usepackage{subcaption}

%%%% links %%%%
% \usepackage[hyphens,spaces,obeyspaces]{url}
\usepackage{xurl}
\usepackage[pdfborder={0 0 0 [0 0]},colorlinks=true,linkcolor=blue,citecolor=blue,filecolor=blue,urlcolor=blue,breaklinks=true]{hyperref}

%%% notas %%%
\usepackage{endnotes}
\renewcommand{\notesname}{Notas}


% colors
\usepackage[dvipsnames]{xcolor}
%%%% code insert (verbatim) %%%%
\usepackage{verbatim}
\definecolor{backcolour}{rgb}{0.969, 0.969, 0.969}
\usepackage{listings}
\renewcommand{\lstlistingname}{Código}% Listing -> Código
\renewcommand{\lstlistlistingname}{Lista de \lstlistingname s}% List of Listings -> Lista of Códigos
%%% listing %%%
\lstset { %
  belowskip=0em,
  backgroundcolor=\color{backcolour},
  keywordstyle=\color{blue},
  linewidth=\textwidth,
  numbers=left,
  numberstyle=\fontsize{9}{9}\color{gray}\ttfamily,
  numbersep=3pt,
  stepnumber=1,
  firstnumber=1,
  xleftmargin=1em,
  framexleftmargin=0in,
  framexrightmargin=0in,
  % frame=single,
  extendedchars=true,
  inputencoding=utf8,
  basicstyle=\ttfamily,
  commentstyle=\ttfamily\itshape\color{gray},
  stringstyle=\ttfamily,
  showspaces=false,
  showstringspaces=false,
  resetmargins=true,
  breaklines=true,
  breakindent=0pt,             
  keepspaces=true,
  upquote=true,
  literate      =        % Support additional characters
      {á}{{\'a}}1  {é}{{\'e}}1  {í}{{\'i}}1 {ó}{{\'o}}1  {ú}{{\'u}}1
      {Á}{{\'A}}1  {É}{{\'E}}1  {Í}{{\'I}}1 {Ó}{{\'O}}1  {Ú}{{\'U}}1
      {à}{{\`a}}1  {è}{{\`e}}1  {ì}{{\`i}}1 {ò}{{\`o}}1  {ù}{{\`u}}1
      {À}{{\`A}}1  {È}{{\'E}}1  {Ì}{{\`I}}1 {Ò}{{\`O}}1  {Ù}{{\`U}}1
      {ä}{{\"a}}1  {ë}{{\"e}}1  {ï}{{\"i}}1 {ö}{{\"o}}1  {ü}{{\"u}}1
      {Ä}{{\"A}}1  {Ë}{{\"E}}1  {Ï}{{\"I}}1 {Ö}{{\"O}}1  {Ü}{{\"U}}1
      {â}{{\^a}}1  {ê}{{\^e}}1  {î}{{\^i}}1 {ô}{{\^o}}1  {û}{{\^u}}1
      {Â}{{\^A}}1  {Ê}{{\^E}}1  {Î}{{\^I}}1 {Ô}{{\^O}}1  {Û}{{\^U}}1
      {œ}{{\oe}}1  {Œ}{{\OE}}1  {æ}{{\ae}}1 {Æ}{{\AE}}1  {ß}{{\ss}}1
      {ç}{{\c c}}1 {Ç}{{\c C}}1 {ø}{{\o}}1  {Ø}{{\O}}1   {å}{{\r a}}1
      {Å}{{\r A}}1 {ã}{{\~a}}1  {õ}{{\~o}}1 {Ã}{{\~A}}1  {Õ}{{\~O}}1
      {ñ}{{\~n}}1  {Ñ}{{\~N}}1  {¿}{{?`}}1  {¡}{{!`}}1
      {°}{{\textdegree}}1 {º}{{\textordmasculine}}1 {ª}{{\textordfeminine}}1
      % acentuação somente funciona com \lstlisting, não deve-se usar
      % \lstinputlisting      
    }


%%%% citation %%%%
\usepackage{cite}

%%%% lists %%%%
\usepackage{enumerate}

%%%% index %%%%
\usepackage{imakeidx}

%%%% miscellaneous %%%%
\usepackage{multicol}
\usepackage{multirow}
\usepackage[normalem]{ulem}
\usepackage{cancel}
% \usepackage{soulutf8}

% tables
\usepackage{booktabs}

%%%%%%%%%%%%%%%%%%%%%%%%%%%%%%%%%%%%%%%%%%%%%%%%%%
% MACROS E NOVOS COMANDOS
%%%%%%%%%%%%%%%%%%%%%%%%%%%%%%%%%%%%%%%%%%%%%%%%%%

% highlight & emphasis
\usepackage{soulutf8}
\renewcommand{\emph}{\textbf}
\newcommand{\hleq}[1]{\color{blue}#1}
%\DeclareTextFontCommand{\emph}{\bfseries}

\newcommand{\hlemph}[1]{\hl{\textbf{#1}}}

% \usepackage{realboxes}
% \newcommand{\hlcode}[1]{\Colorbox{yellow}{#1}}

% math funs & ops
\newcommand{\arc}{\operatorname{arc}}
\newcommand{\arctg}{\operatorname{arctg}}
\newcommand{\cotg}{\operatorname{cotg}}
\newcommand{\cosec}{\operatorname{cosec}}
\newcommand{\cossec}{\operatorname{cossec}}
\newcommand{\dist}{\operatorname{dist}}
\newcommand{\ddiv}{\operatorname{div}}
\newcommand{\mmc}{\operatorname{mmc}}

\newcommand{\p}{\partial}
\newcommand{\dd}{\mathrm{d}}

% cross product (produto vetorial)
\newcommand{\cross}{\times}

\newcommand{\Dom}{\operatorname{Dom}}
\newcommand{\diag}{\operatorname{diag}}
\newcommand{\proj}{\operatorname{proj}}
\newcommand{\rank}{\operatorname{rank}}
\newcommand{\sign}{\operatorname{sign}}
\newcommand{\spn}{\operatorname{span}}
\newcommand{\sen}{\operatorname{sen}}
\newcommand{\senh}{\operatorname{senh}}

\newcommand{\tg}{\operatorname{tg}}
\newcommand{\tr}{\operatorname{tr}}

\newcommand{\elu}{\operatorname{elu}}
\newcommand{\sigmoid}{\operatorname{sigmoid}}

% FENiCS link
\newcommand{\fenics}{\href{https://fenicsproject.org/}{FEniCS}}

% FEniCSx link
\newcommand{\fenicsx}{\href{https://fenicsproject.org/}{FEniCSx}}

% GSL link
\newcommand{\gsl}{\href{https://www.gnu.org/software/gsl/}{GSL}}

% Geogebra link
\newcommand{\geogebra}{\href{https://www.geogebra.org/}{Geogebra}}

% Google
\newcommand{\google}{\href{https://www.google.com.br}{Google}}

% Google Colab
\newcommand{\colab}{\href{https://colab.google.com}{Google Colab}}

% Jupyter
\newcommand{\jupyter}{\href{https://jupyter.org/}{Jupyter}}

% Kaggle
\newcommand{\kaggle}{\href{https://www.kaggle.com/}{Kaggle}}

% Linux
\newcommand{\linux}{\href{https://www.kernel.org/}{Linux}}

% Matplotlib
\newcommand{\matplotlib}{\href{https://matplotlib.org/}{Matplotlib}}

% NumPy link
\newcommand{\numpy}{\href{https://numpy.org/}{NumPy}}

% OpenMP
\newcommand{\omp}{\href{https://www.openmp.org/}{OpenMP}}

% OpenMPI
\newcommand{\ompi}{\href{https://www.open-mpi.org/}{Open MPI}}

% Python link
\newcommand{\python}{\href{https://www.python.org}{Python}}

% PyTorch link
\newcommand{\pytorch}{\href{https://pytorch.org/}{PyTorch}}

% SciPy link
\newcommand{\scipy}{\href{https://scipy.org/}{SciPy}}

% SymPy link
\newcommand{\sympy}{\href{https://www.sympy.org}{SymPy}}

% Spyder link
\newcommand{\spyder}{\href{https://www.spyder-ide.org/}{Spyder}}


%E = 10^
\def\E#1{\mathrm{e}\!#1\!}
\def\e#1{\mathrm{e}\!#1\!}
%NaN
\def\NaN{\mathrm{NaN}\!}

%%%%%%%%%%%%%%%%%%%%%%%%%%%%%%%%%%%%%%%%%%%%%%%%%%

\newcommand{\emconstrucao}{\vspace{0.25cm}Em construção ...\vspace{0.25cm}}

% %%%%%%%%%%%%%%%%%%%%%%%%%%%%%%%%%%%%%%%%%%%%%%%%%%
% \theoremstyle{plain}
% \ifismini
% \newtheorem{teo}{Teorema}[subsection]
% \newtheorem{teorema}{Teorema}[subsection]
% \newtheorem{lem}{Lema}[subsection]
% \newtheorem{lema}{Lema}[subsection]
% \newtheorem{prop}{Proposição}[subsection]
% \newtheorem{proposicao}{Proposição}[subsection]
% \newtheorem{corol}{Corolário}[subsection]
% \newtheorem{corolario}{Corolário}[subsection]
% \theoremstyle{definition}
% \newtheorem{defn}{Definição}[subsection]
% \newtheorem{definicao}{Definição}[subsection]
% \newtheorem{obs}{Observação}[subsection]
% \newtheorem{observacao}{Observação}[subsection]

% \newtheorem{ex}{Exemplo}[subsection]

% \newenvironment{dem}{\begin{proof}}{\end{proof}}
% \newenvironment{demonstracao}{\begin{proof}}{\end{proof}}


% % exercícios para minicursos (document class article)
% \newtheorem{exr}{Exercício}[subsection]

% %Exercícios Resolvidos

% \newtheorem{exeresol}{ER}[subsection]

% \else

% \newtheorem{teo}{Teorema}[section]
% \newtheorem{teorema}{Teorema}[section]
% \newtheorem{lem}{Lema}[section]
% \newtheorem{lema}{Lema}[section]
% \newtheorem{prop}{Proposição}[section]
% \newtheorem{proposicao}{Proposição}[section]
% \newtheorem{corol}{Corolário}[section]
% \newtheorem{corolario}{Corolário}[section]
% \theoremstyle{definition}
% \newtheorem{defn}{Definição}[section]
% \newtheorem{definicao}{Definição}[section]
% \newtheorem{obs}{Observação}[section]
% \newtheorem{observacao}{Observação}[section]

% \newtheorem{ex}{Exemplo}[section]

% \newenvironment{dem}{\begin{proof}}{\end{proof}}
% \newenvironment{demonstracao}{\begin{proof}}{\end{proof}}


% % exercícios para minicursos (document class article)
% \newtheorem{exr}{Exercício}[subsection]

% %Exercícios Resolvidos

% \newtheorem{exeresol}{ER}[section]

% \newcommand{\exerref}[1]{E.\ref{#1}}
% \newcommand{\exeresolref}[1]{ER.\ref{#1}}
% \fi


% giants infos
\input ../preambulo_giants.tex

% % python links
% \input ../preambulo_python.tex


%%%%%%%%%%%%%%%%%%%%%%%%%%%%%%%%%%%%%%%%%%%%%%%%%%
% + INTRUCOES PARA O FORMATO PDF
%%%%%%%%%%%%%%%%%%%%%%%%%%%%%%%%%%%%%%%%%%%%%%%%%%
\ifisbook
\input ../preambulo_book.tex
\fi
%%%%%%%%%%%%%%%%%%%%%%%%%%%%%%%%%%%%%%%%%%%%%%%%%%

%%%%%%%%%%%%%%%%%%%%%%%%%%%%%%%%%%%%%%%%%%%%%%%%%%
% + INTRUCOES PARA O FORMATO HTML
%%%%%%%%%%%%%%%%%%%%%%%%%%%%%%%%%%%%%%%%%%%%%%%%%%
\ifishtml
\input ../preambulo_html.tex
\fi
%%%%%%%%%%%%%%%%%%%%%%%%%%%%%%%%%%%%%%%%%%%%%%%%%%