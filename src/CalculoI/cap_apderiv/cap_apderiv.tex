%Este trabalho está licenciado sob a Licença Atribuição-CompartilhaIgual 4.0 Internacional Creative Commons. Para visualizar uma cópia desta licença, visite http://creativecommons.org/licenses/by-sa/4.0/deed.pt_BR ou mande uma carta para Creative Commons, PO Box 1866, Mountain View, CA 94042, USA.

\chapter{Aplicações da derivada}\label{cap_apderiv}
\thispagestyle{fancy}

\begin{flushright}
  [Vídeo] | [Áudio] | \href{https://phkonzen.github.io/notas/contato.html}{[Contatar]}
\end{flushright}

\ifispython
\begin{obs}\label{obs:cap_apderiv_python}
  Nos códigos \sympy~ apresentados neste capítulo, assumimos o seguinte preâmbulo:
\begin{verbatim}
from sympy import *
var('x',real=True)
\end{verbatim}
\end{obs}
\fi

\section{Regra de L'Hôpital}\label{cap_apderiv_sec_lhospital}

\begin{flushright}
  [Vídeo] | [Áudio] | \href{https://phkonzen.github.io/notas/contato.html}{[Contatar]}
\end{flushright}

A regra de L'Hôpital é uma técnica para o cálculo de limites de indeterminações. Sejam $f$ e $g$ funções deriváveis em um intervalo aberto contendo $x=a$, exceto possivelmente em $x=a$, e
\begin{equation}
  \lim_{x\to a} f(x) = 0\quad\text{e}\quad\lim_{x\to a} g(x) = 0.
\end{equation}
Se, ainda, $\lim_{x\to a} f(x)/g(x)$ existe ou for $\pm\infty$, então
\begin{equation}
  \lim_{x\to a} \frac{f(x)}{g(x)} = \lim_{x\to a} \frac{f'(x)}{g'(x)}.
\end{equation}
Esta é a versão da regra de L'Hôpital para indeterminações do tipo $0/0$. Sem grandes modificações, é diretamente estendida para os casos $x\to a^-$, $x\to a^+$, $x\to \infty$ e $x\to -\infty$.

\begin{ex}
  Vamos calcular o limite
  \begin{equation}
    \lim_{x\to 1} \frac{x-1}{x^2-1}.
  \end{equation}
  \begin{enumerate}[a)]
  \item Pela regra de L'Hôpital.
    \begin{align}
      \lim_{x\to 1} \frac{x-1}{x^2-1} &= \lim_{x\to 1} \frac{(x+1)'}{(x^2-1)'} \\
                                      &= \lim_{x\to 1} \frac{1}{2x} \\
                                      &= \frac{1}{2}.
    \end{align}
  \item Por eliminação do fator comum.
    \begin{align}
      \lim_{x\to 1} \frac{x-1}{x^2-1} &= \lim_{x\to 1} \frac{x-1}{(x-1)(x+1)} \\
                                      &= \lim_{x\to 1} \frac{1}{x+1} \\
                                      &= \frac{1}{2}.
    \end{align}
  \end{enumerate}

  \ifispython
  No \sympy\footnote{Veja a Observação \ref{obs:cap_apderiv_python}.}, temos
\begin{verbatim}
>>> limit((x-1)/(x**2-1),x,1)
1/2
\end{verbatim}
  \fi
\end{ex}

\begin{ex}
  O limite
  \begin{equation}
    \lim_{x\to 2} \frac{x^2-4x+4}{x^3-3x^2+4}
  \end{equation}
  é uma indeterminação $0/0$. Aplicando a regra de L'Hôpital, obtemos
  \begin{equation}
    \lim_{x\to 2} \frac{x^2-4x+4}{x^3-3x^2+4} = \lim_{x\to 2} \frac{\cancelto{0}{2x-4}}{\cancelto{0}{3x^2-6x}},
  \end{equation}
  que também é uma indeterminação do tipo $0/0$. Agora, aplicando a regra de L'Hôpital novamente, obtemos
  \begin{equation}
    \lim_{x\to 2} \frac{2x-4}{3x^2-6x} = \lim_{x\to 2} \frac{2}{6x-6} = \frac{1}{3}.
  \end{equation}
  Portanto, concluímos que
  \begin{equation}
    \lim_{x\to 2} \frac{x^2-4x+4}{x^3-3x^2+4} = \frac{1}{3}.
  \end{equation}  

  \ifispython
  No \sympy\footnote{Veja a Observação \ref{obs:cap_apderiv_python}.}, temos
\begin{verbatim}
>>> limit((x**2-4*x+4)/(x**3-3*x**2+3),x,2)
1/3
\end{verbatim}
  \fi
\end{ex}

\begin{obs}
  A regra de L'Hôpital também pode ser usada para indeterminações do tipo $\infty/\infty$.
\end{obs}

\begin{ex}
  Vamos calcular
  \begin{equation}
    \lim_{x\to \infty} \frac{e^x}{x},
  \end{equation}
  que é uma indeterminação do tipo $\infty/\infty$. Então, aplicando a regra de L'Hôpital, temos
  \begin{equation}
    \lim_{x\to \infty} \frac{e^x}{x} = \lim_{x\to \infty} \frac{e^x}{1} = \infty.
  \end{equation}
\end{ex}

\subsection*{Exercícios resolvidos}

\begin{flushright}
  [Vídeo] | [Áudio] | \href{https://phkonzen.github.io/notas/contato.html}{[Contatar]}
\end{flushright}

\begin{exeresol}
  Calcule
  \begin{equation}
    \lim_{x\to 0^-} \frac{e^x-1}{x^2}.
  \end{equation}
\end{exeresol}
\begin{resol}
  Observamos tratar-se de uma indeterminação do tipo $0/0$, i.e.
  \begin{equation}
    \lim_{x\to 0^-} \frac{\cancelto{0}{e^x-1}}{\cancelto{0}{x^2}}.
  \end{equation}
  Então, aplicando a regra de L'Hôpital, temos
  \begin{equation}
    \lim_{x\to 0^-} \frac{e^x-1}{x^2} = \lim_{x\to 0^-} \frac{\cancelto{1}{e^x}}{\cancelto{0^-}{2x}} = -\infty.
  \end{equation}
\end{resol}

\begin{exeresol}(Indeterminação do tipo $0\cdot\infty$)

  Calcule
  \begin{equation}
    \lim_{x\to \infty} x^{51}e^{-x}.
  \end{equation}
\end{exeresol}
\begin{resol}
  Observamos que
  \begin{equation}
    \lim_{x\to \infty} \cancelto{\infty}{x^{51}}\cancelto{0}{e^{-x}} = \lim_{x\to \infty} \frac{\cancelto{\infty}{x^{51}}}{\cancelto{\infty}{e^{x}}}.
  \end{equation}
  Então, aplicando a regra de L'Hôpital sucessivamente, obtemos
  \begin{align}
    \lim_{x\to\infty} x^{51}e^{-x} &= \lim_{x\to\infty} \frac{x^{51}}{e^x} \\
                                   &= \lim_{x\to\infty} \frac{51\cdot x^{50}}{e^x} \\
                                   &= \lim_{x\to\infty} \frac{51\cdot 50\cdot x^{49}}{e^x} \\
                                   &\vdots \\
                                   &= \lim_{x\to\infty} \frac{51!}{\cancelto{\infty}{e^x}} = 0.
  \end{align}
\end{resol}

\begin{exeresol}(Indeterminação do tipo $\infty - \infty$)

  Calcule
  \begin{equation}
    \lim_{x\to 0^+} \left(\frac{1}{x} - \frac{1}{e^x-1}\right).
  \end{equation}
\end{exeresol}
\begin{resol}
  Trata-se de uma indeterminação do tipo $\infty - \infty$, pois
  \begin{equation}
    \lim_{x\to 0^+} \left(\cancelto{\infty}{\frac{1}{x}} - \cancelto{\infty}{\frac{1}{e^x-1}}\right).
  \end{equation}
  Neste caso, calculando a subtração, obtemos
  \begin{align}
    \lim_{x\to 0^+} \left(\frac{1}{x} - \frac{1}{e^x-1}\right) &= \lim_{x\to 0^+} \frac{e^x-1+x}{xe^x-x},
  \end{align}
  a qual é uma indeterminação do tipo $0/0$. Aplicando a regra de L'Hôpital, obtemos
  \begin{align}
    \lim_{x\to 0^+} \frac{e^x-1-x}{xe^x-x} &= \lim_{x\to 0^+} \frac{\cancelto{0}{e^x-1}}{\cancelto{0}{(x+1)e^x-1}} \\
                                           &= \lim_{x\to 0^+} \frac{e^x}{(x+2)e^x} \\
    &= \lim_{x\to 0^+} \frac{1}{x+2} = \frac{1}{2}.
  \end{align}
\end{resol}

\begin{exeresol}(Indeterminação do tipo $1^\infty$)

  Calcule
  \begin{equation}
    \lim_{x\to 0^+} (1+x)^{1/x}.
  \end{equation}
\end{exeresol}
\begin{resol}
  Trata-se de uma indeterminação do tipo $1^\infty$. Em tais casos, a seguinte estratégia pode ser útil. Nos pontos de continuidade da função logaritmo natural, temos
  \begin{align}
    \ln\left(\lim_{x\to 0^+} (1+x)^{1/x}\right) &= \lim_{x\to 0^+} \ln\left((1+x)^{1/x}\right) \\
                                                &= \lim_{x\to 0^+} \frac{\cancelto{0}{\ln(1+x)}}{\cancelto{0}{x}} \\
                                                &= \lim_{x\to 0^+} \frac{\frac{1}{x+1}}{1} = 1.
  \end{align}
  Ou seja,
  \begin{equation}
    \ln\left(\lim_{x\to 0^+} (1+x)^{1/x}\right) = 1 \Rightarrow \lim_{x\to 0^+} (1+x)^{1/x} = e.
  \end{equation}
\end{resol}

\subsection*{Exercícios}

\begin{flushright}
  [Vídeo] | [Áudio] | \href{https://phkonzen.github.io/notas/contato.html}{[Contatar]}
\end{flushright}

\begin{exer}
  Calcule
  \begin{equation}
    \lim_{x\to -1} \frac{x+1}{x^2+3x+2}.
  \end{equation}
\end{exer}
\begin{resp}
  $1$
\end{resp}

\begin{exer}
  Calcule
  \begin{equation}
    \lim_{x\to \infty} x^{-51}e^x.
  \end{equation}
\end{exer}
\begin{resp}
  $\infty$
\end{resp}

\begin{exer}
  Calcule
  \begin{equation}
    \lim_{x\to 0^+} \left(\frac{1}{x}+\ln x\right).
  \end{equation}
\end{exer}
\begin{resp}
  $\infty$
\end{resp}

\begin{exer}
  Calcule
  \begin{equation}
    \lim_{x\to 0^+} \left(e^x + x\right)^{\frac{1}{2x}}.
  \end{equation}
\end{exer}
\begin{resp}
  $e$
\end{resp}

\section{Extremos de funções}\label{cap_apderiv_sec_extfun}

\begin{flushright}
  [Vídeo] | [Áudio] | \href{https://phkonzen.github.io/notas/contato.html}{[Contatar]}
\end{flushright}

Seja $f$ uma função com domínio $D$. Dizemos que $f$ tem o valor \emph{máximo global}\footnote{Também chamado de máximo absoluto.} $f(a)$ no ponto $x=a$ quando
\begin{equation}
  f(x) \leq f(a),
\end{equation}
para todo $x\in D$. Analogamente, dizemos que $f$ tem o valor \emph{mínimo global}\footnote{Também chamado de mínimo absoluto.} $f(b)$ no ponto $x=b$ quando
\begin{equation}
  f(x) \geq f(b),
\end{equation}
para todo $x\in D$. Em tais pontos, dizemos que a função têm seus valores \emph{extremos globais} (ou extremos absolutos).

\begin{ex}\label{ex:vmaxminabs}
  A função $f(x) = x^2$ tem valor mínimo global no ponto $x=0$ e não assume valor máximo global. A função $g(x) = -x^2$ tem valor máximo global no ponto $x=0$ e não assume valor mínimo global. A função $h(x)=x^3$ não assume valores mínimo e máximo globais. Veja a Figura \ref{fig:ex_vmaxminabs}.

  \begin{figure}[H]
    \centering
    \includegraphics[width=0.32\textwidth]{./cap_apderiv/dados/fig_ex_vmaxminabs/fig_f}~
    \includegraphics[width=0.32\textwidth]{./cap_apderiv/dados/fig_ex_vmaxminabs/fig_g}~
    \includegraphics[width=0.32\textwidth]{./cap_apderiv/dados/fig_ex_vmaxminabs/fig_h}
    \caption{Esboço das funções discutidas no Exemplo \ref{ex:vmaxminabs}.}
    \label{fig:ex_vmaxminabs}
  \end{figure}
\end{ex}

\begin{teo}\normalfont{(Teorema do valor extremo\footnote{Este é uma versão do chamado \href{https://pt.wikipedia.org/wiki/Teorema_de_Weierstrass}{Teorema de Weierstrass}})}\label{teo:valorextremo}
  Se $f$ é uma função contínua em um intervalo fechado $[a, b]$, então $f$ assume tanto um valor máximo como um valor mínimo global em $[a, b]$.
\end{teo}
\begin{dem}
  A demonstração foge dos objetivos deste texto. Caso tenha interesse, consulte \cite{Avila1993a}.
\end{dem}

\begin{ex}\label{ex:fcont}
  Vejamos os seguintes casos:
  \begin{enumerate}[a)]
  \item  A função $\pmb{f(x) = (x-1)^2+1}$ é contínua no intervalo fechado $\displaystyle \left[0, \frac{3}{2}\right]$. Assume valor mínimo global $1$ no ponto $x=1$. Ainda, assume valor máximo global igual a $2$ no ponto $x=0$. Veja Figura \ref{fig:ex_fcont_f}.
  \begin{figure}[H]
    \centering
    \includegraphics[width=0.7\textwidth]{./cap_apderiv/dados/fig_ex_fcont/fig_f}
    \caption{Esboço do gráfico de $f(x) = (x-1)^2+1$ no intervalo $\displaystyle\left[0, \frac{3}{2}\right]$. Veja o Exemplo \ref{ex:fcont} a).}
    \label{fig:ex_fcont_f}
  \end{figure}
\item A função $\pmb{g(x) = \ln x}$ é contínua no intervalo $(0, e]$. Neste intervalo, assume valor máximo global no ponto $x=e$, mas não assume valor mínimo global. Veja Figura \ref{fig:ex_fcont_g}.
  \begin{figure}[H]
    \centering
    \includegraphics[width=0.7\textwidth]{./cap_apderiv/dados/fig_ex_fcont/fig_g}
    \caption{Esboço do gráfico de $g(x) = \ln x$ no intervalo $(0,e]$. Veja o Exemplo \ref{ex:fcont} b).}
    \label{fig:ex_fcont_g}
  \end{figure}
  
\item A função
  \begin{equation}
    h(x) = \left\{
      \begin{array}{ll}
        x &, 0\leq x < 1,\\
        0 &, x=1,
      \end{array}
\right.
\end{equation}
definida no intervalo $[0, 1]$ é descontínua no ponto $x=1$. Neste intervalo, assume valor mínimo global no ponto $x=0$, mas não assume valor máximo global. Veja a Figura \ref{fig:ex_fcont_h}.
  \begin{figure}[H]
    \centering
    \includegraphics[width=0.7\textwidth]{./cap_apderiv/dados/fig_ex_fcont/fig_h}
    \caption{Esboço do gráfico de $h(x)$ no intervalo $[0,1]$. Veja o Exemplo \ref{ex:fcont} c).}
    \label{fig:ex_fcont_h}
  \end{figure}
  \end{enumerate}
\end{ex}

Uma função $f$ tem um valor \emph{máximo local} em um ponto interior $x=a$ de seu domínio, se $f(x) < f(a)$ para todo $x$ em um intervalo aberto em torno de $a$, excluindo-se $x=a$. Analogamente, $f$ tem um valor \emph{mínimo local} em um ponto interior $x=b$ de seu domínio, se $f(x) > f(b)$  para todo $x$ em um intervalo aberto em torno de $b$, excluindo-se $x=b$. Em tais pontos, dizemos que a função têm valores \emph{extremos locais} (ou relativos). Um tal ponto é chamado de \emph{ponto de máximo local} ou \emph{de mínimo local}, conforme o caso.

\begin{ex}\label{ex:vmaxminloc}
  Consideremos a função
  \begin{equation}
    f(x) = \left\{
      \begin{array}{ll}
        -(x+1)^2-2 &, -2\leq x < -\frac{1}{2},\\
        |x| &, -\frac{1}{2} \leq x < 1,\\
        (x-2)^3+2 &, 1\leq x < 3.
      \end{array}
\right.
\end{equation}

  \begin{figure}[H]
    \centering
    \includegraphics[width=0.7\textwidth]{./cap_apderiv/dados/fig_ex_vmaxminloc/fig_f}
    \caption{Esboço do gráfico de $f(x)$ discutida no Exemplo \ref{ex:vmaxminloc}.}
    \label{fig:ex_vmaxminloc}
  \end{figure}
  
Na Figura \ref{fig:ex_vmaxminloc} temos o esboço de seu gráfico. Por inferência, temos que $f$ tem valores máximos locais nos pontos $x=-1$ e $x=-1/2$. No ponto $x=0$ tem um valor mínimo local. Observamos que $x=-2$, $x=2$ e $x=3$ não são pontos de extremos locais desta função. No ponto $x=-2$, $f$ tem seu valor mínimo global. Ainda, $f$ não tem valor máximo global.
\end{ex}

\begin{teo}\normalfont{(Teorema da derivada para pontos extremos locais.)}\label{teo:ptocritico}
  Se $f$ possui um valor extremo local em um ponto $x=a$ e $f$ é diferenciável neste ponto, então
  \begin{equation}
    f'(a) = 0.
  \end{equation}
\end{teo}
\begin{dem}
  Vamos considerar o caso em que $f$ possui um máximo local em $x=a$. Então, segue que
  \begin{gather}
    f'(a) = \lim_{h\to 0^+} \frac{f(x+h)-f(x)}{h} \leq 0\\
    f'(a) = \lim_{h\to 0^-} \frac{f(x+h)-f(x)}{h} \geq 0
  \end{gather}
  Logo, $f'(a) = 0$. Para o caso em que $f$ possui um mínimo local em $x=a$, consulte o Exercício \ref{exer:ptocritico}.
\end{dem}

Deste teorema, podemos concluir que uma função $f$ pode ter valores extremos em:
\begin{enumerate}[a)]
\item pontos interiores de seu domínio onde $f' = 0$,
\item pontos interiores de seu domínio onde $f'$ não existe, ou
\item pontos extremos de seu domínio.
\end{enumerate}

Um ponto interior do domínio de uma função $f$ onde $f'=0$ ou $f'$ não existe, é chamado de \emph{ponto crítico} da função.

\begin{obs}\label{obs:pt_critico_val_extremo}
  Uma função tem valores extremos em pontos críticos ou nos extremos de seu domínio.
\end{obs}


\begin{ex}
  Consideramos a função $f(x)$ discutida no Exemplo \ref{ex:vmaxminloc}. No ponto $x=-1$, $f'(-1)=0$ e $f$ tem valor máximo local neste ponto. Entretanto, no ponto $x=2$, também temos $f'(2)=0$, mas $f$ não tem valor extremo neste ponto.

  No ponto $x=0$, $f'(0)$ não existe e $f$ tem valor mínimo local neste ponto. No ponto, $x=-1/2$, $f'(1/2)$ não existe e $f$ tem valor máximo local neste ponto.

  Nos extremos do domínio, temos que $f$ tem valor mínimo global no ponto $x=-2$, mas não tem extremo global no ponto $x=3$.
\end{ex}

\subsection*{Exercícios resolvidos}

\begin{flushright}
  [Vídeo] | [Áudio] | \href{https://phkonzen.github.io/notas/contato.html}{[Contatar]}
\end{flushright}

\begin{exeresol}\label{exeresol:f_diff}
  Determine os pontos extremos da função $f(x) = (x+1)^2-1$ no intervalo $[-2,1]$.
\end{exeresol}
\begin{resol}
  Os valores extremos de um função podem ocorrer, somente, em seus pontos críticos ou nos extremos de seu domínio. Como $f(x) = (x+1)^2-1$ é diferenciável no intervalo $(-2,1)$, seus pontos críticos são pontos tais que $f'=0$. Para identificá-los, calculamos
  \begin{align}
    f'(x)=0 &\Rightarrow 2(x+1) = 0\\
            &\Rightarrow x = -1.
  \end{align}

  \begin{figure}[H]
    \centering
    \includegraphics[width=0.7\textwidth]{./cap_apderiv/dados/fig_exeresol_f_diff/fig_exeresol_f_diff}
    \caption{Esboço do gráfico da função $f(x) = (x+1)^2-1$ discutida no Exercício Resolvido \ref{exeresol:f_diff}.}
    \label{fig:exeresol_f_diff}
  \end{figure}

  Desta forma, $f$ pode ter valores extremos nos ponto $x=-2$, $x=-1$ e $x=1$. Analisamos, então, o esboço do gráfico da função (Figura \ref{fig:exeresol_f_diff}) e a seguinte tabela:\\
  \begin{center}
  \begin{tabular}[H]{l|ccc}
    $x$ & -2 & -1 & 1 \\\hline
    $f(x)$ & 0 & -1 & 3\\\hline
  \end{tabular}
\end{center}
Daí, podemos concluir que $f$ tem o valor mínimo global (e local) de $f(-1)=-1$ no ponto $x=-1$ e tem valor máximo global de $f(1)=3$ no ponto $x=1$.

\ifispython
Podemos usar o {\python}+{\sympy} para computar os pontos extremos e plotar a função. Por exemplo, com os seguintes comandos:
\begin{lstlisting}
import sympy as sp
x = sp.Symbol('x')
f = sp.lambdify(x, (x+1)**2-1)
# f' == 0
xc = sp.solve(sp.diff(f(x),x), x)
print(f"Pto. crítico xc = {xc}")
print(f"f(-2) = {f(-2)}")
print(f"f({xc[0]}) = {f(xc[0])}")
print(f"f(1) = {f(1)}")
sp.plot(f(x), (x,-2,1))
\end{lstlisting}
\fi
\end{resol}

\begin{exeresol}\label{exeresol:p_infl}
  Determine os pontos extremos da função $f(x)=x^3$ no intervalo $[-1, 1]$.
\end{exeresol}
\begin{resol}
  Como $f$ é diferenciável no intervalo $(-1, 1)$, temos que seus pontos críticos são tais que $f'(x)=0$. Neste caso, temos
  \begin{equation}
    3x^2=0\Rightarrow x=0
  \end{equation}
  é o único ponto crítico de $f$. Entretanto, analisando o gráfico desta função (Figura \ref{fig:exeresol_p_infl}) vemos que $f$ não tem valor extremo local neste ponto. Assim, seus pontos extremos só podem ocorrer nos extremos do domínio $[-1, 1]$. Concluímos que $f(-1)=-1$ é o valor mínimo global de $f$ e $f(1)=1$ é seu valor máximo global.

  \begin{figure}[H]
    \centering
    \includegraphics[width=0.7\textwidth]{./cap_apderiv/dados/fig_exeresol_p_infl/fig_exeresol_p_infl}
    \caption{Esboço do gráfico da função $f(x) = x^3$ discutida no Exercício Resolvido \ref{exeresol:p_infl}.}
    \label{fig:exeresol_p_infl}
  \end{figure}
\end{resol}

\subsection*{Exercícios}

\begin{flushright}
  [Vídeo] | [Áudio] | \href{https://phkonzen.github.io/notas/contato.html}{[Contatar]}
\end{flushright}

\begin{exer}
  Considere que uma dada função $f$ tenha o seguinte esboço de gráfico:

  \begin{center}
    \includegraphics[width=0.7\textwidth]{./cap_apderiv/dados/fig_exer_extfun/fig_exer_extfun}
  \end{center}

  Determine e classifique os pontos extremos desta função.
\end{exer}
\begin{resp}
  $x=-1$ ponto de mínimo global; $x=1$ ponto de máximo local; $x=2$ ponto de mínimo local; $x=\frac{5}{2}$ ponto de máximo global.
\end{resp}

\begin{exer}
  Dada a função $f(x)=x^2-2x+3$ restrita ao intervalo $[-1,2]$, determine:
  \begin{enumerate}[a)]
  \item seu(s) ponto(s) crítico(s).
  \item seu(s) ponto(s) extremo(s) e o(s) classifique.
  \item seu(s) valor(es) extremo(s) e o(s) classifique.
  \end{enumerate}
\end{exer}
\begin{resp}
  a)~$x=1$; b)~$x=-1$ ponto de máximo global; $x=1$ ponto de mínimo local e global; c)~$f(-1)=6$ valor máximo global; $f(1)=2$ valor mínimo local e global;
\end{resp}

\begin{exer}
  Dada a função $f(x)=-x^2+2x+1$ restrita ao intervalo $[0,3]$, determine:
  \begin{enumerate}[a)]
  \item seu(s) ponto(s) crítico(s).
  \item seu(s) ponto(s) extremo(s) e o(s) classifique.
  \item seu(s) valor(es) extremo(s) e o(s) classifique.
  \end{enumerate}
\end{exer}
\begin{resp}
  a)~$x=1$; b)~$x=1$ ponto de máximo local e global; $x=3$ ponto de mínimo global; c)~$f(1)=2$ valor máximo local e global; $f(3)=-2$ valor mínimo global;
\end{resp}

\begin{exer}
  Dada a função $f(x)=x^{3} - 3 x^{2} + 3 x$ restrita ao intervalo $[0, \infty)$, determine:
  \begin{enumerate}[a)]
  \item seu(s) ponto(s) crítico(s).
  \item seu(s) ponto(s) extremo(s) e o(s) classifique.
  \item seu(s) valor(es) extremo(s) e o(s) classifique.
  \end{enumerate}
\end{exer}
\begin{resp}
  a)~$x=1$; b)~$x=0$ ponto de mínimo global;c)~$f(0)=0$ valor mínimo global;
\end{resp}

\begin{exer}
  Dada a função $f(x)=x^{1/3}$ restrita ao intervalo $[-1,1]$, determine:
  \begin{enumerate}[a)]
  \item seu(s) ponto(s) crítico(s).
  \item seu(s) ponto(s) extremo(s) e o(s) classifique.
  \item seu(s) valor(es) extremo(s) e o(s) classifique.
  \end{enumerate}
\end{exer}
\begin{resp}
  a)~$x=0$; b)~$x=-1$ ponto de mínimo global; $x=1$ ponto de máximo global; c)~$f(-1)=-1$ valor mínimo global; $f(1)=1$ valor máximo global;
\end{resp}

\begin{exer}\label{exer:ptocritico}
  Mostre que se $f$ tem um mínimo local em $x=a$ e é diferenciável neste ponto, então $f'(a)=0$.
\end{exer}
\begin{resp}
  Dica: consulte a demonstração do Teorema \ref{teo:ptocritico}.
\end{resp}


\section{Teorema do valor médio}\label{cap_apderiv_sec_valormedio}

\begin{flushright}
  [Vídeo] | [Áudio] | \href{https://phkonzen.github.io/notas/contato.html}{[Contatar]}
\end{flushright}

O teorema do valor médio é uma aplicação do teorema de Rolle.

\subsection{Teorema de Rolle}

\begin{flushright}
  [Vídeo] | [Áudio] | \href{https://phkonzen.github.io/notas/contato.html}{[Contatar]}
\end{flushright}

O Teorema de Rolle fornece uma condição suficiente para que uma dada função diferenciável tenha derivada nula em pelo menos um ponto.

\begin{figure}[H]
  \centering
  \includegraphics[width=0.7\textwidth]{./cap_apderiv/dados/fig_teo_Rolle/fig_teo_Rolle}
  \caption{Ilustração do Teorema de Rolle.}
  \label{fig:teo_Rolle}
\end{figure}

\begin{teo}\normalfont{(Teorema de Rolle)}
  Seja $f$ uma função contínua no intervalo fechado $[a, b]$ e diferenciável no intervalo aberto $(a, b)$. Se
  \begin{equation}
    f(a)=f(b),
  \end{equation}
  então existe pelo menos um {\bf ponto crítico} $c\in (a, b)$ tal que
  \begin{equation}
    f'(c)=0.
  \end{equation}
\end{teo}
\begin{dem}
  A ideia da demonstração é uma consequência dos teoremas \ref{teo:valorextremo} e \ref{teo:ptocritico}. O primeiro, que existem pontos de mínimo e máximos globais  $m, M\in [a, b]$, i.e.
  \begin{equation}
    f(m) \leq f(x) \leq f(M).
  \end{equation}
  Se $m=M$, então $f$ é uma função contínua, donde segue que $f'(x)=0$ para todo $x\in (a, b)$. Agora, se $m\neq M$, então $m$ ou $M$ é um extremo local. Sem perda de generalidade, supomos que $c = m$ seja o mínimo local. Neste caso, o teorema \ref{teo:ptocritico} nos garante que $f'(c)=0$.
\end{dem}

\begin{ex}
  O polinômio $p(x) = x^3 - 4x^2 + 3x + 1$ tem pelo menos um ponto crítico no intervalo $(0,1)$ e no intervalo $(1,3)$. De fato,temos $p(0)=p(1)=1$ e, pelo teorema de Rolle, segue que existe pelo menos um ponto $c\in (0, 1)$ tal que $f'(c)=0$. Analogamente, como também $p(1)=p(3)=1$, segue do teorema que existe pelo menos um ponto crítico no intervalo $(1,3)$. Veja o esboço do gráfico de $p$ na Figura \ref{fig:ex_teo_Rolle}.

  \begin{figure}[H]
    \centering
    \includegraphics[width=0.7\textwidth]{./cap_apderiv/dados/fig_ex_teo_Rolle/fig_ex_teo_Rolle}
    \caption{Esboço do gráfico de $p(x) = x^3 - 4x^2 + 3x + 1$.}
    \label{fig:ex_teo_Rolle}
  \end{figure}
  
  De fato, como todo polinômio é derivável em toda parte, podemos calcular os pontos críticos como segue.
  \begin{align}
    p'(x) = 0 &\Rightarrow 3x^2 - 8x + 3 = 0 \\
              &\Rightarrow x = \frac{8 \pm \cancelto{2\sqrt{7}}{\sqrt{64 - 36}}}{6} \\
              &\Rightarrow x_1 = \frac{4 - \sqrt{7}}{3} \approx 0,45 \quad\text{ou}\quad x_2 = \frac{4 + \sqrt{7}}{3} \approx 2,22.
  \end{align}

  \ifispython
  Podemos usar os seguintes comandos\footnote{Veja a Observação \ref{obs:cap_apderiv_python}.} para computar os pontos críticos de $p$ e plotar seu gráfico:
\begin{verbatim}
>>> p = x**3 - 4*x**2 + 3*x + 1
>>> pc = solve(p.diff()); pc
[-sqrt(7)/3 + 4/3, sqrt(7)/3 + 4/3]
>>> plot(p,(x,-0.5,3.5))
\end{verbatim}
  \fi
\end{ex}

\begin{ex}\label{ex:nteo_Rolle}
  Vejamos os seguintes casos em que o Teorema de Rolle não se aplica:
  \begin{enumerate}[a)]
  \item A função
    \begin{equation}
      f(x) = \left\{
        \begin{array}{ll}
          x &, 0\leq x < 1,\\
          0 &, x=1.
        \end{array}
      \right.
    \end{equation}
    é tal que $f(0)=f(1)=0$, entretanto sua derivada $f'(x)=1$ no intervalo $(0, 1)$. Ou seja, a condição da $f$ ser contínua no intervalo fechado associado é necessária no teorema de Rolle. Veja a Figura \ref{fig:ex_fcont_h_2} para o esboço do gráfico desta função.
    
  \begin{figure}[H]
    \centering
    \includegraphics[width=0.7\textwidth]{./cap_apderiv/dados/fig_ex_fcont/fig_h}
    \caption{Esboço do gráfico da função referente ao Exemplo \ref{ex:nteo_Rolle} a).}
    \label{fig:ex_fcont_h_2}
  \end{figure}
  
\item Não existe ponto tal que a derivada da $g(x)=-|x-1|+1$ seja nula. Entretanto, notemos que $g(0)=g(2)=0$ e $g$ contínua no intervalo fechado $[0, 2]$. O teorema de Rolle não se aplica neste caso, pois $g$ não é derivável no intervalo $(0,2)$, mais especificamente, no ponto $x=1$. Veja a Figura \ref{fig:ex_nteo_Rolle_nderiv}.

  \begin{figure}[H]
    \centering
    \includegraphics[width=0.7\textwidth]{./cap_apderiv/dados/fig_ex_nteo_Rolle_nderiv/fig_ex_nteo_Rolle_nderiv}
    \caption{Esboço do gráfico da função referente ao Exemplo \ref{ex:nteo_Rolle} b).}
    \label{fig:ex_nteo_Rolle_nderiv}
  \end{figure}  
  \end{enumerate}
\end{ex}

\subsection{Teorema do valor médio}

\begin{flushright}
  [Vídeo] | [Áudio] | \href{https://phkonzen.github.io/notas/contato.html}{[Contatar]}
\end{flushright}

O teorema do valor médio é uma generalização do teorema de Rolle.

\begin{figure}[H]
  \centering
  \includegraphics[width=0.7\textwidth]{./cap_apderiv/dados/fig_teo_valmed/fig_teo_valmed}
  \caption{Ilustração do Teorema do valor médio.}
  \label{fig:teo_valor_medio}
\end{figure}

\begin{teo}\normalfont{(Teorema do valor médio\footnote{Também conhecido como Teorema de Lagrange.})}\label{teo:tvm}
  Seja $f$ uma função contínua no intervalo fechado $[a,b]$ e diferenciável no intervalo aberto $(a,b)$. Então, existe pelo menos um ponto $c\in (a,b)$ tal que
  \begin{equation}
    \frac{f(b)-f(a)}{b-a}=f'(c).
  \end{equation}
\end{teo}
\begin{dem}
  O resultado segue da aplicação do Teorema de Rolle \ref{teo:rolle} a seguinte função
  \begin{equation}
    F(x) = f(x) - f(a) - \frac{f(b) - f(a)}{b-a}(x-a)
  \end{equation}
  De fato, $F$ é contínua em $[a, b]$, diferenciável em $(a, b)$ e $F(a) = F(b)$. Logo, existe $c\in (a, b)$ tal que
  \begin{gather}
    F'(c) = 0\\
    f'(c) + \frac{f(b) - f(a)}{b-a} = 0\\
    \frac{f(b) - f(a)}{b-a} = f'(c)
  \end{gather}
\end{dem}

\begin{obs}
  Em um contexto de aplicação, o Teorema do valor médio relaciona a taxa de variação média da função em um intervalo $[a, b]$ com a taxa de variação instantânea da função em um ponto interior deste intervalo.
\end{obs}

\begin{ex}
  A função $f(x)=x^2$ é contínua no intervalo $[0,2]$ e diferenciável no intervalo $(0,2)$. Logo, segue do teorema do valor médio que existe pelo menos um ponto $c\in (0,2)$ tal que
  \begin{equation}
    f'(c)=\frac{f(2)-f(0)}{2-0}=2.
  \end{equation}
  De fato, $f'(x)=2x$ e, portanto, tomando $c=1$, temos $f'(c)=2$.
\end{ex}

\begin{corol}\normalfont{(Funções com derivadas nulas são constantes)}
  Se $f'(x)=0$ para todos os pontos em um intervalo $(a, b)$, então $f$ é constante neste intervalo.
\end{corol}
\begin{dem}
  De fato, sejam $x_1,x_2\in (a, b)$ e, sem perda de generalidade, $x_1<x_2$. Então, temos $f$ é contínua no intervalo $[x_1,x_2]$ e diferenciável em $(x_1,x_2)$. Segue do teorema do valor médio que existe $c\in (x_1,x_2)$ tal que
  \begin{equation}
    \frac{f(x_2)-f(x_1)}{x_2-x_1}=f'(c).
  \end{equation}
  Como $f'(c)=0$, temos $f(x_2)=f(x_1)$. Ou seja, a função vale sempre o mesmo valor para quaisquer dois pontos no intervalo $(a, b)$, logo é constante neste intervalo.
\end{dem}

\begin{corol}\normalfont{(Função com a mesma derivada diferem por uma constante)}\label{corol:apderiv_teomed_2}
  Se $f'(x)=g'(x)$ para todos os pontos em um intervalo aberto $(a,b)$, então $f(x)=g(x)+C$, $C$ constante, para todo $x\in (a,b)$.
\end{corol}
\begin{dem}
  Segue, imediatamente, da aplicação do corolário anterior à função $h(x)=f(x)-g(x)$.
\end{dem}

\begin{corol}\normalfont{(Monotonicidade e o sinal da derivada)}\label{corol:mono_deriv}
  Suponha que $f$ seja contínua em $[a,b]$ e derivável em $(a,b)$.
  \begin{enumerate}[a)]
  \item Se $f'(x)>0$ para todo $x\in (a,b)$, então $f$ é crescente\footnote{$f$ é função crescente em um intervalo $I$, quando $x_1>x_2$ em $I$ implica $f(x_1)>f(x_2)$.} em $[a,b]$.
  \item Se $f'(x)<0$ para todo $x\in (a,b)$, então $f$ é decrescente\footnote{$f$ é função decrescente em um intervalo $I$, quando $x_1>x_2$ em $I$ implica $f(x_1)<f(x_2)$.} em $[a,b]$.
  \end{enumerate}
\end{corol}
\begin{dem}
  Vamos demonstrar o item a), i.e. se $f'(x)>0$ para todo $x\in (a, b)$, então $f$ é crescente em $[a, b]$. Sejam $x_1<x_2$ com $x_1,x_2\in [a, b]$. Observamos que $f$ é contínua em $[x_1, x_2]$ e diferenciável em $(x_1, x_2)$. Logo, pelo Teorema do valor médio \ref{teo:tvm}, temos que existe $c\in (x_1, x_2)$ tal que
  \begin{equation}
    \frac{f(x_2)-f(x_1)}{x_2-x_1} = f'(c)
  \end{equation}
  ou, equivalentemente,
  \begin{equation}
    f(x_2)-f(x_1) = f'(c)\cdot(x_2-x_1).
  \end{equation}
  Como $f'(x)>0$ para todo $x\in (a, b)$ e $x_2-x_1>0$, concluímos que $f(x_2)-f(x_1)>0$, i.e.
  \begin{equation}
    f(x_1) < f(x_2).
  \end{equation}
  Com isso, mostramos que se $x_1<x_2$ com $x_1,x_2\in [a, b]$, então $f(x_1)<f(x_2)$, i.e. $f$ é crescente em $[a, b]$.

  A demonstração do item b) é análoga, consulte o Exercício \ref{exer:mono_deriv}.
\end{dem}

\begin{ex}
  Vamos estudar a monotonicidade da função polinomial $f(x) = x^3 - 4x^2 + 3x + 1$. Na Figura \ref{fig:ex_corol_mono_deriv}, temos o esboço de seu gráfico.
  
  \begin{figure}[H]
    \centering
    \includegraphics[width=0.7\textwidth]{./cap_apderiv/dados/fig_ex_corol_mono_deriv/fig_ex_corol_mono_deriv}
    \caption{Esboço do gráfico de $f(x) = x^3 - 4x^2 + 3x + 1$.}
    \label{fig:ex_corol_mono_deriv}
  \end{figure}

  Podemos usar o Corolário \ref{corol:mono_deriv} para estudarmos a monotonicidade (i.e. intervalos de crescimento ou decrescimento). Isto é, fazemos o estudo de sinal da derivada de $f$. Calculamos
  \begin{equation}
    f'(x) = 3x^2 - 8x + 3.
  \end{equation}
  Logo, temos
  \begin{center}
    \includegraphics[width=0.5\textwidth]{./cap_apderiv/dados/fig_ex_monoderiv_poli/fig_ex_monoderiv_poli}
  \end{center}
  Ou seja, $f'(x) < 0$ no conjunto $\displaystyle \left(-\infty, \frac{4-\sqrt{7}}{3}\right)\cup \left(\frac{4+\sqrt{7}}{3}, \infty\right)$ e $f'(x) < 0$ no conjunto $\displaystyle \left(\frac{4-\sqrt{7}}{3}, \frac{4+\sqrt{7}}{3}\right)$. Concluímos que $f$ é {\bf crescente} nos intervalos $\displaystyle \left(\left.-\infty, \frac{4-\sqrt{7}}{3}\right.\right]$ e $\displaystyle \left[\left.\frac{4+\sqrt{7}}{3}, \infty\right)\right.$, enquanto que $f$ é {\bf decrescente} no intervalo $\displaystyle \left[\frac{4-\sqrt{7}}{3}, \frac{4+\sqrt{7}}{3}\right]$.
\end{ex}

\begin{ex}
  A função exponencial $f(x) = e^x$ é crescente em toda parte. De fato, temos
  \begin{equation}
    f'(x) = e^x > 0,
  \end{equation}
  para todo $x\in\mathbb{R}$.
\end{ex}

\subsection*{Exercícios resolvidos}

\begin{flushright}
  [Vídeo] | [Áudio] | \href{https://phkonzen.github.io/notas/contato.html}{[Contatar]}
\end{flushright}

\begin{exeresol}
  Um carro percorreu 150 km em 1h30min. Mostre que em algum momento o carro estava a uma velocidade maior que 80 km/h.
\end{exeresol}
\begin{resol}
  Seja $s=s(t)$ a função distância percorrida pelo carro e $t$ o tempo, em horas, contado do início do percurso. Do teorema do valor médio, exite tempo $t_1\in (0,~1,5)$ tal que
  \begin{equation}
    f'(t_1) = \frac{s(1,5)-s(0)}{1,5-0} = \frac{150}{1,5} = 100~\text{km/h}.
  \end{equation}
  Ou seja, em algum momento o carro atingiu a velocidade de 100 km/h.
\end{resol}

\begin{exeresol}
  Estude a monotonicidade da função gaussiana $f(x) = e^{-x^2}$.  
\end{exeresol}
\begin{resol}
  Para estudarmos a monotonicidade de uma função, podemos fazer o estudo de sinal de sua derivada. Neste caso, temos
  \begin{equation}
    f'(x) = -2xe^{-x^2}.
  \end{equation}
  Assim, vemos que
  \begin{center}
    \includegraphics[width=0.5\textwidth]{./cap_apderiv/dados/fig_exeresol_gauss_estsinal/fig_exeresol_gauss_estsinal}
  \end{center}
  Concluímos que $f$ é crescente no intervalo $(-\infty, 0)$ e decrescente no intervalo $(0, \infty)$.
\end{resol}

\subsection*{Exercícios}

\begin{flushright}
  [Vídeo] | [Áudio] | \href{https://phkonzen.github.io/notas/contato.html}{[Contatar]}
\end{flushright}

\begin{exer}
  Estude a monotonicidade de $f(x) = x^2 - 2x$.
\end{exer}
\begin{resp}
  Decrescente: $(-\infty, 1]$; Crescente: $[1, \infty)$
\end{resp}

\begin{exer}
  Estude a monotonicidade de $\displaystyle f(x) = \frac{x^3}{3}-x$.
\end{exer}
\begin{resp}
  Decrescente: $[-1, 1]$; Crescente: $(-\infty, -1]$; $[1, \infty)$
\end{resp}

\begin{exer}
  Estude a monotonicidade de $\displaystyle f(x) = \ln x$.
\end{exer}
\begin{resp}
  Crescente: $(0, \infty)$
\end{resp}

\begin{exer}
  Estude a monotonicidade de $\displaystyle f(x) = xe^{-x}$.
\end{exer}
\begin{resp}
  Crescente: $(-\infty, 1)$; Decrescente de $(1, \infty)$
\end{resp}

\begin{exer}
  Demonstre que um polinômio cúbico pode ter no máximo $3$ raízes reais.
\end{exer}
\begin{resp}
  Dica: use o teorema de Rolle.
\end{resp}

\begin{exer}\label{exer:mono_deriv}
  Seja $f$ contínua em $[a,b]$ e derivável em $(a,b)$. Mostre que se $f'(x)<0$ para todo $x\in (a,b)$, então $f$ é decrescente em $[a, b]$. 
\end{exer}
\begin{resp}
  Dica: consulte a demonstração do item a) do Corolário \ref{corol:mono_deriv}.
\end{resp}


\section{Teste da primeira derivada}\label{cap_apderiv_sec_tder1}

\begin{flushright}
  [Vídeo] | [Áudio] | \href{https://phkonzen.github.io/notas/contato.html}{[Contatar]}
\end{flushright}

Na Seção \ref{cap_apderiv_sec_extfun}, vimos que os extremos de uma função ocorrem nos extremos de seu domínio ou em um ponto crítico. Aliado a isso, o Corolário \ref{corol:mono_deriv} nos fornece condições suficientes para classificar os pontos críticos como extremos locais.

Mais precisamente, seja $c$ um ponto crítico de uma função contínua $f$ e diferenciável em todos os pontos de um intervalo aberto $(a, b)$ contendo $c$, exceto possivelmente no ponto $c$. Movendo-se no sentido positivo em $x$:
\begin{itemize}
\item se $f'(x)$ muda de negativa para positiva em $c$, então $f$ possui um mínimo local em $c$;
\item se $f'(x)$ muda de positiva para negativa em $c$, então $f$ possui um máximo local em $c$;
\item se $f'$ não muda de sinal em $c$, então $c$ não é um extremo local de $f$.
\end{itemize}
Veja a Figura \ref{fig:tder1}.

\begin{figure}[H]
  \centering
  \includegraphics[width=0.7\textwidth]{./cap_apderiv/dados/fig_tder1/fig_tder1}
  \caption{Ilustração do teste da primeira derivada com $c$ ponto de máximo local e $d$ ponto de mínimo local.}
  \label{fig:tder1}
\end{figure}

\begin{ex}
  Consideremos a função $\displaystyle f(x)=\frac{x^3}{3}-2x^2+3x+3$. Como $f$ é diferenciável em toda parte, seus pontos críticos são aqueles tais que
  \begin{equation}
    f'(x)=0.
  \end{equation}
  Temos $f'(x) = x^2 - 4x + 3$. Segue, que os pontos críticos são
  \begin{align}
    x^2-4x+3=0 &\Rightarrow x = \frac{4\pm \cancelto{2}{\sqrt{16-12}}}{2} \\
    &\Rightarrow x_1 = 1,\quad x_2=3.
  \end{align}
  Com isso, temos
  \begin{center}
  \begin{tabular}{lccc}\hline
    Intervalo & $x<1$ & $1<x<3$ & $3<x$ \\\hline
    $f'$ & + & - & + \\
    $f$ & crescente & decrescente & crescente\\\hline
  \end{tabular}
\end{center}
  Então, do teste da primeira derivada, concluímos que $x_1=1$ é ponto de máximo local e que $x_2=3$ é ponto de mínimo local.

  \ifispython
  Podemos usar o \sympy~ para computarmos a derivada de $f$ com o comando\footnote{Veja a Observação \ref{obs:cap_apderiv_python}.}
\begin{verbatim}
fl = diff(x**3/3-2*x**2+3*x+3)
\end{verbatim}
  Então, podemos resolver $f'(x)=0$ com o comando
\begin{verbatim}
solve(fl)
\end{verbatim}
  e, por fim, podemos fazer o estudo de sinal da $f'$ com os comandos
\begin{verbatim}
reduce_inequalities(fl<0)
reduce_inequalities(fl>0)
\end{verbatim}
  \fi
\end{ex}

\subsection*{Exercícios resolvidos}

\begin{flushright}
  [Vídeo] | [Áudio] | \href{https://phkonzen.github.io/notas/contato.html}{[Contatar]}
\end{flushright}

\begin{exeresol}
  Determine e classifique os extremos da função
  \begin{equation}
    f(x) = x^4 - 4x^3 + 4x^2.
  \end{equation}
\end{exeresol}
\begin{resol}
  Como o domínio da $f$ é $(-\infty, \infty)$ e $f$ é diferenciável em toda parte, temos que seus extremos ocorrem em pontos críticos tais que
  \begin{equation}
    f'(x)=0.
  \end{equation}
  Resolvendo, obtemos
  \begin{align}
    4x^3-12x^2+8x=0 &\Rightarrow 4x(x^2-3x+2)=0
  \end{align}
  Logo,
  \begin{align}
    4x=0 \quad\text{ou}\quad &x^2-3x+2=0\\
    x_1 = 0                  &x = \frac{3\pm 1}{2}. \\
                             &x_2 = 1,\quad x_3=2
  \end{align}
  Portanto, os ponto críticos são $x_1=0$, $x_2=1$ e $x_3=2$. Fazendo o estudo de sinal da $f'$, temos
  \begin{center}
    \begin{tabular}{lcccc}\hline
                 & $x<0$ & $0<x<1$ & $1<x<2$ & $2<x$ \\\hline
      $4x$       & -       &     +       &     +      &   +  \\
      $x^2-3x+2$ & +       &     +       &     -      &   +   \\
      $f'(x)$    & -       &     +       &     -      &   +   \\
      $f$        & decrescente & crescente & decrescente & crescente \\\hline
    \end{tabular}\\
  \end{center}
  Então, do teste da primeira derivada, concluímos que $x_1=0$ é ponto de mínimo local, $x_2=-2$ é ponto de máximo local e $x_3=-1$ é ponto de mínimo local.

  \ifispython
  Podemos usar os seguintes comandos do \sympy\footnote{Veja a Observação \ref{obs:cap_apderiv_python}.} para resolvermos este exercício:
\begin{verbatim}
# f'
fl = Lambda(x, diff(x**4 - 4*x**3 + 4*x**2,x))
# f'(x) = 0
solve(fl(x))
# fl(x) < 0
reduce_inequalities(fl(x)<0)
# fl(x) > 0
reduce_inequalities(fl(x)>0)
\end{verbatim}
  \fi
\end{resol}

\begin{exeresol}
  Encontre o valor máximo global de $f(x) = (x-1)e^{-x}$.
\end{exeresol}
\begin{resol}
  Como $f$ é diferenciável em toda parte, temos que seu máximo ocorre em ponto crítico tal que
  \begin{align}
    f'(x) = 0 &\Rightarrow (2-x)e^{-x} = 0 \\
              &\Rightarrow 2-x = 0 \\
              &\Rightarrow x = 2.
  \end{align}
  Fazendo o estudo de sinal da derivada, obtemos
  \begin{center}
    \begin{tabular}[H]{lcc}
         & x<0 & 0<x \\\hline
      f' & + & - \\
      f  & crescente & decrescente \\\hline
    \end{tabular}
  \end{center}
  Portanto, do teste da primeira derivada, podemos concluir que $x=2$ é ponto de máximo local. O favor da função neste ponto é $f(2) = e^{-2}$. Ainda, temos
  \begin{align}
    &\lim_{x\to -\infty} (x-1)e^{-x} = -\infty, \\
    &\lim_{x\to \infty} (x-1)e^{-x} = 0.
  \end{align}
  Por tudo isso, concluímos que o valor máximo global de $f$ é $f(2) = e^{-2}$.

  \ifispython
  Podemos usar os seguintes comandos do \sympy\footnote{Veja a Observação \ref{obs:cap_apderiv_python}.} para resolvermos este exercício:
\begin{verbatim}
# f(x)
f = Lambda(x, (x-1)*exp(-x))
# f'(x)
fl = Lambda(x, diff(f(x),x))
# pontos críticos
xc = solve(fl(x))
# f'(x) < 0
reduce_inequalities(fl(x)<0)
# f'(x) > 0
reduce_inequalities(fl(x)>0)
# lim f(x), x->-oo
limit(f(x),x,-oo)
# lim f(x), x->oo
limit(f(x),x,oo)
# f(2)
f(xc[0])
\end{verbatim}
  \fi
\end{resol}

% \begin{exer}\label{exeresol:ap}
%   Um terreno retangular deve ser cercado usando cercas diferentes. Em dois lados opostos deve ser aplicada uma cerca mais resistente que custa \$ 2 o metro, nos outros dois lados deve ser aplicada uma cerca que custa apenas \$ 1 o metro. Quais as dimensões da maior área que pode ser cercada a um custo total de \$ 4000.  
% \end{exer}
% \begin{resol}
%   Denotando por $x$ a largura e por $y$ o comprimento do retângulo, temos que sua área é dada por
%   \begin{equation}
%     A = xy
%   \end{equation}
%   Também, temos que a restrição do custo total nos fornece
%   \begin{equation}
%     2x
%   \end{equation}
% \end{resol}

\subsection*{Exercícios}

\begin{flushright}
  [Vídeo] | [Áudio] | \href{https://phkonzen.github.io/notas/contato.html}{[Contatar]}
\end{flushright}

\begin{exer}
  Use o teste da primeira derivada para encontrar e classificar o(s) ponto(s) extremo(s) de $f(x) = x^2 - 2x$.
\end{exer}
\begin{resp}
  $x=1$ ponto de mínimo global
\end{resp}

\begin{exer}
    Use o teste da primeira derivada para encontrar e classificar o(s) ponto(s) extremo(s) de $\displaystyle f(x) = \frac{x^3}{3}-x$.
\end{exer}
\begin{resp}
  $x_1=-1$ ponto de máximo local; $x_2=1$ ponto de mínimo local;
\end{resp}

\begin{exer}
  Use o teste da primeira derivada para encontrar e classificar o(s) ponto(s) extremo(s) de $\displaystyle f(x) = x^{2/3}(x-1)$.
\end{exer}
\begin{resp}
  $x_1=0$ ponto de máximo local; $x_2=2/5$ ponto de mínimo local;
\end{resp}


\section{Concavidade e o Teste da segunda derivada}\label{cap_apderiv_sec_tder2}

\begin{flushright}
  [Vídeo] | [Áudio] | \href{https://phkonzen.github.io/notas/contato.html}{[Contatar]}
\end{flushright}

O gráfico de uma função diferenciável $f$ é
\begin{enumerate}[a)]
\item {\bf côncavo para cima} em um intervalo aberto $I$, se $f'$ é crescente em $I$;
\item {\bf côncavo para baixo} em um intervalo aberto $I$, se $f'$ é decrescente em $I$.
\end{enumerate}

Assumindo que $f$ é duas vezes diferenciável, temos que a monotonicidade de $f'$ está relacionada ao sinal de $f''$ (a segunda derivada de $f$). Logo, o gráfico de $f$ é
\begin{enumerate}[a)]
\item {\bf côncavo para cima} em um intervalo aberto $I$, se $f'' > 0$ em $I$;
\item {\bf côncavo para baixo} em um intervalo aberto $I$, se $f'' < 0$ em $I$.
\end{enumerate}

\begin{ex}
  Vejamos os seguintes casos:
  \begin{enumerate}[a)]
  \item o gráfico de $f(x) = x^2$ é uma parábola côncava para cima em toda parte. De fato, temos
    \begin{equation}
      f'(x) = 2x,
    \end{equation}
    uma função crescente em toda parte. Também, temos
    \begin{equation}
      f''(x) = 2 > 0,
    \end{equation}
    em toda parte.
  \item o gráfico de $g(x) = -x^2$ é uma parábola côncava para baixo em toda parte. De fato, temos
    \begin{equation}
      g'(x) = -2x,
    \end{equation}
    uma função decrescente em toda parte. Também, temos
    \begin{equation}
      g''(x) = -2 < 0,
    \end{equation}
    em toda parte.
  \item o gráfico da função $h(x) = x^3$ é côncavo para baixo em $(-\infty, 0)$ e côncavo para cima em $(0, \infty)$. De fato, temos
    \begin{equation}
      h'(x) = x^2,
    \end{equation}
    que é uma função decrescente em $(-\infty, 0]$ e crescente em $[0, \infty)$. Também, temos
    \begin{equation}
      h''(x) = 2x
    \end{equation}
    que assume valores negativos em $(-\infty, 0)$ e valores positivos em $(0, \infty)$.
  \end{enumerate}
\end{ex}

Um ponto em que o gráfico de uma função $f$ muda de concavidade é chamado de {\bf ponto de inflexão}. Em tais pontos temos
\begin{equation}
  f'' = 0\quad\text{ou}\quad\nexists f''.
\end{equation}

\begin{ex}
  Vejamos os seguintes casos:
  \begin{enumerate}[a)]
  \item O gráfico da função $f(x) = x^3$ tem $x=0$ como único ponto de inflexão. De fato, temos
    \begin{equation}
      f'(x) = 3x^2
    \end{equation}
    que é diferenciável em toda parte com
    \begin{equation}
      f''(x) = 6x.
    \end{equation}
    Logo, os pontos de inflexão ocorrem quando
    \begin{align}
      f''(x) = 0 &\Rightarrow 6x = 0 \\
                 &\Rightarrow x = 0.
    \end{align}
  \item O gráfico da função $g(x) = \sqrt[3]{x}$ tem $x=0$ como único ponto de inflexão. De fato, temos
    \begin{equation}
      g'(x) = \frac{1}{3}x^{-\frac{2}{3}},\quad x\neq 0.
    \end{equation}
    Segue que
    \begin{equation}
      g''(x) = -\frac{2}{9}x^{-\frac{5}{3}},\quad x\neq 0,
    \end{equation}
    donde $g'' > 0$ em $(-\infty, 0)$e $g'' < 0$ em $(0, \infty)$. Isto é, o gráfico de $g$ muda de concavidade em $x=0$, $\nexists g''(0)$, sendo $g$ côncava para cima em $(-\infty, 0)$ e côncava para baixo em $(0, \infty)$.
\end{enumerate}
\end{ex}

\subsection{Teste da segunda derivada}

\begin{flushright}
  [Vídeo] | [Áudio] | \href{https://phkonzen.github.io/notas/contato.html}{[Contatar]}
\end{flushright}

Seja $x=x_0$ um ponto crítico de uma dada função $f$ duas vezes diferenciável e $f''$ contínua em um intervalo aberto contendo $x=x_0$. Temos
\begin{enumerate}[a)]
\item se $f'(x_0) = 0$ e $f''(x_0) > 0$, então $x=x_0$ é um ponto de mínimo local de $f$;
\item se $f'(x_0) = 0$ e $f''(x_0) < 0$, então $x=x_0$ é um ponto de máximo local de $f$.
\end{enumerate}

\begin{ex}
  A função $f(x) = 2x^3 - 9x^2 + 12x - 2$ tem pontos críticos
  \begin{align}
    f'(x) = 6x^2 - 18x + 12 = 0 &\Rightarrow x^2 - 3x + 2 = 0 \\
                                &\Rightarrow x = \frac{3 \pm \sqrt{1}}{2}\\
                                &\Rightarrow x_1 = 1,\quad x_2 = 2.
  \end{align}
  A segunda derivada de $f$ é
  \begin{equation}
    f''(x) = 12x - 18.
  \end{equation}
  Logo, como $f''(x_1) = f''(1) = -6 < 0$, temos que $x_1 = 1$ é ponto de máximo local de $f$. E, como $f''(x_2) = f''(2) = 6 > 0$, temos que $x_2 = 2$ é ponto de mínimo local de $f$.
\end{ex}

\begin{obs}
  Se $f'(x_0) = 0$ e $f''(x_0) = 0$, então $x=x_0$ pode ser ponto extremo local de $f$ ou não. Ou seja, o teste é inconclusivo.
\end{obs}

\begin{ex}
  Vejamos os seguintes casos:
  \begin{enumerate}[a)]
  \item A função $f(x) = x^3$ tem ponto crítico
    \begin{align}
      f'(x) = 0 &\Rightarrow 3x^2 = 0 \\
                &\Rightarrow x = 0.
    \end{align}
    Neste ponto, temos
    \begin{equation}
      f''(x) = 6x \Rightarrow f''(0) = 0.
    \end{equation}
    Neste caso, $x=0$ não é ponto de extremo local e temos $f'(0) = 0$ e $f''(0) = 0$.
  \item A função $f(x) = x^4$ tem um ponto crítico
    \begin{align}
      f'(x) = 0 &\Rightarrow 4x^3 = 0 \\
                &\Rightarrow x = 0.
    \end{align}
    Neste ponto, temos
    \begin{equation}
      f''(x) = 12x^2 \Rightarrow f''(0) = 0.
    \end{equation}
    Neste caso, $x=0$ é ponto de mínimo local e temos $f'(0)=0$ e $f''(0) = 0$.
  \end{enumerate}
\end{ex}

\subsection*{Exercícios resolvidos}

\begin{flushright}
  [Vídeo] | [Áudio] | \href{https://phkonzen.github.io/notas/contato.html}{[Contatar]}
\end{flushright}

\begin{exeresol}
  Encontre o valor máximo global de $f(x) = (x-1)e^{-x}$.
\end{exeresol}
\begin{resol}
  Como $f$ é diferenciável em toda parte, temos que seu valor máximo (se existir) ocorre em ponto crítico tal que
  \begin{align}
    f'(x) = 0 &\Rightarrow (2-x)e^{-x} = 0 \\
              &\Rightarrow 2-x = 0 \\
              &\Rightarrow x = 2.
  \end{align}
  Agora, usando o teste da segunda derivada, temos
  \begin{align}
    f''(x) = (x-3)e^{-x} \Rightarrow f''(2) = -e^{-2} < 0.
  \end{align}
  Logo, $x=2$ é ponto de máximo local. O favor da função neste ponto é $f(2) = e^{-2}$. Ainda, temos
  \begin{align}
    &\lim_{x\to -\infty} (x-1)e^{-x} = -\infty, \\
    &\lim_{x\to \infty} (x-1)e^{-x} = 0.
  \end{align}
  Por tudo isso, concluímos que o valor máximo global de $f$ é $f(2) = e^{-2}$.

  \ifispython
  Podemos usar os seguintes comandos do \sympy\footnote{Veja a Observação \ref{obs:cap_apderiv_python}.} para resolvermos este exercício:
\begin{verbatim}
>>> f = (x-1)*exp(-x)
>>> fl = diff(f,x)
>>> f = Lambda(x, (x-1)*exp(-x))
>>> fl = Lambda(x, diff(f(x),x))
>>> solve(fl(x))
[2]
>>> fll = Lambda(x, diff(f(x),x,2))
>>> fll(2)
  -2
-e  
>>> f(2), fl(2), fll(2)
 -2       -2
e  , 0, -e  
>>> limit(f(x),x,oo)
0
>>> limit(f(x),x,-oo)
-oo
\end{verbatim}
  \fi
\end{resol}

\begin{exeresol}
  Determine e classifique os extremos da função
  \begin{equation}
    f(x) = x^4 - 4x^3 + 4x^2
  \end{equation}
  restrita ao intervalo de $[-1, 3]$.
\end{exeresol}
\begin{resol}
  Como $f$ é diferenciável em $(-1, 3)$, temos que seus extremos locais ocorrem nos seguintes pontos críticos
  \begin{align}
    f'(x) = 0 &\Rightarrow 4x^3-12x^2+8x = 0 \\
              &\Rightarrow 4x(x^2-3x+2)=0 \\
              &\Rightarrow x_1 = 0, x_2 = 1, x_3 = 2.
  \end{align}
  Calculando a segunda derivada de $f$, temos
  \begin{equation}
    f''(x) = 12x^2 - 24x + 8.
  \end{equation}
  Do teste da segunda derivada, temos
  \begin{align}
    & f''(x_1) = f''(0) = 8 > 0 \Rightarrow x_1=0 ~ \text{pto. mín. local} \\
    & f''(x_2) = f''(1) = -4 < 0 \Rightarrow x_2=1 ~ \text{pto. máx. local} \\
    & f''(x_3) = f''(2) = 8 > 0 \Rightarrow x_3=2 ~ \text{pto. mín. local} 
  \end{align}
  Agora, vejamos os valores de $f$ em cada ponto de interesse.
  \begin{center}
    \begin{tabular}{l|ccccc}
      $x$     & $-1$ & $0$ & $1$ & $2$ & $3$ \\\hline
      $f(x)$  & $9$    & $0$   & $1$   & $0$ & $9$ \\
    \end{tabular}\\
  \end{center}
  Então, podemos concluir que $x=-1$ e $x=3$ são pontos de máximo global (o valor máximo global é $f(-1)=f(3)=9$), $x=1$ é ponto de máximo local, $x=0$ e $x=2$ são pontos de mínimo global (o valor mínimo global é $f(0)=f(2)=0$).
  
  \ifispython
  Podemos usar os seguintes comandos do \sympy\footnote{Veja a Observação \ref{obs:cap_apderiv_python}.} para resolvermos este exercício:
\begin{verbatim}
>>> f = Lambda(x, x**4 - 4*x**3 + 4*x**2)
>>> fl = Lambda(x, diff(f(x),x))
>>> solve(fl(x))
[0, 1, 2]
>>> fll = Lambda(x, diff(fl(x),x))
>>> fll(0), fll(1), fll(2)
(8, -4, 8)
>>> f(-1), f(0), f(1), f(2), f(3)
(9, 0, 1, 0, 9)
\end{verbatim}
  \fi
\end{resol}

\subsection*{Exercícios}

\begin{flushright}
  [Vídeo] | [Áudio] | \href{https://phkonzen.github.io/notas/contato.html}{[Contatar]}
\end{flushright}

\begin{exer}
  Use o teste da segunda derivada para encontrar e classificar o(s) ponto(s) extremo(s) de $f(x) = x^2 - 2x$.
\end{exer}
\begin{resp}
  $x=1$ ponto de mínimo global
\end{resp}

\begin{exer}
  Use o teste da segunda derivada para encontrar e classificar o(s) ponto(s) extremo(s) de $\displaystyle f(x) = \frac{x^3}{3}-x$.
\end{exer}
\begin{resp}
  $x_1=-1$ ponto de máximo local; $x_2=1$ ponto de mínimo local;
\end{resp}

\begin{exer}
  Use o teste da segunda derivada para encontrar e classificar o(s) ponto(s) extremo(s) de $\displaystyle f(x) = x^{2/3}(x-1)$.
\end{exer}
\begin{resp}
  $x_1=0$ ponto de máximo local; $x_2=2/5$ ponto de mínimo local;
\end{resp}

\begin{exer}
  Seja $f(x) = -x^4$. Mostre que $x=0$ é ponto de máximo local de $f$ e que $f'(0) = f''(0) = 0$.
\end{exer}
\begin{resp}
  $f'(x) = -4x^3$, $f'(0)=0$. Pelo teste da 1. derivada, temos que $x=0$ é ponto de máximo local. $f''(x) = -12x^2$, $f''(0)=0$.
\end{resp}
