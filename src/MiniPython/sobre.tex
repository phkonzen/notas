%Este trabalho está licenciado sob a Licença Atribuição-CompartilhaIgual 4.0 Internacional Creative Commons. Para visualizar uma cópia desta licença, visite http://creativecommons.org/licenses/by-sa/4.0/deed.pt_BR ou mande uma carta para Creative Commons, PO Box 1866, Mountain View, CA 94042, USA.

\section{Sobre a Linguagem}\label{sec_sobrepy}

\hl{{\python} é uma \emph{linguagem de programação} de \emph{alto nível} e \emph{multiparadigma}}. Ou seja, é relativamente próxima das linguagens humanas naturais, é desenvolvida para aplicações diversas e permite a utilização de diferentes paradigmas de programação (programação estruturada, orientada a objetos, orientada a eventos, paralelização, etc.).

\begin{itemize}
\item \hlemph{Site oficial}
  \begin{center}
    \href{https://www.python.org/}{https://www.python.org/}
  \end{center}
\end{itemize}

\subsection{Instalação e Execução}

\hl{Para executar um código {\python} em seu computador é necessário instalar um \emph{interpretador}}. No \href{https://www.python.org/}{site oficial}, estão disponíveis para \textit{download} interpretadores gratuitos e com licença livre para uso. Neste minicurso, vamos utilizar \emph{Python 3}.

\subsubsection{Online Notebook}

\hl{Usar um \emph{\textit{Notebook} {\python} \textit{online}} é uma forma rápida e prática de iniciar os estudos na linguagem}. Rodam diretamente em nuvem e vários permitem o uso gratuito por tempo limitado. Algumas opções são:
\begin{itemize}
\item \href{https://deepnote.com}{Deepnote} - \url{https://deepnote.com}
\item \href{https://colab.research.google.com/}{Google Colab} - \url{https://colab.research.google.com/}
\item \href{https://www.kaggle.com/}{Kaggle} - \url{https://www.kaggle.com/}
\item \href{https://www.paperspace.com/notebooks}{Paperspace Gradient} - \url{https://www.paperspace.com/notebooks}
\item \href{https://aws.amazon.com/sagemaker/}{SageMaker} - \url{https://aws.amazon.com/sagemaker}
\end{itemize}

\subsubsection{IDE}

\hl{Usar um \emph{ambiente integrado de desenvolvimento} (IDE, em inglês, \textit{integrated development environment}) é a melhor forma de capturar o todo o potencial da linguagem {\python}}. Algumas alternativas são:
\begin{itemize}
\item \href{https://docs.python.org/3/library/idle.html}{IDLE} - \url{https://docs.python.org/3/library/idle.html}
\item \href{https://www.gnu.org/software/emacs/download.html}{GNU Emacs} - \url{https://www.gnu.org/software/emacs/}
\item \href{https://www.spyder-ide.org/}{Spyder} - \url{https://www.spyder-ide.org/}
\item \href{https://code.visualstudio.com/}{VS Code} - \url{https://code.visualstudio.com/}
\end{itemize}

\subsection{Utilização}

A \hl{execução de códigos {\python}} pode ser feita de três formas básicas:
\begin{itemize}
\item \hl{em modo interativo em um console/\textit{notebook} {\python}};
\item \hl{por execução de um código \texttt{arqnome.py} em um console/\textit{notebook} {\python}};
\item \hl{por execução de um cógido \texttt{arqnome.py} em um terminal do sistema operacional}.
\end{itemize}

\begin{ex}
  Consideramos o seguinte pseudocódigo.

\begin{verbatim}
s = "Ola, mundo!".
imprime(s). (imprime a string s)
\end{verbatim}

Vamos escrevê-lo em {\python} e executá-lo:

\begin{enumerate}[a)]
  \item Em um \textit{notebook}.
  
  Iniciamos um \textit{notebook} {\python} e digitamos o seguinte código em uma célula de entrada.

\begin{lstlisting}
s = "Olá, Mundo!"
#imprime a string s
print(s)
\end{lstlisting}

  Ao executarmos a célula, obtemos a saída

\begin{verbatim}
Olá, Mundo!
\end{verbatim}

  \item Em modo iterativo no console.
  
  Iniciamos um console {\python} em terminal do sistema e digitamos

\begin{verbatim}
$ python3
\end{verbatim}

Aqui, \texttt{\$} é o símbolo de \textit{prompt} de entrada que pode ser diferente a depender do seu sistema operacional. Então, digitamos

\begin{lstlisting}
>>> s = "Olá, Mundo!"
>>> print(s) #imprime a string s
\end{lstlisting}

Observamos que \texttt{>>>} é o símbolo de \textit{prompt} de entrada do console {\python}. A saída 

\begin{lstlisting}
Olá, Mundo!
\end{lstlisting}

aparece logo abaixo da última linha de \textit{prompt} executada. Para encerrar o console, digitamos

\begin{lstlisting}
>>> quit()
\end{lstlisting}
  
  \item Escrevendo o código \verb+ola.py+ e executando-o em um console/\textit{notebook} {\python}.
  
  Primeiramente, escrevemos o código

\begin{lstlisting}
s = "Olá, Mundo!"
print(s) # imprime a string s
\end{lstlisting}

em um IDE (ou em um simples editor de texto) e salvamo-lo no caminho \texttt{/caminho/ola.py}. Então, o executamos no console/\textit{notebook} {\python} com

\begin{lstlisting}
>>> exec(open('/pasta/codigo.py').read())
\end{lstlisting}

A saída é impressa logo abaixo do \textit{prompt}/célula de entrada.
  
  \item Escrevendo o código \verb+ola.py+ e executando-o em terminal do sistema.
  
  Assumindo que o código já esteja salvo no arquivo \texttt{/caminho/ola.py}, podemos executá-lo em um terminal digitando

\begin{lstlisting}
$ python3 /caminho/ola.py
\end{lstlisting}

  A saída é impressa logo abaixo do \textit{prompt} de entrada do sistema.
\end{enumerate}

\end{ex}
