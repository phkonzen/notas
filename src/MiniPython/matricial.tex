%Este trabalho está licenciado sob a Licença Atribuição-CompartilhaIgual 4.0 Internacional Creative Commons. Para visualizar uma cópia desta licença, visite http://creativecommons.org/licenses/by-sa/4.0/deed.pt_BR ou mande uma carta para Creative Commons, PO Box 1866, Mountain View, CA 94042, USA.

\section{Elementos da Computação Matricial}\label{sec_matricial}

Nesta seção, vamos explorar a \hl{{\numpy} (Numerical Python), biblioteca para tratamento numérico de dados}. Ela é extensivamente utilizada nos mais diversos campos da ciência e da engenharia. Aqui, vamos nos restringir a introduzir algumas de suas ferramentas para a computação matricial.

Usualmente, \hl{a biblioteca é importada como segue}

\begin{lstlisting}
import numpy as np
\end{lstlisting}

\subsection{NumPy \texttt{array}}

\hl{Um {\PYTHONnumpyDOTarray} é uma tabela de valores} (vetor, matriz ou multidimensional) e contém informação sobre os dados brutos, indexação e como interpretá-los. \hl{Os elementos são todos do mesmo tipo} (diferente de uma lista {\python}), referenciados pela propriedade \texttt{dtype}. A \emph{indexação} dos elementos pode ser feita por um \texttt{tuple} de inteiros não negativos, por booleanos, por outro {\PYTHONnumpyDOTarray} ou por números inteiros. O \texttt{ndim} de um {\PYTHONnumpyDOTarray} é seu \emph{número de dimensões} (chamadas de \texttt{axes}\endnote{\texttt{axes}, do inglês, plural de \textit{axis}, eixo.}). O {\PYTHONnumpyDOTndarrayDOTshape} é um \texttt{tuple} de inteiros que fornece seu \emph{tamanho (número de elementos) em cada dimensão}. Sua inicialização pode ser feita usando-se listas simples ou encadeadas. Por exemplo,

\begin{lstlisting}
a = np.array([1, 3, -1, 2])
print(a)
\end{lstlisting}

\begin{verbatim}
[ 1  3 -1  2]
\end{verbatim}

\begin{lstlisting}
a.dtype
\end{lstlisting}

\begin{verbatim}
dtype('int64')
\end{verbatim}

\begin{lstlisting}
a.shape
\end{lstlisting}

\begin{verbatim}
(4,)
\end{verbatim}

\begin{lstlisting}
a[2]
\end{lstlisting}

\begin{verbatim}
-1
\end{verbatim}

\begin{lstlisting}
a[1:3]
\end{lstlisting}

\begin{verbatim}
array([ 3, -1])
\end{verbatim}

temos um {\PYTHONnumpyDOTarray} de números inteiros com quatro elementos dispostos em um único \texttt{axis} (eixo). Podemos interpretá-lo como uma representação de um vetor linha ou coluna, i.e.
\begin{equation}
  a = (1, 3, -1, 2)
\end{equation}
vetor coluna ou $a^T$ vetor linha.

Outro exemplo,

\begin{lstlisting}
a = np.array([[1.0,2,3],
              [-3,-2,-1]])
a.dtype
\end{lstlisting}

\begin{verbatim}
dtype('float64')
\end{verbatim}

\begin{lstlisting}
a.shape
\end{lstlisting}

\begin{verbatim}
(2, 3)
\end{verbatim}

\begin{lstlisting}
a[1,1]
\end{lstlisting}

\begin{verbatim}
-2.0
\end{verbatim}

temos um {\PYTHONnumpyDOTarray} de números decimais (\texttt{float}) dispostos em um arranjo com dois \texttt{axes} (eixos). O primeiro \texttt{axis} tem tamanho $2$ e o segundo tem tamanho $3$. Ou seja, podemos interpretá-lo como uma matriz de duas linhas e três colunas. Podemos fazer sua representação algébrica como
\begin{equation}
  a =
  \begin{bmatrix}
    1 & 2 & 3\\
    -3 & -2 & -1 
  \end{bmatrix}
\end{equation}

\begin{exer}
Use {\PYTHONnumpyDOTarray} para alocar:
\begin{enumerate}[a)]
  \item o vetor
  \begin{equation}
    v = \left(-5, \pi, \sen(\pi/3)\right)
  \end{equation}
  \item a matriz
  \begin{equation}
    A = \begin{bmatrix}
    -1 & \displaystyle\frac{1}{3}\\[1em]
    2 & \displaystyle\sqrt{2}\\[1em]
    \displaystyle e^{-1} & -3\\
    \end{bmatrix}
  \end{equation}
\end{enumerate}
\end{exer}
\begin{resp}
  
\begin{lstlisting}
import numpy as np
# a)
v = np.array([-5., np.pi, np.sin(np.pi/3)])
print('v = ', v)
# b)
A = np.array([[-1., 1./3],
              [2., np.sqrt(2)],
              [np.exp(-1.), -3.]])
print('A = \n', A)  
\end{lstlisting}

\end{resp}

\subsubsection{Inicialização de um array}\label{subsubsection:iniarray}

O {\numpy} conta com úteis funções de inicialização de {\PYTHONnumpyDOTarray}. Vejam algumas das mais frequentes:
\begin{itemize}
\item \hl{\PYTHONnumpyDOTzeros}: inicializa um {\PYTHONnumpyDOTarray} com todos seus elementos iguais a zero.
  
\begin{lstlisting}
np.zeros(2)
\end{lstlisting}

\begin{verbatim}
array([0., 0.])
\end{verbatim}

  \item \hl{\PYTHONnumpyDOTones}: inicializa um {\PYTHONnumpyDOTarray} com todos seus elementos iguais a $1$.

\begin{lstlisting}
np.ones((3,2), dtype='int')
\end{lstlisting}

\begin{verbatim}
array([[1, 1],
      [1, 1],
      [1, 1]])
\end{verbatim}

  \item \hl{\PYTHONnumpyDOTempty}: inicializa um {\PYTHONnumpyDOTarray} sem alocar valores para seus elementos\endnote{Atenção! No momento da alocação, os valores dos elementos serão dinâmicos conforme ``lixo'' da memória.}.
  
\begin{lstlisting}
np.empty(3)
\end{lstlisting}

\begin{verbatim}
array([4.9e-324, 1.5e-323, 2.5e-323])
\end{verbatim}

  \item \hl{\PYTHONnumpyDOTarange}: inicializa um {\PYTHONnumpyDOTarray} com uma sequência de elementos\endnote{Similar à função Python \texttt{range}.}.

\begin{lstlisting}
np.arange(1,6,2)
\end{lstlisting}

\begin{verbatim}
array([1, 3, 5])
\end{verbatim}

  \item \hl{\PYTHONnumpyDOTlinspace}\texttt{(a, b[, num=n])}: inicializa um {\PYTHONnumpyDOTarray} como uma sequência de elementos que começa em \texttt{a}, termina em \texttt{b} (incluídos) e contém \texttt{n} elementos igualmente espaçados.

\begin{lstlisting}
np.linspace(0, 1, num=5)
\end{lstlisting}

\begin{verbatim}
array([0.  , 0.25, 0.5 , 0.75, 1.  ])
\end{verbatim}

\end{itemize}

\begin{exer}
  Aloque a matriz escalar
  \begin{equation}
      A = \begin{bmatrix}
        -2 & 0 & 0\\
        0 & -2 & 0\\
        0 & 0 & -2
      \end{bmatrix}
  \end{equation}  
como um {\PYTHONnumpyDOTarray}.
\end{exer}
\begin{resp}
  
\begin{lstlisting}
import numpy as np
A = -2*np.ones((3,3))
print('A = \n', A)
\end{lstlisting}

\end{resp}

\begin{exer}
  Construa um {\PYTHONnumpyDOTarray} para alocar uma partição uniforme com $11$ pontos do intervalo $[0, 1]$. Ou seja, um arranjo $\pmb{x} = (x_1, x_2, \dotsc, x_n)$, de elementos $x_i = (i-1)h$, com passo $h = 1/(n-1)$. 
\end{exer}
\begin{resp}
  
\begin{lstlisting}
import numpy as np
x = np.linspace(0., 1., 11)
print('x = ', x)
\end{lstlisting}

\end{resp}

\subsubsection{Manipulação de \texttt{arrays}}

Outras duas funções importantes no tratamento de \texttt{arrays} são:
\begin{itemize}
\item \hl{\PYTHONnumpyDOTreshape}: permite a alteração da forma de um {\PYTHONnumpyDOTarray}.
  
\begin{lstlisting}
a = np.array([-2,-1])
print(a)
\end{lstlisting}

\begin{verbatim}
[-2 -1]
\end{verbatim}

\begin{lstlisting}
b = a.reshape(2,1)
print(b)
\end{lstlisting}

\begin{verbatim}
[[-2]
 [-1]]
\end{verbatim}

\hl{O {\PYTHONnumpyDOTreshape} também permite a utilização de um coringa \texttt{-1} que será dinamicamente determinado de forma obter-se uma estrutura adequada}. Por exemplo,

\begin{lstlisting}
a = np.array([[1,2],[3,4]])
print(a)
\end{lstlisting}

\begin{verbatim}
[[1 2]
 [3 4]]
\end{verbatim}

\begin{lstlisting}
b = a.reshape((-1,1))
print(b)
\end{lstlisting}

\begin{verbatim}
[[1]
 [2]
 [3]
 [4]]
\end{verbatim}

\item \hl{\PYTHONnumpyDOTtranspose}: computa a transposta de uma matriz.

\begin{lstlisting}
a = np.array([[1,2],[3,4]])
print(a)
\end{lstlisting}

\begin{verbatim}
[[1 2]
 [3 4]]
\end{verbatim}

\begin{lstlisting}
b = a.transpose()
print(b)
\end{lstlisting}

\begin{verbatim}
[[1 3]
 [2 4]]
\end{verbatim}

\item \hl{\PYTHONnumpyDOTconcatenate}: concatena \texttt{arrays}.

\begin{lstlisting}
a = np.array([1,2])
b = np.array([2,3])
c = np.concatenate((a,b))
print(c)
\end{lstlisting}

\begin{verbatim}
[1 2 2 3]
\end{verbatim}

\begin{lstlisting}
a = a.reshape((1,-1))
b = b.reshape((1,-1))
d = np.concatenate((a,b), axis=0)
print(d)
\end{lstlisting}

\begin{verbatim}
[[1 2]
 [2 3]]
\end{verbatim}

  \end{itemize}

\begin{exer}
  Aloque o seguinte vetor como um {\PYTHONnumpyDOTarray}
  \begin{equation}
    \pmb{a} = (2, 3, -1, 1, 4, -5). 
  \end{equation}
  Então, use o método {\PYTHONnumpyDOTreshape} para, a partir de \texttt{b}, alocar a matriz
  \begin{equation}
    A = \begin{bmatrix}
      -1 & 1 \\
      4 & 5
    \end{bmatrix}
  \end{equation}
  como um {\PYTHONnumpyDOTarray}.
\end{exer}
\begin{resp}
  
\begin{lstlisting}
import numpy as np
a = np.array([2, 3, -1, 1, 4, 5])
A = a.reshape((3,-1))
\end{lstlisting}

\end{resp}

\begin{exer}
  Tendo em vista que
  \begin{equation}
    A = \begin{bmatrix}
      \frac{\sqrt{3}}{2} & -\frac{1}{2} \\
      \frac{1}{2} & \frac{\sqrt{3}}{2}
    \end{bmatrix}
  \end{equation}
  é uma matriz ortogonal\endnote{$A$ é dita matriz ortogonal, quando $A^{-1} = A^T$.}, compute $A^{-1}$.
\end{exer}
\begin{resp}
  
\begin{lstlisting}
import numpy as np
A = np.array([[np.sqrt(3)/2, -1./2],
              [1./2, np.sqrt(3)/2]])
Ainv = A.transpose()
\end{lstlisting}

\end{resp}

\begin{exer}
  Considere o seguinte sistema de equações
  \begin{equation}
    \begin{aligned}
      2x_1 - x_2 &= 3 \\
      x_1 + 3x_2 &= -2
    \end{aligned}
  \end{equation}
  Use {\PYTHONnumpyDOTarray} para alocar:
  \begin{enumerate}
    \item a matriz de coeficientes deste sistema.
    \item o vetor dos termos constantes deste sistema.
    \item a matriz estendida deste sistema.
  \end{enumerate}
\end{exer}
\begin{resp}

\begin{lstlisting}
import numpy as np
# a)
A = np.array([[2, -1],
              [1, 3]])
# b)
b = np.array([3, -2])
# c)
E = np.concatenate((A, b.reshape(-1,1)), axis=1)
\end{lstlisting}

\end{resp}

\subsubsection{Operadores Elemento-a-Elemento}\label{cap_matricial_sssec_op_elem-a-elem}

\hl{Os operadores aritméticos disponível no {\python} atuam elemento-a-elemento nos {\PYTHONnumpyDOTarrays}}. Por exemplo,

\begin{lstlisting}
a = np.array([1,2])
b = np.array([2,3])
a+b
\end{lstlisting}

\begin{verbatim}
array([3, 5])
\end{verbatim}

\begin{lstlisting}
a-b
\end{lstlisting}

\begin{verbatim}
array([-1, -1])
\end{verbatim}

\begin{lstlisting}
b*a
\end{lstlisting}

\begin{verbatim}
array([2, 6])
\end{verbatim}

\begin{lstlisting}
a**b
\end{lstlisting}

\begin{verbatim}
array([1, 8])
\end{verbatim}

\begin{lstlisting}
2*b
\end{lstlisting}

\begin{verbatim}
array([4, 6])
\end{verbatim}


\hl{O {\numpy} também conta com várias funções matemáticas elementares que operam elemento-a-elemento em \texttt{arrays}}. Por exemplo,

\begin{lstlisting}
a = np.array([np.pi, np.sqrt(2)])
a
\end{lstlisting}

\begin{verbatim}
array([3.14159265, 1.41421356])
\end{verbatim}

\begin{lstlisting}
np.sin(a)
\end{lstlisting}

\begin{verbatim}
array([1.22464680e-16, 9.87765946e-01])
\end{verbatim}

\begin{lstlisting}
np.exp(a)
\end{lstlisting}

\begin{verbatim}
array([23.14069263,  4.11325038])
\end{verbatim}

\begin{exer}
  Compute os valores da função cosseno para os elementos do vetor
  \begin{equation}
    \pmb{\theta} = \left(0., 30^\circ, 45^\circ, 60^\circ, 90^\circ\right).
  \end{equation}
\end{exer}
\begin{resp}
  
\begin{lstlisting}
import numpy as np
thta = np.array([0., np.pi/6, np.pi/4, 
                 np.pi/3, np.pi/2])
y = np.cos(thta)
\end{lstlisting}

\end{resp}

 
\ifisbook 
\subsubsection*{Respostas dos Exercícios}
\shipoutAnswer
\fi


%%% subsection
\subsection{Elementos da Álgebra Linear}

\hl{O {\PYTHONnumpy} conta com um módulo de álgebra linear}, usualmente importado com

\begin{lstlisting}
import numpy.linalg as npla
\end{lstlisting}

\subsubsection{Vetores}

\hl{Um vetor podem ser alocado usando um {\PYTHONnumpyDOTarray} de um eixo (dimensão)}. Por exemplo,
\begin{gather}
  x = (2, -1),\\
  y = (3, 1, \pi)
\end{gather}
podem ser alocados com

\begin{lstlisting}
x = np.array([2,-1])
print(x)
\end{lstlisting}

\begin{verbatim}
[ 2 -1]
\end{verbatim}

e

\begin{lstlisting}
y = np.array([3, 1, np.pi])
print(y)
\end{lstlisting}

\begin{verbatim}
  [3.         1.         3.14159265]
\end{verbatim}

\begin{exer}
  Aloque cada um dos seguintes vetores como um {\PYTHONnumpyDOTarray}:
  \begin{enumerate}
  \item[a)] $x = (1.2, -3.1, 4)$
  \item[c)] $z = (\pi, \sqrt{2}, e^{-2})$
  \end{enumerate}
\end{exer}
\begin{resp}
  
\begin{lstlisting}
import numpy as np
# a)
x = np.array([1.2, -3.1, 4])
print('x = ', x)
# b)
z = np.array([np.pi, np.sqrt(2.), np.exp(-2)])
print('z = ', z)
\end{lstlisting}

\end{resp}

\subsubsection{Produto Escalar e Norma}

Dados dois vetores
\begin{gather}
  x = (x_0, x_1, \dotsc, x_{n-1}),\\
  y = (y_0, y_1, \dotsc, y_{n-1}),
\end{gather}
define-se o \emph{produto escalar} por
\begin{equation}
  x\cdot y = x_0y_0 + x_1y_1 + \cdots + x_{n-1}y_{n-1}.
\end{equation}
Com o {\numpy}, podemos computá-lo com a função hl{\PYTHONnumpyDOTdot}. Por exemplo,

\begin{lstlisting}
x = np.array([-1, 0, 2])
y = np.array([0, 1, 1])
d = np.dot(x,y)
print(d)
\end{lstlisting}

\begin{verbatim}
2
\end{verbatim}

A norma $l_2$ de um vetor é definida por
\begin{equation}
  \|x\|_2 = \sqrt{\sum_{i=0}^{n-1} x_i^2}.
\end{equation}
O {\numpy} conta com o método \hl{\PYTHONnumpyDOTlinalgDOTnorm} para computá-la. Por exemplo,

\begin{lstlisting}
nrm = npla.norm(y)
print(nrm)
\end{lstlisting}

\begin{verbatim}
4.457533443631058
\end{verbatim}

\begin{exer}
  Faça um código para computar o produto escalar $x\cdot y$ sendo
  \begin{gather}
    x = (1.2, \ln(2), 4),\\
    y = (\pi^2, \sqrt{3}, e)
  \end{gather}
\end{exer}
\begin{resp}

\begin{lstlisting}
import numpy as np
x = np.array([1.2, np.log(2), 4])
y = np.array([np.pi**2, np.sqrt(3), np.e])
d = np.dot(x,y)
\end{lstlisting}

\end{resp}


\subsubsection{Matrizes}\label{sec_alglin}

\hl{Uma matriz pode ser alocada como um {\PYTHONnumpyDOTarray} de dois eixos (dimensões)}. Por exemplo, as matrizes
\begin{align}
  &A =
  \begin{bmatrix}
    2 & -1 & 7\\
    3 & 1 & 0
  \end{bmatrix},\label{sec_alglin:eq:A}\\
  &B =
  \begin{bmatrix}
    4 & 0\\
    2 & 1\\
   -8 & 6
  \end{bmatrix}\label{sec_alglin:eq:B}
\end{align}
podem ser alocadas como segue

\begin{lstlisting}
A = np.array([[2,-1,7],
              [3,1,0]])
print(A)
\end{lstlisting}

\begin{verbatim}
[[ 2 -1  7]
 [ 3  1  0]]
\end{verbatim}

e

\begin{lstlisting}
B = np.array([[4,0],
              [2,1],
              [-8,6]])
print(B)
\end{lstlisting}

\begin{verbatim}
[[ 4  0]
 [ 2  1]
 [-8  6]]
\end{verbatim}

Como já vimos, o {\numpy} conta com operadores elemento-a-elemento que podem ser utilizados na álgebra envolvendo \texttt{arrays}, logo também aplicáveis a matrizes (consulte a Subseção~\ref{cap_matricial_sssec_op_elem-a-elem}). Na sequência, vamos introduzir outras operações próprias deste tipo de objeto.

\begin{exer}
  Aloque cada uma das seguintes matrizes como um {\PYTHONnumpyDOTarray}:
  \begin{enumerate}[a)]
  \item
    \begin{equation}
      A =
      \begin{bmatrix}
        -1 & 2\\
        2 & -4\\
        6 & 0
      \end{bmatrix}
    \end{equation}
  \item $B = A^T$ 
  \end{enumerate}
\end{exer}
\begin{resp}

\begin{lstlisting}
import numpy as np
# a)
A = np.array([[-1, 2],
              [2, -4],
              [6, 0]])
# b)
B = A.transpose()
\end{lstlisting}

\end{resp}

\begin{exer}
  Seja

\begin{lstlisting}
A = np.array([[2,1],[1,1],[-3,-2]])
\end{lstlisting}

  Determine o formato (\texttt{shape}) dos seguintes \texttt{arrays}:
  \begin{enumerate}[a)]
  \item \texttt{A[:,0]}
  \item \texttt{A[:,0:1]}
  \item \texttt{A[1:3,0]}
  \item \texttt{A[1:3,0:1]}
  \item \texttt{A[1:3,0:2]}
  \end{enumerate}
\end{exer}
\begin{resp}
  
\begin{lstlisting}
import numpy as np
# a)
A = np.array([[2, 1],
              [1, 1],
              [-3, -2]])
print('a)', A[:,0].shape)
print('b)', A[:,0:1].shape)
print('c)', A[1:3,0].shape)
print('d)', A[1:3,0:1].shape)
print('e)', A[1:3,0:2].shape)
\end{lstlisting}

\end{resp}

\subsubsection{Inicialização de Matrizes}

Além das inicializações de \texttt{arrays} já estudadas na Subseção \ref{subsubsection:iniarray}, temos mais algumas que são particularmente úteis no caso de matrizes.
\begin{itemize}
  \item \hl{\PYTHONnumpyDOTeye}\texttt{(n)}: retorna a matriz identidade $n\times n$.

\begin{lstlisting}
I = np.eye(3)
print(I)
\end{lstlisting}

\begin{verbatim}
[[1. 0. 0.],
 [0. 1. 0.],
 [0. 0. 1.]]
\end{verbatim}

  \item \hl{\PYTHONnumpyDOTdiag}: extrai a diagonal ou constrói um {\PYTHONnumpyDOTarray} diagonal.

\begin{lstlisting}
D = np.diag([1,2,3])
print(D)
\end{lstlisting}

\begin{verbatim}
[[1, 0, 0],
 [0, 2, 0],
 [0, 0, 3]]
\end{verbatim}

\end{itemize}

\begin{exer}\label{sec_matricial:exer:laplaceprob}
  Aloque a matriz dos coeficientes e o vetor dos termos constantes do seguinte sistema de equações
  \begin{equation}
    \begin{aligned}
      & x_1 = 0 \\
      & -x_{i-1} + 2x_{i} - x_{i+1} = h^2f_i \\
      & x_n = 0
    \end{aligned}
  \end{equation}
  onde $f_i = \pi^2\sin(\pi x_i)$, $x_i = (i-1)h$, $h = 1/(n-1)$, n=5.
\end{exer}
\begin{resp}
  
\begin{lstlisting}
import numpy as np
n = 5
h = 1./(n-1)
x = np.linspace(0., 1., n)
b = np.pi**2*np.sin(np.pi*x)
A = np.diag(2*np.ones(n)) + \
    np.diag(-np.ones(n-1), k=-1) + \
    np.diag(-np.ones(n-1), k=1)
A[0,0] = 1.
A[0,1] = 0.
A[n-1,n-2] = 0.
A[n-1,n-1] = 1.
print('A = \n', A)
print('b = \n', b)
\end{lstlisting}

\end{resp}


\subsubsection{Multiplicação de Matrizes}

A multiplicação da matriz $A = [a_{ij}]_{i,j=0}^{n-1,l-1}$ pela matriz $B = [b_{ij}]_{i,j=0}^{l-1,m-1}$ é a matriz $C = AB = [c_{ij}]_{i,j=0}^{n-1,m-1}$ tal que
\begin{equation}
  c_{ij} = \sum_{k=0}^{l-1} a_{ik}b_{k,j}
\end{equation}
\hl{O {\PYTHONnumpy} tem a função {\PYTHONnumpyDOTmatmul} para computar a multiplicação de matrizes}. Por exemplo, a multiplicação das matrizes dadas em \eqref{sec_alglin:eq:A} e \eqref{sec_alglin:eq:B}, computamos

\begin{lstlisting}
C = np.matmul(A,B)
print(C)
\end{lstlisting}

\begin{verbatim}
[[-50  41],
 [ 14   1]]
\end{verbatim}

\begin{obs}[\hl{\texttt{matmul}, \texttt{*}, \texttt{@}}]
  É importante notar que {\PYTHONnumpyDOTmatmul}\texttt{(A,B)} é a multiplicação de matrizes, enquanto que \texttt{*} consiste na multiplicação elemento a elemento. Alternativamente a {\PYTHONnumpyDOTmatmul}\texttt{(A,B)} pode-se usar \texttt{A @ B}.
\end{obs}

\begin{exer}
  Aloque as matrizes
  \begin{gather}
    C =
    \begin{bmatrix}
      1 & 2 & -1 \\
      3 & 2 & 1 \\
      0 & -2 & -3
    \end{bmatrix}\\
    D =
    \begin{bmatrix}
      2 & 3 \\
      1 & -1 \\
      6 & 4
    \end{bmatrix}\\
    E =
    \begin{bmatrix}
      1 & 2 & 1 \\
      0 & -1 & 3
    \end{bmatrix}
  \end{gather}
  Então, se existirem, compute e forneça as dimensões das seguintes matrizes
  \begin{enumerate}[a)]
  \item $CD$
  \item $D^TE$
  \item $D^TC$
  \item $DE$
  \end{enumerate}
\end{exer}
\begin{resp}
  
\begin{lstlisting}
import numpy as np
C = np.array([[1, 2, -1],
              [3, 2, 1],
              [0, -2, -3]])
D = np.array([[2, 3],
              [1, -1],
              [6, 4]])
E = np.array([[1, 2, 1],
              [0, -1, 3]])
print('a) CD = \n', C@D)
print("b) não existe D'E")
print("c) D'C = \n", C.T@C)
print("d) DE = \n", D@E)
\end{lstlisting}

\end{resp}

\subsubsection{Traço e Determinante de uma Matriz}

O {\PYTHONnumpy} tem a função {\PYTHONnumpyDOTndarrayDOTtrace} para computar o \hlemph{traço} de uma matriz (soma dos elementos de sua diagonal). Por exemplo,

\begin{lstlisting}
A = np.array([[-1,2,0],[2,3,1],[1,2,-3]])
print('tr(A) = ', A.trace())
\end{lstlisting}

\begin{verbatim}
tr(A) =  -1
\end{verbatim}

Já, o \hlemph{determinante} é fornecido no módulo {\PYTHONnumpyDOTlinalg}. Por exemplo,

\begin{lstlisting}
A = np.array([[-1,2,0],[2,3,1],[1,2,-3]])
print('det(A) = ', npla.det(A))
\end{lstlisting}

\begin{verbatim}
det(A) =  25.000000000000007
\end{verbatim}

\begin{exer}
  Compute a solução do seguinte sistema de equações
  \begin{equation}
    \begin{aligned}
      x_1 - x_2 + x_3 &= -2 \\
      2x_1 + 2x_2 + x_3 &= 5 \\
      -x_1 - x_2 + 2x_3 &= -5
    \end{aligned}
  \end{equation}
  pelo método de Cramer{\cramer}.
\end{exer}
\begin{resp}
  
\begin{lstlisting}
import numpy as np
import numpy.linalg as npla
# matriz dos coefs
A = np.array([[1, -1, 1],
              [2, 2, 1],
              [-1, -1, 2]])
# vetor dos termos consts
b = np.array([-2, 5, -5])
# mat aux A1
A1 = A.copy()
A1[:,0] = b
# sol x1
x1 = npla.det(A1)/npla.det(A)
print('x1 = ', x1)
# mat aux A2
A2 = A.copy()
A2[:,1] = b
# sol x2
x2 = npla.det(A2)/npla.det(A)
print('x2 = ', x2)
# mat aux A3
A3 = A.copy()
A3[:,2] = b
# sol x3
x3 = npla.det(A3)/npla.det(A)
print('x3 = ', x3)
\end{lstlisting}

\end{resp}


\subsubsection{Rank e Inversa de uma Matriz}

O \hlemph{rank} de uma matriz é o número de linhas ou colunas linearmente independentes. O {\PYTHONnumpy} conta com a função \hl{\PYTHONnumpyDOTlinalgDOTmatrixrank} para computá-lo. Por exemplo,

\begin{lstlisting}
npla.matrix_rank(np.eye(3))
\end{lstlisting}

\begin{verbatim}
3
\end{verbatim}

\begin{lstlisting}
A = np.array([[1,2,3],[-1,1,-1],[0,3,2]])
npla.matrix_rank(A)
\end{lstlisting}

\begin{verbatim}
2
\end{verbatim}


A \hlemph{inversa de uma matriz} \textit{full rank} pode ser computada com a função \hl{\PYTHONnumpyDOTlinalgDOTinv}. Por exemplo,

\begin{lstlisting}
A = np.array([[1, 2, 3],
              [-1, 1, -1],
              [1, 3, 2]])
Ainv = np.linalg.inv(A)
print('Ainv @ A = \n', Ainv @ A)
\end{lstlisting}


\begin{verbatim}
Ainv @ A = 
 [[ 1.00000000e+00 -2.22044605e-16 -8.88178420e-16]
  [ 0.00000000e+00  1.00000000e+00  0.00000000e+00]
  [ 0.00000000e+00 -2.22044605e-16  1.00000000e+00]]
\end{verbatim}

\begin{exer}
  Compute, se possível, a matriz inversa de cada uma das seguintes matrizes
  \begin{align}
    & B =
    \begin{bmatrix}
      2 & -1\\
      -2 & 1
    \end{bmatrix}\\
    & C =
    \begin{bmatrix}
      -2 & 0 & 1\\
      3 & 1 & -1\\
      2 & 1 & 0
    \end{bmatrix}
  \end{align}
  Verifique suas respostas.
\end{exer}
\begin{resp}
  
\begin{lstlisting}
import numpy as np
import numpy.linalg as npla

def inv(A):
  if (npla.matrix_rank(A) == A.shape[1]):
    return npla.inv(A)
  else:
    print('Matriz não invertível.')
    return None

B = np.array([[2, -1],
              [-2, 1]])
print('inv(B) = \n', inv(B))

A = np.array([[-2, 0, 1],
              [3, 1, -1],
              [2, 1, 0]])
print('inv(A) = \n', inv(A))  
\end{lstlisting}

\end{resp}


\subsubsection{Autovalores e Autovetores de uma Matriz}

Um \hlemph{auto-par} $(\lambda, v)$ de uma matriz $A$, $\lambda$ um escalar chamado de \hlemph{autovalor} e $v\neq 0$ é um vetor chamado de \hlemph{autovetor}, é tal que
\begin{equation}
  A\lambda = \lambda v.
\end{equation}
O {\PYTHONnumpy} tem a função \hl{\PYTHONnumpyDOTlinalgDOTeig} para computar os auto-pares de uma matriz. Por exemplo,

\begin{lstlisting}
lmbda, v = npla.eig(np.eye(3))
print('autovalores = \n', lmbda)
print('autovetores = \n', v)
\end{lstlisting}

\begin{verbatim}
autovalores = 
  [1. 1. 1.]
autovetores = 
  [[1. 0. 0.]
   [0. 1. 0.]
   [0. 0. 1.]]
\end{verbatim}

Observamos que a função retorna um {\PYTHONtuple} de {\PYTHONnumpyDOTarrays}, sendo que o primeiro contém os autovalores (repetidos conforme suas multiplicidades) e o segundo item é a matriz dos autovetores (dispostos nas colunas).

\begin{exer}
  Compute os auto-pares da matriz
  \begin{equation}
    A =
    \begin{bmatrix}
      1 & 3 & 2\\
      3 & 2 & -1\\
      2 & -1 & 1
    \end{bmatrix}.
  \end{equation}
  Então, verifique se, de fato, $Av = \lambda v$ para cada auto-par $(\lambda, v)$ computado.
\end{exer}
\begin{resp}
  
\begin{lstlisting}
import numpy as np
import numpy.linalg as npla
A = np.array([[1, 3, 2],
              [3, 2, -1],
              [2, -1, 1]])
lmbda, v = np.linalg.eig(A)
# testando os auto-pares
print(npla.norm(A @ v[:, 0] - lmbda[0] * v[:, 0]) < 1e-10)
print(npla.norm(A @ v[:, 1] - lmbda[1] * v[:, 1]) < 1e-10)
print(npla.norm(A @ v[:, 2] - lmbda[2] * v[:, 2]) < 1e-10)
\end{lstlisting}
\end{resp}

\ifisbook 
\subsubsection*{Respostas dos Exercícios}
\shipoutAnswer
\fi
