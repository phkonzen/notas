%Este trabalho está licenciado sob a Licença Atribuição-CompartilhaIgual 4.0 Internacional Creative Commons. Para visualizar uma cópia desta licença, visite http://creativecommons.org/licenses/by-sa/4.0/deed.pt_BR ou mande uma carta para Creative Commons, PO Box 1866, Mountain View, CA 94042, USA.

\section{Elementos da Linguagem}\label{sec_elem}

\subsection{Classes de Objetos Básicos}

\hl{{\python} é uma \emph{linguagem} de programação \emph{dinâmica} em que as variáveis/objetos são declaradas/os automaticamente ao receberem um valor/dado}. Por exemplo, consideramos as seguintes instruções

\begin{lstlisting}
x = 2
y = x * 3.0
\end{lstlisting}

Na primeira instrução, a variável \texttt{x} recebe o valor inteiro \texttt{2} e, então, é armazenado na memória do computador como um objeto da \hl{classe {\PYTHONint}} (número inteiro). Na segunda instrução, \texttt{y} recebe o valor decimal $6.0$ (resultado de $2\times 3.0$) e é armazenado como um objeto da \hl{classe {\PYTHONfloat}} (ponto flutuante de 64-{\it bits}). Podemos verificar isso, com as seguintes instruções

\begin{lstlisting}
print(x)
\end{lstlisting}

\begin{verbatim}
  2
\end{verbatim}

\begin{lstlisting}
print(y)
\end{lstlisting}

\begin{verbatim}
  6.0
\end{verbatim}

\begin{lstlisting}
print(type(x), type(y))
\end{lstlisting}

\begin{verbatim}
  <class 'int'> <class 'float'>
\end{verbatim}

\begin{obs}\normalfont{(\hl{Comentários e Continuação de Linha}.)}
  Códigos {\python} admitem \emph{comentários} e \emph{continuação de linha} como no seguinte exemplo

\begin{lstlisting}
# isto é um comentário
s = "isto é uma \
string"
print(s)
\end{lstlisting}

\begin{verbatim}
  isto é uma string
\end{verbatim}

\begin{lstlisting}
  type(s)
\end{lstlisting}

\begin{verbatim}
  <class 'str'>
\end{verbatim}

\end{obs}

\begin{obs}\normalfont{(\hl{Notação científica}.)}
  O {\python} aceita \emph{notação científica}. Por exemplo $5.2\times 10^{-2}$ é digitado da seguinte forma

\begin{lstlisting}
5.2e-2
\end{lstlisting}

\begin{verbatim}
0.052
\end{verbatim}

\end{obs}

\begin{obs}\normalfont{(\hl{\textit{Casting}}.)}
  Quando não há ambiguidade, pode-se fazer a conversão entre objetos de classes diferentes (\textit{casting}). Por exemplo,

\begin{lstlisting}
x = 1
print(x, type(x))
\end{lstlisting}

\begin{verbatim}
  1 <class 'int'>
\end{verbatim}

\begin{lstlisting}
y = float(x)
print(y, type(y))
\end{lstlisting}

\begin{verbatim}
1.0 <class 'float'>  
\end{verbatim}

\end{obs}

Além de objetos numéricos e \textit{string}, {\python} também conta com objetos {\PYTHONlist} (lista), {\PYTHONtuple} ($n$-upla) e {\PYTHONdict} (dicionário). Estudaremos essas classes de objetos mais adiante no minicurso.

\begin{exer}
  Antes de implementar, diga qual o valor de \texttt{x} após as seguintes instruções.

\begin{lstlisting}
x = 1
y = x
y = 0
\end{lstlisting}

Justifique sua resposta e verifique-a.
\end{exer}
\begin{resp}
\texttt{1}
\end{resp}

\begin{exer}
  Implemente um código em que a(o) usuária(o) entra com valores para as variáveis \texttt{x} e \texttt{y}. Então, os valores das variáveis são permutados entre si. Dica: use {\PYTHONinput} para a entrada de dados.
\end{exer}
\begin{resp}

\begin{lstlisting}
x = float(input('x = '))
y = float(input('y = '))
z = x
x = y
y = z
print('x = ', x)
print('y = ', y)
\end{lstlisting}

\end{resp}

\ifisbook 
\subsubsection*{Respostas dos Exercícios}
\shipoutAnswer
\fi

%%% subsection
\subsection{Operações Aritméticas Elementares}

Os operadores aritméticos elementares são:
\begin{itemize}
\item[] \texttt{+} \hlemph{adição}
\item[] \texttt{-} \hlemph{subtração}
\item[] \texttt{*} \hlemph{multiplicação}
\item[] \texttt{/} \hlemph{divisão}
\item[] \texttt{**} \hlemph{potenciação}
\item[] \texttt{\%} \hlemph{módulo}
\item[] \texttt{//} \hlemph{divisão inteira}
\end{itemize}

\begin{ex}
  Estudamos a seguinte computação

\begin{lstlisting}
2+8*3/2**2-1
\end{lstlisting}

\begin{verbatim}
7.0
\end{verbatim}

Observamos que as operações \texttt{**} tem precedência sobre as operações \texttt{*}, \texttt{/}, \texttt{\%}, \texttt{//}, as quais têm precedência sobre as operações \texttt{+, -}. Operações de mesma precedência seguem a ordem da esquerda para direita, conforme escritas na linha de comando. Usa-se parênteses para alterar a precedência entre as operações, por exemplo

\begin{lstlisting}
(2+8*3)/2**2-1
\end{lstlisting}

\begin{verbatim}
5.5
\end{verbatim}

\end{ex}

\begin{obs}\normalfont{(\hl{Precedência das Operações}.)}
Consulte mais informações sobre a precedência de operadores em \href{https://docs.python.org/3/reference/expressions.html#operator-precedence}{Python Docs: Operator Precedence}.
\end{obs}

\begin{exer}\label{exr:bhaskara}
  Compute as raízes do seguinte polinômio quadrático
  \begin{equation}
    p(x) = 2x^2 - 2x - 4
  \end{equation}
  usando a fórmula de Bhaskara{\bhaskara}.
\end{exer}
\begin{resp}

\begin{lstlisting}
a = 2.
b = -2.
c = -4.
dlta = b**2 - 4*a*c
x1 = (-b - dlta**(1./2))/(2*a)
x2 = (-b + dlta**(1./2))/(2*a)
print('x1 = ', x1)
print('x2 = ', x2)
\end{lstlisting}

\end{resp}

O operador \texttt{\%} módulo computa o \emph{resto da divisão} e o operador \texttt{//} a \emph{divisão inteira}, por exemplo

\begin{lstlisting}
5 % 2
\end{lstlisting}

\begin{verbatim}
1
\end{verbatim}

\begin{lstlisting}
5 // 2
\end{lstlisting}

\begin{verbatim}
2
\end{verbatim}

\begin{exer}
  Use o {\python} para computar os inteiros não negativos $q$ e $r$ tais que
  \begin{equation}
    25 = q\cdot 3 + r,
  \end{equation}
  sendo $r$ o menor possível.
\end{exer}
\begin{resp}

\begin{lstlisting}
q = 25//3
print('q = ', q)
r = 25%3
print('r = ', r)
\end{lstlisting}

\end{resp}

\ifisbook 
\subsubsection*{Respostas dos Exercícios}
\shipoutAnswer
\fi


\subsection{Funções e Constantes Elementares}

\hl{O \emph{módulo} Python {\PYTHONmath} disponibiliza várias funções e constantes elementares}. Para usá-las, precisamos importar o módulo em nosso código

\begin{lstlisting}
import math
\end{lstlisting}

Com isso, temos acesso a todas as definições e declarações contidas neste módulo. Por exemplo

\begin{lstlisting}
math.pi
\end{lstlisting}

\begin{verbatim}
3.141592653589793
\end{verbatim}

\begin{lstlisting}
math.cos(math.pi)
\end{lstlisting}

\begin{verbatim}
-1.0
\end{verbatim}

\begin{lstlisting}
math.sqrt(2)
\end{lstlisting}

\begin{verbatim}
1.4142135623730951
\end{verbatim}

\begin{lstlisting}
math.log(math.e)
\end{lstlisting}

\begin{verbatim}
1.0
\end{verbatim}


\begin{obs}\normalfont{(\hl{Função Logaritmo}.)}
  Notamos que {\PYTHONmathDOTlog} é a função logaritmo natural, i.e. $\ln(x) = \log_e(x)$. A implementação {\python} para o logaritmo de base 10 é {\PYTHONmathDOTlog}\texttt{(x, 10)} ou, mais acurado, {\PYTHONmathDOTlogTen}.
\end{obs}

\begin{exer}
  Compute
  \begin{enumerate}[a)]
  \item $\displaystyle \sen\left(\frac{\pi}{4}\right)$
  \item $\displaystyle e^{\log_3(\pi)}$
  \item $\displaystyle \sqrt[3]{-27}$
  \end{enumerate}
\end{exer}
\begin{resp}

\begin{lstlisting}
import math
# a)
a = math.sin(math.pi/4)
print('a) ', a)
# b)
b = 3*math.pi
print('b) ', b)
# c)
c = -(27)**(1/3)
print('c) ', c)
\end{lstlisting}

\end{resp}

\begin{exer}
  Refaça o Exercício~\ref{exr:bhaskara} usando a função {\PYTHONmathDOTsqrt} para computar a raiz quadrada do discriminante.
\end{exer}
\begin{resp}
  
\begin{lstlisting}
import math
a = 2.
b = -2.
c = -4.
dlta = b**2 - 4*a*c
x1 = (-b - math.sqrt(dlta))/(2*a)
x2 = (-b + math.sqrt(dlta))/(2*a)
print('x1 = ', x1)
print('x2 = ', x2)  
\end{lstlisting}

\end{resp}

\ifisbook 
\subsubsection*{Respostas dos Exercícios}
\shipoutAnswer
\fi

%%% subsection
\subsection{Operadores de Comparação Elementares}

Os \hlemph{operadores de comparação} elementares são
\begin{itemize}
\item[]\texttt{==} \hlemph{igual a}
\item[]\texttt{!=} \hlemph{diferente de}
\item[]\texttt{>} \hlemph{maior que}
\item[]\texttt{<} \hlemph{menor que}
\item[]\texttt{>=} \hlemph{maior ou igual que}
\item[]\texttt{<=} \hlemph{menor ou igual que}
\end{itemize}
Estes operadores \hl{retornam os \emph{valores lógicos} {\PYTHONTrue} (verdadeiro) ou {\PYTHONFalse} (falso)}.

Por exemplo, temos

\begin{lstlisting}
x = 2
x + x == 4
\end{lstlisting}

\begin{verbatim}
True
\end{verbatim}

\begin{exer}
  Considere a circunferência de equação
  \begin{equation}
    c: ~(x - 1)^2 + (y + 1)^2 = 1.
  \end{equation}
  Escreva um código em que a(o) usuária(o) entra com as coordenadas de um ponto $P = (x, y)$ e o código verifica se $P$ pertence ao disco determinado por $c$.
\end{exer}
\begin{resp}
  
\begin{lstlisting}
x = float(input('x = '))
y = float(input('y = '))
resp = (x-1.)**2 + (y+1.)**2 <= 1.
print('P em c?', resp)
\end{lstlisting}

\end{resp}

\begin{exer}
  Antes de implementar, diga qual é o valor lógico da instrução

\begin{lstlisting}
math.sqrt(3)**2 == 3
\end{lstlisting}

Justifique sua resposta e verifique!
\end{exer}
\begin{resp}
\texttt{False}
\end{resp}

\ifisbook 
\subsubsection*{Respostas dos Exercícios}
\shipoutAnswer
\fi

%%% subsection
\subsection{Operadores Lógicos Elementares}

Os \hlemph{operadores lógicos} elementares são:
\begin{itemize}
\item[]\texttt{and} \hlemph{\texttt{e} lógico}
\item[]\texttt{or} \hlemph{\texttt{ou} lógico}
\item[]\texttt{not} \hlemph{\texttt{não} lógico}
\end{itemize}

\begin{ex}\normalfont{(\hl{Tabela Booleana do \texttt{and}}.)}
  A tabela booleana{\boole} do \texttt{and} é
  \begin{center}
    \begin{tabular}[H]{ll|l}
      \texttt{A} & \texttt{B} & \texttt{A and B}\\\hline
      \texttt{True} & \texttt{True} & \texttt{True} \\
      \texttt{True} & \texttt{False} & \texttt{False} \\
      \texttt{False} & \texttt{True} & \texttt{False} \\
      \texttt{False} & \texttt{False} & \texttt{False} \\\hline
    \end{tabular}
  \end{center}

  Por exemplo, temos

\begin{lstlisting}
x = 2
(x > 1) and (x < 2)
\end{lstlisting}

\begin{verbatim}
False
\end{verbatim}

\end{ex}

\begin{exer}
  Construa as tabelas booleanas do operador \texttt{or} e do \texttt{not}.
\end{exer}
\begin{resp}

\begin{lstlisting}
print('A B | A or B')
print(True, True, '|', True or True)
print(True, False, '|', True or False)
print(False, True, '|', False or True)
print(False, False, '|', False or False)

print('A | not(A)')
print(True, '|', not(True))
print(False, '|', not(False))
\end{lstlisting}

\end{resp}

\begin{exer}
  Use {\python} para verificar se $1.4 <= \sqrt{2} < 1.5$.
\end{exer}
\begin{resp}

\begin{lstlisting}
import math
resp = 1.4 <= math.sqrt(2) < 1.5
print('1.4 <= math.sqrt(2) < 1.5 ?', resp)
\end{lstlisting}

\end{resp}

\begin{exer}
  Considere um retângulo $r: ~ABDC$ de vértices $A = (1, 1)$ e $D = (2, 3)$. Crie um código em que a(o) usuária(o) informa as coordenadas de um ponto $P = (x, y)$ e o código imprime {\PYTHONTrue} ou {\PYTHONFalse} para cada um dos seguintes itens:
  \begin{enumerate}[a)]
  \item $P\in r$.
  \item $P\in \p r$.
  \item $P\not\in \overline{r}$.
  \end{enumerate}
\end{exer}
\begin{resp}
  
\begin{lstlisting}
x = float(input('x = '))
y = float(input('y = '))
print('P em r', \
      (1 < x < 2) and (1 < y < 3))
print('P na fronteira', \
      ((x == 1 or x == 2) and (1 <= y <= 3)) or \
      (( 1 <= x <= 2) and (y == 1 or y == 3)))
print('P fora do fecho', \
      (x < 1 or x > 2) or (y < 1 or y > 3))
\end{lstlisting}

\end{resp}

\begin{exer}
  Implemente uma instrução para computar o operador \texttt{xor} (ou exclusivo). Dadas duas afirmações \texttt{A} e \texttt{B}, \texttt{A xor B} é {\PYTHONTrue} no caso de uma, e somente uma, das afirmações ser {\PYTHONFalse}, caso contrário é {\PYTHONFalse}.
\end{exer}
\begin{resp}
\texttt{(A or B) and not(A and B)}
\end{resp}

\ifisbook 
\subsubsection*{Respostas dos Exercícios}
\shipoutAnswer
\fi


%%% subsection
\subsection{\texttt{set}}

\hl{Um {\PYTHONset} em {\python} é uma \emph{coleção de objetos} \emph{não ordenada}, \emph{imutável} e \emph{não admite itens duplicados}}. Por exemplo,

\begin{lstlisting}
a = {1, 2, 3}
type(a)  
\end{lstlisting}

\begin{verbatim}
  <class 'set'>
\end{verbatim}

\begin{lstlisting}
b = set((2, 1, 3, 3))
print(b)
\end{lstlisting}

\begin{verbatim}
  {1, 2, 3}
\end{verbatim}

\begin{lstlisting}
a == b
\end{lstlisting}

\begin{verbatim}
True
\end{verbatim}

\begin{lstlisting}
# conjunto vazio
e = set()  
\end{lstlisting}

Acima, alocamos o conjunto $a = \{1,2, 3\}$. Note que o conjunto $b$ é igual a $a$. Observamos que o conjunto vazio deve ser construído com a instrução \texttt{set()} e não com \texttt{\{\}}\endnote{Isso constrói um dicionário vazio, como estudaremos logo mais.}.

\begin{obs}\normalfont{(\hl{Tamanho de uma Coleção de Objetos}.)}
  A função {\PYTHONlen} retorna o número de elementos de uma coleção de objetos. Por exemplo,
  
\begin{lstlisting}
len(a)
\end{lstlisting}

\begin{verbatim}
3
\end{verbatim}

\end{obs}

\emph{Operadores envolvendo conjuntos}:
\begin{itemize}
\item[] \texttt{-} \hl{diferença entre conjuntos}
\item[] \texttt{|} \hl{união de conjuntos}
\item[] \texttt{\&} \hl{interseção de conjuntos}
\item[] \texttt{\^{}} \hl{diferença simétrica}
\end{itemize}

\begin{ex}
  Os conjuntos
  \begin{gather}
    A = \{2, \pi, -0.25, 3, \text{'banana'}\},\\
    B = \{\text{'laranja'}, 3, \operatorname{arc cos}(-1), -1\}
  \end{gather}
  podem ser alocados como \texttt{sets}

\begin{lstlisting}
import math
A = {2, math.pi, -0.25, 3, 'banana'}
B = {'laranja', 3, math.acos(-1), -1}
\end{lstlisting}
    
e, então, podemos computar:

  \begin{enumerate}[a)]
  \item $A\setminus B$

\begin{lstlisting}[framexrightmargin=-1.4em]
a = A - B
print(a)
\end{lstlisting}

\begin{verbatim}
{-0.25, 2, 'banana'}
\end{verbatim}

  \item $A\cup B$

\begin{lstlisting}[framexrightmargin=-1.4em]
b =  A | B
print(b)
\end{lstlisting}

\begin{verbatim}
{-0.25, 2, 3, 3.141592653589793, 'laranja', 'banana', -1}
\end{verbatim}
    
  \item $A\cap B$
  
\begin{lstlisting}[framexrightmargin=-1.4em]
c = A & B
print(c)
\end{lstlisting}

\begin{verbatim}
{3, 3.141592653589793}
\end{verbatim}
    

  \item $A\Delta B = (A\setminus B) \cup (B\setminus A)$
  
\begin{lstlisting}[framexrightmargin=-1.4em]
d = A ^ B
print(d)
\end{lstlisting}


\begin{verbatim}
{-0.25, 2, 'laranja', 'banana', -1}
\end{verbatim}

\end{enumerate}

\end{ex}

\begin{exer}
  Aloque como {\PYTHONset} cada um dos seguintes conjuntos:
  \begin{enumerate}[a)]
    \item O conjunto $A$ dos números $-12 \leq n \leq 6$ e que são divisíveis por 2.
    \item O conjunto $B$ dos números $-12 < n \leq 6$ e que são divisíveis por 3.
  \end{enumerate}
  Então, compute o subconjunto de $A$ e $B$ que contém apenas os números divisíveis por $2$ e $3$.
\end{exer}
\begin{resp}
  
\begin{lstlisting}
A = {-12, -10, -8, -6, \
    -4, -2, 0, 2, 4, 6}
B = {-12, -9, -6, -3, \
    0, 3, 6}
C = A & B
print('C = ', C)
\end{lstlisting}

\end{resp}

\begin{obs}\normalfont{(\hl{Compreensão de \texttt{sets}}.)}\label{obs:compreensão_de_conjuntos}
  {\python} disponibiliza a sintaxe de compreensão de \texttt{sets}. Por exemplo,

\begin{lstlisting}
C = {x for x in A if type(x) == str}
print(C)
\end{lstlisting}

\begin{verbatim}
{'banana'}
\end{verbatim}

\end{obs}

\begin{exer}
  Considere o conjunto
  \begin{equation}
    Z = \{-4, -3, -2, -1, 0, 1, 2, 3, 4\}.
  \end{equation}
  Faça um código {\python} para extrair o subconjunto $\mathcal{P}$ dos números pares do conjunto $Z$. Depois, modifique-o para extrair o subconjunto $\mathcal{I}$ dos números ímpares. Dica: use de compreensão de \texttt{sets}.
\end{exer}
\begin{resp}

\begin{lstlisting}
Z = {-4, -3, -2, -1, \
    0, 1, 2, 3, 4}
P = {p for p in Z if p % 2 == 0}
print('P = ', P)
\end{lstlisting}

\end{resp}

\ifisbook 
\subsubsection*{Respostas dos Exercícios}
\shipoutAnswer
\fi


%%% subsection
\subsection{\texttt{tuple}}

Em {\python}, \hl{{\PYTHONtuple} é uma coleção ordenada e imutável de objetos}. Por exemplo, na sequência alocamos um par, uma tripla e uma quadrupla ordenada usando \texttt{tuples}.

\begin{lstlisting}
a = (1, 2)
print(a, type(a))
\end{lstlisting}

\begin{verbatim}
(1, 2) <class 'tuple'>
\end{verbatim}

\begin{lstlisting}
b = -1, 1, 0
print(b, len(b))
\end{lstlisting}

\begin{verbatim}
(-1, 1, 0) 3
\end{verbatim}

\begin{lstlisting}
c = (0.5, 'laranja', {2, -1}, 2)
print(c)
\end{lstlisting}

\begin{verbatim}
(0.5, 'laranja', {2, -1}, 2)
\end{verbatim}

Os elementos de um {\PYTHONtuple} são indexados, o índice $0$ corresponde ao primeiro elemento, o índice $1$ ao segundo elemento e assim por diante. Desta forma é possível o acesso direto a um elemento de um {\PYTHONtuple} usando-se sua posição. Por exemplo,

\begin{lstlisting}
print(c[2])
\end{lstlisting}

\begin{verbatim}
{2, -1}
\end{verbatim}

Pode-se também extrair uma fatia (um subconjunto) usando-se a notação
\texttt{:}. Por exemplo,

\begin{lstlisting}
d = c[1:3]
print(d)
\end{lstlisting}

\begin{verbatim}
('laranja', {2, -1})
\end{verbatim}

\emph{Operadores básicos}:

\begin{itemize}
\item[] \texttt{+} \hl{concatenação}

\begin{lstlisting}[framexrightmargin=-2.4em]
a = (1, 2) + (3, 4, 5)
print(a)
\end{lstlisting}

\begin{verbatim}
(1, 2, 3, 4, 5)
\end{verbatim}

\item[] \texttt{*} \hl{repetição}

\begin{lstlisting}[framexrightmargin=-2.4em]
b = (1, 2) * 2
\end{lstlisting}

\begin{verbatim}
(1, 2, 1, 2)
\end{verbatim}

\item[] \texttt{in} \hl{pertencimento}

\begin{lstlisting}[framexrightmargin=-2.4em]
c =  1 in (-1, 0, 1, 2)
\end{lstlisting}

\begin{verbatim}
True
\end{verbatim}

\end{itemize}

\begin{exer}
  Use \texttt{sets} para alocar os conjuntos
  \begin{align}
    &A = \{-1, 0, 2\},\\
    &B = \{2, 3, 5\}.
  \end{align}
  Então, compute o produto cartesiano $A\times B=\{(a,b):~a\in A, b\in B\}$. Qual o número de elementos da $A\times B$? Dica: use a sintaxe de compreensão de \texttt{sets} (consulte a Observação \ref{obs:compreensão_de_conjuntos}).
\end{exer}
\begin{resp}

\begin{lstlisting}
A = {-1, 0, 2}
B = {2, 3, 5}
P = {(a,b) for a in A for b in B}
print('P = ', P)
\end{lstlisting}

\end{resp}

\begin{exer}
  Aloque o gráfico discreto da função\endnote{O gráfico de uma função restrita a um conjunto $A$ é o conjunto $\operatorname{G}(f)|_{A} = \{(x,y):~x\in A, y=f(x)\}$.} $f(x) = x^2$ para $x=0, \frac{1}{2}, 1, 2$. Dica: use a sintaxe de compreensão de conjuntos (consulte a Observação \ref{obs:compreensão_de_conjuntos}).
\end{exer}

\begin{resp}

\begin{lstlisting}
X = {0., 0.5, 1., 2.}
G = {(x, x**2) for x in X}
print('G = ', G)
\end{lstlisting}

\end{resp}

\ifisbook 
\subsubsection*{Respostas dos Exercícios}
\shipoutAnswer
\fi


%%% subsection
\subsection{\texttt{list}}

\hl{Um {\PYTHONlist} é uma uma coleção de objetos \emph{indexada} e \emph{mutável}}. Por exemplo,

\begin{lstlisting}
x = [-1, 2, -3, -5]
print(x, type(x))
\end{lstlisting}

\begin{verbatim}
  [-1, 2, -3, -5] <class 'list'>
\end{verbatim}

\begin{lstlisting}
y = [1, 1, 'oi', 2.5]
print(y)
\end{lstlisting}

\begin{verbatim}
[1, 1, 'oi', 2.5]
\end{verbatim}

\begin{lstlisting}
vazia = []
print(len(vazia))
print(len(y))
\end{lstlisting}

\begin{verbatim}
0
4
\end{verbatim}

Os elementos de um {\PYTHONlist} são indexados de forma análoga a um {\PYTHONtuple}, o índice $0$ corresponde ao primeiro elemento, o índice $1$ ao segundo elemento e assim por diante. Bem como, o índice $-1$ corresponde ao último elemento, o $-2$ ao penúltimo e segue. Por exemplo,

\begin{lstlisting}
x[-1] = 3.14
print('x[0] = ', x[0])
print(x = ', x)
\end{lstlisting}

\begin{verbatim}
x[0] = 1
x = [-1, 2, -3, 3.14]
\end{verbatim}

\begin{lstlisting}
x[:3] = [10, -20]
print(x)
\end{lstlisting}

\begin{verbatim}
[10, -20, -3, 3.14]
\end{verbatim}

\hl{Os operadores básicos de concatenação e de repetição também estão disponíveis para um {\PYTHONlist}}. Por exemplo,

\begin{lstlisting}
x = [1,2] + [3, 4, 5]
print(x)
y = [1,2]*2
print(y)
\end{lstlisting}

\begin{verbatim}
[1, 2, 3, 4, 5]
[1, 2, 1, 2]
\end{verbatim}

\begin{obs}
  {\PYTHONlist} conta com várias funções prontas para a execução de diversas tarefas práticas como, por exemplo, inserir/deletar itens, contar ocorrências, ordenar itens, etc. Consulte na web \href{https://docs.python.org/3/tutorial/datastructures.html#more-on-lists}{Python Docs: More on Lists}.
\end{obs}

\begin{obs}\normalfont{(\hl{Alocação \textit{versus} Cópia}.)}
  Estudamos o seguinte exemplo

\begin{lstlisting}
x = [2, 3, 1]
y = x
y[1] = 0
print('x =', x)
\end{lstlisting}

\begin{verbatim}
x = [2, 0, 1]
\end{verbatim}

\hl{Em {\python}, dados têm identificação única}. Logo, neste exemplo, $x$ e $y$ apontam para o mesmo endereço de memória. Modificar $y$ é também modificar $x$ e vice-e-versa. Para desassociar $y$ de $x$, $y$ precisa receber uma cópia de $x$, como segue

\begin{lstlisting}
x = [2, 3, 1]
print('id(x) =', id(x))
y = x.copy()
print('id(y) =', id(y))
y[1] = 0
print('x =', x)
print('y =', y)
\end{lstlisting}

\begin{verbatim}
id(x) = 140476759980864
id(y) = 140476760231360
x = [2, 3, 1]
y = [2, 0, 1]
\end{verbatim}

\end{obs}

\begin{obs}\normalfont{(\hl{Anexar ou Estender}.)}
  Um {\PYTHONlist} tem tamanho dinâmico, permitindo a anexação de um novo item ou sua extensão. A anexação de um item pode ser feita com o método {\PYTHONlistDOTappend}, enquanto que a extensão é feita com {\PYTHONlistDOTextend}. Por exemplo, com o {\PYTHONlistDOTappend} temos

\begin{lstlisting}
l = [1, 2]
l.append((3,4)))
print(l)
\end{lstlisting}

\begin{verbatim}
[1, 2, (3, 4)]
\end{verbatim}

Enquanto, que com o {\PYTHONlistDOTextend} obtemos

\begin{lstlisting}
l = [1, 2]
l.extend((3,4)))
print(l)
\end{lstlisting}
  
\begin{verbatim}
[1, 2, 3, 4]
\end{verbatim}
  
\end{obs}

\begin{exer}
  A solução de
  \begin{equation}
    x^2 - 2 = 0
  \end{equation}
  pode ser aproximada pela iteração do método babilônico
  \begin{align}
    &x_0 = 1,\\
    &x_{i+1} = \frac{1}{2}\left(x_{i} + \frac{2}{x_i}\right)
  \end{align}
  para $i = 0, 1, 2, \ldots$. Aloque uma lista com as quatro primeiras iteradas, i.e. $[x_0, x_1, x_2,\allowbreak x_3, x_4]$. Dica: use {\PYTHONlistDOTappend}.
\end{exer}
\begin{resp}
  
\begin{lstlisting}
x = [1] # x0
x.append(0.5*(x[-1] + 2./x[-1])) # x1
x.append(0.5*(x[-1] + 2./x[-1])) # x2
x.append(0.5*(x[-1] + 2./x[-1])) # x3
x.append(0.5*(x[-1] + 2./x[-1])) # x4
print('x = ', x)
\end{lstlisting}

\end{resp}

\begin{exer}
  Aloque cada um dos seguintes vetores como um {\PYTHONlist}:
  \begin{align}
    x = (-1, 3, -2),\\
    y = (4, -2, 0).
  \end{align}
  Então, compute
  \begin{enumerate}[a)]
  \item $x+y$
  \item $x\cdot y$
  \end{enumerate}
  Dica: use uma compreensão de lista e os métodos {\PYTHONzip} e {\PYTHONsum}.
\end{exer}
\begin{resp}

\begin{lstlisting}
# a)
x = [-1, 3, -2]
y = [4, -2, 0]
xpy = [xy[0]+xy[1] for xy in zip(x,y)]
print('x + y = ', xpy)

# b)
dxy = sum([xy[0]*xy[1] for xy in zip(x,y)])
print('x . y = ', dxy)
\end{lstlisting}

\end{resp}

\begin{exer}
  Uma matriz pode ser alocada como um encadeamento de \texttt{lists}. Por exemplo, a matriz
  \begin{equation}
    M =
    \begin{bmatrix}
      1 & -2 \\
      2 & 3
    \end{bmatrix}
  \end{equation}
  pode ser alocada como a seguinte {\PYTHONlist}

\begin{lstlisting}
M = [[1,-2],[2,3]]
M
\end{lstlisting}

\begin{verbatim}
[[1, -2], [2, 3]]
\end{verbatim}

  Use {\PYTHONlist} para alocar a matriz
  \begin{equation}
    A =
    \begin{bmatrix}
      1 & -2 & 1\\
      8 & 0 & -7\\
      3 & -1 & -2
    \end{bmatrix}
  \end{equation}
  e o vetor
  \begin{equation}
    x = (2, -3, 1),
  \end{equation}
  então compute $Ax$.
\end{exer}
\begin{resp}

\begin{lstlisting}
A = [[1,-2,1],
    [8,0,-7],
    [3,-1,-2]]
x = [2,-3,1]
Ax = [sum([aix[0]*aix[1] for aix in zip(ai, x)]) for ai in A]
print('Ax = ', Ax)
\end{lstlisting}

\end{resp}

\ifisbook 
\subsubsection*{Respostas dos Exercícios}
\shipoutAnswer
\fi


%%% subsection
\subsection{\texttt{dict}}

Um {\PYTHONdict} é um mapeamento de objetos (um dicionário), em que cada item é um par \texttt{chave:valor}. Por exemplo,

\begin{lstlisting}
a = {'nome': 'triangulo', 'perímetro': 3.2}
print(a, type(a))
\end{lstlisting}

\begin{verbatim}
{'nome': 'triangulo', 'perímetro': 3.2} <class 'dict'>
\end{verbatim}


O acesso a um item do dicionário pode ser feito por sua chave, por exemplo,

\begin{lstlisting}
a['nome'] = 'triângulo'
print(a['nome'])
\end{lstlisting}

\begin{verbatim}
'triângulo'
\end{verbatim}

Pode-se adicionar um novo par, simplesmente, atribuindo valor a uma nova chave. Por exemplo,

\begin{lstlisting}
a['vértices'] = {'A': (0,0), 'B': (3,0), 'C': (0,4)}
print('vértice B =', a['vértices']['B'])
\end{lstlisting}

\begin{verbatim}
vértice B = (3,0)
\end{verbatim}


\begin{exer}
  Considere a função afim
  \begin{equation}
    f(x) = 3 - x.
  \end{equation}
  Implemente um dicionário para alocar a raiz da função, a interseção com o eixo $y$ e seu coeficiente angular.
\end{exer}
\begin{resp}

\begin{lstlisting}
f_dict = {'raiz': 3.,
          'y_intercep': 3.,
          'coef_angular': -1.}
print('raiz = ', f_dict['raiz'])
print('y_intercep = ', f_dict['y_intercep'])
print('coef_angular = ', f_dict['coef_angular'])
\end{lstlisting}

\end{resp}

\begin{exer}
  Considere a função quadrática
  \begin{equation}
    g(x) = x^2 - x - 2
  \end{equation}
  Implemente um dicionário para alocar suas raízes, vértice e interseção com o eixo $y$.
\end{exer}
\begin{resp}

\begin{lstlisting}
g_dict = {'raízes': (-1., 2.),
          'vértice': 0.5,
          'y_intercep': 2.}
print('raiz = ', g_dict['raízes'])
print('y_intercep = ', g_dict['y_intercep'])
print('vértice = ', g_dict['vértice'])
\end{lstlisting}

\end{resp}

\ifisbook 
\subsubsection*{Respostas dos Exercícios}
\shipoutAnswer
\fi
