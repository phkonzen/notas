%Este trabalho está licenciado sob a Licença Atribuição-CompartilhaIgual 4.0 Internacional Creative Commons. Para visualizar uma cópia desta licença, visite http://creativecommons.org/licenses/by-sa/4.0/deed.pt_BR ou mande uma carta para Creative Commons, PO Box 1866, Mountain View, CA 94042, USA.

\section{Elementos da Programação Estruturada}\label{sec_progest}

\hl{Na programação estruturada, os comandos de programação são executados em sequência, um novo comando só iniciado após o término do processamento do comando anterior}. Em {\python}, cada linha consiste em um comando, o programa tem início na primeira linha e término na última linha do código. \hl{Instruções de \emph{ramificação} permitem a seleção \textit{on-the-fly} de blocos de comandos, enquanto que instruções de \emph{repetição} permitem a execução repetida de um bloco. A definição de \emph{função} permite a criação de um sub-código (sub-programa) do código}.

\subsection{Ramificação}

\hl{Uma \emph{estrutura de ramificação} é uma instrução para a tomada de decisões durante a execução de um programa}. No {\python}, temos disponível a instrução \texttt{if-[elif-...-elif-else]}.

\subsubsection{\texttt{if}}

Por exemplo, o código abaixo computa as raízes reais do polinômio
\begin{equation}
  p(x) = ax^2 + bx + c,
\end{equation}  
com $a$, $b$ e $c$ alocados no início do código.

\begin{lstlisting}
import math as m
a = 1.
b = -1.
c = -2.
dlta = b**2 - 4.*a*c
if (dlta >= 0.):
  x1 = (-b - m.sqrt(dlta))/(2.*a)
  x2 = (-b + m.sqrt(dlta))/(2.*a)
  print('x1 =', x1)
  print('x2 =', x2)
\end{lstlisting}

\begin{verbatim}
x1 = -1.0
x2 = 2.0
\end{verbatim}

Neste código, o bloco de comandos (linhas 7-10) só é executado, se o discrimante do polinômio seja não-negativo. Verifique! Troque os valores de $a$, $b$ e $c$ de forma que $p$ tenha raízes complexas.

\begin{obs}\normalfont{(\hl{Indentação}.)}
  Nas linhas 7-10 do código anterior, a indentação dos comandos é obrigatória. O bloco de comandos indentados indicam o escopo da instrução \texttt{if}.
\end{obs}

\subsubsection{\texttt{if-else}}

Vamos modificar o código anterior, de forma que as raízes complexas sejam computadas e impressas, quando for o caso.

\begin{lstlisting}
import math as m
a = 1.
b = -4.
c = 8.
dlta = b**2 - 4.*a*c
if (dlta >= 0.):
  # raízes reais
  x1 = (-b - m.sqrt(dlta))/(2.*a)
  x2 = (-b + m.sqrt(dlta))/(2.*a)
else:
  # raízes complexas
  rea = -b/(2.*a)
  img = m.sqrt(-dlta)/(2.*a)
  x1 = rea - img*1j
  x2 = rea + img*1j
print('x1 =', x1)
print('x2 =', x2)
\end{lstlisting}

\begin{verbatim}
x1 = (2-2j)
x2 = (2+2j)
\end{verbatim}

\begin{obs}\normalfont{(\hl{Número Complexo}.)}
  Em {\python}, números complexos podem ser alocados como objetos da classe \texttt{complex}. O número imaginário $i = \sqrt{-1}$ é denotado por \texttt{1j} e um número completo $a + bi$ por \texttt{a + b*1j}.
\end{obs}

\subsubsection{\texttt{if-elif-else}}

A instrução \texttt{elif} é uma conjunção de uma sequência de instruções texttt{else-if}. Vamos modificar o código anterior, de forma a computar o caso de raízes reais duplas de forma própria.

\begin{lstlisting}
import math as m
a = 1.
b = 2.
c = 1.
dlta = b**2 - 4.*a*c
if (dlta > 0.):
  # raízes reais
  x1 = (-b - m.sqrt(dlta))/(2.*a)
  x2 = (-b + m.sqrt(dlta))/(2.*a)
elif (dlta == 0.):
  x1 = x2 = -b/(2.*a)
else:
  # raízes complexas
  rea = -b/(2.*a)
  img = m.sqrt(-dlta)/(2.*a)
  x1 = rea - img*1j
  x2 = rea + img*1j
print('x1 =', x1)
print('x2 =', x2)
\end{lstlisting}

\begin{verbatim}
x1 = -1.0
x2 = -1.0
\end{verbatim}

\begin{exr}
  Desenvolva um código para computar a raiz do polinômio
  \begin{equation}
    f(x) = ax + b
  \end{equation}
  com dados $a$ e $b$. O código deve lidar com todos os casos possíveis, a saber:
  \begin{enumerate}[a)]
    \item única raiz ($a\neq 0$).
    \item infinitas raízes ($a=b=0$).
    \item não existe raíz ($a = 0$ e $b \neq 0$).
  \end{enumerate}
\end{exr}

\begin{exr}
  Desenvolva um código em que dados três pontos $A$, $B$ e $C$ no plano, verifique se $ABC$ determia um triângulo. Caso afirmativo, classifique-o como um triângulo equilátero, isósceles ou escaleno.
\end{exr}

\subsection{Repetição}

\hl{Estruturas de repetição são instruções que permitem que a execução repetida de um bloco de comandos}. São duas instruções disponíveis {\PYTHONwhile} e {\PYTHONfor}.

\subsubsection{\texttt{while}}

\hl{A instrução {\PYTHONwhile} permite a repetição de um bloco de comandos, equanto uma dada condição for verdadeira}. 

Por exemplo, o seguinte código computa e imprimi os elementos da sequência de Fibonacci{\fibonacci}, enquanto forem menores que $10$.

\begin{lstlisting}
n = 1
print(n)
m = 1
print(m)
while (n+m < 10):
  s = m
  m += n
  n = s
  print(m)
\end{lstlisting}

Verifique!

\begin{obs}\normalfont{(\hl{Instruções de Controle}.)}
  As instruções de controle {\PYTHONbreak}, {\PYTHONcontinue} são bastante úteis em várias situações. A primeira, encerra as repetições e, a segunda, pula para uma nova repetição.
\end{obs}

\begin{exr}
  Use {\PYTHONwhile} para imprimir os dez primeiros números ímpares.
\end{exr}

\begin{exr}
  Uma aplicação do Método Babilônico\footnote{Matemática Babilônica, matemática desenvolvida na Mesopotâmia, desde os Sumérios até a queda da Babilônia em 539 a.C.. Fonte: \href{https://pt.wikipedia.org/wiki/Matem\%C3\%A1tica\_babil\%C3\%B4nica}{Wikipédia}.} para a aproximação da solução da equação $x^2-2 = 0$, consiste na iteração
  \begin{gather}
    x_0 = 1,\\
    x_{i+1} = \frac{x_i}{2} + \frac{1}{x_i},\quad i=0,1,2,\ldots
  \end{gather}
  Faça um código com {\PYTHONwhile} para computar aproximação $x_{i}$, tal que $|x_{i}-x_{i-1}|<10^{-7}$.
\end{exr}

\subsubsection{\texttt{for}}

\hl{A instrução {\PYTHONfor} permite a execução iterada de um bloco de comandos}. Dado um objeto iterável, a cada laço um novo item do objeto é tomado. Por exemplo, o seguinte código computa e imprime os primeiros $6$ elementos da sequência de Fibonacci.

\begin{lstlisting}
n = 1
print(f'1: {n}')
m = 1
print(f'2: {m}')
for i in [3,4,5,6]:
  s = m
  m += n
  n = s
  print(f'{i}: {m}')
\end{lstlisting}

Verifique!

\subsubsection{\texttt{range}}

A função {\PYTHONrange}\texttt{([start,]stop[,sep])} é particularmente útil na construção de instruções \texttt{for}. Ela cria um objeto de classe iterável de \texttt{start} (incluído) a \texttt{stop} (excluído), de elementos igualmente separados por \texttt{sep}. Por padrão, \texttt{start=0}, \texttt{sep=1} caso omitidos. Por exemplo, o código anterior por ser reescrito como segue.

\begin{lstlisting}
n = 1
print(f'1: {n}')
m = 1
print(f'2: {m}')
for i in range(3,7):
  s = m
  m += n
  n = s
  print(f'{i}: {m}')
\end{lstlisting}

Verifique!

\begin{exr}
  Com $n$ dado, desenvolva um código para computar o valor da soma harmônica
  \begin{equation}
    \sum_{k=1}^n \frac{1}{k} = \frac{1}{1} + \frac{1}{2} + \cdots + \frac{1}{n}.
  \end{equation}
\end{exr}

\begin{exr}
  Desenvolva um código para computar o fatorial de um dado número natural $n$. Dica: use {\PYTHONmathDOTfactorial} para verificar seu código.
\end{exr}

\subsection{Funções}

\hl{Em {\python}, uma função é definida pela instrução \texttt{def}}. Por exemplo, o seguinte código impleta a função
\begin{equation}
  f(x) = 2x - 3
\end{equation}
e imprime o valor de $f(2)$.

\begin{lstlisting}
  def f(x):
  y = 2*x - 3
  return x

z = f(2)
print(f'f(2) = {z}')
\end{lstlisting}

\begin{verbatim}
f(2) = 2
\end{verbatim}

\begin{obs}
  Para funções pequenas, pode-se utilizar a instrução \texttt{lambda} de funções anônimas. Por exemplo,

\begin{lstlisting}
f = lambda x: 2*x - 3
print(f'f(3) = {f(3)}')
\end{lstlisting}

\begin{verbatim}
f(3) = 3
\end{verbatim}

\end{obs}

\begin{ex}\normalfont{(\hl{Função com Parâmetros}.)}
  O seguinte código, implementa o polinômio de primeiro grau
  \begin{equation}
    p(x) = ax + b,
  \end{equation}
  com parâmteros predeterminados $a=1$ e $b=0$ (função identidade).

\begin{lstlisting}
def p(x, a=1., b=0.):
  y = a*x + b
  return y

print('p(2) =', p(2.))
print('p(2, 3, -5) =', p(2., 3., -5.))
\end{lstlisting}

\end{ex}

\begin{exr}
  Implemente uma função para computar as raízes de um polinômio de grau 2 $p(x) = ax^2 + bx + c$.
\end{exr}

\begin{exr}
  Implemente uma função que computa o produto escalar de dois vetores
  \begin{align}
    &x = (x_0, x_1, \ldots, x_{n-1}),\\
    &y = (y_0, y_1, \ldots, y_{n-1}).
  \end{align}
  Dica: considere que os vetores são alocados com \texttt{lists}.
\end{exr}

\begin{exr}
  Implemente uma função que computa o determinante de matrizes $2\times 2$. Dica: considere que a matriz está alocada com um {\PYTHONlist} encadeado.
\end{exr}

\begin{exr}
  Implemente uma função que computa a multiplicação matriz - vetor $Ax$, com $A$ matriz $2\times 2$ e $x$ um vetor de dois elementos.
\end{exr}
