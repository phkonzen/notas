

\theoremstyle{plain}
\newtheorem{teo}{Teorema}[subsection]
\newtheorem{teorema}{Teorema}[subsection]
\newtheorem{lem}{Lema}[subsection]
\newtheorem{lema}{Lema}[subsection]
\newtheorem{prop}{Proposição}[subsection]
\newtheorem{proposicao}{Proposição}[subsection]
\newtheorem{corol}{Corolário}[subsection]
\newtheorem{corolario}{Corolário}[subsection]
\theoremstyle{definition}
\newtheorem{defn}{Definição}[subsection]
\newtheorem{definicao}{Definição}[subsection]
\newtheorem{obs}{Observação}[subsection]
\newtheorem{observacao}{Observação}[subsection]

\newtheorem{ex}{Exemplo}[subsection]

\newenvironment{dem}{\begin{proof}}{\end{proof}}
\newenvironment{demonstracao}{\begin{proof}}{\end{proof}}


% exercícios para minicursos (document class article)
\newtheorem{exr}{Exercício}[subsection]

%%% PDF %%%
\ifisbook
% exerícios para notas de aula
\usepackage[answerdelayed,lastexercise]{exercise}
\usepackage{chngcntr}
\renewcommand{\AnswerSkipBefore}{-10pt}
\renewcommand{\ExerciseName}{Exercício}
\counterwithin{Exercise}{subsection}
\counterwithin{Answer}{subsection}
\renewcommand{\ExerciseHeader}{\noindent\textbf{\ExerciseName~\ExerciseHeaderNB.}~}
\renewcommand{\AnswerHeader}{\textbf{\ExerciseHeaderNB.}~}
\newenvironment{exer}{\begin{Exercise}}{\end{Exercise}}
\newenvironment{resp}{\begin{Answer}}{\end{Answer}}
\fi

%%% HTML %%%
\ifishtml
\newtheorem{exer}{Exercício}[subsection]
\newtheorem*{resp}{Resposta}  
\newtheorem*{resol}{Solução}
\fi
