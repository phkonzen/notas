%Este trabalho está licenciado sob a Licença Atribuição-CompartilhaIgual 4.0 Internacional Creative Commons. Para visualizar uma cópia desta licença, visite http://creativecommons.org/licenses/by-sa/4.0/deed.pt_BR ou mande uma carta para Creative Commons, PO Box 1866, Mountain View, CA 94042, USA.

\chapter{Fundamentos sobre funções}\label{cap_funcao}
\thispagestyle{fancy}

\begin{flushright}
  [Vídeo] | [Áudio] | \href{https://phkonzen.github.io/notas/contato.html}{[Contatar]}
\end{flushright}

\section{Definição e gráfico de funções}\label{cap_funcao_sec_defgrafico}

\begin{flushright}
  [Vídeo] | [Áudio] | \href{https://phkonzen.github.io/notas/contato.html}{[Contatar]}
\end{flushright}

\subsection{Definição}

\begin{flushright}
  [Vídeo] | [Áudio] | \href{https://phkonzen.github.io/notas/contato.html}{[Contatar]}
\end{flushright}

Uma \pmb{função} de um conjunto $D$ em um conjunto $Y$ é uma regra que associa um único elemento $y\in Y$\footnote{$y\in Y$ denota que $y$ é um elemento do conjunto $Y$.} a cada elemento $x\in D$. Costumeiramente, identificamos uma função por uma letra, por exemplo, $f$ e escrevemos
\begin{equation}
  f:D\mapsto Y, y=f(x)
\end{equation}
para denotar que a função recebe \emph{valor de entrada} em $D$ e fornece \emph{valor de saída} em $Y$, seguindo uma \emph{regra de associação} preestabelecida $y=f(x)$. Usualmente, $D$ é chamado é \emph{conjunto de entrada} e $Y$ de \emph{conjunto de saída}.

\ifispython
\begin{obs}
  No \python, podemos definir uma função abstrata $f$ com o seguinte código
  \begin{lstlisting}
    from sympy import *
    f = Function('f')
  \end{lstlisting}
  Para restringirmos o conjunto de saída aos números reais, usamos
  \begin{lstlisting}
    f = Function('f', real=True)
  \end{lstlisting}  
\end{obs}
\fi

\begin{ex}
  \begin{enumerate}[a)]
  \item $f:\mathbb{R}\mapsto\mathbb{R}$, $y=2x-1$
    
    A função $f$ toma valor de entrada $x$ no conjunto dos números reais $D=\mathbb{R}$ e fornece o valor de saída $y = 2x-1$, também no conjunto dos números reais $Y=\mathbb{R}$. A regra de associação é $y=2x-1$. Seguem alguns exemplos de aplicação:
    \begin{gather}
      f(-1) = 2(-1)-1 = -3\\
      f(\sqrt{2}) = 2\sqrt{2}-1\\
      f(z) = 2z-1,\quad \forall z\in\mathbb{R}
    \end{gather}

    \ifispython
    No \python, podemos definir esta função com o seguinte código
    \begin{lstlisting}
      from sympy import *
      x = Symbol('x', real=True)
      f = Lambda(x, 2*x-1)
    \end{lstlisting}
    Com isso, temos
    \begin{lstlisting}
      In : f(x)
      Out: 2*x - 1
      
      In : f(-1)
      Out: -3
      
      In : f(sqrt(2))
      Out: -1 + 2*sqrt(2)
      
      In : z = Symbol('z', real=True)
      In : 2*z - 1
    \end{lstlisting}
    \fi
    
  \item $g:\mathbb{Z}\mapsto\mathbb{Q}$, $\displaystyle y=\frac{1}{x}$
    
    A função $g$ toma um valor de entrada em $D=\mathbb{Z}$ e fornece o valor de saída $\displaystyle y=\frac{1}{x}$ no conjunto dos números racionais $\mathbb{Q}$. A regra de associação é $\displaystyle y = \frac{1}{x}$. Segue alguns exemplos de aplicação:
    \begin{gather}
      g(2) = \frac{1}{2}\\
      g(-5) = \frac{1}{-5} = -\frac{1}{5}\\
      g(u) = \frac{1}{u},\quad\forall u\in\mathbb{Z}
    \end{gather}

        \ifispython
    No \python, podemos definir esta função com o seguinte código
    \begin{lstlisting}
      from sympy import *
      x = Symbol('x', integer=True)
      g = Lambda(x, 1/x)
    \end{lstlisting}
    Com isso, temos
    \begin{lstlisting}
      In : g(x)
      Out: 1/x
      
      In : g(2)
      Out: 1/2
      
      In : g(-5)
      Out: -1/5
      
      In : u = Symbol('u', integer=True)
      In : g(u)
      Out: 1/u
    \end{lstlisting}
    \fi

  \end{enumerate}
\end{ex}

\begin{obs}
  Ao longo do texto, vamos assumir que as funções são definidas de $\mathbb{R}\mapsto\mathbb{R}$, salvo explicitamente escrito diferente. Assim sendo, vamos passar a usar a notação simplificada
  \begin{equation}
    f:x\mapsto f(x).
  \end{equation}
  Mais ainda, as funções serão descritas diretamente de suas regras associação.
\end{obs}

\ifispython
\begin{obs}
  No \sympy, as computações são realizadas no conjunto dos números complexos. Portanto, deve-se tomar alguns cuidados na interpretação dos resultados. Por exemplo, $\sqrt{-1}\in\mathbb{R}$ e com o \sympy, temos
  \begin{lstlisting}
    In : from sympy import *
    In : sqrt(-1)
    Out: I
  \end{lstlisting}
  onde, \verb+I+ denota o número imaginário $i = \sqrt{-1}$.
\end{obs}
\fi

\subsection{Domínio e imagem}

\begin{flushright}
  [Vídeo] | [Áudio] | \href{https://phkonzen.github.io/notas/contato.html}{[Contatar]}
\end{flushright}

O conjunto $D$ de todos os possíveis valores de entrada da função é chamado de \emph{domínio}. Em notação de conjunto, escrevemos
\begin{equation}
  D_f := \{x\in D:~f(x)\in Y\},
\end{equation}
i.e. o domínio de $f$, denotado por $D_f$, é o conjunto de todos os valores $x\in D$, tal que $f(x)\in Y$\footnote{O valor de saída $f(x)$ pertence ao conjunto $Y$.}.

\begin{ex}\label{ex:funcao_defgrafico_dominio}
  \begin{enumerate}[a)]
  \item $f:x\mapsto f(x), y=x^2$

    Observamos que, dado qualquer valor de entrada $x\in\mathbb{R}$, $x^2$ está definido e é, também, um número real. Desta forma, a função $f$ está definida para todo $x\in\mathbb{R}$, i.e.
    \begin{equation}
      D_f = \mathbb{R}.
    \end{equation}
    Neste caso, dizemos que $f$ está definida em toda parte.

  \item $\displaystyle g:x\mapsto g(x), y=\frac{1}{x}$:

    Lembramos que a divisão por zero não está definida. A expressão $1/x$ está definida para todo número real não nulo, i.e. $x\in\mathbb{R}\setminus\{0\}$. Logo, o domínio de $g$ é
    \begin{equation}
      D_g = \mathbb{R}\setminus\{0\}.
    \end{equation}
    Equivalentemente, escrevemos que $g$ está definida para todo $x\in(-\infty,0)\cup(0,\infty)$, ou ainda, simplesmente para todo $x\neq 0$.

  \item $y=\sqrt{1-x^2}$

    A partir da regra, entendemos que $y$ é função de $x$, i.e. $y\mapsto y(x)$. Aqui, observamos que a raiz quadrada não está definida apenas para números reais não negativos. Logo, esta função está definida para $x$ tal que
    \begin{gather}
      1-x^2 \geq 0\\
      -x^2 \geq -1\\
      x^2 \leq 1\\
      -1 \leq x \leq 1
    \end{gather}
    Concluímos que seu domínio é $x\in (-1, 1)$.
  \end{enumerate}
\end{ex}


Dada uma função $f:D\mapsto Y$, o conjunto de todos os valores $f(x)\in Y$ tal que $x\in D$ é chamado de \pmb{imagem} da função. Em notação de conjunto, temos
\begin{equation}
  I_f = \{y\in Y:~y=f(x)\land x\in D\},
\end{equation}
i.e. o conjunto de todos os valores $y\in Y$ tal que $y=f(x)$ e $x\in D$.

\begin{ex}
  \begin{enumerate}[a)]
  \item $f:\mathbb{R}\to\mathbb{R}, y=x^2$:
    
    Observamos que para qualquer número real $x$, temos $y=x^2\geq 0$. Além disso, para cada número real não negativo $y$, temos que
    \begin{gather}
      x=\sqrt{y} \\
      \Rightarrow x^2 = \left(\sqrt{y}\right)^2\\
      \Rightarrow y = x^2
    \end{gather}
    Logo, concluímos que a imagem de $f$ é
    \begin{equation}
      I_f = \mathbb{R}_{+},
    \end{equation}
    i.e. o conjunto de todos os $y\geq 0$.
    
  \item $y=1/x$:
    
    Primeiramente, observemos que se $y=0$, então não existe número real tal que $0=1/x$. Ou seja, $0$ não pertence a imagem desta função. Por outro lado, dado qualquer número $y\neq 0$, temos que
    \begin{gather}
      x=\frac{1}{y}
      \Rightarrow y = \frac{1}{x}.
    \end{gather}
    Logo, concluímos que a imagem desta função é o conjunto de todos os números reais não nulos, i.e. $(-\infty, 0)\cup (0, \infty)$.
    
  \item $y=\sqrt{1-x^2}$:
    
    No Exemplo \ref{ex:funcao_defgrafico_dominio}, vimos que esta função está definida apenas para $-1 \leq x \leq 1$. Desta forma, temos que
    \begin{gather}
      0 \leq 1-x^2 \leq 1 \\
      \Rightarrow 0\leq \sqrt{1-x^2} \leq 1
    \end{gather}
    Ou seja, a imagem desta função é o intervalo $[0, 1]$.
  \end{enumerate}
\end{ex}

\begin{obs}
  Em aplicações, o domínio e imagem de funções também ficam restritos à modelagem do problema. Por exemplo, pela \href{https://pt.wikipedia.org/wiki/Lei\_geral\_dos\_gases}{Lei geral dos gases}, o produto da pressão $P$ pelo volume $V$ de uma gás é função da temperatura $T$ como segue
  \begin{equation}
    P = \frac{K}{V_0}\cdot T,
  \end{equation}
  onde $V_0$ é o volume fixado do gás e $K>0$ é uma constante que depende do gás. A temperatura é dada em \href{https://pt.wikipedia.org/wiki/Kelvin}{Kelvin}, logo $T\geq 0$. Entendendo a pressão $P$ como função de $T$, temos que o domínio é $T_0 < T < T_1$, onde $T_0$ é a menor temperatura que o gás admite e $T_1$ é a maior temperatura que o gás admite. A imagem é, então, $\frac{K}{V_0}T_0 < P\cdot V < \frac{K}{V_0}T_1$.
\end{obs}

\subsection{Gráfico}

O \emph{gráfico} de uma função $f$ é o conjunto dos pontos ou pares ordenados $(x, f(x))$ tal que $x$ pertence ao domínio da função. Mais precisamente, para uma função $f:D\to \mathbb{R}$, o gráfico é o conjunto
\begin{equation}
  G_f = \{(x, f(x))\in\mathbb{D}\times\mathbb{Y}:~ x\in D_f\}.
\end{equation}

O \pmb{esboço do gráfico} de uma função é, costumeiramente, uma representação geométrica dos pontos de seu gráfico em um plano cartesiano.

\begin{ex}\label{ex:grafico}
  Na sequência, temos os esboços dos gráficos de funções selecionadas.
  \begin{enumerate}[a)]
  \item $f(x) = x^2$

    \begin{figure}[H]
      \centering
      \includegraphics[width=0.7\textwidth]{./cap_funcao/dados/fig_ex_grafico/fig_ex_grafico_x2}
      \caption{Esboço do gráfico de $f(x)=x^2$.}
    \end{figure}

    \ifispython
    No \python, podemos plotar este gráfico com o seguinte código
    \begin{lstlisting}
      from sympy import *
      x = Symbol('x', real=True)
      plot(1/x, (x,-2, 2), ylim=[-6, 6])
    \end{lstlisting}
    \fi

  \item $\displaystyle y=\frac{1}{x}$

    \begin{figure}[H]
      \centering
      \includegraphics[width=0.7\textwidth]{./cap_funcao/dados/fig_ex_grafico/fig_ex_grafico_1x}
      \caption{Esboço do gráfico de $\displaystyle y = \frac{1}{x}$.}
    \end{figure}

    \ifispython
    No \python, podemos plotar este gráfico com o seguinte código
    \begin{lstlisting}
      from sympy import *
      x = Symbols('x', real=True)
      plot(1/x, (x,-2, 2), ylim=[-6, 6])
    \end{lstlisting}
    \fi

  \item $y = \sqrt{1-x^2}$

    \begin{figure}[H]
      \centering
      \includegraphics[width=0.7\textwidth]{./cap_funcao/dados/fig_ex_grafico/fig_ex_grafico_s1x2}
      \caption{Esboço do gráfico de $y = \sqrt{1-x^2}$.}
    \end{figure}

    \ifispython
    No \python, podemos plotar este gráfico com o seguinte código
    \begin{lstlisting}
      from sympy import *
      x = Symbols('x', real=True)
      plot(sqrt(1 - x**2), (x, -1, 1))
    \end{lstlisting}
    \fi
    
  \end{enumerate}
\end{ex}

\subsection{Categorizações de funções}

\begin{flushright}
  [Vídeo] | [Áudio] | \href{https://phkonzen.github.io/notas/contato.html}{[Contatar]}
\end{flushright}

\subsubsection{Funções algébricas}

\emph{Funções algébricas} são funções definidas a partir de somas, subtrações, multiplicações, divisões ou extração de raízes de funções polinomiais. Funções polinomiais e as funções algébricas derivadas são estudas nas próximas seções.

\subsubsection{Funções transcendentes}

\emph{Funções transcendentes} são funções que não são algébricas. Como exemplos, temos as funções trigonométricas, exponencial e logarítmica, as quais são introduzidas nas próximas seções.

\subsubsection{Funções definidas por partes}

\emph{Funções definidas por partes} são funções definidas por diferentes expressões matemáticas em diferentes partes de seu domínio.

Um exemplo fundamental de função definida por partes é a \emph{função valor absoluto}\footnote{Esta função também pode ser definida por $|x| = \sqrt{x^2}$.}
\begin{equation}
  |x| = \left\{
    \begin{array}{ll}
      x &, x\leq 0\\
      -x &, x<0
    \end{array}
\right.
\end{equation}
Vejamos o esboço do seu gráfico dado na Figura \ref{fig:cap_funcao_funabs}.

\begin{figure}[H]
  \centering
  \includegraphics[width=0.7\textwidth]{./cap_funcao/dados/fig_cap_funcao_funabs/fig_cap_funcao_funabs}
  \caption{Esboço do gráfico da função valor absoluto $y=|x|$.}
  \label{fig:cap_funcao_funabs}
\end{figure}

\ifispython
No \sympy, a função valor absoluto é definida por \lstinline!abs()! ou \lstinline!Abs()!. Por exemplo, temos 
\begin{lstlisting}
  In : from sympy import *
  In : abs(-1)
  Out: 1
\end{lstlisting}
\fi

\subsection*{Exercícios}

\begin{flushright}
  [Vídeo] | [Áudio] | \href{https://phkonzen.github.io/notas/contato.html}{[Contatar]}
\end{flushright}

\begin{exer}
  Determine o domínio e a imagem da função identidade, i.e. $f(x) = x$. Então, faça o esboço de seu gráfico.
\end{exer}
\begin{resp}
  Domínio: $\mathbb{R}$; Imagem: $\mathbb{R}$
\end{resp}

\begin{exer}
  Determine o domínio e a imagem da função $f(x) = x^2 + 1$.
\end{exer}
\begin{resp}
  Domínio: $\mathbb{R}$; Imagem: $[1, \infty)$.
\end{resp}

\begin{exer}
  Determine o domínio e a imagem da função $f(x) = 1 - x^2$.
\end{exer}
\begin{resp}
  Domínio: $\mathbb{R}$; Imagem: $(-\infty, 1]$.
\end{resp}

\begin{exer}
  Determine o domínio e a imagem da função
  \begin{equation}
    h(x) = \frac{1}{x-1} - 2.
  \end{equation}
\end{exer}
\begin{resp}
  Domínio: $(-\infty, 1)\cup (1, \infty)$; Imagem: $(-\infty, -2)\cup (-2, \infty)$.
\end{resp}

\begin{exer}
  Determine o domínio e a imagem da função valor absoluto.
\end{exer}
\begin{resp}
  Domínio: $\mathbb{R}$; Imagem: $[0, \infty)$.
\end{resp}

\section{Função afim}\label{cap_funcao_sec_funafim}

\begin{flushright}
  [Vídeo] | [Áudio] | \href{https://phkonzen.github.io/notas/contato.html}{[Contatar]}
\end{flushright}

Uma \emph{função afim} é uma função da forma
\begin{equation}
f(x) = mx + b,
\end{equation}
sendo $m$ e $b$ parâmetros\footnote{números reais.} dados. O parâmetro $m$ é chamado de \pmb{coeficiente angular} e o parâmetro $b$ é chamado de \pmb{coeficiente constante}\footnote{Mais corretamente, coeficiente do termo constante.}.

Quando $m=0$, temos uma \emph{função constante} $f(x) = b$. Esta tem domínio $(-\infty, \infty)$ e imagem $\{b\}$. Quando $b=0$, temos uma \pmb{função linear} $f(x)=mx$, cujo domínio é $(-\infty, \infty)$ e imagem é $(-\infty, \infty)$.

Por outro lado, toda função linear com $m\neq 0$ tem $(-\infty, \infty)$ como domínio e imagem.

\begin{ex}\label{ex:funafim}
  A Figura \ref{fig:ex_funafim} mostra esboços dos gráficos das funções afins $f(x)=-5/2$, $f(x)=2$ e $f(x)=2x-1$.
  
  \begin{figure}[H]
    \centering
    \includegraphics[width=0.7\textwidth]{./cap_funcao/dados/fig_ex_funafim/fig_ex_funafim}
    \caption{Esboços dos gráficos das funções afins $y=-5/2$, $y=2$ e $y=2x-1$ discutidas no Exemplo \ref{ex:funafim}.}
    \label{fig:ex_funafim}
  \end{figure}

  \ifispython
  Com o \sympy, podemos plotar os gráficos destas funções com os seguintes comandos:
\begin{verbatim}
plot(-5/2,(x,-2,2))
plot(2,(x,-2,2))
plot(2*x-1,(x,-2,2))
\end{verbatim}
  \fi
\end{ex}

O lugar geométrico do gráfico de uma função afim é uma reta (ou linha). O coeficiente angular $m$ controla a inclinação da reta em relação ao eixo $x$\footnote{eixo das abscissas}. Quando $m=0$, temos uma reta horizontal. Quando $m>0$ temos uma reta com inclinação positiva (crescente) e, quando $m<0$ temos uma reta com inclinação negativa.

\begin{ex}\label{ex:funlinear}
  A Figura \ref{fig:ex_funlinear} mostra esboços dos gráficos das funções lineares $f_1(x)=\frac{1}{2}x$, $f_2(x) = x$, $f_3(x) = 2x$, $f_4(x)=-2x$, $f_5(x)=-x$ e $f_6(x)=-\frac{1}{2}x$.
  
  \begin{figure}[H]
    \centering
    \includegraphics[width=0.7\textwidth]{./cap_funcao/dados/fig_ex_funlinear/fig_ex_funlinear}
    \caption{Esboços dos gráficos das funções lineares discutidas no Exemplo \ref{ex:funlinear}.}
    \label{fig:ex_funlinear}
  \end{figure}

  \ifispython
  Verifique, plotando os gráficos com o \sympy!
\end{ex}

\begin{figure}[H]
  \centering
  \includegraphics[width=0.7\textwidth]{./cap_funcao/dados/fig_declividade/fig_declividade}
  \caption{Declividade e o coeficiente angular.}
  \label{fig:declividade}
\end{figure}

A inclinação de uma reta é, normalmente, medida pelo ângulo de declividade (veja a Figura \ref{fig:declividade}). Para definirmos este ângulo, sejam $(x_0, y_0)$ e $(x_1, y_1)$, $x_0<x_1$, pontos sobre uma dada reta, gráfico da função afim $f(x)=mx+b$. O ângulo de declividade (ou, simplesmente, a declividade) da reta é, por definição, o ângulo formado pelo segmento que parte de $(x_0, y_0)$ e termina em $(x_1, y_0)$ e o segmente que parte de $(x_0, y_0)$ e termina em $(x_1, y_1)$. Denotando este ângulo por $\theta$, temos
\begin{align}
  \tg\theta &= \frac{y_1-y_0}{x_1-x_0}\\
            &= \frac{mx_1+b-(mx_0+b)}{x_1-x_0}\\
            &= m,
\end{align}
o que justifica chamar $m$ de coeficiente angular.

Quaisquer dois pontos $(x_0, y_0)$ e $(x_1, y_1)$, com $x_0\neq x_1$, determinam uma única função afim (reta) que passa por estes pontos. Para encontrar a expressão desta função, basta resolver o seguinte sistema linear
\begin{align}
  mx_0 + b &= y_0\\
  mx_1 + b &= y_1
\end{align}
Subtraindo a primeira equação da segunda, obtemos
\begin{equation}
  m(x_0-x_1) = y_0-y_1 \Rightarrow m = \frac{y_0-y_1}{x_0-x_1}.
\end{equation}
Daí, substituindo o valor de $m$ na primeira equação do sistema, obtemos
\begin{equation}
  \frac{y_0-y_1}{x_0-x_1}x_0 + b = y_0 \Rightarrow b = -\frac{y_0-y_1}{x_0-x_1}x_0 + y_0.
\end{equation}
Ou seja, a expressão da função linear (equação da reta) que passa pelos pontos $(x_0, y_0)$ e $(x_1, y_1)$ é
\begin{equation}\label{eq:funafim_eq}
  y = \underbrace{\frac{y_0-y_1}{x_0-x_1}}_{m}(x-x_0) + y_0.
\end{equation}

\subsection*{Exercícios resolvidos}

\begin{flushright}
  [Vídeo] | [Áudio] | \href{https://phkonzen.github.io/notas/contato.html}{[Contatar]}
\end{flushright}

\begin{exeresol}
Trace o esboço da reta que representa o gráfico da função afim $f(x) = -x-1$.  
\end{exeresol}
\begin{resol}
  Para esboçar o gráfico de uma função afim, basta traçarmos a reta que passa por quaisquer dois pontos distintos de seu gráfico. Por exemplo, no caso da função $f(x) = -x -1$, temos
  \begin{center}
  \begin{tabular}[H]{r|c}
    $x$ & $y = -x-1$\\\hline
    -1  & 0\\
    1   & -2\\\hline
  \end{tabular}
\end{center}
Assim sendo, marcamos os pontos $(-1, 0)$ e $(1, -2)$ em um plano cartesiano e traçamos a reta que passa por eles. Veja a Figura \ref{fig:exeresol_funafim_grafico}.

\begin{figure}[H]
  \centering
  \includegraphics[width=0.7\textwidth]{./cap_funcao/dados/fig_exeresol_funafim_grafico/fig_exeresol_funafim_grafico}
  \caption{Esboço do gráfico da função afim $f(x)=-x-1$.}
  \label{fig:exeresol_funafim_grafico}
\end{figure}

\ifispython
Com o \sympy, podemos plotar o gráfico da função $f(x)=-x-1$ com o seguinte comando:
\begin{verbatim}
plot(-x-1,(x,-3,3))
\end{verbatim}
\fi
\end{resol}

\begin{exeresol}
  Determine a função afim $f(x) = mx + b$, cujo gráfico contém os pontos $(1, -1)$ e $(2, 1)$.
\end{exeresol}
\begin{resol}
  Vamos usar \eqref{eq:funafim_eq}. Para tanto, tomamos $(x_0, y_0) = (1, -1)$ e $(x_1, y_1) = (2, 1)$. Desta forma, temos
  \begin{equation}
    m = \frac{y_1 - y_0}{x_1 - x_0} = \frac{1 - (-1)}{2 - 1} = 2.
  \end{equation}
  De \eqref{eq:funafim_eq}, temos
  \begin{align}
    f(x) &= m(x-x_0) + y_0\\
         &= 2(x - 1) + (-1) \\
         &= 2x -3.
  \end{align}
  Ou seja, a função afim desejada é $f(x) = 2x - 3$.

  \ifispython
  Com o \sympy, podemos resolver este exercício utilizando o seguinte código\footnote{Veja a Observação \label{obs:cap_funafom_python}.}:
\begin{verbatim}
x0 = 1
y0 = -1

x1 = 2
y1 = 1

m = (y1-y0)/(x1-x0)

print(m*(x-x0) + y0)
\end{verbatim}
  \fi
\end{resol}

\begin{exeresol}
  Verifique se as retas $y = -x - 1$ e $y = 2x - 3$ se interceptam e, caso afirmativo, determine o ponto de interseção.
\end{exeresol}
\begin{resol}
  As retas dadas são gráficos das funções afins $f(x) = -x - 1$ e $g(x) = 2x - 3$. Como os coeficientes angulares de $f(x)$ e $g(x)$ são diferentes, temos que as retas têm ângulos de declividade diferentes e, portanto, são retas concorrentes\footnote{Retas concorrentes são retas que se interceptam em um ponto.}.

  Agora, vamos determinar o ponto de interseção. No ponto de interseção dos gráficos de $f(x)$ e $g(x)$ deve ocorrer que $f(x) = g(x)$. Segue
  \begin{align}
    f(x)=g(x) &\Rightarrow -x-1 = 2x-3\\
              &\Rightarrow 3x = 2\\
              &\Rightarrow x = \frac{2}{3}.
  \end{align}
  Assim, temos que as retas se interceptam no ponto de abscissa $x = 2/3$. Para determinar a ordenada deste ponto, podemos usar qualquer uma das funções. Usando $f(x)$ temos
  \begin{align}
    y = f\left(\frac{2}{3}\right) = 2\frac{2}{3} - 3 = \frac{4 - 9}{3} = -\frac{5}{3}.
  \end{align}
  Concluímos que as retas se interceptam no ponto $(\frac{2}{3}, -\frac{5}{3})$.

    \ifispython
    Com o \sympy, podemos resolver este exercício utilizando o seguinte código\footnote{Veja a Observação \label{obs:cap_funafom_python}.}:
\begin{verbatim}
f = lambda x: -x-1
g = lambda x: 2*x-3

px = solve(f(x)-g(x))[0]
py = f(px)

print(px, py)
\end{verbatim}
  \fi
\end{resol}

\subsection*{Exercícios}

\begin{flushright}
  [Vídeo] | [Áudio] | \href{https://phkonzen.github.io/notas/contato.html}{[Contatar]}
\end{flushright}

\begin{ex}
  Faça um esboço do gráfico de cada uma das seguintes funções:
  \begin{enumerate}[a)]
  \item $f_1(x) = x$
  \item $f_2(x) = -x$
  \item $f_3(x) = x-1$
  \item $f_4(x) = -x+1$
  \end{enumerate}
\end{ex}

\begin{ex}
  Determine a função afim $f(x)=mx+b$, cujo gráfico contém os pontos $(-2, 1)$ e $(0, -2)$.
\end{ex}
\begin{resp}
  $f(x) = -\frac{3}{2}x - 2$
\end{resp}

\begin{ex}
  Determine o ponto de interseção dos gráficos das funções afins $f(x) = 2x + 1$ e $g(x) = 2x -1$.
\end{ex}
\begin{resp}
  não há.
\end{resp}

\section{Função potência}\label{cap_funcao_sec_funpot}

\begin{flushright}
  [Vídeo] | [Áudio] | \href{https://phkonzen.github.io/notas/contato.html}{[Contatar]}
\end{flushright}

Uma função da forma $f(x)=x^n$, onde $n\neq 0$ é uma constante, é chamada de \emph{função potência}.

Funções potências têm comportamentos característicos, conforme o valor de $n$. Quando $n$ é um inteiro positivo ímpar, seu domínio e sua imagem são $(-\infty, \infty)$. Veja a Figura \ref{fig:funpot_impar}.

\begin{figure}[H]
  \centering
  \includegraphics[width=0.7\textwidth]{./cap_funcao/dados/fig_funpot_impar/fig_funpot_impar}
  \caption{Esboços dos gráficos das funções potências $y=x$, $y=x^3$ e $y=x^5$.}
  \label{fig:funpot_impar}
\end{figure}

Funções potências com $n$ positivo par estão definidas em toda parte e têm imagem $[0, \infty)$. Veja a Figura \ref{fig:funpot_par}.

\begin{figure}[H]
  \centering
  \includegraphics[width=0.7\textwidth]{./cap_funcao/dados/fig_funpot_par/fig_funpot_par}
  \caption{Esboços dos gráficos das funções potências $y=x^2$, $y=x^4$ e $y=x^6$.}
  \label{fig:funpot_par}
\end{figure}

Funções potências com $n$ inteiro negativo ímpar não são definidas em $x=0$, tendo domínio e imagem igual a $(-\infty, 0)\cup (0, \infty)$. Também, quando $n$ inteiro negativo par, a função potência não está definida em $x=0$, tem domínio $(-\infty, 0)\cup (0, \infty)$, mas imagem $(0, \infty)$. Veja a Figura \ref{fig:funpot_negativo}.

\begin{figure}[H]
  \centering
  \includegraphics[width=0.5\textwidth]{./cap_funcao/dados/fig_funpot_negativo/fig_funpot_negativo_impar}~
    \includegraphics[width=0.5\textwidth]{./cap_funcao/dados/fig_funpot_negativo/fig_funpot_negativo_par}
  \caption{Esboços dos gráficos das funções potências $y=1/x$ (esquerda), $y=1/x^2$ (direita).}
  \label{fig:funpot_negativo}
\end{figure}

Há, ainda, comportamentos característicos quando $n=1/2$, $1/3$, $3/2$ e $2/3$. Veja a Figura \ref{fig:funpot_racional}.

\begin{figure}[H]
  \centering
  \includegraphics[width=0.5\textwidth]{./cap_funcao/dados/fig_funpot_racional/fig_funpot_racional_par}~
    \includegraphics[width=0.5\textwidth]{./cap_funcao/dados/fig_funpot_racional/fig_funpot_racional_impar}
  \caption{Esboços dos gráficos das funções potências. Esquerda $y=\sqrt{x}$ e $y=\sqrt{x^3}$. Direita: $y=\sqrt[3]{x}$ e $y=\sqrt[3]{x^2}$.}
  \label{fig:funpot_racional}
\end{figure}


\subsection*{Exercícios resolvidos}

\begin{flushright}
  [Vídeo] | [Áudio] | \href{https://phkonzen.github.io/notas/contato.html}{[Contatar]}
\end{flushright}

\begin{exeresol}\label{exeresol:funpot_graf}
  Determine o domínio e faça um esboço do gráfico de cada uma das seguintes funções:
  \begin{enumerate}[a)]
  \item $\displaystyle f(x) = x^{5/2}$;
  \item $\displaystyle g(x) = x^{5/3}$.
  \end{enumerate}
\end{exeresol}
\begin{resol}
  \begin{enumerate}[a)]
  \item Vamos analisar a função $f(x) = x^{5/2}$. Como $x^{5/2} = \sqrt{x^5}$ e não existe a raiz quadrada de número negativo, temos que $x^5$ deve ser não negativo. Daí, $x$ deve ser não negativo. Logo, o domínio de $f(x) = x^{5/2}$ é $[0, \infty)$. Veja o esboço desta função na Figura \ref{fig:exeresol_funpot_graf_a}.

    \begin{figure}[H]
      \centering
      \includegraphics[width=0.5\textwidth]{./cap_funcao/dados/fig_exeresol_funpot_graf/fig_exeresol_funpot_graf_a}
      \caption{Esboço do gráfico de $f(x) = x^{5/2}$.}
      \label{fig:exeresol_funpot_graf_a}
    \end{figure}

    \ifispython
    Para plotar o gráfico de $f(x)$ com o \sympy, basta digitar, por exemplo:
\begin{verbatim}
plot(x**(5/2),(x,0,2))
\end{verbatim}
    \fi
  \item Vamos analisar a função $g(x) = x^{5/3}$. Como $x^{5/3} = \sqrt[3]{x^5}$, não temos restrição sobre os valores de $x$. Logo, o domínio da função $g$ é $(-\infty, \infty)$. Veja o esboço desta função na Figura \ref{fig:exeresol_funpot_graf_b}.

    \begin{figure}[H]
      \centering
      \includegraphics[width=0.5\textwidth]{./cap_funcao/dados/fig_exeresol_funpot_graf/fig_exeresol_funpot_graf_b}
      \caption{Esboço do gráfico de $g(x) = x^{5/3}$.}
      \label{fig:exeresol_funpot_graf_b}
    \end{figure}

    \ifispython
    Para plotar o gráfico de $g(x)$ com o \sympy, digitamos:
\begin{verbatim}
p = plot(x**(5/3),(x,0,2),line_color="blue",show=False)
q = plot(-(-x)**(5/3),(x,-2,0),line_color="blue",show=False)
p.extend(q)
p.show()
\end{verbatim}
    Você sabe o porquê não pode-se usar, simplesmente, o seguinte comando?
\begin{verbatim}
plot(x**(5/3),(x,-2,2))
\end{verbatim}
    \fi
  \end{enumerate}
\end{resol}

\begin{exeresol}\label{exeresol:funpot_intersep}
  Determine a equação da reta que passa pelos pontos de interseção dos gráficos das funções $f(x) = 1/x$ e $g(x) = \sqrt[3]{x}$.
\end{exeresol}
\begin{resol}
  Para determinarmos a reta precisamos, antes, dos pontos de interseção. As funções se interceptam nos pontos de abscissa $x$ tais que
  \begin{align}
    f(x) = g(x) &\Rightarrow \frac{1}{x} = \sqrt[3]{x}\\
                &\Rightarrow 1 = x\sqrt[3]{x}\\
                &\Rightarrow 1 = x\cdot x^{\frac{1}{3}}\\
                &\Rightarrow x^{1+\frac{1}{3}} = 1\\
                &\Rightarrow x^{\frac{4}{3}} = 1\\
                &\Rightarrow x^4 = \sqrt[3]{1}\\
                &\Rightarrow x^4 = 1\\
                &\Rightarrow x_0 = -1\quad\text{ou}\quad x_1=1.
  \end{align}
  Ou seja, os gráficos se interceptam nos pontos de abscissas $x_0 = -1$ e $x_1 = 1$. Veja o esboço dos gráficos das funções na Figura \ref{fig:exeresol_funpot_intersep}. Agora, podemos usar qualquer uma das funções para obter as ordenadas dos pontos de interseção. Usando $f(x)$, temos
  \begin{equation}
    (x_0, y_0) = (x_0, f(x_0)) = (-1, -1)\quad\text{e}\quad (x_1, y_1) = (x_1, f(x_1)) = (1, 1).
  \end{equation}

  \begin{figure}[H]
    \centering
    \includegraphics[width=0.5\textwidth]{./cap_funcao/dados/fig_exeresol_funpot_intersep/fig_exeresol_funpot_intersep}
    \caption{Interseção dos gráficos das funções $f(x) = 1/x$ (azul) e $g(x) = \sqrt[3]{x}$ (vermelho).}
    \label{fig:exeresol_funpot_intersep}
  \end{figure}

  Agora, basta determinarmos a equação da reta que passa pelos pontos $(x_0, y_0) = (-1, -1)$ e $(x_1, y_1) = (1, 1)$. De \eqref{eq:funafim_eq}, temos que a equação da reta é tal que
  \begin{align}
    y = \frac{y_1-y_0}{x_1-x_0}(x-x_0)+y_0 &\Rightarrow y = \frac{1-(-1)}{1-(-1)}(x-(-1))+(-1)\\
                                           &\Rightarrow y = x+1-1 \Rightarrow y = x.
  \end{align}
  Ou seja, a que passa pelos pontos de interseção dos gráficos das funções $f(x)$ e $g(x)$ tem equação $y = x$.

  \ifispython
  Os seguintes comandos, mostrar como podemos resolver este problema usando o \sympy\footnote{Veja a Observaçao \ref{cap_funcao_python}.}:
\begin{verbatim}
f = lambda x: 1/x
# x nao negativo
g1 = lambda x: cbrt(x)
# x negativo
g2 = lambda x: -cbrt(-x)

x0 = solve(f(x)-g2(x))[0]
x1 = solve(f(x)-g1(x))[0]

y0 = f(x0)
y1 = f(x1)

print('y = ',(y1-y0)/(x1-x0)*(x-x0)+y0)
\end{verbatim}
  \fi
\end{resol}

\subsection*{Exercícios}

\begin{flushright}
  [Vídeo] | [Áudio] | \href{https://phkonzen.github.io/notas/contato.html}{[Contatar]}
\end{flushright}

\begin{exer}
  Determine o domínio, a imagem e faça um esboço do gráfico de cada uma das seguintes funções:
  \begin{enumerate}[a)]
  \item $f(x) = x^7$;
  \item $g(x) = x^8$.
  \end{enumerate}
\end{exer}
\begin{resp}
  a) domínio: $(-\infty, \infty)$; imagem: $(-\infty, \infty)$. b) domínio: $(-\infty, \infty)$; imagem: $[0, \infty)$. Dica: use o \href{https://www.sympygamma.com/}{SymPy Gamma} para verificar os esboços de seus gráficos.
\end{resp}

\begin{exer}
  Determine o domínio, a imagem e faça um esboço do gráfico de cada uma das seguintes funções:
  \begin{enumerate}[a)]
  \item $\displaystyle f(x) = \frac{1}{x^7}$;
  \item $\displaystyle g(x) = \frac{1}{x^8}$.
  \end{enumerate}
\end{exer}
\begin{resp}
  a) domínio: $(-\infty, \infty)\setminus\{0\}$; imagem: $(-\infty, \infty)\setminus\{0\}$. b) domínio: $(-\infty, \infty)\setminus\{0\}$; imagem: $(0, \infty)$. Dica: use o \href{https://www.sympygamma.com/}{SymPy Gamma} para verificar os esboços de seus gráficos.
\end{resp}

\begin{exer}
  Determine o domínio, a imagem e faça um esboço do gráfico de cada uma das seguintes funções:
  \begin{enumerate}[a)]
  \item $\displaystyle f(x) = \sqrt{x^2}$;
  \item $\displaystyle g(x) = \sqrt[3]{x^3}$.
  \end{enumerate}
\end{exer}
\begin{resp}
  a) domínio: $(-\infty, \infty)$; imagem: $[0, \infty)$. b) domínio: $(-\infty, \infty)$; imagem: $(-\infty, \infty)$. Dica: use o \href{https://www.sympygamma.com/}{SymPy Gamma} para verificar os esboços de seus gráficos.
\end{resp}


\section{Função polinomial}\label{cap_funcao_sec_funpoli}

\begin{flushright}
  [Vídeo] | [Áudio] | \href{https://phkonzen.github.io/notas/contato.html}{[Contatar]}
\end{flushright}

Uma \emph{função polinomial} (\emph{polinômio}) tem a forma
\begin{equation}
  p(x) = a_nx^n + a_{n-1}x^{n-1} + \cdots + a_1x + a_0,
\end{equation}
onde $a_i$ são coeficientes reais, $a_n\neq 0$ e $n$ é inteiro não negativo, este chamado de \emph{grau do polinômio}.

Polinômios são definidos em toda parte\footnote{Uma função é dita ser definida em toda parte quando seu domínio é $(\infty, \infty)$}. Polinômios de grau ímpar tem imagem $(-\infty, \infty)$. Entretanto, a imagem polinômios de grau par dependem de cada caso. Iremos estudar mais propriedades de polinômios ao longo do curso de cálculo. Veja a Figura \ref{fig:poli_graficos}.

\begin{figure}[H]
  \centering
  \includegraphics[width=0.5\textwidth]{./cap_funcao/dados/fig_poli_graficos/fig_poli_impar}~
    \includegraphics[width=0.5\textwidth]{./cap_funcao/dados/fig_poli_graficos/fig_poli_par}
  \caption{Esboços dos gráficos das funções polinomiais. Esquerda $p(x) = x^{3} - 2.5 x^{2} - 1.0 x + 2.5$. Direita: $q(x) = x^{4} - 3.5 x^{3} + 1.5 x^{2} + 3.5 x - 2.5$.}
  \label{fig:poli_graficos}
\end{figure}

Quando $n=0$, temos um polinômio de grau 0 (ou uma função constante). Quando $n=1$, temos um polinômio de grau 1 (ou, uma função afim). Ainda, quando $n=2$ temos uma \emph{função quadrática} (ou \emph{polinômio quadrático}) e, quando $n=3$, temos uma \emph{função cúbica} (ou \emph{polinômio cúbico}).

\subsection{Função quadrática}

\begin{flushright}
  [Vídeo] | [Áudio] | \href{https://phkonzen.github.io/notas/contato.html}{[Contatar]}
\end{flushright}

Os polinômios de grau 2 são, também, chamados de \pmb{funções quadráticas}, i.e. funções da forma
\begin{equation}
  f(x) = ax^2 + bx + c,
\end{equation}
onde $a$ é chamado de \pmb{coeficiente do termo quadrático}, $b$ o \pmb{coeficiente do termo linear} e $c$ o \pmb{coeficiente do termo constante}.

Os zeros de uma função quadrática podem ser calculados pela \pmb{fórmula de Bhaskara}
\begin{equation}\label{eq:Bhaskara}
  x_0, x_1 = \frac{-b \pm \sqrt{b^2 - 4ac}}{2a}.
\end{equation}


O esboço do gráfico de uma função quadrática é uma \pmb{parábola côncava para cima} quando $a > 0$ e, \pmb{côncava para baixo} quando $x < 0$. Veja a Figura \ref{fig:funquad_concavidade}.

\begin{figure}[H]
  \centering
  \includegraphics[width=0.5\textwidth]{./cap_funcao/dados/fig_funquad_concavidade/fig_funquad_concavidade_cima}~
    \includegraphics[width=0.5\textwidth]{./cap_funcao/dados/fig_funquad_concavidade/fig_funquad_concavidade_baixo}
  \caption{Esboço dos gráficos das funções quadráticas $f(x) = x^2-x-2$ (esquerda) e $g(x)=-x^2+x+2$ (direita).}
  \label{fig:funquad_concavidade}
\end{figure}

O \pmb{vértice} da função quadrática $f(x)$ com coeficiente quadrático positivo (com coeficiente quadrático negativo) é o ponto no qual ela atinge seu {\bf valor máximo (mínimo)} em todo o seu domínio natural. Quando $f$ têm zeros reais, o ponto de abscissa do vértice é o ponto médio entre os zeros $x_0$ e $x_1$ da função, i.e. o vértice $V = (x_v, y_v)$ é tal que
\begin{equation}
  x_v = \frac{x_0 + x_1}{2},\quad\text{e}\quad y_v = f(x_v). 
\end{equation}
O valor $x_v$ é a abscissa do ponto em que a função quadrática $f$ atinge o valor máximo (valor mínimo) $y_v$.


\subsection*{Exercícios resolvidos}

\begin{flushright}
  [Vídeo] | [Áudio] | \href{https://phkonzen.github.io/notas/contato.html}{[Contatar]}
\end{flushright}

\begin{exeresol}
  Determine os zeros do polinômio $f(x) = x^3-x^2-2x$.
\end{exeresol}
\begin{resol}
  Determinar os zeros da função $f$ significa entrar todos os valores de $x$ tais que $f(x)=0$ (estes são as abscissas dos pontos nos quais o gráfico de $f$ intersepta o eixo das abscissas). Temos
  \begin{align}
    f(x)=0 &\Rightarrow x^3-x^2-2x=0\\
           &\Rightarrow x(x^2-x-2)=0\\
           &\Rightarrow x=0\quad\text{ou}\quad x^2-x-2=0.
  \end{align}
  Então, usando a fórmula de Bhaskara \eqref{eq:Bhaskara} na equação $x^2-x-2=0$, obtemos
  \begin{align}
    x &= \frac{-b\pm\sqrt{b^2-4ac}}{2a} \\
      &= \frac{1\pm\sqrt{1-4\cdot 1\cdot (-2)}}{2}\\
      &= \frac{1\pm\sqrt{9}}{2}\\
      &= \frac{1\pm 3}{2}\\
      &= -1\quad\text{ou}\quad 2
  \end{align}
  Com isso, temos que os zeros da função $f$ ocorrem nos pontos $x_0 = -1$, $x_1=0$ e $x_2=2$.

  \ifispython
  Com o \sympy, podemos calcular os zeros da função $f$ com o seguinte comando:
\begin{verbatim}
solve(x**3-x**2-2*x)
\end{verbatim}
  \fi
\end{resol}

\begin{exeresol}
  Determine o valor mínimo da função $f(x) = x^2 - x - 2$.
\end{exeresol}
\begin{resol}
  Como $f$ é uma função quadrática com coeficiente quadrático positivo, temos que seu gráfico é uma parábola côncava para cima. Logo, $f$ atinge seu valor mínimo no seu vértice. Por sorte, os zeros de $f$ são $x_0 = -1$ e $x_1 = 2$. Logo, o vértice tem abscissa
  \begin{equation}
    x = \frac{x_0 + x_1}{2} = \frac{1}{2}.
  \end{equation}
  Ou seja, a abscissa do ponto de mínimo de $f$ é $1/2$ e seu valor mínimo é
  \begin{equation}
    f\left(\frac{1}{2}\right) = \left(\frac{1}{2}\right)^2-\frac{1}{2}-2 = \frac{1-2-8}{4} = -\frac{9}{4}.
  \end{equation}

  \ifispython
  Podemos resolver este exercício com o seguinte código \sympy:
\begin{verbatim}
f = lambda x: x**2-x-2
z = solve(f(x))
f((z[0]+z[1])/2)
\end{verbatim}
  \fi
\end{resol}

\subsection*{Exercícios}

\begin{flushright}
  [Vídeo] | [Áudio] | \href{https://phkonzen.github.io/notas/contato.html}{[Contatar]}
\end{flushright}

\begin{exer}
  Determine os zeros do polinômio $f(x) = -x^3+x^2+2x$.
\end{exer}
\begin{resp}
  $-1$, $0$, $2$
\end{resp}

\begin{exer}
  Determine o valor máximo da função $f(x) = -x^2 + x + 2$.
\end{exer}
\begin{resp}
  $9/4$
\end{resp}

\section{Função racional}\label{cap_funcao_sec_funracio}

\begin{flushright}
  [Vídeo] | [Áudio] | \href{https://phkonzen.github.io/notas/contato.html}{[Contatar]}
\end{flushright}

Uma \emph{função racional} tem a forma
\begin{equation}
  f(x) = \frac{p(x)}{q(x)},
\end{equation}
onde $p(x)$ e $q(x)\not\equiv 0$ são polinômios.

Funções racionais não estão definidas nos zeros de $q(x)$. Além disso, suas imagens dependem de cada caso. Estudaremos o comportamento de funções racionais ao longo do curso de cálculo. Como exemplo, veja a Figura \ref{fig:racional_grafico} para um esboço do gráfico da função racional
\begin{equation}
  f(x) = \frac{x^2-x-2}{x^3-x^2+x-1}.
\end{equation}

\begin{figure}[H]
  \centering
  \includegraphics[width=0.8\textwidth]{./cap_funcao/dados/fig_racional_grafico/fig_racional_grafico}
  \caption{Esboço do gráfico da função racional $f(x) = \frac{x^{2} - x - 2}{x^{3} - x^{2} + x - 1}$.}
  \label{fig:racional_grafico}
\end{figure}

Com o estudo do cálculo de limites, veremos que a reta $y = 0$ (eixo das abscissas) é uma assíntota horizontal e a reta $x=1$ (reta tracejada) é uma assíntota vertical ao gráfico desta função. Esta singularidade no ponto $x=1$ está relacionada ao fato de que o denominador se anula em $x=1$. Ainda, para $x\neq 1$ temos
\begin{equation}
  \frac{x^3 - x^2 + x - 1}{x-1} = x^2 + 1,
\end{equation}
Com isso, podemos concluir que o domínio da função $f(x)$ é $\mathbb{R}\setminus\{0\}$.


\subsection*{Exercícios resolvidos}

\begin{flushright}
  [Vídeo] | [Áudio] | \href{https://phkonzen.github.io/notas/contato.html}{[Contatar]}
\end{flushright}

\begin{exer}
  Determine o domínio da função racional
  \begin{equation}
    f(x) = \frac{x^3-x^2+x-1}{x^2-1}.
  \end{equation}
\end{exer}
\begin{resol}
  Como $f(x)$ é uma função racional, ela não está definida nos zeros do polinômio que constitui seu denominador. I.e., nos pontos
  \begin{equation}
    x^2-1=0\Rightarrow x = \pm 1.
  \end{equation}
  Logo, o domínio de $f(x)$ é o conjunto $\mathbb{R}\setminus\{-1, 1\}$.
\end{resol}

\begin{exer}
  Determine o domínio e faça o esboço do gráfico da função racional
  \begin{equation}
    g(x) = \frac{x-1}{x-1}.
  \end{equation}
\end{exer}
\begin{resol}
  Tendo em vista que o denominador se anula em $x=1$, o domínio de $g$ é $(-\infty, 0)\cup (0, \infty)$. Agora, para fazermos um esboço de seu gráfico, observamos que $g(x)=1$ para $x\neq 1$. I.e., $g$ é uma função constante para valores de $x\neq 1$ e não está definida em $x=1$. Veja a Figura \ref{fig:exeresol_funracio_graf} para o esboço do gráfico da função $g$.

  \begin{figure}[H]
    \centering
    \includegraphics[width=0.8\textwidth]{./cap_funcao/dados/fig_exeresol_funracio_graf/fig_exeresol_funracio_graf}
    \caption{Esboço do gráfico da função $g(x) = (x-1)/(x-1)$.}
    \label{fig:exeresol_funracio_graf}
  \end{figure}

  \ifispython
  Com o \sympy, o comando
\begin{verbatim}
plot((x-1)/(x-1),(x,-2,2))
\end{verbatim}
  plota uma linha constante, sem identificar a singularidade em $x=1$. Isto ocorre, pois os gráficos com o \sympy~ são obtidos a partir de uma amostra discreta de pontos. Ocorre que esta amostra pode não conter as singularidades. No caso de conter, a execução pode não plotar o gráfico e retornar um erro.

  Devemos ficar atentos a esboços de gráficos obtidos no computador, muitas vezes os gráficos podem estar errados. Cabe ao usuário identificar e analisar pontos e região de interesse.
  \fi
\end{resol}

\subsection*{Exercícios}

\begin{flushright}
  [Vídeo] | [Áudio] | \href{https://phkonzen.github.io/notas/contato.html}{[Contatar]}
\end{flushright}

\begin{exer}
  Determine o domínio da função racional
  \begin{equation}
    f(x) = \frac{x^2-1}{x^3-x}.
  \end{equation}
\end{exer}
\begin{resp}
  $\mathbb{R}\setminus\{0\}$
\end{resp}


\section{Funções trigonométricas}\label{cap_funcao_sec_funtri}

\begin{flushright}
  [Vídeo] | [Áudio] | \href{https://phkonzen.github.io/notas/contato.html}{[Contatar]}
\end{flushright}

\subsection{Seno e cosseno}

\begin{flushright}
  [Vídeo] | [Áudio] | \href{https://phkonzen.github.io/notas/contato.html}{[Contatar]}
\end{flushright}

As funções trigonométricas seno $y=\sen(x)$ e cosseno $y=\cos(x)$ podem ser definidas a partir do círculo trigonométrico (veja a Figura \ref{fig:cos_seno}). Seja $x$ o ângulo\footnote{Em geral utilizaremos a medida em radianos para ângulos.} de declividade da reta que passa pela origem do plano cartesiano (reta $r$ na Figura \ref{fig:cos_seno}). Seja, então, $(a,b)$ o ponto de interseção desta reta com a circunferência unitária\footnote{Circunferência do círculo de raio 1.}. Então, definimos:
\begin{equation}
  \sen(x) = a,\qquad \cos(x) = b.
\end{equation}
A partir da definição, notemos que ambas funções têm domínio $(-\infty, \infty)$ e imagem $[-1, 1]$.

\begin{figure}[H]
  \centering
  \includegraphics[width=0.8\textwidth]{./cap_funcao/dados/fig_cos_seno/fig_cos_seno}
  \caption{Funções seno e cosseno no círculo trigonométrico.}
  \label{fig:cos_seno}
\end{figure}

Na Figura \ref{fig:cos_seno_valores} podemos extrair os valores das funções seno e cosseno para os ângulos fundamentais. Por exemplo, temos
\begin{align}
  &\sen\left(\frac{\pi}{6}\right) = \frac{1}{2},\qquad \cos\left(\frac{\pi}{6}\right) = \frac{\sqrt{3}}{2},\\
  &\sen\left(\frac{3\pi}{4}\right) = \frac{\sqrt{2}}{2},\qquad \cos\left(\frac{\pi}{4}\right) = -\frac{\sqrt{2}}{2},\\
  &\sen\left(\frac{8\pi}{6}\right) = -\frac{\sqrt{3}}{2},\qquad \cos\left(\frac{8\pi}{6}\right) = -\frac{1}{2},\\
  &\sen\left(\frac{11\pi}{6}\right) = -\frac{1}{2},\qquad \cos\left(\frac{11\pi}{6}\right) = \frac{\sqrt{3}}{2},\\
\end{align}
\ifispython
As funções seno e cosseno estão definidas no \sympy~ como \verb+sin+ e $\verb+cos+$, respectivamente. Por exemplo, para computar o seno de $\pi/6$, digitamos:
\begin{verbatim}
sin(pi/6)
\end{verbatim}
\fi

\begin{figure}[H]
  \centering
  \includegraphics[width=0.8\textwidth]{./cap_funcao/dados/fig_cos_seno_valores/fig_cos_seno_valores}
  \caption{Funções seno e cosseno no círculo trigonométrico.}
  \label{fig:cos_seno_valores}
\end{figure}

Uma \emph{função} $f(x)$ é dita \emph{periódica} quando existe um número $p$, chamado de período da função, tal que
\begin{equation}
  f(x+p) = f(x)
\end{equation}
para qualquer valor de $x$ no domínio da função. Da definição das funções seno e cosseno, notemos que ambas são periódicas com período $2\pi$, i.e.
\begin{equation}
  \sen(x+2\pi) = \sen(x),\qquad \cos(x+2\pi) = \cos(x),
\end{equation}
para qualquer valor de $x$.

Na Figura \ref{fig:cos_seno_graficos}, temos os esboços dos gráficos das funções seno e cosseno.

\begin{figure}[H]
  \centering
  \includegraphics[width=0.5\textwidth]{./cap_funcao/dados/fig_cos_seno_graficos/fig_seno_grafico}~
  \includegraphics[width=0.5\textwidth]{./cap_funcao/dados/fig_cos_seno_graficos/fig_cosseno_grafico}
  \caption{Esboços dos gráficos das funções seno (esquerda) e cosseno (direita).}
  \label{fig:cos_seno_graficos}
\end{figure}

\subsection{Tangente, cotangente, secante e cossecante}

\begin{flushright}
  [Vídeo] | [Áudio] | \href{https://phkonzen.github.io/notas/contato.html}{[Contatar]}
\end{flushright}

Das funções seno e cosseno, definimos as funções \emph{tangente}, \emph{cotangente}, \emph{secante} e \emph{cossecante} como seguem:
\begin{align}
  \tg(x) := \frac{\sen(x)}{\cos(x)},\qquad \cotg(x) := \frac{\cos(x)}{\sen(x)},\\
  \sec(x) := \frac{1}{\cos(x)},\qquad \cosec(x) := \frac{1}{\sen(x)}.
\end{align}

\ifispython
No \sympy, as funções tangente, cotangente, secante e cossecante podem ser computadas com as funções $\verb+tan+$, $\verb+cot+$, $\verb+sec+$ e $\verb+csc+$, respectivamente. Por exemplo, podemos computar o valor de $\cosec(\pi/4)$ com o comando
\begin{verbatim}
csc(pi/4)
\end{verbatim}
\fi

Na Figura \ref{fig:co_tg_graficos}, temos os esboços dos gráficos das funções tangente e cotangente. Observemos que a função tangente não está definida nos pontos $(2k+1)\pi/2$, para todo $k$ inteiro. Já, a função cotangente não está definida nos pontos $k\pi$, para todo $k$ inteiro. Ambas estas funções têm imagem $(-\infty, \infty)$ e período $\pi$.

\begin{figure}[H]
  \centering
  \includegraphics[width=0.5\textwidth]{./cap_funcao/dados/fig_co_tg_graficos/fig_tg_grafico}~
  \includegraphics[width=0.5\textwidth]{./cap_funcao/dados/fig_co_tg_graficos/fig_cotg_grafico}
  \caption{Esboços dos gráficos das funções tangente (esquerda) e cotangente(direita).}
  \label{fig:co_tg_graficos}
\end{figure}

Na Figura \ref{fig:co_sec_graficos}, temos os esboços dos gráficos das funções secante e cossecante. Observemos que a função secante não está definida nos pontos $(2k+1)\pi/2$, para todo $k$ inteiro. Já, a função cossecante não está definida nos pontos $k\pi$, para todo $k$ inteiro. Ambas estas funções têm imagem $(-\infty, 1]\cup [1, \infty)$ e período $\pi$.

\begin{figure}[H]
  \centering
  \includegraphics[width=0.5\textwidth]{./cap_funcao/dados/fig_co_sec_graficos/fig_sec_grafico}~
  \includegraphics[width=0.5\textwidth]{./cap_funcao/dados/fig_co_sec_graficos/fig_cosec_grafico}
  \caption{Esboços dos gráficos das funções tangente (esquerda) e cotangente(direita).}
  \label{fig:co_sec_graficos}
\end{figure}

\subsection{Identidades trigonométricas}

\begin{flushright}
  [Vídeo] | [Áudio] | \href{https://phkonzen.github.io/notas/contato.html}{[Contatar]}
\end{flushright}

Aqui, vamos apresentar algumas identidades trigonométricas que serão utilizadas ao longo do curso de cálculo. Comecemos pela identidade fundamental
\begin{equation}
  \sen^2 x + \cos^2 x = 1.
\end{equation}
Desta decorrem as identidades
\begin{align}
  &\tg^2(x) + 1 = \sec^2 x,\\
  &1 + \cotg^2(x) = \cosec^2(x).
\end{align}

Das seguintes fórmulas para adição/subtração de ângulos
\begin{align}
  &\cos(x\pm y) = \cos(x)\cos(y) \mp \sen(x)\sen(y),\\
  &\sen(x\pm y) = \sen(x)cos(y) \pm \cos(x)\sen(y),
\end{align}
seguem as fórmulas para ângulo duplo
\begin{align}
  &\cos(2x) = \cos^2x - \sen^2x,\\
  &\sen(2x) = 2\sen x\cos x.
\end{align}

Também, temos as fórmulas para o ângulo metade
\begin{align}
  &\cos^2 x = \frac{1 + \cos 2x}{2},\\
  &\sen^2 x = \frac{1 - \cos 2x}{2}.\label{eq:id_trig_cos_x2}
\end{align}

\subsection*{Exercícios resolvidos}

\begin{flushright}
  [Vídeo] | [Áudio] | \href{https://phkonzen.github.io/notas/contato.html}{[Contatar]}
\end{flushright}

\begin{exeresol}
  Mostre que
  \begin{equation}
    \cos x - 1 = -2\sen^2 \frac{x}{2}.
  \end{equation}
\end{exeresol}
\begin{resol}
  A identidade trigonométrica
  \begin{equation}
    \sen^2 x = \frac{1 - \cos 2x}{2},
  \end{equation}
  aplicada a metade do ângulo, fornece
  \begin{equation}
    \sen^2 \frac{x}{2} = \frac{1 - \cos x}{2}.
  \end{equation}
  Então, isolando $\cos x$, obtemos
  \begin{align}
    \sen^2 \frac{x}{2} = \frac{1 - \cos x}{2} &\Rightarrow 1 - \cos x = 2\sen^2 \frac{x}{2}\\
                                              &\Rightarrow \cos x - 1 = -2\sen^2 \frac{x}{2}.
  \end{align}
\end{resol}

\subsection*{Exercícios}

\begin{flushright}
  [Vídeo] | [Áudio] | \href{https://phkonzen.github.io/notas/contato.html}{[Contatar]}
\end{flushright}

\begin{exer}
  Mostre que $\sen x$ é  uma função ímpar, i.e.
  \begin{equation}
    \sen x = \sen(-x)
  \end{equation}
  para todo número real $x$.
\end{exer}
\begin{resp}
  Dica: analise o ciclo trigonométrico.
\end{resp}

\begin{exer}
  Mostre que $\cos x$ é  uma função par, i.e.
  \begin{equation}
    \cos x = -\cos(-x)
  \end{equation}
  para todo número real $x$.
\end{exer}
\begin{resp}
  Dica: analise o ciclo trigonométrico.
\end{resp}

\section{Operações com funções}\label{cap_funcao_sec_opfun}

\begin{flushright}
  [Vídeo] | [Áudio] | \href{https://phkonzen.github.io/notas/contato.html}{[Contatar]}
\end{flushright}

\subsection{Somas, diferenças, produtos e quocientes}

\begin{flushright}
  [Vídeo] | [Áudio] | \href{https://phkonzen.github.io/notas/contato.html}{[Contatar]}
\end{flushright}

Sejam dadas as funções $f$ e $g$ com domínio em comum $D$. Então, definimos as funções
\begin{itemize}
\item $(f\pm g)(x) := f(x) \pm g(x)$ para todo $x\in D$;
\item $(fg)(x) := f(x)g(x)$ para todo $x\in D$;
\item $\displaystyle \left(\frac{f}{g}\right)(x) := \frac{f(x)}{g(x)}$ para todo $x\in D$ tal que $g(x)\neq 0$.
\end{itemize}

\begin{ex}
  Sejam $f(x)=x^2$ e $g(x)=x$. Temos:
  \begin{itemize}
  \item $(f+g)(x) = x^2 + x$ e está definida em toda parte.
  \item $(g-f)(x) = x - x^2$ e está definida em toda parte.
  \item $(fg)(x) = x^3$ e está definida em toda parte.
  \item $\left(\frac{f}{g}\right)(x) = \frac{x^2}{x}$ e tem domínio $(-\infty, \infty)\setminus \{0\}$\footnote{Observemos que não podemos simplificar o $x$, pois a função $y=x$ é diferente da função $y=x^2/x$.}.
  \end{itemize}
\end{ex}

\subsection{Funções compostas}

\begin{flushright}
  [Vídeo] | [Áudio] | \href{https://phkonzen.github.io/notas/contato.html}{[Contatar]}
\end{flushright}

Sejam dadas as funções $f$ e $g$. Definimos a \emph{função composta} de $f$ com $g$ por
\begin{equation}
  (f\circ g)(x) := f(g(x)).
\end{equation}
Seu domínio consiste dos valores de $x$ que pertençam ao domínio da $g$ e tal que $g(x)$ pertença ao domínio da $f$.

\begin{ex}
  Sejam $f(x) = x^2$ e $g(x) = x+1$. A função composta $(f\circ g)(x) = f(g(x)) = f(x+1) = (x+1)^2$.
\end{ex}

\subsection{Translações, contrações, dilatações e reflexões de gráficos}

\begin{flushright}
  [Vídeo] | [Áudio] | \href{https://phkonzen.github.io/notas/contato.html}{[Contatar]}
\end{flushright}

Algumas operações com funções produzem resultados bastante característico no gráfico de funções. Com isso, podemos usar estas operações para construir gráficos de funções mais complicadas a partir de funções básicas.

\subsection{Translações}

\begin{flushright}
  [Vídeo] | [Áudio] | \href{https://phkonzen.github.io/notas/contato.html}{[Contatar]}
\end{flushright}

Dada uma função $f$ e uma constante $k\neq 0$, temos que a o gráfico de $y = f(x) + k$ é uma translação vertical do gráfico de $f$. Se $k>0$, observamos uma translação vertical para cima. Se $k<0$, observamos uma translação vertical para baixo.

\begin{ex}
  Seja $f(x) = x^2$. A Figura \ref{fig:ex_trans_vert}, contém os esboços dos gráficos de $f(x)$ e $f(x)+k = x^2+k$ para $k=1$.

  \begin{figure}[H]
    \centering
    \includegraphics[width=0.7\textwidth]{./cap_funcao/dados/fig_ex_transvert/fig_ex_transvert}
    \caption{Esboço do gráfico de $f(x) = x^2$ e $f(x)+k$ com $k=1$.}
    \label{fig:ex_trans_vert}
  \end{figure}

  \ifispython
  O seguinte código \verb+Python+, faz os esboços dos gráficos de $f(x)$ e $f(x)+k$:
\begin{verbatim}
k = 1
f = lambda x: x**2

p = plot(f(x),(x,-2,2),line_color="gray",show=False)
q = plot(f(x)+k,(x,-2,2),line_color="blue",show=False)
p.extend(q)
p.title = ("$k = %1.1f$" % k)
p.xlabel = '$x$'
p.ylabel = '$y$'
p[0].label = "$f(x) = x^2$"
p[1].label = "$f(x)+k$"
p.save('fig.png')

fig = p._backend.fig
ax = fig.axes[0]
ax.grid()
ax.legend(loc="upper right")
fig.savefig('fig.png', bbox_inches='tight')
\end{verbatim}
  Podemos alterar o valor de $k$ e a função $f$ para vermos o efeito das translações verticais.
  \fi
\end{ex}

Translações horizontais de gráficos podem ser produzidas pela soma de uma constante não nula ao argumento da função. Mais precisamente, dada uma função $f$ e uma constante $k\neq 0$, temos que o gráfico de $y=f(x+k)$ é uma translação horizontal do gráfico de $f$ em $k$ unidades. Se $k>0$, observamos uma translação horizontal para a esquerda. Se $k<0$, observamos uma translação horizontal para a direita.

\begin{ex}
  Seja $f(x) = x^2$. A Figura \ref{fig:ex_transhoriz}, contém os esboços dos gráficos de $f(x)$ e $f(x+k) = (x+k)^2$ para $k=1$.

  \begin{figure}[H]
    \centering
    \includegraphics[width=0.7\textwidth]{./cap_funcao/dados/fig_ex_transhoriz/fig_ex_transhoriz}
    \caption{Esboço do gráfico de $f(x) = x^2$ e $f(x+k)$ com $k=1$.}
    \label{fig:ex_transhoriz}
  \end{figure}

  \ifispython
  O seguinte código \verb+Python+, faz os esboços dos gráficos de $f(x)$ e $f(x+k)$:
\begin{verbatim}
k = 1
f = lambda x: x**2

p = plot(f(x),(x,-3,3),line_color="gray",show=False)
q = plot(f(x+k),(x,-3,3),line_color="blue",show=False)
p.extend(q)
p.title = ("$k = %1.1f$" % k)
p.xlabel = '$x$'
p.ylabel = '$y$'
p[0].label = "$f(x) = x^2$"
p[1].label = "$f(x+k)$"
p.save('fig.png')

fig = p._backend.fig
ax = fig.axes[0]
ax.grid()
ax.legend(loc="upper right")
fig.savefig('fig.png', bbox_inches='tight')
\end{verbatim}
  Podemos alterar o valor de $k$ e a função $f$ para vermos o efeito das translações horizontais.
  \fi
\end{ex}

\subsection{Dilatações e contrações}

\begin{flushright}
  [Vídeo] | [Áudio] | \href{https://phkonzen.github.io/notas/contato.html}{[Contatar]}
\end{flushright}

Sejam dados uma função $f$ e uma constante $\alpha$. Então, o gráfico de:
\begin{itemize}
\item $y = \alpha f(x)$ é uma dilatação vertical do gráfico de $f$, quando $\alpha > 1$;
\item $y = \alpha f(x)$ é uma contração vertical do gráfico de $f$, quando $0<\alpha < 1$;
\item $y = f(\alpha x)$ é uma contração horizontal do gráfico de $f$, quando $\alpha > 1$;
\item $y = f(\alpha x)$ é uma dilatação horizontal do gráfico de $f$, quando $0<\alpha < 1$.
\end{itemize}

\begin{ex}
  Seja $f(x) = x^2$. A Figura \ref{fig:ex_dilavert}, contém os esboços dos gráficos de $f(x)$ e $\alpha\cdot f(x) = \alpha \cdot x^2$ para $\alpha = 2$.

  \begin{figure}[H]
    \centering
    \includegraphics[width=0.7\textwidth]{./cap_funcao/dados/fig_ex_dilavert/fig_ex_dilavert}
    \caption{Esboço do gráfico de $f(x) = x^2$ e $\alpha\cdot f(x)$ com $\alpha=2$.}
    \label{fig:ex_dilavert}
  \end{figure}

  \ifispython
  O seguinte código \verb+Python+, faz os esboços dos gráficos de $f(x)$ e $\alpha\cdot f(x)$:
\begin{verbatim}
alpha = 2
f = lambda x: x**2

p = plot(f(x),(x,-2,2),line_color="gray",show=False)
q = plot(alpha*f(x),(x,-2,2),line_color="blue",show=False)
p.extend(q)
p.title = ("$\\alpha = %1.1f$" % alpha)
p.xlabel = '$x$'
p.ylabel = '$y$'
p[0].label = "$f(x) = x^2$"
p[1].label = "$\\alpha\\cdot f(x)$"
p.save('fig_ex_dilavert.png')

fig = p._backend.fig
ax = fig.axes[0]
ax.grid()
ax.legend(loc="upper right")
fig.savefig('fig_ex_dilavert.png', bbox_inches='tight')
\end{verbatim}
  Podemos alterar o valor de \verb+alpha+ e a função \verb+f+ para vermos o efeito das dilatações/contrações verticais.
  \fi
\end{ex}

\begin{ex}
  Seja $f(x) = x^2-2x+1$. A Figura \ref{fig:ex_dilahoriz}, contém os esboços dos gráficos de $f(x)$ e $f(\alpha\cdot x) = (\alpha \cdot x)^2-2(\alpha\cdot x) + 1$ para $\alpha = \frac{1}{2}$.

  \begin{figure}[H]
    \centering
    \includegraphics[width=0.7\textwidth]{./cap_funcao/dados/fig_ex_dilahoriz/fig_ex_dilahoriz}
    \caption{Esboço do gráfico de $f(x) = x^2-2x+1$ e $f(\alpha\cdot x)$ com $\alpha=\frac{1}{2}$.}
    \label{fig:ex_dilahoriz}
  \end{figure}

  \ifispython
  O seguinte código \verb+Python+, faz os esboços dos gráficos de $f(x)$ e $f(\alpha\cdot x)$:
\begin{verbatim}
alpha = 0.5
f = lambda x: x**2-2*x+1

p = plot(f(x),(x,-2,4),line_color="gray",show=False)
q = plot(f(alpha*x),(x,-2,4),line_color="blue",show=False)
p.extend(q)
p.title = ("$\\alpha = %1.1f$" % alpha)
p.xlabel = '$x$'
p.ylabel = '$y$'
p[0].label = "$f(x) = x^3$"
p[1].label = "$f(\\alpha\\cdot x)$"
p.save('fig_ex_dilahoriz.png')

fig = p._backend.fig
ax = fig.axes[0]
ax.grid()
ax.set_yticks(range(0,9))
ax.legend(loc="upper right")
fig.savefig('fig_ex_dilahoriz.png', bbox_inches='tight')
\end{verbatim}
  Podemos alterar o valor de \verb+alpha+ e a função \verb+f+ para vermos o efeito das dilatações/contrações horizontais.
  \fi
\end{ex}


\subsection{Reflexões}

\begin{flushright}
  [Vídeo] | [Áudio] | \href{https://phkonzen.github.io/notas/contato.html}{[Contatar]}
\end{flushright}

Seja dada uma função $f$. O gráfico da função $y = -f(x)$ é uma reflexão em torno do eixo das abscissas do gráfico da função $f$. Já, o gráfico da função $y = f(-x)$ é uma reflexão em torno do eixo das ordenadas do gráfico da função $f$.

\begin{ex}
  Seja $f(x) = x^2-2x+2$. A Figura \ref{fig:ex_reflex}, contém os esboços dos gráficos de $f(x)$ e $-f(x) = -x^2+2x-2$.

  \begin{figure}[H]
    \centering
    \includegraphics[width=0.7\textwidth]{./cap_funcao/dados/fig_ex_reflex/fig_ex_reflex}
    \caption{Esboço do gráfico de $f(x) = x^2-2x+2$ e $-f(x)$.}
    \label{fig:ex_reflex}
  \end{figure}

  \ifispython
  O seguinte código \verb+Python+, faz os esboços dos gráficos de $f(x)$ e $-f(x)$:
\begin{verbatim}
f = lambda x: x**2-2*x+2

p = plot(f(x),(x,-1,3),line_color="gray",show=False)
q = plot(-f(x),(x,-1,3),line_color="blue",show=False)
p.extend(q)
p.xlabel = '$x$'
p.ylabel = '$y$'
p[0].label = "$f(x) = x^2-2x+2$"
p[1].label = "$-f(x)$"
p.save('fig_ex_reflex.png')

fig = p._backend.fig
ax = fig.axes[0]
ax.grid()
ax.set_yticks(range(-5,6))
ax.legend(loc="upper right")
fig.savefig('fig_ex_reflex.png', bbox_inches='tight')
\end{verbatim}
  Podemos alterar a função \verb+f+ para vermos o efeito das reflexões em torno de eixo das abscissas.
  \fi
\end{ex}


\begin{ex}
  Seja $f(x) = x^2-2x+2$. A Figura \ref{fig:ex_refley}, contém os esboços dos gráficos de $f(x)$ e $f(-x) = x^2+2x+2$.

  \begin{figure}[H]
    \centering
    \includegraphics[width=0.7\textwidth]{./cap_funcao/dados/fig_ex_refley/fig_ex_refley}
    \caption{Esboço do gráfico de $f(x) = x^2-2x+2$ e $f(-x)$.}
    \label{fig:ex_reflex}
  \end{figure}

  \ifispython
  O seguinte código \verb+Python+, faz os esboços dos gráficos de $f(x)$ e $f(-x)$:
\begin{verbatim}
f = lambda x: x**2-2*x+2

p = plot(f(x),(x,-1,3),line_color="gray",show=False)
q = plot(f(-x),(x,-3,1),line_color="blue",show=False)
p.extend(q)
q = plot(-1,(x,-3,3),line_color="none",show=False)
p.extend(q)
p.xlabel = '$x$'
p.ylabel = '$y$'
p[0].label = "$f(x) = x^2-2x+2$"
p[1].label = "$f(-x)$"
p[2].label = ""
p.save('fig_ex_refley.png')

fig = p._backend.fig
ax = fig.axes[0]
ax.grid()
ax.set_yticks(range(-1,6))
ax.legend(loc="upper right")
fig.savefig('fig_ex_refley.png', bbox_inches='tight')
\end{verbatim}
  Podemos alterar a função \verb+f+ para vermos o efeito das reflexões em torno de eixo das ordenadas.
  \fi
\end{ex}

\subsection*{Exercícios resolvidos}

\begin{flushright}
  [Vídeo] | [Áudio] | \href{https://phkonzen.github.io/notas/contato.html}{[Contatar]}
\end{flushright}

\begin{exeresol}
  Sejam
  \begin{equation}
    f(x) = \frac{x^2 - \sqrt{x-1}}{x}\quad\text{e}\quad g(x) = x^2 + 1.
  \end{equation}
  Determine a função composta $(f\circ g)$ e seu domínio.
\end{exeresol}
\begin{resol}
  Começamos determinando a função composta
  \begin{align}
    (f\circ g)(x) &:= f(g(x))\\
                  &= f(x^2 + 1)\\
                  &= \frac{(x^2 + 1)^2 - \sqrt{x^2+1-1}}{x^2 + 1}\\
                  &= \frac{x^4 + 2x^2 + 1 - \sqrt{x^2}}{x^2 + 1}\\
                  &= \frac{x^4 + 2x^2 + 1 - |x|}{x^2 + 1}.
  \end{align}
  Agora, observamos que $g$ está definida em toda parte e tem imagem $[1, \infty)$. Como o domínio da $f$ é $[1, \infty)$, temos que $(f\circ g)$ está definida em toda parte.
\end{resol}

\begin{exeresol}
  Faça o esboço do gráfico de $f(x) = 2(x-1)^3+1$.
\end{exeresol}
\begin{resol}
  Começamos trançando o gráfico de $f_1(x) = x^3$. Então, obtemos o gráfico de $f_2(x) = (x-1)^3$ por translação de uma unidade à direita. O gráfico de $f_3(x) = 2(x-1)^3$ é obtido por dilatação vertical de 2 vezes. Por fim, o gráfico de $f_4(x) = 2(x-1)^3+1$ é obtido por translação de uma unidade para cima. Veja a Figura \ref{fig:exeresol_opfun_graf}.

  \begin{figure}[H]
    \centering
    \includegraphics[width=0.49\textwidth]{./cap_funcao/dados/fig_exeresol_opfun_graf/fig_exeresol_opfun_graf_1}~
    \includegraphics[width=0.49\textwidth]{./cap_funcao/dados/fig_exeresol_opfun_graf/fig_exeresol_opfun_graf_2}\\
    \includegraphics[width=0.49\textwidth]{./cap_funcao/dados/fig_exeresol_opfun_graf/fig_exeresol_opfun_graf_3}
    \caption{Construção do esboço do gráfico de $f(x) = 2(x-1)^3+1$.}
    \label{fig:exeresol_opfun_graf}
  \end{figure}
\end{resol}

\subsection*{Exercícios}

\begin{flushright}
  [Vídeo] | [Áudio] | \href{https://phkonzen.github.io/notas/contato.html}{[Contatar]}
\end{flushright}

\begin{exer}
  Sejam $f(x) = \sqrt{x}+1$ e $g(x) = x^2 -1$. Determine a função $(f\circ g)$ e seu domínio.
\end{exer}
\begin{resp}
  $(f\circ g)(x) = \sqrt{x^2-1}+1$; domínio: $(-\infty, 1]\cup [1, \infty)$.
\end{resp}

\begin{exer}
  Faça um esboço do gráfico de $g(x) = 2x^3 - 1$.
\end{exer}
\begin{resp}
  Dica: verifique sua resposta com um pacote de matemática simbólica, por exemplo, com o \sympy.
\end{resp}

\section{Propriedades de funções}\label{cap_funcao_sec_funprop}

\begin{flushright}
  [Vídeo] | [Áudio] | \href{https://phkonzen.github.io/notas/contato.html}{[Contatar]}
\end{flushright}

\subsection{Funções crescentes ou decrescentes}

\begin{flushright}
  [Vídeo] | [Áudio] | \href{https://phkonzen.github.io/notas/contato.html}{[Contatar]}
\end{flushright}

Uma da função $f$ é dita ser crescente quando $f(x_1)<f(x_2)$ para todos $x_1<x_2$ no seu domínio. É dita não decrescente quando $f(x_1)\leq f(x_2)$ para todos os $x_1<x_2$ no seu domínio. Analogamente, é dita decrescente quando $f(x_1)>f(x_2)$ para todos $x_1<x_2$. E, por fim, é dita não crescente quando $f(x_1)\geq f(x_2)$ para todos $x_1<x_2$, sempre no seu domínio.

\begin{ex}
  Vejamos os seguintes casos:
  \begin{itemize}
  \item A \emph{função identidade} $f(x)=x$ é crescente.
  \item A seguinte função definida por partes
    \begin{equation}
      f(x) = \left\{
        \begin{array}{ll}
          x+1 &,x\leq 0,\\
          2 &,0<x\leq 1,\\
          (x-1)^2+2 &, x>1
        \end{array}
\right.
\end{equation}
é não decrescente.
  \end{itemize}
\end{ex}

Também, definem-se os conceitos análogos de uma função ser crescente ou decrescente em um dado intervalo.

\begin{ex}
  A função $f(x) = x^2$ é uma função decrescente no intervalo $(-\infty, 0]$ e crescente no intervalo $[0, \infty)$.
\end{ex}

\subsection{Funções pares ou ímpares}

\begin{flushright}
  [Vídeo] | [Áudio] | \href{https://phkonzen.github.io/notas/contato.html}{[Contatar]}
\end{flushright}

Uma dada \emph{função} $f$ é dita \emph{par} quando $f(x)=f(-x)$ para todo $x$ no seu domínio. Ainda, é dita \emph{ímpar} quando $f(x)=-f(-x)$ para todo $x$ no seu domínio.

\begin{ex}
  Vejamos os seguintes casos:
  \begin{itemize}
  \item $f(x) = x^2$ é uma função par.
  \item $f(x) = x^3$ é uma função par.
  \item $f(x) = \sen x$ é uma função ímpar.
  \item $f(x) = \cos x$ é uma função par.
  \item $f(x) = x+1$ não é par nem ímpar.
  \end{itemize}
\end{ex}

\subsection{Funções injetoras}

\begin{flushright}
  [Vídeo] | [Áudio] | \href{https://phkonzen.github.io/notas/contato.html}{[Contatar]}
\end{flushright}

Uma dada {\bf função} $f$ é dita {\bf injetora} quando $f(x_1)\neq f(x_2)$ para todos $x_1\neq x_2$ no seu domínio.

\begin{ex}
  Vejamos os seguintes casos:
  \begin{itemize}
  \item $f(x) = x^2$ não é uma função injetora.
  \item $f(x) = x^3$ é uma função injetora.
  \item $f(x) = e^x$ é uma função injetora.
  \end{itemize}
\end{ex}

Função injetoras são funções invertíveis. Mais precisamente, dada uma função injetora $y = f(x)$, existe uma única função $g$ tal que
\begin{equation}
  g(f(x)) = x,
\end{equation}
para todo $x$ no domínio da $f$. Tal função $g$ é chamada de \emph{função inversa} de $f$ é comumente denotada por $f^{-1}$.\footnote{Observe que, em geral, $f^{-1} \neq \frac{1}{f}$.}

\begin{ex}
  Vamos calcular a função a função inversa de $f(x) = x^3 + 1$. Para tando, escrevemos
  \begin{equation}
    y = x^3 + 1.
  \end{equation}
  Então, isolando $x$, temos
  \begin{equation}
    x = \sqrt[3]{y - 1}.
  \end{equation}
  Desta forma, concluímos que $f^{-1}(x) = \sqrt[3]{x-1}$. Verifique que $f^{-1}(f(x)) = x$ para todo $x$ no domínio de $f$!
\end{ex}

\begin{obs}
 Os gráficos de uma dada função injetora $f$ e de sua inversa $f^{-1}$ são simétricos em relação a \emph{reta identidade} $y=x$.
\end{obs}

\subsection*{Exercícios resolvidos}

\begin{flushright}
  [Vídeo] | [Áudio] | \href{https://phkonzen.github.io/notas/contato.html}{[Contatar]}
\end{flushright}

\begin{exeresol}
  Defina os intervalos em que a função $f(x) = -|x+1|$ é crescente ou decrescente.
\end{exeresol}
\begin{resol}
  A função $f$ é uma translação à esquerda, seguida de uma reflexão em torno do eixo das abscissas da função $f(x) = |x|$. Veja a Figura \ref{fig:exeresol_funprop_mono}.

  \begin{figure}[H]
    \centering
    \includegraphics[width=0.6\textwidth]{./cap_funcao/dados/fig_exeresol_funprop_mono/fig_exeresol_funprop_mono.png}
    \caption{Esboço do gráfico de $f(x) = -|x+1|$.}
    \label{fig:exeresol_funprop_mono}
  \end{figure}

  Do esboço do gráfico de $f$, podemos inferir que $f$ é crescente no intervalo $(-\infty, -1]$ e decrescente no intervalo $[-1, \infty)$.
\end{resol}

\begin{exeresol}
  Analise a paridade da função $tg(x)$.
\end{exeresol}
\begin{resol}
  Da paridade das funções seno e cosseno, temos
  \begin{equation}
    \tg(-x) = \frac{\sen(-x)}{\cos(-x)} = \frac{-\sen x}{\cos x} = -\frac{\sen x}{\cos x} = -\tg x.
  \end{equation}
  Logo, a tangente é uma função ímpar.
\end{resol}

\begin{exeresol}
  Calcule a função inversa de $f(x) = \sqrt{x+1}$.
\end{exeresol}
\begin{resol}
  Para obtermos a função inversa de uma função $f$, resolvemos $y = f(x)$ para $x$. Ou seja,
  \begin{align}
    y = f(x) &\Rightarrow y = \sqrt{x+1}\\
             &\Rightarrow y^2 = x+1\\
             &\Rightarrow x = y^2 - 1.
  \end{align}
  Logo, temos $f^{-1}(x) = x^2 - 1$ restrita ao conjunto imagem da $f$, i.e. o domínio de $f^{-1}$ é $[0, \infty)$.
\end{resol}

\subsection*{Exercícios}

\begin{flushright}
  [Vídeo] | [Áudio] | \href{https://phkonzen.github.io/notas/contato.html}{[Contatar]}
\end{flushright}

\begin{exer}
  Determine os intervalos de crescimento ou decrescimento da função
  \begin{equation}
    f(x) = \left\{
      \begin{array}{ll}
        (x+1)^2 &, -\infty < x \leq 1,\\
        -x+5 &, 1 \leq x < \infty
      \end{array}
\right.
  \end{equation}
\end{exer}
\begin{resp}
  decrescente: $(-\infty, -1]\cup [1, \infty)$; crescente: $[-1, 1]$.
\end{resp}

\begin{exer}
  Analise a paridade da função $\cosec x$.
\end{exer}
\begin{resp}
  função ímpar
\end{resp}

\begin{exer}
  Seja $f(x) = 2\sqrt{x-1}-1$. Calcule $f^{-1}$ e determine seu domínio.
\end{exer}
\begin{resp}
  $f^{-1}(x) = \frac{1}{2}x^2 + x - \frac{1}{2}$; domínio $[-1, \infty)$
\end{resp}

\section{Funções exponenciais}\label{cap_funcao_sec_funexp}

\begin{flushright}
  [Vídeo] | [Áudio] | \href{https://phkonzen.github.io/notas/contato.html}{[Contatar]}
\end{flushright}

Uma \emph{função exponencial} tem a forma
\begin{equation}
  f(x) = a^x,
\end{equation}
onde $a\neq 1$ é uma constante positiva e é chamada de \emph{base} da função exponencial.

Funções exponenciais estão definidas em toda parte e têm imagem $(0, \infty)$. O gráfico de uma função exponencial sempre contém os pontos $(-1,1/a)$, $(0,1)$ e $(1,a)$. Veja a Figura \ref{fig:exponencial_graficos}.

\begin{figure}[H]
  \centering
  \includegraphics[width=0.5\textwidth]{./cap_funcao/dados/fig_exponencial_graficos/fig_exponencial_2}~
  \includegraphics[width=0.5\textwidth]{./cap_funcao/dados/fig_exponencial_graficos/fig_exponencial_12}
  \caption{Esboços dos gráficos de funções exponenciais: (esquerda) $f(x) = a^x$, $a>1$; (direita) $g(x) = a^x$, $0<a<1$.}
  \label{fig:exponencial_graficos}
\end{figure}

\begin{obs}
  Quando a base é o número de Euler $e \approx 2,718281828459045$, chamamos $f(x) = e^x$ de função exponencial natural.

  \ifispython
  No \sympy\footnote{Veja a Observação \ref{obs:cap_funcao_python}}, o número de Euler é obtido com a constante \verb+E+:
\begin{verbatim}
>>> float(E)
2.718281828459045
\end{verbatim}
  \fi
\end{obs}

\subsection*{Exercícios resolvidos}

\begin{flushright}
  [Vídeo] | [Áudio] | \href{https://phkonzen.github.io/notas/contato.html}{[Contatar]}
\end{flushright}

\begin{exeresol}
  Faça um esboço do gráfico de $f(x) = e^{-2x+1}-1$.
\end{exeresol}
\begin{resol}
  Primeiramente, observamos que $f(x) = e^{-2x+1}-1 = e^{-2\left(x-\frac{1}{2}\right)}-1$. Então, partindo do gráfico de $e^{-x}$, fazemos uma translação de $\frac{1}{2}$ unidades à direita, seguida de uma contração horizontal de $\frac{1}{2}$ vezes e, por fim, uma translação para baixo de uma unidade. Veja a Figura \ref{fig:exeresol_funexp_graf}.

  \begin{figure}[H]
    \centering
    \includegraphics[width=0.49\textwidth]{./cap_funcao/dados/fig_exeresol_funexp_graf/fig_exeresol_funexp_graf_1}~
    \includegraphics[width=0.49\textwidth]{./cap_funcao/dados/fig_exeresol_funexp_graf/fig_exeresol_funexp_graf_2}\\
    \includegraphics[width=0.49\textwidth]{./cap_funcao/dados/fig_exeresol_funexp_graf/fig_exeresol_funexp_graf_3}
    \caption{Esboço do gráfico de $f(x) = e^{-2x+1}-1$.}
    \label{fig:exeresol_funexp_graf}
  \end{figure}
\end{resol}

\subsection*{Exercícios}

\begin{flushright}
  [Vídeo] | [Áudio] | \href{https://phkonzen.github.io/notas/contato.html}{[Contatar]}
\end{flushright}

\begin{exer}
  Faça um esboço do gráfico de $f(x) = 2e^{x-1}+2$.
\end{exer}
\begin{resp}
  Dica: use um pacote de matemática simbólica para verificar sua resposta.
\end{resp}

\section{Funções logarítmicas}\label{cap_funcao_sec_funlog}

\begin{flushright}
  [Vídeo] | [Áudio] | \href{https://phkonzen.github.io/notas/contato.html}{[Contatar]}
\end{flushright}

A \emph{função logarítmica} $y = \log_a x$, $a>0$ e $a\neq 1$, é a função inversa da função exponencial $y = a^x$. Veja a Figura \ref{fig:log_graficos}. O domínio da função logarítmica é $(0,\infty)$ e a imagem $(-\infty, \infty)$.

\begin{figure}[H]
  \centering
  \includegraphics[width=0.5\textwidth]{./cap_funcao/dados/fig_log_graficos/fig_log_2}~
  \includegraphics[width=0.5\textwidth]{./cap_funcao/dados/fig_log_graficos/fig_log_12}
  \caption{Esboços dos gráficos de funções logarítmicas: (esquerda) $y = \log_a x$, $a>1$; (direita) $y = \log_a x$, $0<a<1$.}
  \label{fig:log_graficos}
\end{figure}

\begin{obs}
  Quando a base é o número de Euler $e \approx 2,718281828459045$, chamamos $y = \log_e x$ de função exponencial natural e denotamo-la por $y = \ln x$.

  \ifispython
  No \sympy, podemos computar $\log_a x$ com a função \verb+log(x,a)+. O $\ln x$ é computado com $\verb+log(x)+$.
  \fi
\end{obs}

\begin{obs}
  Vejamos algumas propriedades dos logaritmos:
  \begin{itemize}
  \item $\displaystyle \log_a x = y \Leftrightarrow a^y = x$;
  \item $\displaystyle \log_a 1 = 0$;
  \item $\displaystyle \log_a a = 1$;
  \item $\displaystyle \log_a a^x = x$;
  \item $\displaystyle a^{\log_a^x} = x$;
  \item $\displaystyle \log_a xy = \log_a x + \log_a y$;
  \item $\displaystyle \log_a \frac{x}{y} = \log_a x - \log_a y$;
  \item $\displaystyle \log_a x^r = r\cdot\log_a x$.
  \item $\displaystyle \log_a x = \frac{\log_b x}{\log_b a}$
  \end{itemize}
\end{obs}

\subsection*{Exercícios resolvidos}

\begin{flushright}
  [Vídeo] | [Áudio] | \href{https://phkonzen.github.io/notas/contato.html}{[Contatar]}
\end{flushright}

\begin{exeresol}
  Faça o esboço do gráfico de $f(x) = \ln(x+2)+1$ e determine seu domínio.
\end{exeresol}
\begin{resol}
  Para fazermos o esboço do gráfico de $f(x) = \ln(x+2)+1$, podemos começar com o gráfico de $f_1(x) = \ln x$. Então, podemos transladá-lo 2 unidades à esquerda, de forma a obtermos $f_2(x) = \ln(x+2) = f_1(x+2)$. Por fim, transladamos o gráfico de $f_2(x)$ uma unidade para cima, obtendo o esboço do gráfico de $f(x) = \ln(x+2)+1=f_2(x)+1$. Veja a Figura \ref{fig:exeresol_lograf}.

  \begin{figure}[H]
    \centering
    \includegraphics[width=0.5\textwidth]{./cap_funcao/dados/fig_exeresol_lograf/fig_exeresol_lograf}
    \caption{Esboço do gráfico de $f(x) = \ln(x+2)+1$.}
    \label{fig:exeresol_lograf}
  \end{figure}

Ainda, o domínio de $\ln x$ é $(0, \infty)$. Como, $f(x) = \ln(x+2)+1$ é uma translação de duas unidades à esquerda e uma para cima de $\ln x$, temos que o domínio de $f(x)$ é $(-2, \infty)$.
\end{resol}

\begin{exeresol}
  Resolva a seguinte equação para $x$
  \begin{equation}
    \ln(x+2) + 1 = 1. 
  \end{equation}
\end{exeresol}
\begin{resol}
  Podemos calcular a solução pelos seguintes passos:
  \begin{align}
    \ln(x+2)+1=1 &\Rightarrow \ln(x+2)=0\\
                 &\Rightarrow x+2=e^0\\
                 &\Rightarrow x=1-2=-1.\\
  \end{align}

  \ifispython
  Com o \sympy, podemos computar a solução com o seguinte comando:
\begin{verbatim}
solve(Eq(log(x+2)+1,1),x)
\end{verbatim}
  \fi
\end{resol}

\subsection*{Exercícios}

\begin{flushright}
  [Vídeo] | [Áudio] | \href{https://phkonzen.github.io/notas/contato.html}{[Contatar]}
\end{flushright}

\begin{exer}
  Faça o esboço do gráfico de $f(x) = \log(x-2)-1$ e determine seu domínio.
\end{exer}
\begin{resp}
  Dica: use um pacote computacional de matemática simbólica para verificar o esboço de seu gráfico. Domínio: $(2, \infty)$.
\end{resp}

\begin{exer}
  Resolva a seguinte equação para $x$
  \begin{equation}
    \ln(x+1)^2=0.
  \end{equation}
\end{exer}
\begin{resp}
  $0$
\end{resp}
