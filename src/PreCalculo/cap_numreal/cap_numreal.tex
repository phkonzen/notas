%Este trabalho está licenciado sob a Licença Atribuição-CompartilhaIgual 4.0 Internacional Creative Commons. Para visualizar uma cópia desta licença, visite http://creativecommons.org/licenses/by-sa/4.0/deed.pt_BR ou mande uma carta para Creative Commons, PO Box 1866, Mountain View, CA 94042, USA.

\chapter{Números reais}\label{cap_numreal}
\thispagestyle{fancy}

\begin{flushright}
  [Vídeo] | [Áudio] | \href{https://phkonzen.github.io/notas/contato.html}{[Contatar]}
\end{flushright}

\section{Conjuntos numéricos}\label{cap_numreal_sec_funconj}

\begin{flushright}
  [Vídeo] | [Áudio] | \href{https://phkonzen.github.io/notas/contato.html}{[Contatar]}
\end{flushright}

\subsection{Definição de conjunto}

\begin{flushright}
  [Vídeo] | [Áudio] | \href{https://phkonzen.github.io/notas/contato.html}{[Contatar]}
\end{flushright}

Um \emph{conjunto} $A$ é uma coleção de elementos ou objetos. Quando $x$ é um elemento do conjunto $A$, denotamos
\begin{equation}
  x\in A,
\end{equation}
lê-se x pertence ao conjunto A. Já, a notação
\begin{equation}
  x\not\in A
\end{equation}
é usada para denotar que $x$ não pertence ao $A$.

Usualmente, um conjunto é descrito usando a notação
\begin{equation}
  A = \{x:~\text{condição}\},
\end{equation}
lê-se $A$ é o conjunto dos elementos $x$ tais que $x$ satisfaz a condição.

\begin{ex}
  O conjunto $A$ formado por números positivos pode ser denotado por
  \begin{equation}
    A = \{x: x>0\}.
  \end{equation}
  Ainda, observamos que $2\in A$, $\sqrt{2}\in A$, mas $-1\not\in A$. Você saberia escolher mais elementos que pertençam ou que não pertençam a $A$?

  \ifispython
  No \python, podemos definir este conjunto com
  \begin{lstlisting}
    from sympy import *
    A = ConditionSet(x, x>0)
  \end{lstlisting}
  o que nos fornece
  \begin{lstlisting}
    In : 2 in A
    Out: True
    In : sqrt(2) in A
    Out: True
    In : -1 in A
    Out: False
  \end{lstlisting}
  \fi
\end{ex}

\subsubsection{Conjunto finito}

Um \emph{conjunto finito} é todo aquele que contém um número finito de elementos. Tais conjuntos podem ser descritos de forma simplificada como segue
\begin{equation}
  A = \{a_1, a_2, \ldots, a_n\},
\end{equation}
neste caso, temos um conjunto com $n$ elementos. Analogamente, um conjunto que contenha infinitos elementos é chamado de \emph{conjunto infinito}.

\begin{obs}
  \begin{equation}
    A = \{-1,3,2\}
  \end{equation}
  é o conjunto que contém apenas pelos números $-1$, $3$ e $2$.

  \ifispython
  No \python, podemos definir tal conjunto com o seguinte código
  \begin{lstlisting}
    from sympy import *
    A = FiniteSet(-1, 3, 2)
  \end{lstlisting}
  Com este, obtemos
  \begin{lstlisting}
    In : -1 in A
    Out: True
    In : sqrt(2) in A
    Out: False
  \end{lstlisting}
  \fi
\end{obs}

\subsubsection{Conjunto vazio}

O conjunto que não contém elemento algum é chamado de \emph{conjunto vazio} e é denotado por $\emptyset$ ou por $\{\}$.

\begin{ex}
  O conjunto $A$ de todos os números negativos e positivos é vazio, i.e.
  \begin{equation}
    A = \emptyset
  \end{equation}
\end{ex}

\ifispython
\begin{obs}
  No \python, podemos definir o conjunto vazio com
  \begin{lstlisting}
    from sympy import *
    A = EmptySet
  \end{lstlisting}
\end{obs}
\fi

\subsubsection{Igualdade de conjuntos}

Dois \emph{conjuntos} $A$ e $B$ são \emph{iguais}, quando todos os elementos $A$ pertencem a $B$ e vice-versa. Em notação matemática, escrevemos $A=B$ quando
\begin{equation}
  x\in A \Leftrightarrow x\in B,
\end{equation}
lê-se $x\in A$ se, e somente se, $x\in B$.

\begin{ex}
  \begin{enumerate}[a)]
  \item São iguais os conjuntos
    \begin{gather}
      A = \{-1, 3, 2\}\\
      B = \{3, 2, -1\},
    \end{gather}
    i.e. $A = B$.

    \ifispython
    No \python, temos
    \begin{lstlisting}
      from sympy import *
      A = FiniteSet(-1, 3, 2)
      B = FiniteSet(3, 2, -1)
    \end{lstlisting}
    Com este, obtemos
    \begin{lstlisting}
      In : A == B
      Out: True
    \end{lstlisting}
    \fi

  \item São diferentes os conjuntos
    \begin{gather}
      C = \{-3, -2, -1, 0\}\\
      D = \{-3, -1, 0, 2\},
    \end{gather}
    i.e. $C\neq D$.

    \ifispython
    No \python, temos
    \begin{lstlisting}
      from sympy import *
      A = FiniteSet(-3, -2, -1, 0)
      B = FiniteSet(-33, -1, 0, 2)
    \end{lstlisting}
    Com este, obtemos
    \begin{lstlisting}
      In : C != D
      Out: True
    \end{lstlisting}
    \fi

  \end{enumerate}
\end{ex}

\subsubsection{Subconjuntos}

Dizemos que $A$ é subconjunto de $B$, quando todos os elementos de $A$ pertencem a $B$. Neste caso, denotamos
\begin{equation}
  A \subset B.
\end{equation}
Mais precisamente, $A\subset B$ quando
\begin{equation}
  x\in A \Rightarrow x\in B,
\end{equation}
lê-se $x\in A$ implica $x\in B$. O mesmo pode ser denotado por $B\supset A$.

\subsection{Operações entre conjuntos}

\begin{flushright}
  [Vídeo] | [Áudio] | \href{https://phkonzen.github.io/notas/contato.html}{[Contatar]}
\end{flushright}

\subsubsection{União de conjuntos}

Sejam $A$ e $B$ dois conjuntos dados. A união do conjunto $A$ com o conjunto $B$ é o conjunto $A\cup B$ que contém todos os elementos de $A$ e todos os elementos de $B$. Mais precisamente, temos
\begin{equation}
  A\cup B = \{x:~x\in A \lor x\in B\},
\end{equation}
lê-se o conjunto dos elementos $x$ tais que $x\in A$ ou $x\in B$.

\begin{ex}
  Se
  \begin{equation}
    A = \{-1, 3, 2\}\\
    B = \{-2, 0\},
  \end{equation}
  então
  \begin{equation}
    A\cup B = \{-2, -1, 0, 2, 3\}.
  \end{equation}

  \ifispython
  No \python, temos
  \begin{lstlisting}
    from sympy import *
    A = FiniteSet(-1, 3, 2)
    B = FiniteSet(-2, 0)
  \end{lstlisting}
  \begin{lstlisting}
    In : Union(A, B)
    Out: FiniteSet(-2, -1, 0, 2, 3)
  \end{lstlisting}
  \fi
\end{ex}

\subsubsection{Interseção de conjuntos}

Sejam $A$ e $B$ dois conjuntos dados. A interseção do conjunto $A$ com o conjunto $B$ é o conjunto $A\cap B$ que contém os elementos que pertencem simultaneamente a ambos os conjuntos $A$ e $B$. Mais precisamente, temos
\begin{equation}
  A\cap B = \{x:~x\in A \land x\in B\},
\end{equation}
lê-se o conjunto dos elementos $x$ tais que $x\in A$ e $x\in B$.

\begin{ex}
  Se
  \begin{equation}
    A = \{-1, 3, 2\}\\
    B = \{3, 0\},
  \end{equation}
  então
  \begin{equation}
    A\cap B = \{3\}.
  \end{equation}

  \ifispython
  No \python, temos
  \begin{lstlisting}
    from sympy import *
    A = FiniteSet(-1, 3, 2)
    B = FiniteSet(3, 0)
  \end{lstlisting}
  \begin{lstlisting}
    In : Intersection(A, B)
    Out: FiniteSet(3)
  \end{lstlisting}
  \fi
\end{ex}

\subsubsection{Diferença entre conjuntos}

Sejam $A$ e $B$ dois conjuntos dados. A diferença (ou complemento relativo) do conjunto $A$ com o conjunto $B$ é o conjunto $A\setminus B$ que contém os elementos que pertencem ao $A$ e não pertencem ao conjunto $B$. Mais precisamente, temos
\begin{equation}
  A\setminus B = \{x:~x\in A \land x\not\in B\},
\end{equation}
lê-se o conjunto dos elementos $x$ tais que $x\in A$ e $x\not\in B$.

\begin{ex}
  Se
  \begin{equation}
    C = \{-3, -2, -1, 0\}\\
    D = \{-3, -1 0, 2\},
  \end{equation}
  então
  \begin{equation}
    C\setminus D = \{-2\}.
  \end{equation}

  \ifispython
  No \python, temos
  \begin{lstlisting}
    from sympy import *
    C = FiniteSet(-3,-2,-1,0)
    D = FiniteSet(-3,-1,0,2)
  \end{lstlisting}
  \begin{lstlisting}
    In : C - D
    Out: FiniteSet(-2)
  \end{lstlisting}
  \fi
\end{ex}

\subsubsection{Produto cartesiano}

Sejam $A$ e $B$ dois conjuntos. O produto cartesiano de $A$ com $B$ é o conjunto $A\times B$, cujos elementos são os \emph{pares ordenados} $(x, y)$ com $x\in A$ e $y\in B$. Mais precisamente, temos
\begin{equation}
  A\times B = \{(x, y):~x\in A \land y\in B\},
\end{equation}
lê-se o conjunto dos pares ordenados $(x, y)$ tais que $x\in A$ e $x\in B$.

\begin{obs}
  Um par ordenado $(x, y)$ é um conjunto formado por $x$ e $y$, no qual a posição dos elementos importa. Por exemplo, temos
  \begin{gather}
    (3, -1) \neq (-1, 3),
  \end{gather}
  enquanto que
  \begin{gather}
    \{3, -1\} = \{-1, 3\}.
  \end{gather}

  \ifispython
  No \python, escrevemos
  \begin{lstlisting}
    from sympy import *
    A = (3, -1)
    B = (-1, 3)
  \end{lstlisting}
  então
  \begin{lstlisting}
    In : A = (3, -1)
    In : B = (-1, 3)
    In : A == B
    Out: False
  \end{lstlisting}
  \fi
\end{obs}

\begin{ex}
  Se
  \begin{equation}
    A = \{-3, -2, -1\}\\
    B = \{0, 1\},
  \end{equation}
  então
  \begin{gather}
    A\times B = \{(-3,0), (-2, 0), (-1, 0), \\
    (-3, 1), (-2, 1), (-1, 1)\}.
  \end{gather}

  \ifispython
  No \python, temos
  \begin{lstlisting}
    from sympy import *
    A = FiniteSet(-3,-2,-1)
    B = FiniteSet(0, 1)
    C = ProductSet(A, B)
  \end{lstlisting}
  então
  \begin{lstlisting}
    In : (-3, 1) in C 
    Out: True
  \end{lstlisting}
  \fi
\end{ex}

\emconstrucao ...

\subsection*{Exercícios}

\emconstrucao

\section{Conjunto dos números reais}\label{cap_numreal_conjreal}

\emconstrução

\subsection{Conjunto de números}

\emconstrução

\subsection{Intervalos}

\emconstrução

\subsection*{Exercícios}

\emconstrucao

\section{}
