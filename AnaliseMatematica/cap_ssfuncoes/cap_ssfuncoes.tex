%Este trabalho está licenciado sob a Licença Atribuição-CompartilhaIgual 4.0 Internacional Creative Commons. Para visualizar uma cópia desta licença, visite http://creativecommons.org/licenses/by-sa/4.0/ ou mande uma carta para Creative Commons, PO Box 1866, Mountain View, CA 94042, USA.

\chapter{Sequências e séries de funções}\label{cap:sequências_e_séries_de_funções}\index{sequência de funções}\index{séries de funções}
\thispagestyle{fancy}

\section{Sequência de funções}\index{sequência de funções}

\begin{defn}
  Uma sequência de funções $(f_n)_{n\in\mathbb{N}}$ é um conjunto de funções $f_n:D\to\mathbb{R}$, $y=f_n(x)$, indexadas por $n\in\mathbb{R}$. Comumente, utiliza-se a notação $(f_n(x))_{n\in\mathbb{N}}$ (ou, simplesmente, $f_n(x)$) para explicitar que trata-se de uma sequência de funções.
\end{defn}

\begin{obs}
  Salvo explicitado ao contrário, ao longo deste capítulo assumiremos que as funções que compõe uma dada sequência têm todas o mesmo domínio.
\end{obs}

\begin{ex}
  Vejamos os seguintes exemplos:
  \begin{enumerate}[a)]
  \item $f_n(x) = x+1/n$, $n\in\mathbb{N}$, é uma sequência de funções afins.
  \item $g_n(x) = x^n$ é uma sequência de polinômios.
  \item $h_n(x) = 1 + x + x^2 + \cdots + x^n$ é, também, uma sequência\footnote{Um sequência deste tipo também é chamada de série de funções, como definiremos logo adiante no texto.} de polinômios.
  \end{enumerate}
\end{ex}

\subsection{Limite pontual}\index{limite!pontual}

\begin{defn}\normalfont{Limite pontual}
  Diz-se que uma sequência de funções $(f_n(x))_{n\in\mathbb{R}}$, $f_n:D\to\mathbb{R}$, \emph{converge pontualmente}\index{convergência!pontual} (ou simplesmente\index{convergência!simples}) para uma função $f(x)$, $f:D\to\mathbb{R}$, se, dado qualquer $\varepsilon>0$, para cada $x\in D$, existe $N$ tal que
  \begin{equation}
    n>N \Rightarrow |f_n(x)-f(x)|<\varepsilon.
  \end{equation}
\end{defn}

\begin{ex}\label{ex:convergência_pontual}
  Vejamos os seguintes casos.
  \begin{enumerate}[a)]
  \item A sequência de funções $f_n(x) = x + 1/n$ converge pontualmente para a função identidade $f(x)=x$. De fato, sejam $\varepsilon>0$ e $x$ no domínio da $f$. Escolhendo $N > 1/\varepsilon$, temos
    \begin{equation}
      n>N \Rightarrow |f_n(x) - f(x)| = \left|x+\frac{1}{n} - x\right| = \left|\frac{1}{n}\right| < \frac{1}{N} < \varepsilon.
    \end{equation}
  \item A sequência de funções $g_n(x) = x/n$ converge pontualmente para a função nula $f(x) \equiv 0$. De fato, sejam $\varepsilon>0$ e $x$ no domínio da $f$. Escolhendo $N > |x|/\varepsilon$, temos
    \begin{equation}
      n>N \Rightarrow \left|\frac{x}{n} - 0\right| < \frac{|x|}{n} < \frac{|x|}{N} < \varepsilon.
    \end{equation}
  \end{enumerate}
\end{ex}

\subsection{Convergência uniforme}

\begin{defn}\normalfont{Convergência uniforme}
  Diz-se que uma sequência de funções $(f_n(x))_{n\in\mathbb{R}}$, $f_n:D\to\mathbb{R}$, \emph{converge uniformemente}\index{convergência!uniforme} para uma função $f(x)$, $f:D\to\mathbb{R}$, se, dado qualquer $\varepsilon>0$, existe $N$ tal que
  \begin{equation}
    x\in D, n>N \Rightarrow |f_n(x)-f(x)|<\varepsilon.
  \end{equation}
\end{defn}

\begin{ex}\label{ex:convergência_uniforme}
  Vejamos os seguintes casos:
  \begin{enumerate}[a)] 
  \item No Exemplo~\ref{ex:convergência_pontual}~a), vimos que a
    sequência de funções $f_n(x) = x + 1/n$ é pontualmente convergente
    para a função $f(x)$. Agora, observando a demonstração vemos que a
    convergência é também uniforme. Verifique!
  \item A sequência de funções $f_n(x) = x/n$ é pontualmente
    convergente para $f(x)\equiv 0$, mas não é uniformemente
    convergente. De fato, dado $\varepsilon>0$ e $x$ no domínio,
    escolhemos $N>|x|/\varepsilon$ e, então, temos
    \begin{equation}
      n>N \Rightarrow |f_n(x) - f(x)| = \left|\frac{x}{n}-0\right|\leq \frac{|x|}{N} < \varepsilon.
    \end{equation}
    Isto mostra a convergência pontual. Entretanto, por exemplo,
    tomemos $\varepsilon=1$. Então, dado qualquer $n\in \mathbb{N}$
    escolhemos $x=2/n$. Logo, temos
    \begin{equation}
      |f_n(x) - f(x)| = \left|\frac{x}{n}\right| = \frac{x}{n} = 2 > \varepsilon.
    \end{equation}
    Isto mostra que a convergência de $f_n\to 0$ não é uniforme.
  \end{enumerate}
\end{ex}

\begin{teo}\normalfont{(Critério de convergência de Cauchy)}\index{teorema de!Cauchy}\index{critério de!Cauchy}
  Uma sequência de funções $f_n:D\to\mathbb{R}$ é uniformemente convergente para uma função $f:D\to\mathbb{R}$ se, e somente se, dado qualquer $\varepsilon>0$, exista $N\in\mathbb{N}$ tal que
  \begin{equation}
    x\in D, ~n,m>N \Rightarrow |f_n(x) - f_m(x)| < \varepsilon.
  \end{equation}
\end{teo}
\begin{dem}
  Mostraremos, separadamente, que a condição é necessária e suficiente.
  \begin{enumerate}[a)]
  \item Necessidade. Seja $\varepsilon>0$. Por hipótese, $f_n(x) \to f(x)$ uniformemente, ou seja, existe $N$ tal que
    \begin{equation}
      x\in D, n>N \Rightarrow |f_n(x) - f(x)| < \frac{\varepsilon}{2}.
    \end{equation}
Logo, para este mesmo $N$, temos
\begin{equation}
  x\in D, n,m > N \Rightarrow |f_n(x)-f_m(x)| \leq |f_n(x)-f(x)|+|f_m(x)-f(x)| < \varepsilon.
\end{equation}
Isto mostra que $(f_n)$ satisfaz o critério de Cauchy.
  \item Suficiência. Comecemos construindo nosso candidato a limite. Para cada $x\in D$, $(f_n(x))$ é uma sequência de números reais que, pela hipótese, satisfaz o critério de Cauchy e, portanto, $f_n(x)$ converge quando $n\to \infty$. Seja, então, $f:D\to \mathbb{R}$ a função tal que $f(x)$ é o limite de $f_n(x)$ para cada $x\in D$.
    Mostraremos, agora, que $f_n$ converge uniformemente para $f$. Seja dado $\varepsilon>0$. Por hipótese, existe $N$ tal que
    \begin{equation}\label{eq:afirmação_Cauchy}
      x\in D, ~n,m>N \Rightarrow |f_n(x) - f_m(x)| < \frac{\varepsilon}{2}.
    \end{equation}
Agora, observemos que $f_n(x) - f_m(x) \to f_n(x) - f(x)$ pontualmente quando $m\to \infty$. Logo, passando ao limite na afirmação~\eqref{eq:afirmação_Cauchy}, temos
\begin{equation}
  x\in D, n>N \Rightarrow |f_n(x) - f(x)| \leq \frac{\varepsilon}{2} < \varepsilon,
\end{equation}
o que mostra a convergência uniforme.
  \end{enumerate}
\end{dem}

\begin{ex}
  No exemplo anterior (Exemplo~\ref{ex:convergência_uniforme}~a)) vimos que $f_n(x) = x + 1/n$ converge uniformemente para $f(x) = x$. Aqui, mostraremos que $f_n$ satisfaz o critério de Cauchy para sequência de funções.
  Seja $\varepsilon>0$. Observemos que $1/n\to 0$ quando $n\to \infty$ e, portanto, satisfaz o critério de Cauchy para sequências de números. A saber, existe $N$ tal que
  \begin{equation}
    n,m > N \Rightarrow \left|\frac{1}{n} - \frac{1}{m}\right| < \varepsilon.
  \end{equation}
  Daí, temos também que
  \begin{equation}
    x\in\mathbb{R}, ~n,m>n \Rightarrow |f_n(x) - f_m(x)| = \left|x+\frac{1}{n} - x - \frac{1}{m}\right| = \left|\frac{1}{n} - \frac{1}{m}\right| < \varepsilon.
  \end{equation}
O que concluí que $f_n$ satisfaz o critério de Cauchy.
\end{ex}

\subsection*{Exercícios}

\begin{exer}
  Mostre que a sequência de funções $f_n:\mathbb{R}\setminus \{0\}$, $f_n(x) = 1/(nx)$, converge pontualmente para a função nula $f(x) \equiv 0$.
\end{exer}

\begin{exer}
  Mostre que a sequência de funções $f_n(x) = \cos(x/n)$ converge pontualmente para função constante $f(x)\equiv 1$.
\end{exer}

\begin{exer}
  Mostre que a sequência de funções $f_n(x) = e^{(x-n)^2}$ é pontualmente convergente para $f(x)\equiv 0$.
\end{exer}

\begin{exer}
  Mostre que a sequência de funções $f_n(x) = e^{(x-n)^2}$ não é uniformemente convergente para $f(x)\equiv 0$.
\end{exer}

\begin{exer}
  Mostre que a sequência de funções $f_n:[-1, 1]\to\mathbb{R}$, $f_n(x) = e^{x/n}$, é uniformemente convergente para a função $f(x)\equiv 1$.
\end{exer}

\begin{exer}
  Mostre que a sequência de funções $f_n(x) = x^2/(1 + nx^2)$ satisfaz o critério de convergência de Cauchy para sequências de funções. Dica: observe que $0\leq f_n(x) < 1/n$.
\end{exer}

\section{Algumas consequências da convergência uniforme}

\begin{teo}\label{teo:sf_continuas}
  Seja $f_n:D\to\mathbb{R}$ uma sequência de funções contínuas. Se $f_n$ converge uniformemente para uma função $f:D\to\mathbb{R}$, então $f$ é contínua.
\end{teo}
\begin{dem}
  Primeiramente, observemos que para quaisquer $x,a\in D$ e $n\in \mathbb{N}$, temos
  \begin{equation}\label{eq:sf_desigualdade_triangular}
    \begin{split}
      |f(x)-f(a)| &= |f(x)-f_n(x)+f_n(x)-f_n(a)+f_n(a)-f(a)|\\
      &\leq |f(x)-f_n(x)| + |f_n(x)-f_n(a)| + |f_n(a)-f(a)|.
    \end{split}
  \end{equation}
Seja, então, $\varepsilon>0$. Como $f_n$ converge uniformemente para $f$, existe $N\in\mathbb{N}$ tal que, para todo $n>N$ temos
\begin{equation}
  |f(x)-f_n(x)| < \frac{\varepsilon}{3} \qquad\text{e}\qquad |f_n(a)-f(a)| < \frac{\varepsilon}{3}.
\end{equation}
Fixemos um $n>N$. Como $f_n$ é contínua, para cada $a\in D$, existe $\delta>0$ tal que
\begin{equation}
  x\in D, |x-a|<\delta \Rightarrow |f_n(x) - f_n(a)| < \frac{\varepsilon}{3}.
\end{equation}
Portanto, usando a desigualdade em~\eqref{eq:sf_desigualdade_triangular}, vemos que para cada $a\in D$, existe $\delta>0$ tal que
\begin{equation}
  x\in D, |x-a|<\varepsilon \Rightarrow |f(x) - f(a)| < \varepsilon.
\end{equation}
Isto mostra a continuidade de $f$.
\end{dem}

\begin{teo}\label{teo:sf_integravel}
  Seja $f_n:[a, b]\to\mathbb{R}$ uma sequência de funções contínuas. Se $f_n$ converge uniformemente para uma função $f:[a, b]\to\mathbb{R}$, então
  \begin{equation}
    \lim \int_a^b f_n(x)\,dx = \int_a^b [\lim f_n(x)]\,dx = \int_a^b f(x)\,dx.
  \end{equation}
\end{teo}
\begin{dem}
  Seja $\varepsilon>0$. Como $f_n$ converge uniformemente para $f$, temos que existe $N\in\mathbb{N}$ tal que
  \begin{equation}
    x\in [a, b], n>N \Rightarrow |f_n(x) - f(x)| < \frac{\varepsilon}{(b-a)}.
  \end{equation}
Além disso, do teorema anterior (Teorema~\ref{teo:sf_continuas}), temos que $f$ é contínua em $[a, b]$ e, portanto, assim como $f_n$, é integrável neste intervalo (veja Teorema~\ref{teo:integrabilidade_de_f_contínua}). Por tudo isso, temos
\begin{equation}
  \begin{split}
    n>N \Rightarrow \left|\int_a^b f_n(x)\,dx - \int_a^b f(x)\,dx\right| &= \left|\int_a^b |f_n(x) - f(x)|\,dx\right|\\
    &\leq \frac{\varepsilon}{(b-a)}(b-a) = \varepsilon.
  \end{split}
\end{equation}
O que concluí a demonstração.
\end{dem}

\begin{teo}\label{teo:sf_deriváveis}
  Seja $f_n:[a, b]\to\mathbb{R}$ uma sequência de funções com derivadas contínuas em $[a, b]$, tal que $f_n'$ converge uniformemente para uma função $g:[a, b]\to\mathbb{R}$. Suponhamos, ainda, para algum ponto $c\in [a, b]$, $f_n(c)$ é uma sequência convergente. Então, $f_n$ converge uniformemente para uma função $f$ para a qual $f' = g$, i.e.
  \begin{equation}
    \frac{d}{dx}\lim f_n(x) = \lim \frac{d}{dx}f_n(x).
  \end{equation}
\end{teo}
\begin{dem}
  Do teorema fundamental do cálculo (Teorema~\ref{teo:teo_fundamental_do_cálculo}), para todo $x\in [a, b]$ temos
  \begin{equation}\label{eq:sf_tfc1}
    f_n(x) = f_n(c) + \int_c^x f'_n(t)\,dt.
  \end{equation}
Como, por hipótese, $f_n(c)$ é convergente e $f_n'$ é uniformemente convergente para a função $g$, temos, do teorema anterior (Teorema~\ref{teo:sf_integravel}), que tomando o limite de $n\to \infty$ nesta última equação, obtemos
  \begin{equation}\label{eq:sf_tfc2}
    f(x) = f(c) + \int_c^x g(t)\,dt.
  \end{equation}
Observemos que $f' = g$. Fica de exercício, mostrar que $f_n$ converge uniformemente para $f$, o que completa a demonstração.
\end{dem}

\subsection*{Exercícios}

\begin{exer}
  Complete a demonstração do Teorema~\ref{teo:sf_deriváveis}.
\end{exer}