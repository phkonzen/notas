%Este trabalho está licenciado sob a Licença Atribuição-CompartilhaIgual 4.0 Internacional Creative Commons. Para visualizar uma cópia desta licença, visite http://creativecommons.org/licenses/by-sa/4.0/ ou mande uma carta para Creative Commons, PO Box 1866, Mountain View, CA 94042, USA.

\chapter{Integração}\label{cap:integração}\index{integração}
\thispagestyle{fancy}

\section{Integral de Riemann}\index{integral de!Riemann}

\begin{defn}\normalfont{(Partição de um intervalo)}\index{partição}
  Uma \emph{partição} $P$ de um intervalo $[a, b]$ é um conjunto ordenado da forma
  \begin{equation}
    P([a, b]) = \{a=x_0 < x_1 < x_2 < \ldots < x_n=b\}.
  \end{equation}
O valor $|P| = \max_{1\leq i\leq n} \Delta x_i$, $\Delta x_i = x_i - x_{i-1}$, é chamado de \emph{norma da partição}\index{norma da partição}.
\end{defn}

\begin{defn}\normalfont{(Integral de Riemann)}
  A \emph{integral de Riemann}\index{integral de Riemann} de uma função $f:D\to\mathbb{R}$, $y=f(x)$, num intervalo $[a, b]\subset D$, quando existe, é o número $I$ tal que
  \begin{equation}
     I = \lim_{n\to \infty} \sum_{i=1}^n f(\xi_i)\Delta x_i,
  \end{equation}
onde arbitrariamente $\xi_i\in [x_i, x_{i-1}]$ e $\Delta x_i = x_{i}-x_{i-1}$ são tomados considerando todas as possíveis partições $P([a, b]) = \{a=x_0, x_1, x_2, \dotsc, x_n=b\}$, com $|P|\to 0$ quando $n\to 0$.
\end{defn}

\begin{obs}
  As somas parciais
  \begin{equation}
    S_n = \sum_{i=1}^n f(\xi_i)\Delta x_i
  \end{equation}
que aparecem na definição da integral de Riemann são chamadas de \emph{somas de Riemann}\index{somas de!Riemann}.
\end{obs}

\subsection*{Exercícios}


