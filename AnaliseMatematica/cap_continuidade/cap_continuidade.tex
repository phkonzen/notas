%Este trabalho está licenciado sob a Licença Atribuição-CompartilhaIgual 4.0 Internacional Creative Commons. Para visualizar uma cópia desta licença, visite http://creativecommons.org/licenses/by-sa/4.0/ ou mande uma carta para Creative Commons, PO Box 1866, Mountain View, CA 94042, USA.

\chapter{Continuidade}\label{cap:continuidade}\index{continuidade}
\thispagestyle{fancy}

\begin{defn}\normalfont{(Continuidade)}
  Sejam $f:D\to\mathbb{R}$, $y=f(x)$, e $a$ um ponto de acumulação de $D$. Dizemos que $f$ é \emph{contínua} no ponto $a$ se as sguintes condições são satisfeitas:
  \begin{enumerate}[a)]
  \item $a\in D$.
  \item existe o limite de $f(x)$ com $x\to a$.
  \item $f(x)\to f(a)$ quando $x\to a$.
  \end{enumerate}
Ainda, dizemos que $f$ é uma \emph{função contínua}\index{função contínua} (ou, simplesmente, contínua) quando $f$ é contínua em todos os pontos de seu domínio.
\end{defn}

\begin{ex}
  Vejamos os seguintes casos:
  \begin{enumerate}[a)]
  \item A função $\displaystyle f(x)=\frac{x^2-1}{x+1}$ não é contínua no ponto $x=-1$, pois 
  \end{enumerate}
\end{ex}
