%Este trabalho está licenciado sob a Licença Atribuição-CompartilhaIgual 4.0 Internacional Creative Commons. Para visualizar uma cópia desta licença, visite http://creativecommons.org/licenses/by-sa/4.0/ ou mande uma carta para Creative Commons, PO Box 1866, Mountain View, CA 94042, USA.

\chapter{Limites}\label{cap:limites}\index{limites!de funções}
\thispagestyle{fancy}

\section{Noções de topologia}

\begin{defn}\normalfont{(Ponto interior)}
  Diz-se que $x$ é um \emph{ponto interior}\index{ponto interior} de um dado conjunto $C$ quando existe um intervalo $(a, b)$ que contém $x$ e está contido em $C$, i.e. $x\in (a, b)\subset C$. O conjunto de todos os pontos interiores de $C$ é chamado de seu \emph{interior}\index{conjunto!interior}.
\end{defn}

\begin{ex}
  \begin{enumerate}[a)]
  \item Todo elemento de um intervalo aberto $(a, b)$ é ponto interior deste.
  \item O interior de um dado intervalo fechado $[a, b]$ é o intervalo aberto $(a, b)$.
  \end{enumerate}
\end{ex}

\begin{defn}\normalfont{(Conjunto aberto)}
  Diz se que $C$ é \emph{conjunto aberto}\index{conjunto aberto} quando todos seus elementos são pontos interiores.
\end{defn}

\begin{ex}\label{ex:conjunto_aberto}
  Vejamos os seguintes casos:
  \begin{enumerate}[a)]
  \item O intervalo $(a, b) := \{x\in\mathbb{R};~a<x<b\}$ é um conjunto aberto. De fato, dado $x\in (a, b)$ podemos tomar $0 < \epsilon < \min\{x-a,b-x\}$ de forma que $x\in (x-\epsilon, x+\epsilon)\subset (a, b)$.
  \item O intervalo $(a, b]$ não é aberto, pois $b\in (a, b]$ não é ponto interior.
  \item O conjunto vazio $\emptyset$ é um conjunto aberto. Com efeito, se o conjunto $\emptyset$ não é aberto, então existe um elemento $x\in\emptyset$ que não é ponto interior de $\emptyset$, o que é um absurdo pois $\emptyset$ não contém elementos por definição.
  \item O conjunto dos números racionais $\mathbb{Q}$ não é aberto.
  \end{enumerate}
\end{ex}

\begin{defn}\normalfont{(Vizinhança)}
  Uma \emph{vizinhança}\index{vizinhança} de um dado ponto $x$ é qualquer conjunto $V$ que contenha $x$ como ponto interior. Também, a \emph{vizinhança simétrica}\index{vizinhança!simétrica} de um ponto $x\in\mathbb{R}$ é todo intervalo $V_\epsilon(x) := (x-\epsilon, x+\epsilon)$ com $\epsilon>0$. Mais estrito, a \emph{vizinhança perfurada}\index{vizinhança!perfurada} de $x\in\mathbb{R}$ é uma vizinhança de $x$ que não contém $x$. Aproveitamos para fixar a notação:
  \begin{equation*}
    V'_\epsilon(x) := V_\epsilon(x)\setminus \{x\} = \{y\in\mathbb{R};~0<|x-y|<\epsilon\}.
  \end{equation*}
\end{defn}

\begin{ex}
  Podemos reescrever o Exemplo~\ref{ex:conjunto_aberto} da seguinte forma. Um intervalo $(a, b)$ é um conjunto aberto, pois para cada $x\in (a, b)$ podemos escolher $0<\epsilon<\min\{x-a,b-x\}$ tal que $V_\epsilon(x)\subset (a, b)$.
\end{ex}

\begin{defn}\normalfont{(Ponto de acumulação)}
  Um ponto $x$ é chamado de \emph{ponto de acumulação}\index{ponto de acumulação} de um dado conjunto $C$ quando toda vizinhança de $x$ contém infinitos pontos de $C$.
\end{defn}

\begin{ex}
  Vejamos os seguintes casos:
  \begin{enumerate}[a)]
  \item O número $a$ é ponto de acumulação do intervalo $(a, b]$ não degenerado. De fato, dado $\epsilon>0$, temos $(a, a+\epsilon)\subset V_\epsilon(a)$ e $(a, a+\epsilon)\cap (a, b]$ é um conjunto infinito.
  \item Zero é o único ponto de acumulação do conjunto $\{1, 1/2, 1/3, \dotsc, 1/n, \ldots\}$.
  \end{enumerate}
\end{ex}

\begin{defn}\normalfont{(Ponto isolado)}
  Diz que $x$ é \emph{ponto isolado}\index{ponto!isolado} de um dado conjunto $C$ quando $x\in C$ não é ponto de acumulação de $C$. Diz-se que um conjunto é \emph{discreto}\index{conjunto!discreto} quando todos seus elementos são pontos discretos.
\end{defn}

\begin{ex}
  Vejamos os seguintes casos:
  \begin{enumerate}[a)]
  \item O conjunto dos números naturais $\mathbb{N}$ é discreto.
  \item O conjunto dos números racionais $\mathbb{Q}$ não é discreto.
  \item O conjunto $\{1, 1/2, 1/3, \dotsc, 1/n, \ldots\}$ é discreto.
  \end{enumerate}
\end{ex}

\begin{defn}\normalfont{(Ponto aderente)}
  Dizemos que $x$ é \emph{ponto aderente}\index{ponto aderente} de um dado conjunto $C$ quando toda vizinhança de $x$ contém algum ponto de $C$. O conjunto de todos os pontos aderentes de $C$ é chamado de \emph{fecho}\index{fecho} (ou, conjunto de aderência\index{conjunto de!aderência}) de $C$, o qual denotamos por $\overline{C}$.
\end{defn}

\begin{obs}
  Observe que todo ponto de um conjunto é aderente ao mesmo, bem como, todos os seus pontos de acumulação.
\end{obs}

\begin{ex}
  Vejamos os seguintes casos:
  \begin{enumerate}[a)]
  \item O fecho de $(a, b]$ é o intervalo fechado $[a, b]$.
  \item O conjunto dos números reais $\mathbb{R}$ é o fecho do conjunto dos números racionais $\mathbb{Q}$, i.e. $\overline{Q} = \mathbb{R}$.
  \end{enumerate}
\end{ex}

\begin{defn}\normalfont{Conjunto fechado}
  Dizemos que um conjunto $C$ é \emph{fechado}\index{conjunto!fechado} quando é igual ao seu fecho, i.e. $C = \overline{C}$.
\end{defn}

\begin{ex}
  Vejamos os seguintes casos:
  \begin{enumerate}[a)]
  \item O intervalo $[a, b]$ é um conjunto fechado.
  \item O conjunto vazio $\emptyset$ é fechado. Por quê?
  \item O conjunto dos números reais $\mathbb{R}$ é fechado.
  \item O conjunto dos números racionais $\mathbb{Q}$ não é fechado.
  \end{enumerate}
\end{ex}

\begin{defn}\normalfont{(Conjunto denso)}
  Dizemos que um conjunto $A$ é \emph{denso}\index{denso} no conjunto $B$, quando todo ponto aderente de $\overline{A} \subset B$. 
\end{defn}

\begin{ex}
  O conjunto dos números racionais $\mathbb{Q}$ é denso no conjunto dos números reais $\mathbb{R}$.
\end{ex}

\subsection{Exercícios}

\begin{exer}
  Seja dado um conjunto $C$. Mostre que $x$ é ponto de acumulação de $C$ se, e somente se, toda vizinhança de $x$ contém pelo menos um elemento de $C$ diferente de $x$.
\end{exer}
\begin{resp}
  Basta considerar sucessivas vizinhanças $V_{1/n}(x)$ com $n\in\mathbb{R}$.
\end{resp}

\begin{exer}
  Seja dado um conjunto $C$. Mostre que $x$ é ponto isolado de $C$ se, e somente se, existe uma vizinhança de $x$ tal que $(V(x)\setminus\{x\})\cap C = \emptyset$.
\end{exer}
\begin{resp}
  A implicação segue imediatamente por negação.
\end{resp}

\section{Limites}\index{limite}

\begin{defn}\normalfont{(Limite)}
  Sejam uma função $f:D\to\mathbb{R}$, $y=f(x)$, e $a$ um ponto de acumulação de $D$. Diz-se que $L\in\mathbb{R}$ é o \emph{limite}\index{limite de!função} de $f(x)$ com $x$ tendendo a $a$ se, para todo $\varepsilon>0$, existe $\delta>0$ tal que
  \begin{equation}
    x\in D, 0<|x-a|<\delta \Rightarrow |f(x) - L| < \varepsilon.
  \end{equation}
Quando isso ocorre, escrevemos
\begin{equation}
  \lim_{x\to a} f(x) = L,
\end{equation}
ou ainda, simplesmente, $f(x)\to L$ quando $x\to a$.
\end{defn}

\begin{ex}
  Vejamos os seguintes casos:
  \begin{enumerate}[a)]
  \item Temos $\lim_{x\to 1}x-1=0$. Isto segue imediatamente, pois, neste caso, $f(x)=x-1$, $a=1$, $L=0$ e, então, dado $\varepsilon>0$, tomamos $\delta=\varepsilon$ de forma que
    \begin{equation}
      x\in\mathbb{R}, 0<|x-1|<\delta \Rightarrow |x-1 - 0|<\varepsilon.
    \end{equation}
  \item A função não precisa estar definida no ponto em o limite é tomado. Por exemplo, $\lim_{x\to 1} \frac{x^2-1}{x+1} = 0$. Verifique!
  \end{enumerate}
\end{ex}

\begin{obs}
  Quando nos referirmos a expressão ``x tende a $a$'' (ou similares), estaremos sempre assumindo que $a$ é um ponto de acumulação do domínio da função de interesse.
\end{obs}

\subsection{Propriedades do limite}

\begin{teo}
  Se $f:\mathbb{D}\to\mathbb{R}$, $y=f(x)$, com $\lim_{x\to a}f(x)=L$, então $\lim_{x\to a}|f(x)|=|L|$.
\end{teo}
\begin{dem}
  Seja $\varepsilon>0$. Por hipótese, existe $\delta>0$ tal que $x\in D$, $0<|x-a|<\delta$ implica $|f(x)-L|<\varepsilon$. Tomando, então, um tal $\delta$ e observando que $\left||f(x)|-|L|\right| < |f(x)-L|$, temos que para todo $x\in D$, $0<|x-a|<\delta$, ocorre $\left||f(x)|-|L|\right|<\varepsilon$.
\end{dem}

\begin{teo}\label{teo:lim_imagem}
  Se $f:\mathbb{D}\to\mathbb{R}$, $y=f(x)$, com $\lim_{x\to a}f(x)=L$ e $A < L < B$, então existe $\delta>0$ tal que $x\in D$, $0<|x-a|<\delta$ implica $A<f(x)<B$.
\end{teo}
\begin{dem}
  De fato, por hipótese, para cada $\varepsilon>0$, existe $\delta>0$ tal que $x\in D$, $0<|x-a|<\delta$ implica $|f(x)-L|<\varepsilon$. Então, o resultado segue escolhendo um tal $\delta$ quando $\varepsilon=\min\{L-A, B-L\}$.
\end{dem}

\begin{corol}\normalfont{(Permanência do sinal)}
  Se $f:\mathbb{D}\to\mathbb{R}$, $y=f(x)$, com $\lim_{x\to a}f(x)=L>0$ ($L<0$), então existe $\delta>0$ tal que $x\in D$, $0<|x-a|<\delta$, implica $f(x)>0$ ($f(x)<0)$.
\end{corol}
\begin{dem}
  Quando $L>0$ ($L<0$) basta escolher $A=0$ ($B=0$) no teorema anterior.
\end{dem}

\begin{teo}\normalfont{(Operações com limites)}
  Sejam $f_1,f_2:D\to\mathbb{R}$, $y=f_1(x)$, $y=f_2(x)$, com $\lim_{x\to a}f_1(x)=L_1$ e $\lim_{x\to a}f_2(x)=L_2$, então (omitindo que $x\to a$)
  \begin{enumerate}[a)]
  \item $\lim [f_1(x) + f_2(x)] = \lim f_1(x) + \lim f_2(x)$.
  \item para todo $k\in\mathbb{R}$, temos $\lim kf_1(x) = k\lim f_1(x)$.
  \item $\lim f_1(x)f_2(x) = \lim f_1(x) \cdot \lim f_2(x)$.
  \item $\displaystyle \lim \frac{f_1(x)}{f_2(x)} = \frac{\lim f_1(x)}{\lim f_2(x)}$, quando $L_2\neq 0$.
  \end{enumerate}
\end{teo}
\begin{dem}
  Seja dado $\varepsilon>0$.
  \begin{enumerate}[a)]
  \item Seja $\delta>0$ tal que $x\in D$, $0<|x-a|<\delta$ implica $|f_1(x)-L_1|<\varepsilon/2$ e $|f_2(x)-L_2|<\varepsilon/2$. Logo, para tais $\delta$ e $x$ temos 
    \begin{equation}
      |(f_1(x)+f_2(x)) - (L_1+L_2)| \leq |f_1(x)-L_1|+|f_2(x)-L_2| < \frac{\varepsilon}{2}+\frac{\varepsilon}{2} = \varepsilon.
\end{equation}
  \item O resultado é imediato para $k=0$. Sejam $k\neq 0$ e $\delta>0$ tal que $x\in D$, $0<|x-a|<\delta$ implica $|f_1(x)-L_1|<\varepsilon/|k|$. Então, para tais $\delta$ e $x$ temos $|kf_1(x)-kL_1| = |k||f_1(x)-L_1| < |k|\varepsilon/|k| = \varepsilon$.
  \item Sejam $M>0$ e $\delta>0$ tal que $x\in D$, $0<|x-a|<\delta$ implica $|f_1(x)-L_1| < \varepsilon/(2|L_2|)$, $|f_1(x)| < M$ (veja Teorema~\ref{teo:lim_imagem}) e $|f_2(x)-L_2| < \varepsilon/(2M)$. Então
    \begin{equation}
      \begin{split}
        |f_1(x)f_2(x)-L_1L_2| &= |f_1(x)f_2(x)-f_1(x)L2+f_1(x)L_2-L_1L_2|\\
        &= |f_1(x)(f_2(x)-L_2)+(f_1(x)-L_1)L_2|\\
        &\leq |f_1(x)||f_2(x)-L_2|+|f_1(x)-L_1||L_2|\\
        &< M\frac{\varepsilon}{2M}+\frac{\varepsilon}{2|L_2|}|L_2| = \varepsilon.
      \end{split}
    \end{equation}
  \item De c), basta mostrar que $1/f_2(x)\to 1/L_2$ quando $x\to a$. Para tando, seja $\delta>0$ tal que $x\in D$, $0<|x-a|<\delta$ implica $|f_2(x)-L_2|<\frac{\varepsilon L_2^2}{2}$ e $|f_2(x)|>|L_2|/2$ (veja Teorema~\ref{teo:lim_imagem}). Então, para tais $\delta$ e $x$ temos
    \begin{equation}
      \begin{split}
        \left|\frac{1}{f_2(x)}-\frac{1}{L_2}\right| &= \frac{|f_2(x)-L_2|}{|f_2(x)L_2|}\\
       &< \frac{\frac{\varepsilon L_2^2}{2}}{|L_2|\frac{|L_2|}{2}} = \varepsilon.
      \end{split}
    \end{equation}
  \end{enumerate}
\end{dem}

\subsection*{Exercícios}

\begin{exer}
  Mostre que se $f:\mathbb{D}\to\mathbb{R}$, $y=f(x)$, com $\lim_{x\to a}f(x)=L$, então existe $\delta>0$ tal que $f(x)$ é limitada em $V'_\delta(a)\cap D$.
\end{exer}
\begin{resp}
  Veja o Teorema~\ref{teo:lim_imagem}.
\end{resp}

\begin{exer}
  Mostre que se $f:\mathbb{D}\to\mathbb{R}$, $y=f(x)$, com $\lim_{x\to a}f(x)=L>0$, então $\lim_{x\to a}\sqrt{f(x)}=\sqrt{L}$.
\end{exer}
\begin{resp}
  Use o Teorema~\ref{teo:lim_imagem} observando que
  \begin{equation}
    |\sqrt{f(x)}-\sqrt{L}| = |(\sqrt{f(x)}-\sqrt{L})\frac{\sqrt{f(x)}+\sqrt{L}}{\sqrt{f(x)}+\sqrt{L}}| = \frac{|f(x) - L|}{|\sqrt{f(x)}+\sqrt{L}|}.
  \end{equation}
\end{resp}